\documentclass[10pt]{article}
\setlength{\parskip}{0.25\baselineskip}
\usepackage[margin=1in]{geometry} 
\usepackage{amsmath,amsthm,amssymb, graphicx, multicol, array}
\usepackage[font=small,labelfont=bf]{caption}

\newcommand{\supp}{{\text{supp}}} 
\newcommand{\bv}{{\text{BV}}}
\newcommand{\ac}{{\text{AC}}}

\newenvironment{problem}[2][]{\begin{trivlist}
\item[\hskip \labelsep {\bfseries #1}\hskip \labelsep {\bfseries #2.}]}{\end{trivlist}}

\begin{document}
 
\title{Homework \#3}
\author{Eric Tao\\
Math 233: Homework \#3}
\maketitle

\begin{problem}{Question 1}

Let $u$ be a harmonic function on a region $\Omega$. What can we say about the set of points such that $\nabla u = 0$, that is, the set of points where $u_x = u_y = 0$?

\end{problem}
\begin{proof}[Solution]

Recall that if $u$ is a real harmonic function, then we may identify it as the real part of a holomorphic function $f(x,y) = u(x,y) + i v(x,y)$ locally. Suppose $u_x = u_y = 0$. Then, by the Cauchy-Riemann equations, we have that at these points, $v_x = v_y = 0$. Further, identifying $f'(z) = \partial f(z)$ for $\partial = \frac{1}{2}\left( \frac{\partial}{\partial x} - i \frac{\partial}{\partial y} \right)$, we have that:

$$ f'(z) = \partial f(z) = \frac{1}{2}\left( \frac{\partial}{\partial x} - i \frac{\partial}{\partial y} \right) (u + iv) = \frac{1}{2} \left[ (u_x+ v_y) + i(v_x - u_y) \right]$$

So, we have that at points where $u_x = u_y = 0$, we have that $f'(z) = 0$. But, since $f$ is holomorphic on this neighborhood, so is $f'$. Therefore, $\{ (x,y) : \nabla u(x,y) = 0 \}$ is either all of the neighborhood, or has no limit points. Since $\Omega$ is a region, we can always patch our entire region with overlapping neighborhoods, so this extends to all of $\Omega$.

Now, if $u$ is a complex-valued harmonic function, we simply identify it as $u = w + iv$, where $w, v$ are the real and imaginary portions. It should be clear that if $u$ is harmonic, so must $w,v$ as:

$$u_{xx} + u_{yy} = w_{xx} + i v_{xx} + w_{yy} + v_{yy} = (w_{xx}+ w_{yy}) + i (v_{xx} + v_{yy}) = 0 \implies w_{xx}+ w_{yy} = 0, v_{xx} + v_{yy} = 0 $$

Then, suppose $u_x = u_y = 0$. At such points, we would have that $u_x = w_x + i v_x = 0, u_y = w_y + i v_y = 0 \implies w_x = w_y = 0, v_x = v_y = 0$. But, by the previous work, since $v,w$ are real harmonic functions, they either have no limit points, or are the full space. It should be clear then, that the set of points where $\nabla u = 0$ is simply the union of these sets. It too may only be the full space or not have limit points, as if it did, then we could construct a subsequence of points coming from either the set where $\nabla v = 0$, or  $\nabla w = 0$, which would imply that the original set had a limit point, a contradiction.
\end{proof}

\begin{problem}{Question 2}

Let $u,v$ be real harmonic functions on a plane region $\Omega$. Under what conditions is $uv$ harmonic?

Further, show that $u^2$ may not be harmonic on $\Omega$, unless $u$ is constant. 

Further, for which $f \in \mathcal{H}(\Omega)$ is $| f|^2$ harmonic?

\end{problem}

\begin{proof}[Solution]

We start by proving that if we take the Laplacian of $uv$, $\Delta(uv)$, then this is equal to $2 \nabla u \cdot \nabla v$:

$$ \Delta(uv) = (uv)_{xx} + (uv)_{yy} = (u_x v + uv_x)_x + (u_yv + uv_y)_y = u_{xx}v + u_x v_x + u_x v_x + u v_{xx} + u_{yy} v + u_y v_y + u_y v_y + u v_{yy} $$

Because $u,v$ are harmonic, we know that $u_{xx} + u_{yy} =0, v_{xx} + v_{yy} = 0$, so:

$$ = v(u_{xx} + v_{xx}) + 2 u_x v_x + u (v_{xx} + v_{yy}) + 2 u_y v_y = 2(u_x v_x + u_y v_y) = 2\langle u_x, u_y \rangle \cdot \langle v_x, v_y \rangle = 2 \nabla u \cdot \nabla v$$

Here, it should be clear then that if $u^2$ is not constant, then $u^2$ is not harmonic. We have that $\Delta(u^2) = \Delta(uu) = 2 \nabla u \cdot \nabla u = 2 | \nabla u |^2$. So, suppose $u$ is harmonic, then for $\Delta(u^2) = 0$, this implies that $|\nabla u| = 0$ for all $z \in \Omega$. However, this implies immediately that $u$ is constant, and we have the contrapositive.

Now, of course, if $u$ or $v$ is constant, suppose $u = a$ is constant, then of course $uv = av$ is harmonic, being a scalar multiple of a harmonic function. So, assume $u,v$ both non-constant.

Define the set $A = \{ z \in \Omega : \nabla u (z) = 0 \text{ or } \nabla v(z) = 0 \}$. By the first problem, we know that neither of those sets have limit points in $\Omega$. Since both of those are closed conditions, $A$ is the union of two closed sets, and thus closed. Thus, consider $\Omega' = \Omega \setminus A$. 

This is an open set, of course, being open minus closed, or equivalently, open intersect open. Further, it must be connected, since the points of $A$ have no limit points, and are at most countable. Suppose $x,y \in \Omega'$, and consider a path between them in $\Omega$. This may have at most countably many disconnections when we move to $\Omega'$. Since $A$ has no limit points, we may restrict down into a small enough punctured disk around any connection and take a path there - this punctured disk must be completely contained within $\Omega'$ due to $A$ having no limit points. Since we have merely countably many of these issues, we are assured that we can patch this. Finally, this must be dense because let $U$ be any open set in $\Omega$. Choose any $a \in U$. There exists a disk $D(a,r) \subset U$, with uncountable cardinality. But, $A$ is merely countable, thus $D(a,r) \setminus A \not = \emptyset$. Thus, since $A \cup \Omega' = \Omega$, we must have that $D(a,r) \cap \Omega' \not  = \emptyset$. Thus, we have that $\Omega'$ is a region.

Now, we have that since $\Delta(uv) = 0$, we must have that $u_x v_x + u_y v_y = 0 \implies u_x v_x = -u_y v_y$. Since we wish $uv$ to be harmonic, this must hold for all $z \in \Omega'$, which leads us to two cases, since $u_x, u_y, v_x, v_y \not = 0$ on $\Omega'$:

Case 1:

$$ \begin{cases} v_x = -\lambda u_y \\ v_y = \lambda u_x \end{cases}$$

It should be clear that due to the definition of $\Omega'$, that $\lambda \not = 0 $. In particular, since $u,v$ are harmonic on $\Omega$, they are continuous on all of $\Omega$, with continuous first derivatives. Thus, these must actually hold for all of $\Omega$, since $u_x,u_y, v_x, v_y$. Thus, we can say that the function

$$ f = \lambda u + iv $$ is holomorphic, since these are exactly the Cauchy-Riemann equations for $u' = \lambda u, v' = v$. Thus, in this case, $uv$ is harmonic if we may find a $\lambda$ such that $u,v$ are real and imaginary parts of a holomorphic function.

Case 2:

$$ \begin{cases} u_x = -\lambda u_y \\ v_y = \lambda v_x \end{cases}$$

Consider the first equation. This implies that $u_{xx} = -\lambda u_{yx}$ and $u_{yy} = -\frac{1}{\lambda} u_{xy}$. Thus, in such a case, since $u$ is harmonic, we must have that:

$$ u_{xx} + u_{yy} = 0 \implies  -\lambda u_{yx} - \frac{1}{\lambda} u_{xy} = 0 \implies u_{xy} = 0$$

Similarly:

$$ v_{xx} + v_{yy}= 0 \implies \lambda v_{yx} + \frac{1}{\lambda} v_{xy} = 0 \implies v_{xy} = 0$$

However, since $u_x, u_y \not = 0$ on $\Omega'$, this implies that $u_x = f(x)$ since $u_{xy} = 0$ and $u_y = g(y)$ since $u_{yx}= 0$. Then, we must have that $u =  F(x) + G(y)$ for $F' = f, G' = g$, and due to harmonicity, we further have that $f'(x) + g'(y) = 0$. This can only be true on all of $\Omega'$ if $f', g'$ are constant, which implies that $F, G$ are at most quadratics. However, since we started with $u_x = -\lambda u_y$, this implies that $F'(x) = -\lambda G'(y)$, and if $F,G$ are polynomials, this implies then that $F', G'$ are constants and thus $F,G$ are linear. Thus, we have that:

$$u = -\lambda ax + a y + b$$

Running through the same logic with $v$, we see that:

$$ v =cx +\lambda c y + d $$

However, here, we notice that:

$$\begin{cases} u_x = -\lambda a \\ u_y = a \\ v_x = c \\ v_y = \lambda c \end{cases}$$

Choosing $\lambda' = - \frac{c}{a}$, we see that:

$$\begin{cases} -\lambda' u_y = \frac{c}{a} a = c = v_x \\ \lambda' u_x = -\frac{c}{a} \cdot - \lambda a =\lambda c = v_y \end{cases} $$

and thus we are back in case 1. Thus, in either case, we see that $uv$ is harmonic for $u,v$ non-constant if there exists a $\lambda \not = 0$ such that $\lambda u + iv$ is holomorphic.

Now, let $f \in \mathcal{H}(\Omega)$, and consider $|f|^2$. Explicitly taking derivatives:

$$\frac{\partial^2}{\partial x^2}|f|^2 = \frac{\partial^2}{\partial x^2}( u^2 + v^2 )= \frac{\partial}{\partial x} (2u u_x + 2v v_x) = 2( u_x^2 + u u_{xx} + v_x^2 + v v_{xx}) $$

Of course then, the same equation will hold for the $y$, just switching the labels. Thus:

$$ 2( u_x^2 + u u_{xx} + v_x^2 + v v_{xx}) + 2( u_y^2 + u u_{yy} + v_y^2 + v v_{yy}) = 2( u(u_{xx} + u_{yy}) + u_x^2 + u_y^2 + v(v_{xx} + v_{yy}) + v_x^2 + v_y^2 ) = 2(u_x^2 + u_y^2 + v_x^2 + v_y^2)$$

where we've used the fact that because $u,v$ come from the real, imaginary parts of a holomorphic function, $u,v$ are harmonic.

Now, applying the Cauchy-Riemann equations, we obtain:

$$2(u_x^2 + u_y^2 + v_x^2 + v_y^2) = 2(2 v_x^2 + 2 v_y^2)  = 4 (v_x^2 + v_y^2) = 4(u_x^2 + u_y^2) $$

However, since $u$ is a real-valued function, so must be $u_x, u_y$. Then, since $u_x^2, u_y^2 \geq 0$, for this to be harmonic, we must have $u_x,u_y = 0$. But that implies that $u$ and thus $v$, are constants. Thus, we have that $|f|^2$ is harmonic iff $f$ is constant.
\end{proof}

\begin{problem}{Question 3}

Suppose $f$ is a complex function on a region $\Omega$, and both $f, f^2$ are harmonic on $\Omega$. Prove that either $f, \overline{f}$ must be holomorphic on $\Omega$.

\end{problem}

\begin{proof}[Solution]

It is clear that if $f = a \in\mathbb{C}$, that is, constant, then $f, f^2$ are harmonic and $f, \overline{f}$ are both holomorphic. Thus, we restrict ourselves to $f$ non-constant.

Now, we see that:

$$ \Delta (f^2) = (2ff_x)_x + (2ff_y)_y = 2[ f_x^2 + f_y^2] = 2[ (f_x + i f_y)(f_x - i f_y)] = 2\overline{\partial} f \partial f$$

where, as in the text, we identify:

$$  \partial = \frac{1}{2} \left( \frac{\partial}{\partial x}- i  \frac{\partial}{\partial y}\right) , \overline{\partial} = \frac{1}{2} \left( \frac{\partial}{\partial x}+ i  \frac{\partial}{\partial y}\right)$$

Where we've used the fact that $f$ is harmonic to say that $f(f_{xx} + f_{yy}) = 0$ Now, since $f^2$ is harmonic, we have that $\Delta(f^2) = 0$, which implies that at every point in $\Omega$, either $\partial f = 0$ or $\overline{\partial} f = 0$. Now, consider $\partial f, \overline{\partial}f$. In particular, consider the quanitity $\overline{\partial}( \partial f)$:

%$$ \partial f = \frac{1}{2} (f_x - i f_y), \overline{\partial} f = \frac{1}{2} (f_x+ i f_y)$$

%We notice, these can only be simultaneously 0 only if $f_x = f_y = 0$. Then, by problem 1 of this homework, we have that the set:

%$$Z =  \{ z \in \Omega : \partial f (z) = \overline{\partial} f (z) = 0 \} $$

%contains no limit points, because we have assumed $f$ to be non-constant.

%Now, we note a key point: Consider the quantity $\overline{\partial}( \partial f)$:

$$ \overline{\partial} (\partial f) = \frac{1}{2} \left( \frac{\partial}{\partial x}+ i  \frac{\partial}{\partial y}\right)  \frac{1}{2} (f_x- i f_y) = \frac{1}{4} (f_{xx} + i f_{xy} - i f_{yx} + f_{yy}) = 0 $$

That is, for $f$ harmonic, $\partial f$ is holomorphic on $\Omega$, because the Cauchy-Riemann equations hold. In particular, its zero set is either all of $\Omega$, or a countable subset without limit points. If its zero set is all of $\Omega$, we are done, since if we expand out $\partial f$, we find that:

$$ \partial f = \frac{1}{2} (f_x - i f_y) = \frac{1}{2} (u_x + i v_x - i u_y + v_y) = 0 \implies \begin{cases} u_x  = - v_y \\ u_y = v_x \end{cases}$$

and thus we have that:

$$\overline{\partial}( \overline{f}) =  \frac{1}{2} \left( \frac{\partial}{\partial x}+ i  \frac{\partial}{\partial y}\right) (u - i v) = \frac{1}{2} ( u_x - i v_x + i u_y + v_y) = 0 $$

that is, $\overline{f}$ is holomorphic. Otherwise, suppose $Z = \{ z \in \Omega : \partial f(z) = 0 \}$ has no limit points. Since $f$ harmonic, at least one of $\partial f, \overline{\partial} f = 0$ so on $\Omega \setminus Z$, $\overline{\partial} f = 0$. But, because $Z$ has no limit points, by continuity, $\overline{\partial} f = 0$ actually on all of $\Omega$, where we know this must be continuous, because it is the linear combination of continuous functions. Then, we would have that:

$$ \overline{\partial} f = \frac{1}{2} (f_x+ i f_y) = \frac{1}{2} (u_x + i v_x + i u_y - v_y) \implies  \begin{cases} u_x  = v_y \\ u_y = -v_x \end{cases}$$

and thus

$$ \overline{\partial}(f) = \frac{1}{2} \left( \frac{\partial}{\partial x}+ i  \frac{\partial}{\partial y}\right) (u + i v) = \frac{1}{2} ( u_x + i v_x + i u_y - v_y) = 0$$

that is, $f$ is holomorphic. 

\end{proof}

\begin{problem}{Question 4}

Let $\Omega$ be a region, and $f_n \in \mathcal{H}(\Omega)$ for all $n$. Set $u_n = \Re(f_n)$, and suppose $u_n$ converges uniformly on compact subsets of $\Omega$ and that there exists $z \in \Omega$ such that $f_n(z)$ converges. Prove that $f_n$ converges uniformly on compact subsets of $\Omega$.

\end{problem}
 
\begin{proof}[Solution]

By hypothesis, there exists a $z_0 \in \Omega$ such that $f_n(z_0)$ converges. Since $\Omega$ is open, we may choose an $R > 0$ such that $\overline{D}(z_0,R) \subset \Omega$, since if the disk $D(a,r)$ is contained in $\Omega$, the closed disk $\overline{D}(a,r/2)$ is as well.

Since this is a compact set, and $u_n$ converges uniformly on compact sets, if we set $u = \lim_{n \to \infty} u_n(z)$ for $z \in \overline{D}(z_0, R)$, by theorem 11.11, we have that $u$ is harmonic. Since $u$ is harmonic on $D(z_0,R)$ and continuous on the boundary, we have that by 11.9 that $u$ is the real part of a holomorphic function defined by:

$$ f(z_0 + z) = \frac{1}{2\pi} \int_0^{2\pi} \frac{R e^{it} + z}{R e^{it} - z} u(z_0 + Re^{it}) dt$$

for $|z| < R$.

In the same way, we see that since each $u_n$ is harmonic on the same disk, we may find a sequence of holomorphic functions $g_n$ such that:

$$ g_n(z_0 + z) = \frac{1}{2\pi} \int_0^{2\pi} \frac{R e^{it} + z}{R e^{it} - z} u_n(z_0 + Re^{it}) dt$$

But, $u_n$ is also the real part of $f_n$, also a holomorphic function. Thus, by 11.10, these holomorphic functions may only differ by an imaginary additive constant, and we may say that there exists $c_n \in \mathbb{R}$ such that $f_n = g_n + i c_n$.

First, we wish to show that $g_n \to f$ uniformly for any $r < R$, the closed disk $\overline{D}(z_0,r)$. Let $\epsilon > 0$ be given. Well, by definition, we have that for any point $|z| < r$:

$$ |f - g_n | =  \left |\frac{1}{2\pi} \int_0^{2\pi} \frac{R e^{it} + z}{R e^{it} - z} u(z_0 + Re^{it}) dt - \frac{1}{2\pi} \int_0^{2\pi} \frac{R e^{it} + z}{R e^{it} - z} u_n(z_0 + Re^{it}) dt\right| = $$

$$ \frac{1}{2\pi} \left| \int_0^{2\pi}  \frac{R e^{it} + z}{R e^{it} - z} u(z_0 + Re^{it})  -  \frac{R e^{it} + z}{R e^{it} - z} u_n(z_0 + Re^{it}) dt \right| \leq $$
$$ \frac{1}{2\pi}  \int_0^{2\pi} \left|  \frac{R e^{it} + z}{R e^{it} - z} \right| \left|  u(z_0 + Re^{it}) -  u_n(z_0 + Re^{it})\right| dt$$

Here, we notice that in terms of moduli, $|Re^{it} + z|  \leq |Re^{it}| + |z| \leq R + r$, and similarly, $|Re^{it} - z| \geq R - r$. Thus, we have the estimate:

$$  \left|  \frac{R e^{it} + z}{R e^{it} - z} \right| \leq \frac{R + r}{R - r} $$ for all $| z| < r$. Further, since $\overline{D}(a,r)$ is compact, we have that $u_n \to u$ uniformly.Then, choose $N$ such that for all $n > N$, $|u(z) - u_n(z)| < \epsilon \frac{R - r}{R + r}$. Then, for any $n > N$ and for every $z \in \overline{D}(a,r)$, we have that:

$$| f(z)  - g_n(z) | \leq  \frac{1}{2\pi}  \int_0^{2\pi} \left|  \frac{R e^{it} + z}{R e^{it} - z} \right| \left|  u(z_0 + Re^{it}) -  u_n(z_0 + Re^{it})\right| dt \leq \frac{1}{2\pi}  \int_0^{2\pi}  \frac{R + r}{R - r} \epsilon \frac{R - r}{R + r}  dt = $$

$$ \frac{1}{2\pi}  \int_0^{2\pi} \epsilon dt = \frac{1}{2\pi} \epsilon 2\pi = \epsilon$$

Thus, we have that $g_n \to f$ uniformly for every $\overline{D}(a,r), r < R$.

Next, we restrict our focus to $z_0$. We have that $f_n = g_n + ic_n$. Thus, at $z_0$, since $g_n(z_0) \to f(z_0)$ because of what we showed above, we have that:

$$ f(z_0) = \lim_{n \to \infty} f_n(z_0) = \lim_{n \to \infty} (g_n(z_0) + i c_n ) = \lim_{n \to \infty} g_n(z_0) + i \lim_{n \to \infty} c_n = f(z_0) + i \lim_{n \to\infty} c_n$$

Thus, we have that $\lim_{n \to \infty} c_n$ exists and is equal to 0.

Now, we look at any closed disk $\overline{D}(z_0,r), r < R$ again, and look at $f_n$ this time. Let $\epsilon > 0$ be given. We have that:

$$ \Vert f - f_n \Vert = \Vert f - g_n - i c_n \Vert \leq \Vert f - g_n \Vert + \Vert c_n \Vert$$

Since we have that $g_n \to f$ uniformly,there exists $N_g$ such that for all $n > N_g$, $\Vert f - g_n \Vert < \epsilon/2$. Since $c_n \to 0$ is a sequence of constant numbers, there exists $N_c$ such that for all $n > N_c$, $\Vert c_n \Vert < \epsilon/2$. Then, for any $n > \max(N_g, N_c)$:

$$ \Vert f - f_n \Vert \leq \Vert f - g_n \Vert + \Vert c_n \Vert <  \frac{\epsilon}{2} + \frac{\epsilon}{2} = \epsilon$$

Thus, we have that $f_n \to f$ uniformly for any $\overline{D}(z_0,r), r < R$.

Define $\Omega_1, \Omega_2$ via the following:

$$\Omega_1 = \{ z \in \Omega : \{ f_n(z) \} \text{ converges } \} $$

$$\Omega_2 = \{ z \in \Omega : \{ f_n(z) \} \text{ does not converge } \} $$

From what we've shown above, $\Omega_1$ must be open, since we've shown that there exists a disk around a convergent point $z_0$ such that for any concentric, closed disk contained within this neighborhood, $f_n$ converges uniformly on the closed disk.

However, we notice that $\Omega_2$ must also be open, because we chose a closed disk $\overline{D}(z_0,R)$ to analyze and the complement of such in $\Omega$ is open. Thus, the union of such complements is also open. Further, by definition, $\Omega_1 \cap \Omega_2 = \emptyset$. 

Thus, because $\Omega$ is a region, we must have that either $\Omega = \Omega_1$, or $\Omega = \Omega_2$, and by hypothesis, we see that $z_0 \in \Omega_1 \implies \Omega = \Omega_1$, and thus $f_n \to f$ on all of $\Omega$.

Then, the result is clear. Let $K$ be any compact subset of $\Omega$. For each $k \in K$, we may find $r_k > 0$ such that $D(k,r_k) \subset \Omega$. Consider $\cup_k D(k,r_k)$. Clearly, this is a open cover of $K$, so by the compactness of $K$, there exists a finite subcover

$$ K \subset  \cup_{i =1}^n D(k_i, r_{k_i}) \subset \Omega $$

Then, let $\epsilon > 0$ be given. For each $i$, choose $r_i$ such that $r_i < r_{k_i}$, but that $\cup_{i=1}^n D(k_i, r_i)$ remains a cover of $K$. We may do this because of homework 1, finding the minimum distance between $K$ and the complement of $ \cup_{i =1}^n D(k_i, r_{k_i}) $, a closed set. Then, by the work above, we have that on each $\overline{D}(k_i, r_i)$, that since $f_n \to f$ uniformly, there exists $N_i$ such that for all $n > N_i$, $\Vert f - f_n \Vert < \epsilon$ on $D(k_i, r_i)$. We notice, that there are only finitely many $N_i$ and thus it achieves a maximum. Thus, of course, we have that for $N = \max_i N_i$, for any $n > N$:

$$ \Vert f - f_n \Vert_K  = \Vert f - f_n \Vert_{\overline{D}(k_j, r_j)} < \epsilon $$

for some $k_j, r_j$ since they cover $K$. Thus, $f_n \to f$ uniformly on compact subsets of $\Omega$. 

\end{proof}
  

\begin{problem}{Question 5}
Let $\Omega$ be a region, $K$ a compact subset of $\Omega$, and fix some $z_0 \in \Omega$. Let $u$ be any positive harmonic function. Prove that there exists $\alpha, \beta > 0$ such that

$$ \alpha u(z_0) \leq u(z) \leq \beta u(z_0) $$

for all $z \in K$.

If $\{ u_n \}$ is a sequence of positive harmonic functions in $\Omega$, and $u_n(z_0) \to 0$, describe the behavior of $\{ u_n \}$ on the rest of $\Omega$. Repeat this process for if $u_n(z_0) \to \infty$. Show that $\{ u_n \}$ must be positive.

\end{problem}

\begin{proof}[Solution]

First, fix some $z_0 \in \Omega$. Let $u$ be any positive harmonic function on $\Omega$. Let $z \in \Omega$ be any other point, and let $\gamma$ be a path, $\gamma^* \subset \Omega$ such that $\gamma(0) = z_0, \gamma(1)= z$, which exists since $\Omega$ is connected. Further, assume that $\gamma$ has a finite length. Such a path must exist.  Since the path is a compact set, and the complement of $\Omega$ is closed, this implies that we may find a $R > 0$ such that $D(\zeta,R) \subset \Omega$ for all $\zeta \in \gamma^*$.

Now, consider $\overline{D}(z_0,R/2) \subset \Omega$. If $z \in D(z_0,R/3)$, then we can say that, for $r = |z - z_0|$, that:

$$ \frac{R/2 - r}{R/2 + r} u(z_0) \leq u(z) \leq \frac{R/2 + r}{R/2 - r} u(z_0) \implies   \frac{R/2 - r}{R/2 + r}  \leq \frac{u(z)}{u(z_0)} \leq \frac{R/2 + r}{R/2 - r} \implies  \frac{1}{5} \leq \frac{u(z)}{u(z_0)} \leq 5 $$

Where we use the fact that $\frac{R/2 - r}{R/2 + r}$ is a decreasing function, since as $r \to R/2$, $R/2 - r$ decreases, and $R/2 + r$ increases, so the fraction decreases, so it takes on its minimum value at $r = R/3$. The same logic applies for the upper bound, as an increasing function.

Otherwise, take the boundary $\partial D(z_0,R/3)$. Since $z$ is not contained within $D(z_0,R/2) \supset  D(z_0,R/3)$, there exists at least some point $\zeta \in \gamma^*$ such that $\zeta \in \gamma^* \cap \partial D(z_0,R/3)$. If there are multiple such $\zeta$, we choose the one corresponding to the largest parameter in $\gamma(t)$. Calling this point $\zeta_1$, we have that:

$$  \frac{R/2 - R/3}{R/2 + R/3} u(z_0) \leq u(\zeta_1) \leq \frac{R/2 +R/3}{R/2 - R/3} u(z_0) \implies \frac{1}{5} u(z_0) \leq u(\zeta_1) \leq 5u(z_0) $$

Now, we repeat this process for $\zeta_1$ taking the role of $z_0$. If $z$ is contained within $D(\zeta_1,R/3)$, then we have that, for $r = |z - \zeta_1|$ again:

$$  \frac{R/2 - r}{R/2 + r} u(\zeta_1) \leq u(z) \leq \frac{R/2 + r}{R/2 - r} u(\zeta_1) \implies  \frac{1}{5}^2  u(z_0) \leq u(z) \leq  5^2u(z_0) \implies  \frac{1}{5}^2  \leq  \frac{u(z)}{u(z_0)} \leq  5^2 $$

Otherwise, choose $\zeta_2$ in the same fashion as $\zeta_1$, by looking at the boundary  $\partial D(\zeta_1,R/3)$. Importantly,this process must terminate in a finite amount of steps - in particular, it must terminate in at most $\lceil L/(R/3)\rceil = \lceil \frac{3L}{R} \rceil$ steps, where $L$ is the path length of $\gamma$. Thus, letting $n =  \lceil \frac{3L}{R} \rceil$ , we have for our estimate, that:

$$  \frac{1}{5}^n  \leq  \frac{u(z)}{u(z_0)} \leq  5^n$$

We notice that this is actually independent of $u$, and is strictly a function of the geometry.

Consider the functions defined for $z \in \Omega$:

$$ B(z) = \sup_u \frac{u(z)}{u(z_0)}$$

$$ b(z) = \inf_u \frac{u(z)}{u(z_0)} $$



\end{proof}

\end{document}