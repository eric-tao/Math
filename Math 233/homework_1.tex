\documentclass[10pt]{article}
\setlength{\parskip}{0.25\baselineskip}
\usepackage[margin=1in]{geometry} 
\usepackage{amsmath,amsthm,amssymb, graphicx, multicol, array}
\usepackage[font=small,labelfont=bf]{caption}

\newcommand{\supp}{{\text{supp}}} 
\newcommand{\bv}{{\text{BV}}}
\newcommand{\ac}{{\text{AC}}}

\newenvironment{problem}[2][]{\begin{trivlist}
\item[\hskip \labelsep {\bfseries #1}\hskip \labelsep {\bfseries #2.}]}{\end{trivlist}}

\begin{document}
 
\title{Homework \#1}
\author{Eric Tao\\
Math 233: Homework \#1}
\maketitle

\begin{problem}{Question 1}
The following fact was tacitly used in this chapter: if $A, B$ are disjoint subsets of the plane, $A$ is compact, $B$ is closed, then there exists a $\delta > 0$ such that, for all $\alpha \in A$, $\beta \in B$, $| \alpha - \beta | \geq \delta > 0$. Prove this for $A, B \subset X$ for $X$ an arbitrary metric space. 

\end{problem}
\begin{proof}[Solution]

Let $X$ be a metric space, $A \subseteq X$ compact, $B \subseteq X$ closed, $A \cap B = \emptyset$

Suppose not. Then, there exist pairs of points $(\alpha_n, \beta_n)$ such that $d(\alpha_n,\beta_n) < \frac{1}{n}$. Now, consider the sequence of points $\{ \alpha_n \}_{n=1}^\infty$. Since $A$ is compact, we know that there exists a subsequence $\{ \alpha_{n_k} \}_{k=1}^\infty$, convergent to $\alpha$. 

Let $\epsilon > 0$ be given. Since $\alpha_{n_k} \to \alpha$, choose $N_k$ such that $d(\alpha, \alpha_{n_k}) < \frac{\epsilon}{2}$ for all $n_k > N_k$. Choose $N$ such that $\frac{1}{n} < \frac{\epsilon}{2}$ for all $n > N$. Choose $M_k$ such that $M = \max(N, N_k)$. Assume $m > M, m \in \{ n_k \}_{k=1}^\infty$. Consider the sequence of $\{ \beta_{n_k} \}_{k=1}^\infty$, and in particular, consider:

$$ d(\alpha, \beta_m) \leq d(\alpha, \alpha_m) + d(\alpha_m, \beta_m) < \frac{\epsilon}{2} + \frac{\epsilon}{2} = \epsilon$$

Thus, we have that $\beta_{n_k} \to \alpha$. Since $\{ \beta_{n_k} \}_{k=1}^\infty\subset B$, a closed set, $ \alpha \subset B$, because closed sets contain its limit points. But, this is a contradiction. Thus, $\delta > 0$ exists.
\end{proof}

\begin{problem}{Question 3}

Suppose $f, g$ are entire functions, and suppose that for all $z \in \mathbb{C}$, that $| f(z) | \leq | g(z)|$. What conclusion can you draw?

\end{problem}

\begin{proof}[Solution]

Claim: for some $m \in \mathbb{C}$, $f = mg$.

First suppose $g = 0$. Then, since $|f| \leq |g| = 0$, this implies that $f = 0$ everywhere. Then, of course $f = mg$, for actually any $m$.

Now, suppose not. Then, define $Z(g) = \{ z \in \mathbb{C} : g(z) = 0 \}$, that is, the zero set of $g$, and consider the function $h = \frac{f}{g}$. By the algebra of holomorphic functions, we have that $h$ is holomorphic on at least $\mathbb{C} \setminus Z(g)$.

Because $\mathbb{C}$ is of course a connected open set, we have the result that $Z(g)$ has no limit points in $\mathbb{C}$. Then, let $a \in Z(g)$. Because $a$ is not a limit point, there exists $r > 0$ such that $D(a,r) \cap Z(g) = \emptyset$. We have then that $h$ is holomorphic on $D(a,r) \setminus \{ a \}$, a region. Further, on $\mathbb{C} \setminus Z(g)$, we have that $|h| = \frac{|f|}{|g|} \leq 1$. So, in particular, on $D'(a,\frac{r}{2}) = \{ z \in \mathbb{C} : 0 < |z - a| < \frac{r}{2} \} \subseteq \mathbb{C} \setminus Z(g)$, we have that $h$ is bounded. Then, by Theorem 10.20 from Rudin, we have that $f$ has a removable singularity at $a$.

Now, we recall from Theorem 10.18, that $Z(g)$ is at most countable. So, we may patch $h$ countably many times at each point in $Z(g)$ to produce a holomorphic function everywhere, which we call $\tilde{h}$. Further, since $\tilde{h}$ is holomorphic, it must be continuous everywhere. Thus, since $|\tilde{h}(z)| \leq 1$ at every point other than $z \in Z(g)$, we must have that $|\tilde{h}(z)| \leq 1$ everywhere by continuity. Thus, we have that $\tilde{h}$ is a bounded, entire function, and by Liouville's Theorem, it must be constant, that is, $\tilde{h} = k$ for some $k \in \mathbb{C}$. Then, we have that at least on $\mathbb{C} \setminus Z(g)$, that $f(z) = k g(z)$.

However, $k g(z)$ is certainly holomorphic, and it agrees with $f(z)$ almost everywhere, which of course is a set with limit points in $\Omega$. Thus, $f = kg$ everywhere.


\end{proof}

\begin{problem}{Question 4}
Suppose that $f$ is an entire function, and

$$ | f(z) | \leq A + B|z|^k$$

for all $z$, where $A, B, k$ are positive real numbers. Prove that $f$ must be polynomial.

\end{problem}

\begin{proof}[Solution]

Because $f$ is entire, it is analytic, specifcally at $a = 0$, with infinite radius of convergence. Then, we may rewrite $f$ as:

$$f(z) = \sum_{n=0}^\infty c_n z^n $$

Now, we apply Theorem 10.22. We have that:

$$ \sum_{n=0}^\infty |c_n |^2 r^{2n} = \frac{1}{2\pi} \int_{-\pi}^\pi | f(re^{i\theta})|^2 d\theta $$

Here, we use our hypothesis. Since we have that $ | f(z) | \leq A + B|z|^k$, we must have that:

$$ |f(r e^{i\theta}) | \leq A + B |re^{i\theta}|^k = A + Br^k$$

Thus, using our first equation then, we have a bound:

$$  \sum_{n=0}^\infty |c_n |^2 r^{2n}  = \frac{1}{2\pi} \int_{-\pi}^\pi | f(re^{i\theta})|^2 d\theta \leq  \frac{1}{2\pi} \int_{-\pi}^\pi (A + Br^k)^2 d\theta = (A + Br^k)^2 $$

Now, suppose we have that $c_n \not = 0$ for some $n > k$. Then, we would have that:

$$\frac{|c_n| r^{2n}}{(A + Br^k)^2} = \frac{|c_n| r^{2(n - k)}}{(\frac{A}{r^k} + B)^2}$$

Now, since $f$ is entire and thus the radius of convergence is infinite,  we may take the limit as $r \to \infty$. But, since $n > k$, we have that:

$$ \lim_{r \to \infty} \frac{|c_n| r^{2(n - k)}}{(\frac{A}{r^k} + B)^2} = \infty$$

Then, $c_n = 0$ for every $n > k$. Then, this implies that we have that

$$ f(z) = \sum_{n=0}^{ \lfloor k \rfloor } c_n z^n $$

and since this holds everywhere, with finite degree, $f$ is polynomial.

\end{proof}

\begin{problem}{Question 6}
There is a region $\Omega$ such that $\text{exp}(\Omega) = D(1,1)$. Show that the exponential function is one-to-one on $\Omega$, but that there are many such $\Omega$. Fix one, and define $\text{log}(z)$, for $|z-1| < 1$ to be $w \in \Omega$ such that $e^w = z$. Prove that $\text{log}'(z) = \frac{1}{z}$. Further, find the coefficients $a_n$ in

$$ \frac{1}{z} = \sum_{n=0}^\infty a_n(z-1)^n$$

and hence, find the coefficients $c_n$ in the expansion

$$ \text{log} z = \sum_{n=0}^\infty c_n (z-1)^n$$.

In what other discs can this be done?
\end{problem}

\begin{proof}[Solution]

First, we find the shape of one such $\Omega$. We notice that $D(1,1) = \{ z : |z-1| < 1 \}$. Thus, we would have that, for $x, y \in \mathbb{R}, z = x + yi$:

$$ |e^z - 1| < 1 \implies |  e^{x}(\cos(y) + i \sin(y)) - 1 | < 1 \implies | e^{x}\cos(y) - 1 + i e^{x} \sin{y} | < 1 \implies \sqrt{ e^{2x} - 2e^{x}\cos(y) + 1} < 1 $$

However, we make one more observation, that first of all $e^z$ is cyclic in the imaginary component $y$, with a period of $2\pi$. Further, we have that in terms of the radial component, $D(1,1)$ is completely contained within $(-\pi/2,\pi/2)$, and that other regions may be found, but they are separated by integer multiples of $2\pi$ and therefore disconnected from this one. So, then, we have that we may describe our region as $\Omega = \{ x + yi :  \sqrt{ e^{2x} - 2e^{x}\cos(y) + 1} < 1. y \in (-\pi/2,\pi/2) \}$. We notice that this must be 1:1 because the exponential $e^z = e^x(\cos(y) + i \sin(y))$ must be 1:1 on $y \in (-\pi/2,\pi/2)$:

$$ e^x(\cos(y)+ i \sin(y)) = e^{x'} (\cos(y') + i \sin(y') \implies \begin{cases}e^x \cos(y) = e^{x'} \cos(y') \\ e^{x}\sin(y)  = e^{x'}\sin(y')\end{cases}$$

$$\implies \begin{cases}e^{2x} \cos^2(y) = e^{2x'} \cos^2(y') \\ e^{2x}\sin^2(y)  = e^{2x'}\sin^2(y')\end{cases} \implies e^{2x} = e^{2x}( \cos^2(y) + \sin^2(y) ) =e^{2x'}( \cos^2(y') + \sin^2(y') ) = e^{2x'} $$

Thus, we have that $x = x'$. Now, we note that on $(-\pi/2, \pi/2)$, $\sin$ is 1:1, so therefore 

$$ e^x \sin(y) = e^{x'} \sin(y') \implies \sin(y) = \sin(y') \implies y= y'$$

Thus, we have that $e^z$ is one-to-one on this region.

As noted earlier though, we notice that we can find another region easily - $\Omega' = \{ x + yi :  \sqrt{ e^{2x} - 2e^{x}\cos(y) + 1} < 1. y \in (3\pi/2,5\pi/2) \}$ is certainly another valid region, and there are actually infinitely many, separated by $2\pi n, n \in \mathbb{N}$.

Now, choose an arbitrary one of these $\Omega$. Define $\log z = w \in \Omega$ such that $e^w = z$, where $z \in D(1,1)$. It should be clear that because the exponential in injective on $\Omega$, $\log$ must be injective on $D(1,1)$.

%Fix some point $z_0 \in D(1,1)$. Clearly, we have that the exponential is holomorphic on $\Omega$, since it is entire, and since it is non-0 everywhere, and its own derivative, we have that $e'(z_0) \not = 0$. Then, we have a neighborhood on 

Fix some $z_0 \in D(1,1)$. Since the exponential is one-to-one on $\Omega$, this corresponds to $\log(z_0) = w_0$. Then, for arbitrary $w \in \Omega$ and $z \in D(1,1)$, we have that:


$$ \frac{\log(z) - \log(z_0)}{z - z_0} = \frac{ w - w_0}{e^w - e^{w_0}} $$

just by the injective nature of these functions and using the fact that $z = e^w, w = \log(z)$.

Then, by the continuity of the exponential, we have that as $w \to w_0$, that $z \to z_0$. Thus, we have that, by taking the limit of both sides as $w \to w_0, z \to z_0$:

$$ \frac{\log(z) - \log(z_0)}{z - z_0} = \frac{ w - w_0}{e^w - e^{w_0}} \implies \lim_{z\to z_0} \frac{\log(z) - \log(z_0)}{z - z_0} = \lim_{w\to w_0} \frac{ w - w_0}{e^w - e^{w_0}} \implies \log'(z_0) = \frac{1}{e^{w_0}}$$

But, by our definition of $\log$, $e^{w_0}$ is exactly $z_0$. So $\log'(z_0) = \frac{1}{z_0}$ as desired. 

Well, now we notice that $\frac{1}{z}$ is holomorphic on regions that exclude the origin, thus we can use the corollary to 10.6 to compute coefficients to our power series. Since $$\frac{d^n}{dz^n}\frac{1}{z} = \frac{(-1)^n n!}{ z^{n+1}}$$

we have that

$$n! c_n = f^{(n)}(1) \implies c_n = \frac{1}{n!}\frac{(-1)^n n!}{ 1^{n+1}} = -1^n$$

So, we have that

$$ \frac{1}{z} = \sum_{n=0}^\infty (-1)^n(z-1)^n$$

Doing term by term integration, then, we have that the n-th term becomes:

$$\int (-1)^n(z-1)^n dz = \frac{(-1)^n}{n+1}(z-1)^{n+1} $$

up to a constant. So, then, we have that:

$$\log(z) = c_0 + \sum_{n=0}^\infty \frac{(-1)^n}{n+1}(z-1)^{n+1}$$

where $c_0$ is some constant connected to the choice of $\Omega$.

We remark that this procedure is not special to the disk $D(1,1)$, but rather, is permissable on any disk that does not include the origin, as if it does, there is no $z$ such that $e^z = 0$. In such a case, the inverse function defined on a region missing the origin would have a pole at $0$, and discs that include $0$ may come from a single region, and may be hard to restrict to a disconnected domain. For example, the disk $D(0,1)$ has, as a preimage under the exponential, $\{ x + yi : x \leq 0 \}$, which has a many to 1 relation with $D(0,1)$.

\end{proof}


\begin{problem}{Question 7}

Let $f \in \mathcal{H}(\Omega)$. Under certain conditions on $z, \Gamma$, we have that:

$$ f^{(n)}(z) = \frac{n!}{2\pi i} \int_\Gamma \frac{f(\zeta)}{(\zeta - z)^{n+1}} d\zeta$$

for $n \in \mathbb{N}$. State these, and prove the formula.

\end{problem}

\begin{proof}[Solution]

Before we start, we will prove a necessary result. We will prove that for a closed cycle $\Gamma$, an open set $\Omega$, and $z \in \Omega \setminus \Gamma^*$, that $\int_{\Gamma} (z - \zeta)^{-m} d\zeta = 0$ for $m  > 1$. 

Well, since $z \not \in \Gamma^*$, we have that $(z - \zeta)^{-m}$ is continuous, and thus integrable on $\Gamma^*$. In particular, since we have that $m > 1$, it has exactly anti-derivative $F(\zeta) = \frac{(z-\zeta)^{-m+1}}{-m+1}$. Then, if $\Gamma = \gamma_1 + \gamma_2 + ... + \gamma_n$, and if the endpoints of $\gamma_i$ are $\alpha_i, \beta_i$, we can rewrite this as:

$$ \int_{\Gamma} (z - \zeta)^{-m} d\zeta = \sum_{i=1}^n \int_{\gamma_i} (z - \zeta)^{-m} d\zeta = \sum_{i=1}^n F(\beta_i) - F(\alpha_i) = 0 $$

because since $\Gamma$ is a closed cycle, we must have that $\beta_i = \alpha_{i+1}$ with the understanding that $\beta_n = \alpha_1$. So, this is a telescoping sum and vanishes, at least when $m > 1$. Note that although we may prove this for really, more generally, $m \not = 1$ in the same manner, this is all we need here.

We will need that $z \in \Omega \setminus \Gamma^*$, so that we can take a contour integral over $\Gamma$, as well as $\text{Ind}_\Gamma(z) = 1$ and $\text{Ind}_\Gamma(\alpha) = 0$ when $\alpha \not \in \Omega$ for use in Cauchy's theorem.

First, fix some $z \in \Omega \setminus \Gamma^*$, and choose some $n \in \mathbb{N}$. Define the related function $P(\zeta)$ on $\Omega$ via:

$$ P(\zeta) = f(z) + f'(z)(\zeta - z) + ... + \frac{f^{(n-1)}(z)}{(n-1)!} (\zeta - z)^{n-1} $$

First, we show that $f(\zeta) - P(\zeta) = (\zeta - n)^n h(\zeta)$:

Since $f$ is holomorphic and thus analytic, we can write a power series for $f$ around $z$:

$$ f(\zeta) = f(z) + f'(z) (\zeta - z) + ... =  \sum_{n=0}^\infty \frac{f^{(i)}(z)}{i!} (\zeta - z)^i $$

where the coefficients come from the corollary to theorem 10.6.

Then, we can compute:

$$f(\zeta) - P(\zeta) =  \sum_{i=0}^\infty \frac{f^{(i)}(z)}{i!} (\zeta - z)^i - \sum_{i=0}^n \frac{f^{(i)}(z)}{i!} (\zeta - z)^i = \sum_{i=n}^\infty  \frac{f^{(i)}(z)}{i!} (\zeta - z)^i = (\zeta - z)^n \sum_{i=n}^\infty  \frac{f^{(i)}(z)}{i!} (\zeta - z)^{i-n}  $$

Identifying $h(\zeta) =  \sum_{i=n}^\infty  \frac{f^{(i)}(z)}{i!} (\zeta - z)^{i-n}$, we notice that:

$$ h(z) = \sum_{i=n}^\infty \frac{f^{(i)}(z)}{i!} (z - z)^{i-n} = \frac{f^{(n)}(z)}{n!}$$.

Next, we claim that:

$$\int_\Gamma \frac{P(\zeta)}{(\zeta - z)^{n+1}} d\zeta = 0$$.

Well:

$$\int_\Gamma \frac{P(\zeta)}{(\zeta - z)^{n+1}} d\zeta = \int_\Gamma \sum_{i=0}^{n-1} \frac{f^{(i)}(z) (\zeta - z)^{i}}{i!(\zeta - z)^{n+1}} d\zeta = \sum_{i=0}^{n-1} \int_\Gamma  \frac{f^{(i)}(z)}{i!(\zeta - z)^{n+1 - i}} d\zeta =  \sum_{i=0}^{n-1} \frac{f^{(i)}(z)}{i!} \int_\Gamma (\zeta - z)^{i - n - 1} d\zeta  $$ 

Since for all $0 \leq i \leq n-1$, we have that $-n-1 \leq i -n - 1 \leq -2$ and since $z \not \in \Gamma^*$, we apply the lemma we proved at the beginning to show that $ \int_\Gamma (\zeta - z)^{i - n - 1} d\zeta = 0$ for all $i$. Thus, the entire integral vanishes, and we have that:

$$ \int_\Gamma \frac{P(\zeta)}{(\zeta - z)^{n+1}} d\zeta = 0$$

Now, consider the following quantity:

$$ \int_{\Gamma} \frac{f(\zeta)}{(\zeta - z)^{n+1}} d\zeta = \int_\Gamma \frac{ P(\zeta) + (\zeta - n)^n h(\zeta)}{(\zeta - z)^{n+1}} d\zeta = \int_\Gamma \frac{ P(\zeta))}{(\zeta - z)^{n+1}} d\zeta +  \int_\Gamma \frac{ h(\zeta)(\zeta - n)^n}{(\zeta - z)^{n+1}} d\zeta  =  \int_\Gamma \frac{ h(\zeta)}{(\zeta - z)} d\zeta $$

where we applied the facts that $\int_\Gamma \frac{P(\zeta)}{(\zeta - z)^{n+1}} d\zeta = 0$ and $f(\zeta) - P(\zeta) = (\zeta - n)^n h(\zeta) \implies f(\zeta)  =  P(\zeta) + (\zeta - n)^n h(\zeta) $.

Now, since $h$ is analytic, it must be holomorphic on $\Omega$. Further, due to our conditions on $z$ and $\Gamma$, we may apply Cauchy's theorem to claim that:

$$h(z) * \text{Ind}_\Gamma(z) = \frac{1}{2\pi i} \int_\Gamma \frac{h(\zeta)}{\zeta - z} = \frac{1}{2\pi i}  \int_{\Gamma} \frac{f(\zeta)}{(\zeta - z)^{n+1}} d\zeta $$

Now, we use the condition that $\text{Ind}_\Gamma(z) = 1$, and the result that $h(z) = \frac{f^{(n)}(z)}{n!}$ to get that:

$$ \frac{f^{(n)}(z)}{n!} = \frac{1}{2\pi i}  \int_{\Gamma} \frac{f(\zeta)}{(\zeta - z)^{n+1}} d\zeta \implies f^{(n)}(z) = \frac{n!}{2\pi i}  \int_{\Gamma} \frac{f(\zeta)}{(\zeta - z)^{n+1}}$$

Since the choice of $n$ was arbitrary, we can use this procedure for any natural number $n$, the desired result.


\end{proof}

\end{document}