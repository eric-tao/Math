\documentclass[10pt]{article}
\setlength{\parskip}{0.25\baselineskip}
\usepackage[margin=1in]{geometry} 
\usepackage{amsmath,amsthm,amssymb, graphicx, multicol, array}
\usepackage[font=small,labelfont=bf]{caption}
 
\newenvironment{problem}[2][Problem]{\begin{trivlist}
\item[\hskip \labelsep {\bfseries #1}\hskip \labelsep {\bfseries #2.}]}{\end{trivlist}}

\begin{document}
 
\title{Assignment}
\author{Eric Tao\\
Math 240: Homework \#3}
\maketitle
 
\begin{problem}{3.1}

Let $X \subset \mathbb{A}^n, Y \subset \mathbb{A}^m$ be algebraic sets. Consider $X \times Y \subset \mathbb{A}^n \times \mathbb{A}^m \cong \mathbb{A}^{n+m}$

(a) Show that $X \times Y$ is an algebraic subset of $\mathbb{A}^{m+n}$.

(b) Show that if either $X$ or $Y$ are reducible, then $X \times Y$ is reducible.

(c) Show that if both $X,Y$ are irreducible, then $X \times Y$ is irreducible.

(d) Compute the dimension of $X \times Y$ in terms of the dimensions of $X$ and $Y$.

\end{problem}

\begin{proof}[Solution]

(a)

Let $I = <f_1, f_2,...,f_i>$ be the radical ideal in $k[x_1,...,x_n]$ such that $V(I) = X \subseteq \mathbb{A}^n$, and let $J = < g_1,g_2,...,g_j>$ be the radical ideal in $k[x_1,...,x_m]$ such that $V(J) = Y \subseteq \mathbb{A}^m$. Consider, the ideal generated by $\overline{I} = <f_1, f_2,...,f_i>\subseteq k[x_1,...,x_{m+n}]$ where we take the $f_k$ to be variables in the first $n$ variables. The zero set of this ideal will have points that look like $V(\overline{I}) = \{ (x,y) : x \in X, y \in \mathbb{A}^m \} = X \times \mathbb{A}^m$ due to $X$ being the zero set in the first $n$ variables, and the other variables being free. Analogously, we have the same to be true for $Y$, that is, we may take $\overline{J} = <g_1,...,g_j> \subseteq k[x_1,...,x_{m+n}]$ where we take the polynomials to only be in the last $m$ variables, and free otherwise, and $V(\overline{J}) = \{ (x,y) : x \in \mathbb{A}^n, y \in Y \} = \mathbb{A}^n \times Y$. Consider now the ideal generated by $<\overline{I}, \overline{J} >  = < f_1,...,f_i, g_1,...,g_j>$. We claim that $V(<\overline{I}, \overline{J} >)  = X \times Y$, as a point vanishes on any polynomial in the span when it belongs to a point both in $X$ and in $Y$ as a Cartesian product.

Let $x \in V(<\overline{I}, \overline{J} >)$. Then, $x$ is the zero of every polynomial in this ideal. In particular, looking at the generators, this implies that $x$ is 0 on every polynomial in the generators. This implies that $x \in \{ (x,y) : x \in X, y \in \mathbb{A}^m \}$ and $x \in  \{ (x,y) : x \in \mathbb{A}^n, y \in Y \} $, which implies that $x$ is in their intersection, $x \in ( X \times \mathbb{A}^m) \cap (\mathbb{A}^n \times Y) = X \times Y$. Now, suppose $z = (x,y) \in X \times Y$, where we associate the first $n$ coordinates with $X$ and the last $m$ with $Y$. Since $f_k((x,y)) = 0$ for all $k$, since $X \times Y \subseteq X \times \mathbb{A}^m$ and same for the $g_k$, we have that $z$ is in the zero set of $(<\overline{I}, \overline{J} >)$. Thus, we have that $X \times Y$ is the zero set of some ideal of polynomials, and is thus algebraic.

(b)

Suppose, without loss of generality, that $X$ is reducible. Then, $ X = X_1 \cup X_2$, for $X_1, X_2$ closed, and $X_1 \not = X$, $X_2 \not = X$. Then, we may find two ideals in $k[x_1,...,x_n]$, such that $Z(I_1) = X_1$ and $Z(I_2) = X_2$. By the structure we set up in part (a), then, we can see that we can construct $X_1 \times Y$ and $X_2 \times Y$ from the ideals $< \overline{I_1}, \overline{J} >$ and $< \overline{I_2}, \overline{J} >$. Then, we have that $X \times Y = X_1 \times Y \cup X_2 \times Y$ due to our hypothesis that $X = X_1 \cup X_2$, and because $X \not = X_1$ or $X \not = X_2$, $X \times Y \not = X_1 \times Y$ and  $X \times Y \not = X_2 \times Y$.

(c)

Suppose we have closed sets $Z_1, Z_2 \subseteq  Z_1 \cup Z_2$ such that $Z_1 \cup Z_2 = X \times Y$. Consider the subset $S_y = X \times \{ y \}$ for some fixed element $y \in Y$. This must be contained within one of $Z_1$ or $Z_2$ as, suppose not, then we would have $X_1 \times \{y\} \subseteq Z_1$ and $X_2 \times \{y\} \subseteq Z_2$, and we've found two closed sets $X_1, X_2$ that union to $X$, but neither are the full space. WLOG, suppose $X \times \{y \} \subseteq Z_1$. Now, consider the set of $\{ y \in Y : X \times \{ y \} \subseteq Z_1\}$. If this is all of $Y$, then we are done, otherwise, suppose not. Then, we have that due to the irreducibility of $Y$, then $\{ y \in Y : X \times \{ y \} \subseteq Z_2\} = Y$, as otherwise we've found two closed sets that join up to $Y$ and neither are all of $Y$. But if that's true, then by construction, $Z_2  = X \times Y$ and we are done.

(d)

Let $I$ be the ideal in $k[x_1,...,x_n]$ such that $V(I) = X  \subseteq \mathbb{A}^n$, and let $J = < g_1,g_2,...,g_j>$ be the radical ideal in $k[x_1,...,x_m]$ such that $V(J) = Y \subseteq \mathbb{A}^m$. Consider the ideal in $k[x_1,x_2,...,x_{m+n}]$ generated by the image $\overline{I}$ under inclusion where we associate the variables in $I$ with the first $m$ variables. In particular, the generators are exactly the same, since we just include them into a larger space. This must also be true for the image of $I_y$, where we associate those $n$ variables with the last $n$ variables. Now, consider the degree of $k[x_1,x_2,...,x_{m+n}]/<\overline{I}, \overline{J}>$. In particular, we notice here that because the generators of $\overline{I}$ are polynomials only in the first $m$ variables, and $\overline{J}$ in the last $n$ variables, then modding out by $\overline{I}$ will not affect the last $m$ variables, and vice versa for $\overline{J}$. In particular, if we call $\dim(X) = a, \dim(Y) = b$, construct the map that sends $k[x_1,x_2,...,x_{m+n}] \to k[y_1,y_2,...,y_{a+b}]$ that sends a polynomial $f(x_1...,x_n) \to [f]$ modulo $I$ identified with the first $a$ variables, a polynomial $f(x_{n+1},...,x_{m+n}) \to [g]$ modulo $J$ identified with the last $b$ variables, and extend this linearly. This must be surjective, as we know that $k[x_1,...,x_n]/I \cong k[x_1,..,x_a]$ as k-algebras by definition of the degree, and same with the last $b$ variables. Then, for any monomial in $g \in  k[y_1,y_2,...,y_{a+b}]$, we can idenitfy a coset $[a] \in k[x_1,...,x_n]/I$ and one in $[b] \in k[x_1,...,x_m]/J$ such that $[a]*[b] = g$ from our identification, which we can lift to a polynomial in $k[x_1,x_2,...,x_{m+n}]$. Further, the kernel is exactly those things that go to the zero coset in either, which is exactly the ideal generated by $<\overline{I}, \overline{J}>$. So, then, the dimensionality of $X \times Y$ is the trascendental degree of $k[x_1,x_2,...,x_{m+n}]/<\overline{I}, \overline{J}>$ which by our isomorphism is $\dim(X) + \dim(Y)$.

\end{proof}

\begin{problem}{3.2}

Let $X$ be a closed set in $\mathbb{A}^n$. Consider $\mathbb{A}^n$ as a linear subspace of $\mathbb{A}^m$, for $m > n$ by taking the last $m - n$ coordinates equal to $0$. Let $P = (0,...,1) \in \mathbb{A}^m$. Define the set $Y \subseteq \mathbb{A}^m$ via:

$$ Y = \{ Q \in \mathbb{A}^m \setminus \{ P \} : \text{ the line } \overline{PQ} \cap X \not = \emptyset \} \cup \{ P \}$$

(a) Show that $Y$ is an algebraic subset of $\mathbb{A}^m$.

(b) Compute the dimension of $Y$.

\end{problem}

\begin{proof}[Solution]

(a)

Let $a = (a_1,...,a_m)$ be a point in $Y$, and consider the line $\overline{ap}$ that we can express parametrically as $h(t) = at + p(1-t)$ for $t \in k$. Consider the intersection of the line with a point in $x$, which must exist due to construction. By our embedding, we have that the points in $X$ have form $(x_1,...,x_n,0,...0)$. Then, we know that this intersects $X$ when $a_n*t + (1-t) = 0$, which implies that $t = \frac{1}{1 - a_m}$. Then, we have the following set of coordinates on the line: $h_i = \frac{a_i}{1 - a_m}$ for $i \leq n$, $h_j = 0$ for $n < j < m$, and $h_m$ is free.

Now, since $X$ is an algebraically closed set in $\mathbb{A}^n$, take $X = V(I)$ for some ideal $I \subseteq k[x_1,...,x_n]$. In particular, for each $f \in I$ we may then write $f(\frac{x_1}{1 - x_m},\frac{x_2}{1 - x_m},...,\frac{x_n}{1 - x_m}) \in  k[x_1,...,x_n]$. We may clear the denominators by multiplying through via $(1-x_m)^k$ for some $k$ to recover $(1-x_m)^kf(\frac{x_1}{1 - x_m},\frac{x_2}{1 - x_m},...,\frac{x_n}{1 - x_m}) \in k[x_1,...,x_n,x_m]$. Because this vanishes on $X$ and we've merely introduced a parametrization on a line intersecting $X$, $f$ vanishes on any copy of $X$ scaled along our line through $\overline{ap}$. Therefore, the cone $Y$ is algebraic.
%First, fix a point $x_0 \in X$, consider lines of the form $\overline{x_0p}$. We can write these in parametric form as $x_0t + p(1-t)$ where $t$ takes on any value in $k$. Expanding this, in terms of the coordinates of $\mathbb{A}^m$, we have that $x_i = x_{0_i}t$ for $i \leq n$, $x_i = 0$ for $n < i < m$ and $x_m = (1-t)$. Rearranging, we see that we can if $x_m = 1-t$, then $t = 1 - x_m$. Then, we may find polynomials of the form $x_i - x_{0_i}(1 - x_m)$ for $i \leq n$ and $x_j$ for $n < j < m$. We see that by construction, the line is the zero set of these polynomials. Since $x_j$ are polynomials, then $x_j = 0$ for all  $n < j < m$, which matches. We take $x_m$ to be free, and parametrize it via $t$. Then, we have that the other polynomials take on forms like $x_i - x_{0_i}(1 - t)$, so this is 0 when $x_i  =  x_{0_i}(1 - t)$. We see that when $t$ attains the value $1$, we have that $x_i = 0$ and $x_m = 1$, so we have the point $p$. We also see when $t$ attains the value $0$, that $x_i = x_{0_i}$ and $x_m = 0$, which is our point $x_0$. Since we can see that we are only dependent on $x_m$ in one variable, then we must have a line and we recover our two points. Now, take $I_x = < x_i - x_{0_i}(1 - x_m), x_j>$ where $i < n, n < j< m$. Then, we can construct the ideal $\cap_{x \in X} I_x$. This cannot be empty, because every ideal contains $0$. Further, it cannot be trivial, as there are points in the corresponding space that do not belong to any line through $\overline{xp}$ for any $x$. Take $(1,0,0,....,1)$ for example. 

%Now, we claim that $Y$ is the union of all such lines from $p$ to points $x \in X$. First, take the line $\overline{xp}$. Clearly, this has non trivial intersection with $X$ as $x \in X$. Then, this line is contained within $Y$. Now, suppose we have a point $y \in Y$. Case 1: $y = p$. Then, take any point $x \in X$, the line $\overline{xp}$ contains $p$, so this is within the union of lines from points in $X$ to $p$. Now, suppose not. Then, by definition, the line $\overline{yp}$ has non trivial intersection with $X$. Then, there exists an $x_0 \subseteq \overline{yp}$. But then, consider the line $\overline{x_0p}$. Since 2 points determine a line, and $x_0, p \in \overline{yp}$ and $\overline{x_0p}$, then $ \overline{yp} = \overline{x_0p}$, so $y \in \overline{x_0p}$, and thus, $y$ is in some line from a point $x \in X$ to $p$. Thus, these sets are equal, and since each line is algebraic from the work above, it corresponds to the intersection of such ideals because $I(V_1 \cup V_2) = I(V_1) \cap I(V_2)$ from 2.2.12 in Osserman. 

(b)

Since the dimension of $X$ is well defined, take $X = V(I)$ as above, we know that $k[x_1,...,x_n]/ I$ has trascendental degree dim($X$) over $k$ as a $k$-algebra. In particular, since the point $P$ is linearly independent of every point in $X$, we claim that $\dim(Y) > \dim(X)$. However, we also see from part (a) that we can express the zero set of $Y$ as a polynomial in $k[x_1,...,x_n,x_m]$, and thus, one extra variable, therefore the transcendental degree can be at most one more than the transcendental degree of $X$ because we add one new variable. Thus, $\dim(Y) = \dim(X) + 1$.

 $k[x_1,...,x_n,x_m]/ I$

\end{proof}

\begin{problem}{3.3}

Fix a polynomial $f_0 \in k[x_1,...,x_n]$ without multiple irreducible factors. Define $A = \{ \frac{g}{f_0^k} \}$ for $k \in \mathbb{N}$, where we identify $\frac{g}{f_0^k} \sim \frac{g'}{f_0^{k'}}$ if $gf_0^{k'} = g' f_0^k$.

(a) Show that $A$ is a ring with a natural addition and multiplication.

(b) Show that there exists a natural injective morphism $k[x_1,...,x_n] \to A$.

(c) Show that the prime ideals of $A$ are in bijection with the prime ideals of $k[x_1,...,x_n]$ that do not contain $f_0$.

(d) Let $U$ be the open set of $\mathbb{A}^n$ given by $f_0 \not = 0$. Show that the ring of regular functions in $U$ is identified with $A$.

\end{problem}

\begin{proof}[Solution]
(a)

Well, we claim that we can define an addition via:

$$ \frac{g}{f_0^k} + \frac{g'}{f_0^{k'}} = \frac{f_0^{k'} g + f_0^k g'}{f_0^{k + k'}}$$

and a multiplication via:

$$  \frac{g}{f_0^k} * \frac{g'}{f_0^{k'}} = \frac{gg'}{f_0^{k + k'}} $$

Firstly, we wish to check that this is well-defined. Suppose we have that $ \frac{g}{f_0^k} \sim \frac{g''}{f_0^{k''}} \implies gf_0^{k''} = g'' f_0^k$. Consider $\frac{f_0^{k'} g + f_0^k g'}{f_0^{k + k'}}$ and $\frac{f_0^{k'} g'' + f_0^{k''} g'}{f_0^{k'' + k'}}$. In particular, consider $(f_0^{k'} g + f_0^k g) * (f_0^{k'' + k'})$ and  $(f_0^{k'} g'' + f_0^{k''} g') * (f_0^{k + k'})$. We have that:

$$(f_0^{k'} g + f_0^k g') * (f_0^{k'' + k'}) = (f_0^{2k' + k''}g + f_0^{k + k' + k''} g') = (f_0^{2k'}g''f_0^k + f_0^{k + k' + k''} g') =  (f_0^{2k' + k}g'' + f_0^{k + k' + k''} g') $$

and

$$ (f_0^{k'} g'' + f_0^{k''} g') * (f_0^{k + k'}) = (f_0^{k + 2k'} g'' + f_0^{k + k' + k''} g') $$

So addition is well-defined.

Doing the same for multiplication, we have that:

$$ gg'f_0^{k'' + k'} = (gf_0^{k''}) g' f_0^{k'} = g'' f_0^k g' f_0^{k'} = g'' g'f_0^{k' + k} $$

and

$$ g''g'f_0^{k + k'} $$

Now, we have that these operations are associative, commutative, etc from the inheritance on polynomial addition/multiplication. So, we need only check that we have a group under addition, a multiplicative identity exists, and that multiplication distributes over addition.

We identify $\frac{0}{1}$ as an additive identity, where we denote $0,1$ as the zero and one from $k[x_1,...,x_n]$, because $\frac{0}{1} + \frac{g}{f_0^k} = \frac{0 + g}{f_0^k} = \frac{g}{f_0^k}$

It should be clear that every element has an additive inverse. For any $\frac{g}{f_0^k}$, take $\frac{-g}{f_0^k}$, where $-g$ is the additive inverse of $g$ in $k[x_1,...,x_n]$. Then, we have:

$$ \frac{g}{f_0^k} + \frac{-g}{f_0^k} = \frac{f_0^kg + f_0^k (-g)}{f_0^{2k}} = \frac{f_0^kg + (-g)}{f_0^{2k}} = \frac{0}{f_0^{2k}} $$

which we see as equivalent to $\frac{0}{f_0^0} = \frac{0}{1}$, as $ 1 * 0 = f_0^{2k} * 0$.

We identify $\frac{1}{1}$ as a multiplicative identity, as $\frac{g}{f_0^k} * \frac{1}{1} = \frac{g*1}{f_0^k * 1} = \frac{g}{f_0^k}$.

Now, we just need to show that this multiplication distributes over addition. Here, we notice that $\frac{f_0^k}{f_0^k} \sim \frac{1}{1}$.

$$ \frac{g}{f_0^k} ( \frac{g'}{f_0^{k'}} + \frac{g''}{f_0^{k''}}) = \frac{g}{f_0^k} * \frac{f_0^{k'} g'' + f_0^{k''} g'}{f_0^{k'' + k'}} = \frac{gg''f_0^{k'} + gg'f_0^{k''}}{f_0^{k + k' + k''}}* \frac{f_0^k}{f_0^k} = \frac{gg''f_0^{k+k'} + gg'f_0^{k+k''}}{f_0^{(k + k')}f_0^{( k+ k'')}} =  \frac{gg''}{f_0^{k+k''}} +  \frac{gg'}{f_0^{k+k'}} $$

Thus, $A$ has a ring structure.

(b)

Take the morphism $i: k[x_1,...,x_n] \to A$ that sends a polynomial $f \to \frac{f}{1}$. 

We may verify that this is actually a ring hom: 

$$i(f) + i(g) = \frac{f}{1} + \frac{g}{1} = \frac{f*1 + g*1}{1*1} = \frac{f+g}{1} = i(f+g)$$

and

$$ i(f)i(g) = \frac{f}{1} * \frac{g}{1} = \frac{f*g}{1*1} = \frac{fg}{1} = i(fg)$$.

Further, this must be injective. Suppose $i(f) = i(f')$. Then, we have that $\frac{f}{1} = \frac{f'}{1}$. This implies that $\frac{f}{1} + \frac{-f'}{1} = \frac{0}{1} \implies \frac{f - f'}{1} = \frac{0}{1}$. Then, we have that $f - f' = 0 \in k[x_1,...,x_n]$, and thus that $f = f'$.

(c)

This is exactly the same as what we did in class with localizations of rings. However, let's write it out.

Firstly, via our injective morphism $ \phi: k[x_1,...,x_n] \to A$, we can bring any prime ideal $P$ to $\phi(P)$ via $f \to \frac{f}{1}$. Clearly, if $f_0^k \in P$, then $\phi(P)$ is trivial, since $\frac{f_0^k}{1} \in \phi(P)$ and $\frac{f_0^k}{1} * \frac{1}{f_0^k} = \frac{f_0^k}{f_0^k} \sim \frac{1}{f_0^0} = \frac{1}{1}$ since $f_0^k = f_0^0 f_0^k = f_0^k * 1  = f_0^k$. Then, a unit is in $\phi(P)$ and $\phi(P) = A$. So we can assume first that $f_0^k \not \in P$ for any $k \geq 0$. Now, consider the prime ideal spanned by $\phi(P)$, $S = \{ f_0^k : k \in \mathbb{N} \}$ called $S^{-1}P$. From class, we showed that this must have form $\{ [ \frac{p}{f_0^k} ] : p \in P, k \in \mathbb{N} \}$. These must be distinct, because suppose not, then, two images have the same span, but because our original morphism was injective, they must come from the same elements in $k[x_1,...,x_n]$. Further, these clearly pull back to the same preimage under $\phi^{-1}$, since for any $p \in P$, we havethat $[\frac{p}{1}] \in S^{-1}P$, so $p \in \phi^{-1}(S^{-1}P)$, and if $p \in \phi^{-1}(S^{-1}P)$, then we can say $p =  \phi^{-1}(a)$ for some $a \in S^{-1}P$. But by the construction, $a = \frac{p}{1}$, which implies that $p \in P$. 

Now, instead, suppose we have a prime ideal in $A$ called $P_A$. We can see that $\phi^{-1}(P_A)$ is a prime ideal. Suppose $x,y \in \phi^{-1}(P_A)$, then $\phi(x-y) = \phi(x) - \phi(y) \in P_A$, so $x-y \in \phi^{-1}(P_A)$. and we have that it is a subring. Further, let $f \in k[x_1,...,x_n]$, and take $x \in \phi^{-1}(P_A)$. Then, $\phi(xf) = \phi(x) \phi(f) \in P_A$, because $\phi(x) \in P_A$ and $P_A$ an ideal, so $xf \in \phi^{-1}(P_A)$. So this has the property of multiplicative absorption. Further, $f_0^k$ cannot be in this because if it were, then $\phi(f_0^k) \in P_A$, which we've shown to be a unit, and then $P_A$ is trivial. Now, consider $S^{-1}\phi^{-1}(P_A)$. It is clear that $S^{-1}\phi^{-1}(P_A) \subseteq P_A$. Now suppose we have a $\frac{a}{b} \in P_A$. Then, we notice that since $\frac{b}{1} \in A$, then $\frac{ab}{b} \sim \frac{a}{1} \in P_A$. Then, we have that $a \in \phi^{-1}(P_A)$, and then of course, $\frac{a}{b} \in S^{-1}\phi^{-1}(P_A)$, because $\frac{a}{1} \in \phi( \phi^{-1}(P_A))$. Further, this must be injective because if $\phi^{-1}(P) = \phi^{-1}(P')$, then traveling back by $\phi$ into $A$, they would generate the same span, and if the generating set for two ideals are equal, the original ideals were equal. So, by these two constructions, we have a one to one association that takes a prime ideal in $A$ to $\phi^{-1}(A)$ and a prime ideal in $k[x_1,...,x_n]$ to the prime ideal spanned by its image in $A$.

(d)

Recall that the ring of regular functions on $U \subseteq \mathbb{A}^n$ is defined as functions $\phi: V \to \mathbb{A}^1$ for $V \subseteq U$ an open neighborhood and acting via $\phi: x \to \frac{f}{g}(x)$, that is, via some rational functions of polynomials such that $g(x) \not = 0$ on $V$. Here, we notice something: since $f_0$ has only a single irreducible factor, call it $h$, then $h \in \text{rad}(<f_0>)$. In particular, since $h$ is the only irreducible polynomial that divides $f_0$, then $<h> = \text{rad}(<f_0>)$ and $f_0 = c h^j$ for some field element $c$ and some number $j$. Then, for any regular function, we may take a $g$ that's defined on the whole space that vanishes on the zeros of $f_0$.Now, let $f/g$ be a regular function defined on all of $U$, which we can do, because if it were defined piecewise, we may find a smaller open set that it agrees on piece by piece.. Then, if $g$ is non-constant, $g$ may only be $0$ on the zeroes of $f_0$. Then, we have that $g$ is exactly a power of $h$. In particular, since $k[x_1,...,x_n]$ is a UFD, we may identify a power $i$ such that $g = c'h^i$ and for $c'$ a field element. Then, we may identify a regular function $f/g$ as $m/f_0^k$, where$k$ is the smallest natural such that $jk - i > 0$, and $m = c'^{-1}c^k f h^{jk - i}$. We may confirm this: $gm = c'h^{i} * c'^{-1}c^k f h^{jk -i} = c^kfh^{jk} = f_0^k f$. 
 

\end{proof}


\end{document}