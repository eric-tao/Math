\documentclass[10pt]{article}
\setlength{\parskip}{0.25\baselineskip}
\usepackage[margin=1in]{geometry} 
\usepackage{amsmath,amsthm,amssymb, graphicx, multicol, array}
\usepackage[font=small,labelfont=bf]{caption}
 
\newenvironment{problem}[2][Problem]{\begin{trivlist}
\item[\hskip \labelsep {\bfseries #1}\hskip \labelsep {\bfseries #2.}]}{\end{trivlist}}

\begin{document}
 
\title{Assignment}
\author{Eric Tao\\
Math 240: Homework \#4}
\maketitle
 
\begin{problem}{4.1}

Let $X \subset \mathbb{A}^n, Y \subset \mathbb{A}^m$ be affine varieties, $\phi: X \to Y$ a regular map of affine varieties.

(a) Show that $\phi$ is continuous with respect to the Zariski topology.

(b) Show that the graph of $\phi$ is an affine subvariety of $X \times Y \subset \mathbb{A}^n \times \mathbb{A}^m$.

\end{problem}

\begin{proof}[Solution]

(a)

Let $V_y$ be a closed set in $Y$, and suppose it is realized as the zero set of $J \subseteq k[y_1,...,y_m]$. Because $\phi$ is a morphism of affine spaces, we may express it as $\phi(x_1,...,x_n) = (f_1(x_1,...,x_n),...,f_m(x_1,...x_n))$ for $f_i$ members of the coordinate ring $A(X)$. It also induces a map $\phi^*: A(Y) \to A(X)$ that sends a polynomial $g(y_1,...,y_m) \to g(f_1(x_1,...,x_n),...,f_m(x_1,...,x_n))$. Call $<\phi^*(J)>$ the ideal generated by the image under $\phi^*$ of the generators of $J$ in $k[x_1,...,x_n]$. We claim that $\phi^{-1}(V_y) = Z(<\phi^*(J)>)$, that is, the inverse image of $V_y$ is the zero set of the ideal generated by the image of $J$.

First, suppose we have $x \in \phi^{-1}(V_y)$. Then, we know that $\phi(x)$ vanishes on $J$, specifically, it vanishes on the generators of $J$. Looking at the action of $\phi$, this implies that, for generators $g_j$ of $J$, $g_j(f_1(x),...,f_m(x)) = 0$. However, these are exactly the image of the generators of $J$ under the map $\phi^*$. Then, $x$ must vanish under each $\phi^*(g_j)$, which implies that $x \in  Z(<\phi^*(J)>)$.

Now, suppose we have $x \in Z(<\phi^*(J)>)$. Then, for each generator $g_j \in J$, $x$ vanishes, so we have that $g_j(f_1,...,f_m) = 0$ at $x$. Now, consider the point $\phi(x) = (f_1,f_2,...,f_m)$. By what we just said about being in the zero set, this point vanishes at each $g_j$, and since $V_y$ is realized as the vanishing locus of $J$, and $\phi(x) \in V_y$. Thus, $x \in \phi{-1}(V_y)$. Thus, we have that $\phi^{-1}(V_y) = Z(<\phi^*(J)>)$, so $\phi^{-1}(V_y)$ is closed. Since the inverse image of closed is closed, we have that $\phi$ is continuous.

(b)

Consider the map given from $\psi: X \to X \times Y$ given by $\psi(x_1,...,x_n) = (x_1,...,x_n,f_1(x_1,...,x_n),...,f_m(x_1,...,x_n))$ where $\phi$ is a regular map of affine varieties, and $\phi$ is realized as $\phi(x) = (f_1(x),...,f_m(x))$ for $f_i$ regular functions on $X$. Well, $x_1,...,x_n$ are regular functions on $X$, and so are $f_i$, by construction. Then, since we can realize $\psi$ as a collection of regular functions, $\psi$ is a regular map. From part (a), we have then that $\psi$ is cts, and, since $X$ is an affine variety, and thus an irreducible set, we have that $\psi(X)$ is irreducible. Now, we need only show that this is closed, i.e. realized as the zero locus of polynomials in $k[x_1,...,x_n,y_1,...y_m]$. For each $y_i$, we notice that since $y_i = f_i$ for some regular function on $X$, then $y_i = g_i/h_i$ for $g_i,h_i \in k[x_1,...,x_n]$. Then, let $I \subseteq k[x_1,..,x_n]$ be an ideal such that $Z(I) = X$, and let $i: k[x_1,...,x_n] \to k[x_1,...,x_n,y_1,...y_m]$ be the natural inclusion. We claim that we may realize $X \times \phi(X)$ as the locus of the ideal generated by $< i(I), h_i y - g_i >$ for all $1 \leq i \leq m$. It is clear that $X \times \phi(X) \subseteq Z(< i(I), h_i y - g_i >)$, because of how we map $X \to \phi(X)$ via regular functions. Take any point in $X \times \phi(X)$. This has form $(x_1,...,x_n,f_1,...,f_m)$ where $f_i$ is shorthand for $f_i(x_1,..,x_n)$. Then, this must vanish on the generators of $i(I)$ because $X$ is the zero set of $I$, and must vanish on each $ h_i y - g_i $, by construction of the regular functions. Now, suppose we have a point in the zero set $ Z(< i(I), h_i y - g_i >)$. Then, in particular, it must vanish on its generators. Then, for each $x_i$, we know that since the image of the generators of $I$ lives in the ideal, the $x_i$'s must belong to $X$. Further, looking at each $y_i$, we have that $h_i y = g_i$. Since by construction, $h_i \not = 0$ on $X$, then we can say that $y_i$ takes on the value of a regular function $g_i/h_i$. However, that is exactly the image of $\phi$. Then, we have set equality, so the graph is a closed set, and irreducible from what we said earlier, and therefore is an affine variety.

\end{proof}

\begin{problem}{4.2}

Let $C$ be the affine curve in $\mathbb{A}^2$ with equation $y^2 = x^3$, and $\phi: \mathbb{A}^1 \to C$ given by $\phi(t) = (t^2,t^3)$.

(a) Show that $\phi$ is continuous in the Zarisky topology.

(b) Show that $\phi$ is a bijection, and that its inverse is also continuous in the Zariski topology.

(c) Show that the rings of functions of $\mathbb{A}^1$ and $C$ are not isomorphic.

(d) Show that $\phi$ is not an isomorphism.


\end{problem}

\begin{proof}[Solution]

(a) 

We see that $\phi$ can be realized as regular functions on $\mathbb{A}^1$ to $C$, with $f_1(t) = t^2$ and $f_2(t) = t^3$, which we are told that $C$  is an affine curve. Then, by question 1, $\phi$ is continuous.

(b)

%It is clear that $\phi$ must be injective, because $t^3$ is injective. 

Suppose $(t^2,t^3) = (t'^2,t'^3)$. If we're in the reals, $t^3$ is injective, so we are done. Otherwise, suppose we work in the complex numbers. Then, we may express $t = r e^{i\theta}, t' = r'e^{i \theta'}$, for $r,r',\theta,\theta'$ real numbers. Then, we have the following equations:

$$r^2 = r'^2, r^3 = r'^3, 2\theta = 2\theta' + 2\pi k, 3\theta = 3\theta' + 2\pi k'$$

for $k,k' \in \mathbb{Z}$. We see that $r = r'$ since the cubic is injective for real numbers. Further, we see that we have $\theta = \theta' + \pi k, \theta = \theta' + \frac{2}{3} \pi k' \implies \pi k = \frac{2}{3} \pi k' \implies 3k = 2k'$, or $k = k' = 0$. Then, $3 | k'$ and $2 | k$, so we have that $\pi k$  and $\frac{2}{3}\pi k'$ must be a multiple of $2\pi$, so $e^{i \theta} = e^{i \theta'}$. Thus, $\phi$ is injective.

Now, suppose we have $(x_0,y_0) \in \mathbb{A}^2 : y_0^2 = x_0^3$. Let $z = y_0^2 = x_0^3$. Since we can take $k$ to be algebraically closed, view $x_0$ as one of the third roots of $z$ and $y_0$ to be one of the second roots of $z$. If $z = r e^{i\theta}$, then either $x_0 = \sqrt{r} e^{i (\theta + k)/3}$ for $k= 0,2\pi, 4\pi$, and $y_0 = \sqrt[3]{r} e^{i (\theta + k') /2}$ for $k' = 0, 2\pi$. Choose $t_0 = \sqrt[6]{r} e^{i (\theta + k'')/6}$, where we have:

$$k'' = \begin{cases} 0, & k' = 0, k  = 0\\
2\pi, & k' = 2\pi, k = 2\pi \\
4\pi, & k' = 0, k = 4\pi \\ 
6\pi, & k' = 2\pi, k = 0 \\
8\pi, & k' = 0, k = 2\pi \\
10\pi, & k' = 2\pi, k = 4\pi \\ \end{cases} $$

It is not hard to see that for each combination of $k,k'$, for the choice of $k''$, $t_0^2 = x_0$ and $t_0^3 = y_0$ as desired.

Thus, $\phi$ is surjective, and moreover, bijective.

We may now construct $\phi^{-1}$ in this way, sending a point $(x_0,y_0) \to t_0$ via $y_0/x_0$ for an open set $x_0 \not = 0$ and $0$ at $(0,0)$. We can see this works as the inverse of the action of $\phi$, since for $x_0 \not = 0$, $y_0/x_0 = t^3/t^2 = t$, and for $x_0 = 0$, $x_0 = y_0 = t = 0$. This must be continuous in the Zariski topology - let $V \subseteq \mathbb{A}^1$ be a closed subset. Then, $V$ is either all of $ \mathbb{A}^1$ or a finite set in $ \mathbb{A}^1$. If $V$ is all of $C$, since $\phi$ is a bijection of sets, $\phi^{-1}$ is as well, so $(\phi^{-1})^{-1}(V) = C$. Else, by cardinality, $(\phi^{-1})^{-1}(V)$ is a finite set of points in $C$. But, since closed sets in $C$ are finite collections of points as well, this is closed. Thus $\phi^{-1}$ is continous.

(c)

%Suppose we permit an isomorphism $\psi: k[x,y]/(y^2 - x^3) \to k[x]$. Then, consider the irreducible elements $[x],[y] \in  k[x,y]/(y^2 - x^3)$, which must be irreducible because suppose we have polynomials such that $[f]*[g] = [x]$. Then, in the full polynomial ring, we have $[h(y^2-x^3) + f] *  [h'(y^2-x^3) + g] = h''(y^2 - x^3) + x $ for some polynomials $h,h', h''$.

Suppose we permit a surjective homomorphism $\psi: k[x,y] \to k[x]$ such that $y^2 - x^3 \in ker(\psi)$. Suppose $\psi(x) = f, \psi(y) = g$. Since $y^2 - x^3$ is in the kernel, then we have that $g^2 - f^3 = 0$. This may only be true then if $2 * \deg(g) = 3 * \deg(f)$, so then either $\deg(f) = \deg(g) = 0$, or $2 | \deg(f)$ and $3 | \deg(g)$. In the first case, if that were true, then our hom cannot be surjective, as then by hom properties, if $\deg(\psi(x)) = \deg(\psi(y)) = 0$, then $\deg(h) = 0$ for any $h \in k[x,y]$. Then, suppose $2 | \deg(f)$. In particular, we can say that $\deg(f) \geq 2$. Then, since we may take $k$ to be algebraically closed, every polynomial splits in $k[x]$, so we may find $a \in k$ such that $(x-a) | f$. Now, by hypothesis, $\psi$ is surjective, so then for the quantity $(x-a)*h = f$, we may find $\psi^{-1}(x-a) * \psi^{-1}(h) = x$. However, $x$ is irreducible in $k[x,y]$. This means that either $\psi^{-1}(x-a)$ or $\psi^{-1}(h)$ is invertible. In the first case, if $\psi^{-1}(x-a)$ is invertible, then that would imply $x-a$ invertible in $k[x]$, since we could find $z : z *  \psi^{-1}(x-a) = 1 \in k[x,y]$, and taking the map $\psi$, we would find an inverse of $(x-a)$. But $x-a$ is irreducible in $k[x]$, so that is impossible. In the second case, if $\psi^{-1}(h)$ were invertible, then again, we could find $z : z*\psi^{-1}(h) = 1 \implies \psi(z)* h = 1$ for some $z \in k$, since the invertible elements of $k[x,y]$ are field elements. But then the degree of $h$ is 0, because the invertible elements in $k[x]$ are also field elements, which contradicts the degree of $f$. Thus, we cannot permit any surjective hom from $k[x,y] \to k[x]$ such that $(y^2 - x^3)$ is in the kernel, so we cannot have permit an isomorphism from $k[x,y]/(y^2 - x^3) to k[x]$.

(d)

Suppose not. Suppose $\phi$ were an isomorphism. Then, we can find a $\phi^{-1}$ a morphism of varieties. We then look at the respective induced maps, $\phi^*, \phi^{-1^*}$. These would provide maps on the coordinate rings such that they are inverses of each other. But, from part (c), we see that there cannot exist an isomorphism between $A(C)$ and $k[x]$. Thus, we cannot find a $\phi^{-1}$ a morphism of varieties, and $\phi$ cannot be an isomorphism.


\end{proof}

\begin{problem}{4.3}

(a) Show that the ring of functions of $\mathbb{A}^2 \setminus \{ (0,0) \}$ is $k[x,y]$.

(b) Show that the maximal ideal $<x,y>$ does not correspond to any point of $\mathbb{A}^2 \setminus \{ (0,0) \}$.

(c) Show that $\mathbb{A}^2 \setminus \{ (0,0) \}$ is not an affine variety.

\end{problem}

\begin{proof}[Solution]
 
(a)
Consider the regular map acting via inclusion $i: \mathbb{A}^2 \setminus \{ (0,0) \} \to \mathbb{A}^2$. This is a regular map given by $(x,y) \to (x,y)$, which induces an inclusion map from $\mathcal{O}(\mathbb{A}^2) \to \mathcal{O}(\mathbb{A}^2 \setminus \{ (0,0) \})$. Since we know that $A(\mathbb{A}^2) \cong  \mathcal{O}(\mathbb{A}^2)$, and the zero locus of $\mathbb{A}^2$ is the zero ideal, we see that $\mathcal{O}(\mathbb{A}^2) \cong k[x,y]/(0) \cong k[x,y]$. However, we now claim that this is all of the regular functions on $\mathcal{O}(\mathbb{A}^2 \setminus \{ (0,0) \})$. Since $\mathbb{A}^2$ is a variety, this open set is dense, so we need only show that there are no more regular functions, i.e. there are no new denominators that are introduced. We may only have new denominators for polynomials that vanish at exactly $(0,0)$. Suppose $f(x,y)$ vanishes only at $(0,0)$. Then, consider the polynomial in 1 variable $f(1,y)$. This is a polynomial in one variable, and since we said it only vanishes at $(0,0)$, it has no roots in $y$ at $x = 1$, and thus must be a constant in $y$. We may make the same argument for $f(x,1)$. Then, $f$ is constant, a contradiction. Therefore, the only regular functions on $ \mathcal{O}(\mathbb{A}^2 \setminus \{ (0,0) \})$ have form $f/g$ where $g$ is a non-zero constant. But these are exactly the regular functions on $\mathcal{O}(\mathbb{A}^2)$. Thus, the map is actually surjective as well, and is actually an isomorphism. So we have that $k[x,y] \cong \mathcal{O}(\mathbb{A}^2)  \cong  \mathcal{O}(\mathbb{A}^2 \setminus \{ (0,0) \})$.

(b)

Recall from the Galois correspondence, that maximal ideals correspond to minimal closed sets, i.e. singleton points. Then, we can say that the maximal ideal $<x,y>$ corresponds uniquely to $Z(<x,y>) = \{ (0,0) \}$ as a member of the affine plane, as the only point that disappears for every polynomial that either $x,y$ divides it is $(0,0) \in \mathbb{A}^2$. However, $(0,0) \not \in \mathbb{A}^2 \setminus \{ (0,0) \}$ by construction, and since it is the unique point in the enclosing variety, there cannot be another point in the subvariety.

(c)

Suppose it were an affine variety. Then, by part (a), the inclusion map to the affine variety $\mathbb{A}^2$ induces an isomorphism of the ring of functions $k[x,y] \cong \mathcal{O}(\mathbb{A}^2 \setminus \{ (0,0) \})$. Then, if it were affine, since the ring of functions are isomorphic under the inclusion map, we could find an inverse of the ring of functions, inducing an inverse regular map. However, the inclusion map is not bijective, so such a inverse map cannot exist. Therefore, $\mathbb{A}^2 \setminus \{ (0,0) \}$ may not be an affine variety.

\end{proof}

\begin{problem}{4.4}

Let $A$ be a ring, $P_1 \subseteq P_2$ be prime ideals. Define $A_P$ the localization of $A$ by the multiplicatively closed set $A \setminus P$. Show that $A_{P_1}$ is isomorphic to the localization of $A_{P_2}$ by a suitable multiplicatively closed set.

\end{problem}

\begin{proof}[Solution]
 
%Construct the multiplicatively closed set $S = \{ \frac{1}{1}, \frac{p_2}{1} : p_2 \in P_2 \setminus P_1 \}$. First, this is multiplicatively closed because, let $\frac{p_2}{1}$ and $\frac{p_2'}{1}$ be two different elements in $S$. Then, we look at $\frac{p_1p_2}{1}$. Since $P_1$ is prime, $p_1p_2 \in P_2$. However, since neither $p_1, p_2 \in P_1, p_1p_1 \not \in P_1$. Therefore, $p_1p_2 \in P_2 \setminus P_1$, and thus $S$ is multiplicatively closed.

%Now, consider the map from $A_{P_1} \to S^{-1}A_{P_2}$ via sending $\frac{r}{s} \to \frac{\frac{r}{s}}{\frac{1}{1}}$ if $s \in A \setminus P_2$, and to $\frac{\frac{r}{1}}{\frac{s}{1}}$ when $s \in P_2 \setminus P_1$. 

Consider the ideal in $A_{P_2}$ of the form $I = \{ p_1/r : p_1 \in P_1, r \in A \setminus P_2 \}$. This is an ideal because if we take any other element $r'/s' \in A_{P_2}$, then we have that $r'p_1 \in P_1$, since $P_1$ is an ideal, so $r'p_1/s'r$ is of the form $p_1/r$. Further, it actually has a subring structure, as for any $p_1/r$, $-p_1 \in P_1$, so $-p_1/r$ is a member. And we have $p_1/r + p_1'/r' = (r'p_1 + rp_1')/rr'$. But since $P_1$ an ideal, $r'p_1 + rp_1' \in P_1$, so $ (r'p_1 + rp_1')/rr'$ has the right form as well. Further, this is actually a prime ideal. Suppose we have that $r/s * r'/s' \in I$. Then, we look at $rr'/ss'$. Since this is in $I$, we know that $rr' \in P_1$. But, because $P_1$ is prime, this implies that one of $r,r' \in P_1$. But then, one of $r/s$ or $r'/s'$ is in $I$. So this $I$ is in fact prime.

Now, consider the localization of $A_{P_2}$ with respect to $A_{P_2} \setminus I$, and call it $R$ for my sanity. We claim that we can construct an isomorphism from $A_{P_1} \to R$ by sending $r/s \to (r/1)/(s/1)$. First, we check that this is well-defined. Suppose we have that $r/s \sim r'/s'$, that is, we have $s'' \in A \setminus P_1 : s''(rs' - r's) = 0$. Consider $(r'/1)/(s'/1)$ and $(r/1)/(s/1)$. In particular, consider the quantity $s''/1 * [(r/1)(s'/1) - (r'/1)(s/1)] = s''/1 * [(rs'/1) - (r's/1)] = [s''(rs' - r's)]/1 = 0$. and because $s'' \in A \setminus P_1$, $s''/1 \in A_{P_2} \setminus I$, though really I could've said that if $s'' \in A$ in general. Regardless, we see that the choice of representative does not affect the map, so we are well-defined.

We see that the zero elements in $R$ have the form $(0/s)/a$, where $a \in A_{P_2} \setminus I$. Then, the kernel of our map must have form $(0/s) \in A_{P_1}$. So then, we need only check that this map is surjective. Let $(r/s)/(r'/s') \in R$, where $r \in A, s,s' \in A \setminus P_2, r' \in A \setminus P_1$. We claim that $(r/s)/(r'/s') \sim (rs'/1)/(r's/1)$. Consider the quantity $(1/1)[ (r/s)(r's/1) - (r'/s')(rs'/1)] = (rr's/s) - (rr's'/s') = (rr'ss' - rr's's)/ss' = 0/ss'$, which is the $0$ in $A_{P_2}$. Then, we may find a preimage in $A_{P_1}$, in particular, the element $rs'/r's$. Since $r \in A, s' \in A \setminus P_2$, $rs' \in A$ due to rings being multiplicatively closed. And, since $r' \in A \setminus P_1$, $s \in A \setminus P_2$, $r's \in A \setminus P_1$, since $A \setminus P_1$ is multiplicatively closed due to the primality of $P_1$. Then, this is a valid element of $A_{P_1}$, and thus this map is a surjective map with trivial kernel, thus surjective.

\end{proof}


\end{document}