\documentclass[10pt]{article}
\setlength{\parskip}{0.25\baselineskip}
\usepackage[margin=1in]{geometry} 
\usepackage{amsmath,amsthm,amssymb, graphicx, multicol, array}
 
\newenvironment{problem}[2][Problem]{\begin{trivlist}
\item[\hskip \labelsep {\bfseries #1}\hskip \labelsep {\bfseries #2.}]}{\end{trivlist}}

\begin{document}
 
\title{First Assignment}
\author{Eric Tao\\
Math 240: Homework \#1}
\maketitle
 
\begin{problem}{1.1}

(a) Let $f: R \rightarrow S$ be a morphism of rings, with $\text{Ker}(f) = I$. Let $J$ be an ideal in $S$. Show that $f^{-1}(J) = \{ x \in R | f(x) \in J\}$ is an ideal in $R$ such that $I \subseteq f^{-1}(J)$.

(b) Let $f: R \rightarrow S$ be a surjective morphism of rings. If $I$ is an ideal in $R$, show that $f(I)$ is an ideal in $S$.

(c) Let $R$ be a ring, $I$ an ideal of $R$. Show that the ideals of $R/I$ are in one-to-one correspondence with the ideals of $R$ that contain $I$.

(d) Find all ideals of $\mathbb{Z}_{12}$
\end{problem}

\begin{proof}[Solution]
(a)

Take $J$ to be a an ideal in $S$. Because ideals are subrings, we can see that $0_S \in J$, where $0_S$ is the zero element in $S$. Then, $f^{-1}(0_S) \in f^{-1}(J)$. In particular, since $I = \text{Ker}(f) = \{ r \in R | f(r) = 0_S\}$, we have $I \subseteq  f^{-1}(J)$. Now, we confirm that $J$ is a subring. Take $j,j' \in f^{-1}(J)$. Consider the sum $j + (-j')$. $f(j + (-j')) = f(j) + f(-j') = f(j) - f(j')$, which is in $J$, since $f(j), f(j')$ in $J$. Therefore, $j + (-j')$ is in $f^{-1}(J)$ and thus it is a subring.

Now, fix a $j \in f^{-1}(J)$, and a $r \in R$. We have the following, by ring morphism properties that $ f(rj) = f(r)*f(j)$. Due to $f(j) \in J$ being an ideal, we also have that $f(r) * f(j) \in J$. So, we have that $f(rj) \in J$, which tells us that $rj \in  f^{-1}(J)$. Since the choice of $r,j$ was arbitrary, this is true for all $j \in f^{-1}(J)$ and $r \in R$, i.e. $f^{-1}(J)$ is multiplicatively closed. So $f^{-1}(J)$ is a subring closed under multiplication, and thus an ideal.

(b)

Firstly, we will prove that $f(I)$ is a subring of $S$. Clearly, it is non-empty, as $0_R \in I$, therefore $0_S \in f(I)$. Now, take $s,s' \in f(I)$. Since $s,s'$ in the image of I, we have that there exists $i,i'$ such that $f(i) = s, f(i') = s'$. Now, we have that $f(i + (-i')) = f(i) + f(-i') = f(i)  + (- f(i')) = s + (-s')$. Since $i + (-i') \in I$, then this shows that $s + (-s') \in f(I)$, and is a subring.

Now, take any $j \in f(I)$ and any $s \in S$. Since $f$ is surjective, there exists an $r \in R$ such that $f(r) = s$. Because $j$ is in the image of $I$, there exists $i \in I$ such that $f(i) = j$. Now, because $I$ is an ideal, $ir \in I$. Then, we have that $f(ir) = f(i)f(r) = js$, that is, $js$ is in the image of $I$. Since the choice of $j$ and $s$ was arbitrary, this works for all such $j,s$ and thus $f(I)$ is closed under multiplication. Therefore, $f(I)$ is an ideal.

(c)

Define a map $f: R \rightarrow R/I$ that sends $f(r) = \overline{r} = \{ r + I | r \in R \}$, that is, its coset of $I$. This is a surjective ring morphism, with $\text{Ker}(f) = I$. 

From part (b), because $f$ is surjective, we see that for any ideal $V \subseteq R$, that $f(V) = U \in R/I$ is an ideal. 

Further, from part (a), for any ideal of $U \subseteq R/I$, there exists an ideal $V$ in $R$ that contains $I$ such that $f (V) = U$.

Now, assume we have two ideals $I \subseteq V_1, V_2 \subseteq R$ and that $f(V_1) = f(V_2)$. Let $v_1 \in V_1$. Then, since $f(V_1) = f(V_2)$, we have that $v_1 + i = v_2 + i'$ for some $i,i' \in I$ and some $v_2 \in V_2$. Then, rearranging, we have that $v_1 = v_2 +i' - i$. But, since $I \subseteq V_2$, and $V_2$ is an subring, i.e. closed under addition, this implies that $v_1 \in V_2$. Since this choice of $v_1 \in V_1$ was arbitrary, this means that $V_1 \subseteq V_2$. Using the same argument, we see that $V_2 \subseteq V_1$, and thus $V_1 = V_2$.

Thus, $f$ is an surjective and injective map that brings ideals of $R$ that contain $I$ to ideals of $R/ I$ and the sets are in one-to-one correspondence.

(d)

The ideals of $\mathbb{Z}/12\mathbb{Z}$ are: $\mathbb{Z}/12\mathbb{Z}$, $(2)$, $(3)$, $(4)$, $(6)$, $\{ 0 \}$.

We can see this because of course we have the trivial ideals, the entire space and just $0$.

Then, we notice that if $\text{gcd}(n,12) = 1$, by B\'ezout's identity, there exists $an + 12b = 1$ in $\mathbb{Z}$, which, under modulo 12, becomes $an = 1$, that is, $n$ is invertible. But this implies then if $I \subseteq \mathbb{Z}/12\mathbb{Z}$ an ideal, and there exists $n\in I$ with $\text{gcd}(n,12) = 1$, then $1 \in I$ and thus $I = \mathbb{Z}/12\mathbb{Z}$.

Then, the other non-trivial ideals can only be the subrings additively generated by the elements of $\mathbb{Z}/12\mathbb{Z}$ non-coprime to $12$, and we can see quickly that $(2)$, $(3)$, $(4)$, $(6)$ are multiplicatively closed under multiplication by elements of $\mathbb{Z}/12\mathbb{Z}$.

\end{proof}

\begin{problem}{1.2}
Let $R$ be a ring. Call an element $a \in R$ nilpotent, if there exists $n > 0$ such that $a^n = 0$.

(a) Show that the set of nilpotent elements $N$ is an ideal in $R$.

(b) Show that $R/N$ has no non-zero nilpotent elements.

\end{problem}

\begin{proof}[Solution]

(a)
Firstly, since $0^n = 0$ for all $n \in \mathbb{N}$, $0 \in N$. Now, let $r,s$ be non-zero elements of $N$. Since they are nilpotent, take $n_r, n_s$ such that $r^{n_r} = 0, s^{n_s} = 0$. Consider

$$(r-s)^{n_r + n_s} = \Sigma_{k=0}^{n_r + n_s} c_k r^{k} (-s)^{n_r + n_s -k}$$ 
for some coefficient $c_k$, where we understand $2r = r + r$. We notice that if $k < n_r$, then $n_r + n_s - k \geq n_s$. Similarly,if $n_r + n_s -k < n_s$, then $k \geq n_r$. Thus:

$$(r-s)^{n_r + n_s} = \Sigma_{k=0}^{n_r + n_s} c_k r^{k} (-s)^{n_r + n_s -k} = \Sigma_{k=0}^{n_r - 1} c_k r^{k}* 0 + \Sigma_{k=n_r}^{n_r + n_s} c_k 0 * (-s)^{n_r + n_s -k} = 0$$

Therefore, for any $r,s \in N$, $r-s \in N$, and therefore, $N$ is a subring of $R$.

Now, let $t \in R$, and $r \in N$ with $r^{n_r} = 0$. Then, consider $tr$. $(tr)^{n_0} = t^{n_0} r^{n_0} = 0$, and $tr \in N$. Since this can be applied for any $r \in N$ and any element of $R$, N is multiplicatively closed in $R$, and thus an ideal.

(b)

Suppose $R/N$ has a nilpotent element, that is, $\overline{x}^{n_x} = 0 \in R/N$. Then, this implies that for a representative of $\overline{x}$, $x \in R$, $x ^ {n_x} \in N$. But, that implies that $(x^{n_x})$ is nilpotent as being a member of $N$. Since $(x^{n_x})$ is nilpotent, there exists $m_x$ such that $(x^{n_x})^{m_x} = 0$. But that implies that $x^{n_x*m_x} = 0$, which implies that $x \in N$ itself. Therefore, any nilpotent element of $\overline{x} = 0 \in R/N$.

\end{proof}

\begin{problem}{1.3}
Let $K$ be an algebraically closed field, $R = K[x_1,x_n,...,x_n]$, the ring of polynomials in $n$ variables. Recall that for a set of polynomials $f_n \in R$,

$$ V(f_1,...,f_k) = \{(a_1,...,a_n) \in K^n | f_i(a_1,...,a_n) = 0 \text{ for each } i = 1,...k \}$$

with a similar definition when one replaces the set of polynomials by an ideal $I \subseteq R$.


(a) Show that if $f \in R$, $f \not=0$, then $V(f) \not=\mathbb{A}^n$.

(b) Show that if $f \in R$, $f$ non-constant, then $V(f) \not= \emptyset$.

(c) Use part (a) to show that $\mathbb{A}^n$ is irreducible.

\end{problem}

\begin{proof}[Solution]
(a)

Construct the family of functions $g_{x_{2_i},...x_{n_i}}(x_1)$ for $i \in I$, $i$ not necessarily countable, such that  $g_{x_{2_i},...x_{n_i}}(x_1) = f(x_1,x_{2_i},...,x_{n_i})$. We claim that at least one such $g$ is not identically 0. Suppose not. Then, for every $(x_{2_i},...,x_{n_i}), g(x_1) = 0$ for all $x_1$. Then, this implies that for all $(x_1,...,x_n) \in A_k^n$, we have that $f(x_1,...,x_n) = 0$, a contradiction.

Then, we have a $g_j = g_{x_{2_j},...x_{n_j}}(x_1)$ for some $j$ such that $g_j$ is not identically 0. Since $g_j$ is not identically 0, there exists some $x_{1_j}$ such that $g_j(x_{1_j}) \not= 0$. Then, the point $(x_{1_j},...,x_{n_j}) \not \in V(f)$, and thus $V(f) \not=\mathbb{A}^n$.

(b)

In a similar argument as above, construct the family now over all $x_i$, fixing one $x_i$ at a time. That is, denote $g_{\{x_{m_i}\}}(x_j)$ for $i \in I$ to be the level curves of constant $x_i$ where $i \not = j$ and consider the collections of  $g_{\{x_{m_i}\}}(x_j)$ for $j = 1,..n$. Claim that there is at least one level curve in this set that is not constant.

Suppose not. Then, let $(a_1,...a_n)$ and $(b_1,...,b_n)$ be two arbitrary points in $\mathbb{A}^n$. Then, they are connected by the level curves $g_{(a_2,a_3,...a_n)}(x_1),g_{(b_1,a_3,...a_n)}(x_2),...g_{(b_1,...b_{k-1},a_{k+1},...,a_n)}(x_k),g_{(b_1,...b_{n-1})}(x_n)$ via the line segments that have the form $f_m: [0,1] \rightarrow \mathbb{R}^n$ with $f_m(t) = t(b_1,...b_{m-1},b_m,a_{m+1},...,a_n) + (b_1,...,b_{m-1},a_m,a_{m+1},...a_n)$. But, because $f$ is constant on all of these level curves, $f$ is constant on all of these line segments, therefore $f(a_1,...a_n) = (b_1,...,b_n)$. Since the choice of points was arbitrary, this is true for all points in $\mathbb{A}^n$, and then $f$ is constant, a contradiction.

Choose a level curve that is not constant $g_{\{x_{m_k}\}}(x_j)$. This is a non-constant polynomial in one variable, over an algebraically closed field. This implies that there exists at least some $x_{j_0}$ such that $g_{\{x_{m_k}\}}(x_{j_0}) = 0$. Then, $g_{\{x_{m_k}\}}(x_{j_0}) = f(x_{1_k},....x_{j_0},...x_{n_k}) = 0$, and thus $(x_{1_k},....x_{j_0},...x_{n_k}) \in V(f)$ i.e. $V(f) \not = \emptyset$.

(c)

Let $X_1, X_2$ be closed sets such that $\mathbb{A}^n = X_1 \cup X_2$. Then, we have $\mathbb{A}^n = V(f_1,...f_m) \cup V(g_1,...g_n)$ for some indices $m \in I, n\in J$. Then, from what we proved in class, we can take $V(f_1,...f_m) \cup V(g_1,...g_n) = V(f_1g_1,f_2g_1,...,f_ig_1,f_1,g_2,...,f_mg_n)$. But, also from class,  $V(f_1g_1,f_2g_1,...,f_mg_1,f_1,g_2,...,f_ng_m) = \cap_{i \in I, j\in J} V(f_ig_j)$. Since this is an intersection of sets, it follows then that $ V(f_ig_j) = \mathbb{A}^n$ for all $i,j$. But, by part (a) then, $f_ig_j = 0$. We claim that either $V(f_1,...f_m)$ or $V(g_1,...g_n)$. Suppose $f_1,...f_m = 0$ for all $i$. Then we're done, as $V(f_1,...f_i) = V(0) = \mathbb{A}^n$. Else, there exists $f_{i_0} \not= 0$ for some $i_0$. However, for all $n \in J$, $f_{i_0}g_n = 0$, thus $g_n = 0$ and  $V(g_1,...g_n) = \mathbb{A}^n$.

\end{proof}

\begin{problem}{1.4}
Let $X$ be a topological space, and let $Y\subseteq X$. Call $Y$ dense in $X$ if for every non-empty open set $U \subseteq X$, $U \cap Y$ is non-empty.

Call a topological space $X$ irreducible if for any closed sets $X_1, X_2$, $X = X_1 \cup X_2 \implies X_1 = X \text{ or } X_2 = X$.

Let $X$ be a topological space in the following:

(a) If $Y \subseteq X$, show that $Y$ is dense in $X$ if and only if the closure $\overline{Y}$ of $Y$ in $X$ satisfies $\overline{Y} = X$.

(b) Show that $X$ is irreducible if and only if every non-empty open subset $U \subseteq X$ is dense in $X$. 

(c) If $Y \subseteq X$ such that $Y$ is irreducible and $Y$ is dense in $X$, then $X$ is also irreducible.

(d) If $Y \subseteq X$ such that $Y$ is dense in $X$ and $X$ is irreducible, then $Y$ is also irreducible.

(e) If $f: A \rightarrow B$ is a continuous map of topological spaces and $X \subseteq A$ is an irreducible set, show that $f(X) \subseteq B$ is an irreducible set.

(f) If $Y \subseteq X$ satisfies that, for every $P \in X \setminus Y$, there exists a topological space $Z$ and a continuous map $f : Z \rightarrow X$ with $P \in f(Z)$ and $f^{-1}(Y)$ dense in $Z$, then $Y$ is dense in $X$.

\end{problem}

\begin{proof}[Solution]
(a)

By construction, $\overline{Y} \subseteq X$ as it is the closure with respect to $X$. Now, suppose $Y$ is dense in $X$. Let $ x \in X \setminus Y$. Let $V$ be any neighborhood of $x$. Then, since $V$ contains an open set $U$ such that $x \in U$, and $Y$ is dense, then there exists $y \in Y$ such that $y \in U$. Since the choice of neighborhood was arbitrary, this is true for every neighborhood, and every $x \in X \setminus Y$ is a limit point of $Y$. Then, that implies that $X \subseteq \overline{Y}$. Thus, $X = \overline{Y}$.

Now, suppose $\overline{Y} = X$. Let $U \subseteq X$ be an open set. Let $ u \in U$. If $u \in Y$, then we are done. Otherwise, suppose $u \in X \setminus Y$. But then, by hypothesis, $u \in \overline{Y}$, so take a small enough neighborhood $V$ of $u$ such that $V \subseteq U$. Since $u$ is in the closure of $Y$ and not in $Y$, $u$ must be a limit point of $Y$, so there exists $y \in Y$ such that $y \in V$. But, by construction, $y \in U$. Thus, for any arbitrary open set $U \subseteq X$, there exists $u \in U$ such that $u \in Y$ and therefore $Y$ is dense in $X$.

(b)

Suppose $X$ is irreducible. Let $U$ be a non-empty open subset of $X$. Consider the quantity $U^c \cup \overline{U}$, where $U^c$ is the compliment of $U$ in $X$. It should be clear that $U^c \cup \overline{U} = X$. $U^c \cup \overline{U}\subseteq X$ follows from construction, and $X \subseteq U^c \cup \overline{U}$ as for $x \in X$, $x \in U \subset \overline{U}$ or $x \in U^c$. Further, $\overline{U}$ is closed by construction, and since $U$ is open, its complement is closed. Thus, we have $X$ as a union of closed sets. Further, since $U$ is non-empty, $U^c$ cannot be $X$, therefore $\overline{U} = X$. But, by part (a), then $U$ is dense in $X$.

Now, suppose we have every non-empty subset $U \subseteq X$ dense in $X$, and suppose $X = X_1 \cup X_2$ for $X_1,X_2$ closed. If $X_1 = X$, then we are done, else, consider the open set $X_1^c$, non-empty. We have then that $X_1^c \subseteq X_2$. By the properties of the closure of $U$ being the smallest such closed set that is a superset of $U$, we have that  $\overline{X_1^c} \subseteq X_2$. But, by part (a), we have that $\overline{X_1^c} = X$, so we have that $X \subseteq X_2$. We also have $X_2 \subseteq X$ from the original union. Thus, if $X_1 \not= X$, $X_2 = X$.

(c)

Let $U$ be a non-empty open subset of $X$. Then, we can consider the closed sets $\overline{U}, U^c$. In particular, consider the closed sets in $Y$, $\overline{U}\cap Y, U^c \cap Y$ and, $Y = (\overline{U}\cap Y) \cup (U^c \cap Y)$ 

Since Y is irreducible, we have either that  $U^c \cap Y = Y$ or $\overline{U}\cap Y = Y$.

Suppose $U^c \cap Y = Y$. However, since $Y$ is dense in $X$, and $U$ is open, there exists $y_u \in Y$ such that $y_u \in U$. But then, $y_u \not \in U^c$, therefore $y_u \not \in U^c \cap Y$, a contradiction.

Then, we must have $\overline{U}\cap Y = Y$. But then we have that $Y \subseteq U$ in $X$. And, in particular, since $Y$ is dense, so must be $U$. Thus, every non-empty open subset of $X$ is dense, and $X$ is irreducible.

(d)

Suppose there exists closed sets in $Y$, $V_1, V_2 \subseteq Y$ such that $V_1 \cup V_2 = Y$. From the subspace topology, we have closed sets in $X$, $V'_1, V'_2 \subseteq X$ such that $V'_1 \cap Y = V_1, V'_2 \cap Y = V_2$. Since $Y$ is dense in $X$, $\overline{Y} = X$. But  $Y \subseteq V'_1 \cup V'_2$, which is a closed set, and the closure is the smallest such closed set that contains $Y$, so we have $\overline{Y} = X \subseteq V'_1 \cup V'_2$. Then, $X = V'_1 \cup V'_2$. Since $X$ is irreducible, this means that either $V'_1 = X$ or $V'_2 = X$. Suppose $V'_1 = X$. Then, $V_1 = V'_1 \cap Y = Y$, and a similar calculation follows for $V'_2$. Because the choice of closed sets $V_1, V_2$ in $Y$ is arbitrary, this means this is true for any such closed sets that cover $Y$, and thus $Y$ is irreducible. 

(e)

Let $U \subseteq f(X)$ be a non-empty open set in $f(X)$. Because $f$ is continuous, $f^{-1}(U)$ is an open set in $A$, and because it is wholly contained within $X$, is also an open set in $X$. Now, since $X$ is irreducible, this implies that $\overline{f^{-1}(U)} = X$. But also, we have, due to the continuity of $f$, that $f(X) = f(\overline{f^{-1}(U)}) \subseteq \overline{U}$. However, by definition, the closure of $U$ in $f(X)$ is a subset of $f(X)$ itself. Therefore $\overline{U} = f(X)$, and we have that $U$ is dense in $f(X)$ by part (a). Since the choice of $U$ was arbitrary, this is true for all $U \subseteq f(X)$, and thus by part (b), $f(X)$ is irreducible.

(f)

For each $P \in X \setminus Y$, because there exists $Z_p$ with $f_p^{-1}(Y)$ dense in $Z_p$, we have that the closure$\overline{f_p^{-1}(Y)} = Z_p$. Then, due to the continuity of $f_p$,  we have that $f(Z_p) = f(\overline{f_p^{-1}(Y)}) \subseteq \overline{Y}$. In particular, this tells us that $P \in \overline Y$. Since we can do this for all $P \in X \setminus Y$, this implies that $X \setminus Y \subseteq \overline{Y}$. Further, by definition, $Y \subseteq \overline{Y}$. Therefore, we have that $X = Y \cup (X \setminus Y) \subseteq \overline{Y}$, so $X \subseteq \overline{Y}$ and, because the closure in $X$ is a clear subset of $X$, we have $X = \overline{Y}$. Then, by part (a), we have that $Y$ is dense in $X$.
\end{proof}


\end{document}