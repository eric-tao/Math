\documentclass[10pt]{article}
\setlength{\parskip}{0.25\baselineskip}
\usepackage[margin=1in]{geometry} 
\usepackage{amsmath,amsthm,amssymb, graphicx, multicol, array}
\usepackage[font=small,labelfont=bf]{caption}
\usepackage{float}
\usepackage{bbm}
\usepackage{dsfont}
\newcommand{\supp}{{\text{supp}}} 
\newcommand{\bv}{{\text{BV}}}
\newcommand{\ac}{{\text{AC}}}
\newcommand{\vol}{{\text{Vol}}}


\ExplSyntaxOn
\NewDocumentCommand{\cycle}{ O{\;} m }
 {
  (
  \alec_cycle:nn { #1 } { #2 }
  )
 }

\seq_new:N \l_alec_cycle_seq
\cs_new_protected:Npn \alec_cycle:nn #1 #2
 {
  \seq_set_split:Nnn \l_alec_cycle_seq { , } { #2 }
  \seq_use:Nn \l_alec_cycle_seq { #1 }
 }
\ExplSyntaxOff

\newenvironment{problem}[2][]{\begin{trivlist}
\item[\hskip \labelsep {\bfseries #1}\hskip \labelsep {\bfseries #2.}]}{\end{trivlist}}

\begin{document}
 
\title{Homework \#}
\author{Eric Tao\\
Math 285: Homework \#}
\maketitle

\begin{problem}{Question 1}

\end{problem}

\begin{proof}[Solution]
\end{proof}

\begin{problem}{Question 2}


\end{problem}

\begin{proof}[Solution]

\end{proof}

\begin{problem}{Question 3}


\end{problem}

\begin{proof}[Solution]


\end{proof}

\begin{problem}{Question 4}



\end{problem}

\begin{proof}[Solution]


\end{proof}

\begin{problem}{Question 5}

\end{problem}

\begin{proof}[Solution]

\end{proof}



\begin{problem}{Question 6}


\end{problem}

\begin{proof}[Solution]

\end{proof}

\end{document}