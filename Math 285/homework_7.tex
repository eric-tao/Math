\documentclass[10pt]{article}
\setlength{\parskip}{0.25\baselineskip}
\usepackage[margin=1in]{geometry} 
\usepackage{amsmath,amsthm,amssymb, graphicx, multicol, array}
\usepackage[font=small,labelfont=bf]{caption}
\usepackage{float}
\usepackage{bbm}
\usepackage{dsfont}
\newcommand{\supp}{{\text{supp}}} 
\newcommand{\bv}{{\text{BV}}}
\newcommand{\ac}{{\text{AC}}}
\newcommand{\vol}{{\text{Vol}}}
\newtheorem{theorem}{Theorem}
\newtheorem{lemma}[theorem]{Lemma}


\ExplSyntaxOn
\NewDocumentCommand{\cycle}{ O{\;} m }
 {
  (
  \alec_cycle:nn { #1 } { #2 }
  )
 }

\seq_new:N \l_alec_cycle_seq
\cs_new_protected:Npn \alec_cycle:nn #1 #2
 {
  \seq_set_split:Nnn \l_alec_cycle_seq { , } { #2 }
  \seq_use:Nn \l_alec_cycle_seq { #1 }
 }
\ExplSyntaxOff

\newenvironment{problem}[2][]{\begin{trivlist}
\item[\hskip \labelsep {\bfseries #1}\hskip \labelsep {\bfseries #2.}]}{\end{trivlist}}

\begin{document}
 
\title{Homework \#7}
\author{Eric Tao\\
Math 285: Homework \#7}
\maketitle

\begin{problem}{Question 1}

(a)

Let $A$, $B$ be disjoint, closed sets on a manifold $M$. Find a $C^\infty$ function $f$ such that $f |_A = 1$ and $f |_B = 0$.

(b)

Let $M$ be a manifold, and let $A \subset M$ be closed, $U \subset M$ be open, such that $A \subset U$. Show that there exists a $C^\infty$ function $f$ on $M$ such that $f|_A = 1 $ and that the support of $f$ is contained within $U$.


\end{problem}

\begin{proof}[Solution]
(a)

Firstly, by Theorem 13.6, we know that relative to the open sets $\{ M_A = M \setminus A, M_B = M \setminus B \}$, there exists a $C^\infty$ partition of unity$\{ \rho_A, \rho_B \}$ where $\rho_A$ is subordinate to $M_A$, and analogously for $\rho_B$.

Clearly then, since the support of $\rho_B$ is a subset of $M_B$, we have that $\rho_B \big|_B = 0$, as $B \cap M_B = \emptyset$. Furthermore, we can say that on $A$, we have the analogous result that $\rho_A \big|_A = 0$. 

Then, we notice that due to these results, that because this is a partition of unity, we have that:

$$ (\rho_A + \rho_B)\big|_{A} = 1 \implies  \rho_A \big|_A + \rho_B \big|_A = 1 \implies \rho_B \big|_A = 1$$

Thus, $\rho_B$ is a $C^\infty$ function that is identically 0 on $B$ and identically $1$ on $A$, as desired.

(b)

Take the disjoint closed sets $M \setminus U$ and $A$. Then, by part (a), there exists a $C^\infty$ function $f$ such that $f\big|_{M \setminus U} = 0$ and $f \big|_{A} = 1$.

Furthermore, from construction of $f$ being a partition of unity subordinate to $M \setminus (M \setminus U)$, by deMorgan's laws, we have that 
$$M \setminus (M \setminus U) = M \cap (M \setminus U)^c = M \cap (M \cap U^c)^c = M \cap (M^c \cup U^{c^c}) = M \cap (\emptyset \cup U) = M \cap U = U$$

Thus, by definition, since $f$ is subordinate to $U$, the support of $f$ is contained within $U$, as requested. 

\end{proof}

\begin{problem}{Question 2}

Let $F:N \to M$ be a $C^\infty$ map of manifolds. Let $h: M \to \mathbb{R}$ be a $C^\infty$ function. Show that the support of $F^*(h)$ is a subset of $F^{-1}(\text{supp}(h))$.

\end{problem}

\begin{proof}[Solution]

First, we recall that the pullback is exactly $F^*(h) = h \circ F$.

Now, first, we want to show that $(F^*h)^{-1}(\mathbb{R}^\times) \subset F^{-1}(\text{supp}(h))$.

So, suppose $p \in N$, such that $F^*h(p) \not = 0$. Then, we have that $h \circ F(p) = h(F(p)) \not = 0$, which implies that $F(p) \in \text{supp}(h)$. Thus, $p \in F^{-1}(\text{supp}(h))$. Since the choice of $p$ were arbitrary, this implies that $(F^*h)^{-1}(\mathbb{R}^\times) \subset F^{-1}(\text{supp}(h))$.

Now, by definition, we know that $\text{supp}(F^*h) = \text{cl}[(F^*h)^{-1}(\mathbb{R}^\times)]$, that is, it is the closure of the preimage. Furthermore, we notice that because $F$ is $C^\infty$, $F$ is continuous. Since the $\text{supp}(h)$ is a closed set in $M$, being a closure, then $F^{-1}(\text{supp}(h))$ is a closed set in $N$, by continuity.

Hence, since $(F^*h)^{-1}(\mathbb{R}^\times) \subseteq F^{-1}(\text{supp}(h))$ and $F^{-1}(\text{supp}(h))$ is closed, we have that $\text{supp}(F^*h) = \text{cl}[(F^*h)^{-1}(\mathbb{R}^\times)] \subseteq F^{-1}(\text{supp}(h))$, via the characterization of the closure of a set $S$ being the smallest closed set containing $S$.

\end{proof}

\begin{problem}{Question 3}

Define $x^1, y^1,....,x^n, y^n$ as the standard coordinates of $\mathbb{R}^{2n}$. Define the unit sphere $S^{2n-1}$ in an ambient $\mathbb{R}^{2n}$ cut out by the equation $\sum_{i=1}^n (x^i)^2 + (y^i)^2 = 1$.

Show that 

$$ X= \sum_{i=1}^n -y^i \frac{\partial}{\partial x^i} + x^i \frac{\partial}{\partial y^i} $$ is a smooth vector field that does not vanish anywhere on $S^{2n-1}$.

\end{problem}

\begin{proof}[Solution]

First, we prove part of problem 11.1, that is:

\begin{lemma}

Define the unit sphere $S^n \subset \mathbb{R}^{n+1}$ as being cut out by the equation $\sum_{i=1}^{n+1} (x^i)^2 = 1$. For a point $p  = (p^1,...,p^{n+1}) \in S^n$, and a tangent vector at $p$,

$$ X_p = \sum a^i \frac{\partial}{\partial x^i} \bigg|_p \in T_p(\mathbb{R}^{n+1}) $$

show that $\sum a^i p^i = 0$ if and only if $X_p$ is tangent to $S^n$ at $p$.

\end{lemma}

\begin{proof}

We follow the discussion in Section 11.5.

Since we have the equation for an $n$-sphere, we have then an equation for the tangent space for $S^n$ at a point $p$ as being, in terms of a tangent vector $a = \langle a^1,...,a^{n+1} \rangle$ for $TS^n$:

$$ \sum_{j=1}^{n+1} \frac{\partial}{\partial x^j} \left( \sum_{i=1}^{n+1} (x^i)^2  \right)(p)a^j  = 0 \iff$$

$$\sum_{j=1}^{n+1}  \sum_{i=1}^{n+1} \delta^i_j 2x^i (p) a^j  = 0  \iff$$

$$\sum_{j=1}^{n+1} 2p^j a^j = 0 \iff \sum_{j=1}^{n+1} p^j a^j = 0 $$

So, we have that if $\sum_{j=1}^{n+1} p^j a^j = 0$, then $a = \langle a^1,...,a^{n+1} \rangle$ is the image of some tangent vector in $TS^{n}$ under the differential of the inclusion, and thus $X_p$ is tangent to $S^n$ at $p$ as desired. Note that to be clear, the tangent vector $a$ should be thought of the image of the differential of the inclusion of the $n$ sphere into $\mathbb{R}^{n+1}$, and so we may identify it with its image in $T\mathbb{R}^{n+1}$.

\end{proof}


Thus, we fix a point $p = (p^1,...,p^{2n}) \in S^{2n-1}$, and we wish to show that the vector field

$$ X =  \sum_{i=1}^n -y^i \frac{\partial}{\partial x^i} + x^i \frac{\partial}{\partial y^i} $$

is tangent to $S^{n-1}$.

By the lemma above, we need only check if $\sum_{j=1}^{n} -y^i\big|_p p^{2i-1} + x^i\big|_p p^{2i} = 0$ at a point $p$. However, this is clear:

$$\sum_{j=1}^{n} -y^i\big|_p p^{2i-1} + x^i\big|_p p^{2i} =\sum_{j=1}^{n} -p^{2i} p^{2i-1} + p^{2i-1} p^{2i} = \sum_{j=1}^n 0 = 0$$

Since the choice of $p$ was arbitrary, this works for all $p$, and since $X$ is tangent at every point $p\in S^{2n-1}$, it defines a vector field on $S^{2n-1}$, where we see smoothness as the lift from $\mathbb{R}^{2n} \to T\mathbb{R}^{2n}$ is smooth by being polynomial in each coordinate, and taking an appropriate restriction.

However, it is clear that it is nowhere vanishing, as at any point $p$, we have that:

$$X_p = \sum_{i=1}^n -y^i \frac{\partial}{\partial x^i}\bigg|_p + x^i \frac{\partial}{\partial y^i}\bigg|_p = \sum_{i=1}^n -p^{2n} \frac{\partial}{\partial x^i}\bigg|_p + p^{2n-1} \frac{\partial}{\partial y^i}\bigg|_p$$

Therefore, $X_p = 0$ if and only if $p^i = 0$ for all $i$. However, the point $(0,...,0) \not \in S^{2n-1}$, and thus $X$ does not vanish on $S^{2n-1}$.

\end{proof}

\begin{problem}{Question 4}

Let $M = \mathbb{R} \setminus \{ 0 \}$. Let $X = \frac{d}{dx}$ on $M$. Find the maximal integral curve of $X$ with initial point $x = 1$.


\end{problem}

\begin{proof}[Solution]

Clearly, if we have that $c(t) = x(t)$, in order to be an integral curve, we must have that:

$$ x'(t) = 1 \implies x(t) = t + a$$

for indeterminant $a\in \mathbb{R}$.

Clearly, to satisfy the initial condition, we then must have:

$$x(0) = 0 + a = 1 \implies a = 1$$

And thus, our integral curve desired has equation $c(t) = x(t) = t + 1$.

Now, here, we want to determine what the domain should be.

We notice that since $c$ is a smooth curve, in particular, it is continuous. Thus, because the domain of the curve is an interval in $\mathbb{R}$, hence connected, because the continous image of a connected set is also connected (Proposition A.42), we must have that the image of the curve is connected.

Thus, since we notice that we may find connected components on $M$ under the subspace topology as $(-\infty, 0), (0, \infty)$, the natural maximal integral curve is simply the preimage of the connected component containing our initial point.

Thus, we see that our desired domain is:

$$c^{-1}((0, \infty)) = (-1, \infty)$$

Hence, our maximal curve is:

$$c: (-1, \infty) \to M \text{ via } t \mapsto t+1 $$

A remark,  if we consider curves to only have finite endpoints, we may not find a maximal curve. Since, suppose $c_0: (a,b) \to \mathbb{R}$ were a maximal candidate, that sends $t \mapsto t+1$.

The problem here is that we can always construct $c_1: (a, b+1) \to \mathbb{R}$ with the same map. Since we can do this for any finite $b$, there is no maximal integral curve with finite endpoints as its domain for such a manifold, initial point, and vector field.



\end{proof}

\begin{problem}{Question 5}

Find the integral curves of the vector field:

$$X_{(x,y)} = x \frac{\partial}{\partial x} - y \frac{\partial}{\partial y}  $$

in an ambient $\mathbb{R}^2$.
\end{problem}


\begin{proof}

Define the initial condition as the point $x_0, y_0$.

For $c(t)$ to be an integral curve, if we define $c(t) = (x(t), y(t))$, and denoting a derivative with respect to only time with a $'$ we have the following system of equations:

$$\begin{cases} x'(t) = x \\ y'(t) = -y \end{cases}$$

We recognize this as first order differential equations, with the following solution set:

$$ \begin{cases} x(t) = Ae^{t} \\ y(t) = Be^{-t} \end{cases} $$

for indeterminants $A, B \in \mathbb{R}$.

Using our initial conditions, we see that:

$$c(0) = (x(0), y(0)) = (x_0, y_0) \implies A = x_0, B = y_0$$

Thus, relative to the initial point $p = (x_0, y_0)$, we see that integral curves for the vector field $X = \langle x, -y \rangle$ take on the form:

$$ c(t) = c_t(p) = (x_0 e^{t}, y_0 e^{-t}) $$

Here, we note that the interval curve can take on arbitrary intervals of $\mathbb{R}$ as its domain.

\end{proof}

\begin{problem}{Question 6}

Let $f, g$ be $C^\infty$ functions and $X, Y$ be $C^\infty$ vector fields, all on a manifold $M$. Show that:

$$ [fX, gY] = fg[X,Y] + f(Xg)Y - g(Yf)X$$

\end{problem}

\begin{proof}[Solution]

Fix some $h \in C^\infty(M)$, and consider the action of $[fX, gY]h$. We have that:

$$ [fX, gY]h = fX(gY(h)) - gY(fX(h)) $$

Now, using the fact that a vector field is a derivation, we can use the Leibniz rule on $X(g*[Y(h)])$ and $Y(f*X(h))$:

$$ fX(gY(h)) - gY(fX(h)) = f(Xg)(Yh)+ fg(XY(h)) - g(Yf)(Xh) - gf(YX(h)) $$

Regrouping a little, we can see that because $[X,Y] := XY - YX$, and $fg = gf$ being scalar functions:

$$ f(Xg)(Yh)+ fg(XY(h)) - g(Yf)(Xh) - gf(YX(h))  = ( fg(XY(h))  - gf(YX(h))) +  f(Xg)(Yh) - g(Yf)(Xh) =$$
$$ fg(XY - YX)h +  f(Xg)(Yh) - g(Yf)(Xh) = fg[X,Y]h +  f(Xg)(Yh) - g(Yf)(Xh)$$

Thus, we see that:

$$  [fX, gY]h =  fg[X,Y]h +  f(Xg)(Yh) - g(Yf)(Xh) $$

Since the choice of $h$ was arbitrary, varying over $h$, we obtain an equality of vector fields:

$$  [fX, gY] =  fg[X,Y] +  f(Xg)Y - g(Yf)X$$

as desired.

\end{proof}

\end{document}