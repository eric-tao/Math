\documentclass[10pt]{article}
\setlength{\parskip}{0.25\baselineskip}
\usepackage[margin=1in]{geometry} 
\usepackage{amsmath,amsthm,amssymb, graphicx, multicol, array}
\usepackage[font=small,labelfont=bf]{caption}
\usepackage{float}
\usepackage{bbm}
\usepackage{dsfont}
\newcommand{\supp}{{\text{supp}}} 
\newcommand{\bv}{{\text{BV}}}
\newcommand{\ac}{{\text{AC}}}
\newcommand{\vol}{{\text{Vol}}}

\theoremstyle{nonumberplain}% default
\newtheorem{theorem}{Theorem}
\newtheorem*{lemma}{Lemma}

\ExplSyntaxOn
\NewDocumentCommand{\cycle}{ O{\;} m }
 {
  (
  \alec_cycle:nn { #1 } { #2 }
  )
 }

\seq_new:N \l_alec_cycle_seq
\cs_new_protected:Npn \alec_cycle:nn #1 #2
 {
  \seq_set_split:Nnn \l_alec_cycle_seq { , } { #2 }
  \seq_use:Nn \l_alec_cycle_seq { #1 }
 }
\ExplSyntaxOff

\newenvironment{problem}[2][]{\begin{trivlist}
\item[\hskip \labelsep {\bfseries #1}\hskip \labelsep {\bfseries #2.}]}{\end{trivlist}}

\begin{document}
 
\title{Homework \#9}
\author{Eric Tao\\
Math 285: Homework \#9}
\maketitle

\begin{problem}{Question 1}

Consider $S^1$ as a subset of the unit circle. We see that it admits multiplication, given by:

$$ e^{it} \cdot e^{iu} = e^{i(t+u)} $$

for $u, t \in \mathbb{R}$. 

Viewed as complex numbers, we can also express complex multiplication by an element $\cos t + i \sin t \in S^1$ as:

$$ (\cos t + i \sin t) (x + iy) = ((\cos t) x - (\sin t)y) + i(( \sin t) x + (\cos t) y) $$

Then for $g = (\cos t, \sin t)  \in S^1 \subset \mathbb{R}^2$, the left multiplication is given by:

$$ l_g(x,y) = (( \cos t)x - (\sin t)y, (\sin t)x + (\cos t)y)$$

Let $\omega = -y dx + x dy$ be a 1-form on $S^1$. Prove that $l_g^* \omega = \omega$ for all $g \in S^1$.

\end{problem}

\begin{proof}[Solution]

We recall that for any pullback, by 17.10 and 17.9:

$$ l_g^*( -y dx + x dy) = -l_g^*(y) d(l_g^*(x)) + l_g^*(x)  d(l_g^*(y)) $$

Fixing a $g = (\cos t, \sin t)$, we see that $l_g^*$ has the action of sending $x \mapsto (\cos t) x - (\sin t)y, y \mapsto  (\sin t)x + (\cos t)y$.

Thus, we have that:

$$  -l_g^*(y) d(l_g^*(x)) + l_g^*(x)  d(l_g^*(y)) = -  [(\sin t)x + (\cos t)y] d( (\cos t) x - (\sin t)y) + [(\cos t) x - (\sin t)y] d( (\sin t)x + (\cos t)y) $$

Doing some more algebra, we see that this is equal to:

$$  -  [(\sin t)x + (\cos t)y] [\cos t dx - \sin t dy ] +  [(\cos t) x - (\sin t)y]  [(\sin t)dx + (\cos t)dy] =  $$

$$- \sin t \cos t (x dx) + \sin^2 t (x dy) - \cos^2 t (y dx) + \sin t \cos t (y dy) + \cos t \sin t (x dx) + \cos^2 t (x dy) - \sin^2 t (y dx) - \sin t \cos t (y dy)  = $$

$$ (-\sin t \cos t + \cos t \sin t ) x dx + \sin^2 t + \cos^2 t x dy + (\sin t \cos t - \sin t \cos t)y dy - (\sin^2 t + \cos ^2 t ) y dx = $$

$$ - y dx + x dy = \omega$$

Since the choice of $g$ were arbitrary, as we used no properties of $t$ other than trigonometric identities, we have that $l_g^*\omega = \omega$ for all $g \in S^1$.

\end{proof}

\begin{problem}{Question 2}

We say that a sum $\sum \omega_\alpha$ of differential $k$-forms on a manifold $M$ is locally finite if the collection of supports $\{ \text{supp} \: \omega_\alpha \}$ is locally finite. Suppose that we have two locally finite sums $\sum \omega_\alpha, \sum \tau_\beta$ , and $f \in C^\infty(M)$.

(a) Show that for every $p \in M$, there exists a neighborhood $p \in U$ such that $\sum \omega_\alpha$ is finite.

(b) Show that $\sum (\omega_\alpha + \tau_\alpha)$ is a locally finite sum, and

$$ \sum (\omega_\alpha + \tau_\alpha) = \sum \omega_\alpha + \sum \tau_\alpha$$

(c) Show that $\sum f\omega_\alpha$ is a locally finite sum, and that:

$$ \sum f \cdot \omega_\alpha = f \cdot \left( \sum \omega_\alpha\right) $$

\end{problem}

\begin{proof}[Solution]

(a)

Fix some $p \in M$. Since the supports of $\omega_\alpha$ are locally finite, we know that there exists a neighborhood $U$ of $p$ such that only finitely many of $\text{supp}(\omega_\alpha) \cap U \not = \emptyset$. Choose such a neighborhood $U$, and index the supports with non-empty intersection $\text{supp}(\omega_{\alpha_i})$ for $i \in (1,...,n)$.

Then, consider the value of $\sum \omega_\alpha$. Of course, if the support has trivial intersection with $U$, then $\omega_\alpha = 0$ on $U$, as the support of a $k$-form is defined as the closure of the complement of the zero set, hence the complement of the support is contained within the zero set.

Then, we see that $\sum \omega_\alpha = \sum_{i=1}^n \omega_{\alpha_i}$. But, at each point $q \in U$, each $\omega_{\alpha_i}$ is a alternating $k$-tensor, hence a $k$-linear real function. Then, at $q$, this is a finite sum of finite values, hence finite.

Since this is true for all $q \in U$, since the choice of $q$ was arbitrary, this is finite for all of $U$, and we may say that $\sum \omega_\alpha = \sum_{i=1}^n \omega_{\alpha_i} < \infty$ on all of $U$.

(b)

We start by proving a lemma:

\begin{lemma}
Let $\omega, \tau$ be $k$-forms on a manifold $M$.

Then, we have that:

$$ \text{supp}(\omega + \tau) \subseteq \text{supp}(\omega) \cup \text{supp}(\tau)$$

\end{lemma}

\begin{proof}

Clearly, we know that as sets:

$$ \{ p \in M : \omega_p + \tau_p \not = 0 \} \subseteq \{ p \in M : \omega_p \not = 0 \} \cup \{ q \in M : \tau \not 0 \} $$

as of course, if $\omega_p + \tau_p \not = 0$, then at least one of $\omega_p, \tau_p$ is non-0.

Since we have that $ \{ p \in M : \omega_p \not = 0 \} \subseteq \text{supp}(\omega)$, being the closure, and same with $\tau$, we see that:

$$ \{ p \in M : \omega_p + \tau_p \not = 0 \} \subseteq \text{supp}(\omega) \cup \text{supp}(\tau) $$

Now, since $\text{supp}(\omega)$ is closed, and same with $\text{supp}(\tau)$, we see that the right hand side is a closed set, being the finite union of closed sets.

Then, since the closure of the left-hand side, is the smallest closed set that contains the left-hand side, and that the right hand side is a closed set, we have that:

$$ \text{supp}(\omega + \tau) = \text{cl}( \{ p \in M : \omega_p + \tau_p \not = 0 \}) \subseteq  \text{supp}(\omega) \cup \text{supp}(\tau) $$

as desired.


\end{proof}

We start by proving the equality at a fixed point. Let $p$ be an arbitrary point in $M$. Since $ \sum \omega_\alpha$ is locally finite, we may find a neighborhood $U$ such that the intersection with $\text{supp}( \omega_\alpha)$ is trivial for all but $\alpha_i$, $i \in (1,...,n)$. Similarly, there exists $V$ such that the interesection with $\text{supp}(\tau_\beta)$ is trivial for all but $\beta_j$, $j \in (1,...,m)$.

Then, on $U \cap V$, we have that:

$$ \sum \omega_\alpha + \sum \tau_\beta = \sum_{i=1}^n \omega_{\alpha_i} + \sum_{j=1}^m \tau_{\beta_j} = \sum_{i=1}^{\max(m,n)} \omega_{\alpha_i} + \tau_{\beta_j} $$

where WLOG if $m < n$, we pad with $ \beta_{m+1},...,\beta_n = 0$.

Thus, on $p \in U \cap V$, we have that:
 
$$ \sum \omega_\alpha + \sum \tau_\beta = \sum_{i=1}^n \omega_{\alpha_i} + \sum_{j=1}^m \tau_{\beta_j} = \sum_{i=1}^l \omega_{\alpha_i} + \tau_{\beta_i} = \sum \omega_\alpha  + \tau_\beta $$

Moreover, we see that on $U \cap V$, only $\omega_{\alpha_i}, \tau_{\beta_j}$ potentially non-0 for $i \in (1,...,n), j \in (1,...,m)$. Hence, for whatever numbering $\sum_\alpha \omega_\alpha + \tau_\alpha$ takes on, only these can have supports with non-trivial interesection. Since by our lemma, the support of $\omega + \tau$ is contained within the union of the supports, only up to $m+n$ distinct $\omega_\alpha + \tau_\alpha$ may have supports with non-trivial intersection. Thus, $\sum_\alpha \omega_\alpha + \tau_\alpha$ is locally finite at $p$.

However, the choice of $p$ was trivial. Thus, we can repeat this argument for every point $p \in M$, and thus the sum is locally finite on all of $M$, and the equality holds on all of $M$.

(c)

In the same vein as (b), we first look at the support. It should be clear that for a single $k$-form $\omega$, that we have set-wise:

$$ \{ p \in M : \omega_p = 0 \} \subseteq \{ p \in M :  f \cdot \omega_p = 0 \}$$ as if $\omega_p = 0$, of course $f \cdot \omega_p = 0$. 

Now, because complements reverse the inclusion, this implies that:

$$   \{ p \in M :  f \cdot \omega_p \not= 0 \}^c \subseteq \{ p \in M : \omega_p \not = 0 \}$$

Then, since the support of $\omega$ is the closure, we have that:

$$  \{ p \in M :  f \cdot \omega_p \not= 0 \}^c \subseteq \text{supp}(\omega)$$

And finally, since the right hand side is a closed set, the closure of the left hand set must be contained within the right hand side, and so we have that:

$$ \text{supp}(f \cdot \omega) \subseteq \text{supp}(\omega) $$

Now, look at the sum $ f \cdot \sum \omega_\alpha$. Fix some $p \in M$. Since $\sum \omega_\alpha$ is locally finite, we may find a neighborhood such that only finitely many $\omega_\alpha$ have supports which intersect $U$ non-trivially. Label these forms $\omega_{\alpha_i}$ for $i \in (1,...,n)$.

Then, on $U$, we have that

$$ f \cdot \sum \omega_\alpha = f \cdot \sum_{i=1}^n \omega_{\alpha_i}$$

as on $U$, since the intersection with the support is trivial, if $\alpha \not \in \{ \alpha_i \}_{i=1}^n$, then $\omega_{\alpha} = 0$ identically on $U$.

Thus, since this is now a finite sum, we may distribute into the sum, and we find that we have a neighborhood with $p$ such that:

$$ f \cdot \sum \omega_\alpha = f \cdot \sum_{i=1}^n \omega_{\alpha_i} = \sum_{i=1}^n f \cdot \omega_{\alpha_i} = \sum_{\alpha} f \cdot \omega_\alpha$$

where the last equality comes from the fact that $\omega_\alpha = 0 \implies f \cdot \omega_\alpha = 0$ for all $\alpha \not \in  \{ \alpha_i \}_{i=1}^n$ because, by the statement proved about the containment of the supports, $f \cdot \omega_\alpha$ must be locally finite, as only the $\text{supp}(\omega_{\alpha_i})$ have non-trivial intersection with $U$, and thus only the $\text{supp}(f \cdot \omega_{\alpha_i})$ could have non-trivial intersection.

Finally, since this procedure may be done for every point $p$, this sum becomes an equality of functions across all $p \in M$ and is locally finite everywhere, and so we conclude that:

$$ f \cdot \sum \omega_\alpha = \sum f \cdot \omega_\alpha $$

with $\sum f\omega_\alpha$ as locally finite.

\end{proof}

\begin{problem}{Question 3}

Let $U$ be the open set $(0, \infty) \times (0, \pi) \times (0, 2\pi) \subseteq \mathbb{R}^3$, with the spherical coordinates $(\rho, \phi, \theta)$. Define $F: U \to \mathbb{R}^3$ via:

$$ F(\rho, \phi, \theta) = ( \rho \sin \phi \cos \theta, \rho \sin \phi \sin \theta, \rho \cos \phi)$$

Let $x, y, z$ be the standard coordinates on $\mathbb{R}^3$. Show that:

$$ F^*(dx \wedge dy \wedge dz) = \rho^2 \sin\phi \: d\rho \wedge d\phi \wedge d\theta $$

\end{problem}

\begin{proof}[Solution]

First, we recall that by proposition 18.10, the pullback commutes with the wedge product. Thus, we have that:

$$  F^*(dx \wedge dy \wedge dz) =  F^*(dx) \wedge F^*(dy) \wedge F^*(dz)$$

Now, by proposition 17.9, since $F$ is clearly $C^\infty$, each coordinate being the product of $C^\infty$ functions, we see that:

$$   F^*(dx) \wedge F^*(dy) \wedge F^*(dz) =   dF^*(x) \wedge dF^*(y) \wedge dF^*(z)= d(x \circ F) \wedge d(y \circ F) \wedge d (z \circ F) = $$

$$ d(\rho \sin \phi \cos \theta) \wedge  d(\rho \sin \phi \sin \theta) \wedge d(\rho \cos \phi) $$

Expanding, we compute:

$$  d(\rho \sin \phi \cos \theta) \wedge  d(\rho \sin \phi \sin \theta) \wedge (\rho \cos \phi) = (\sin \phi \cos\theta d\rho + \rho \cos \phi \cos \theta d\phi - \rho \sin \phi \sin \theta d\theta) \wedge $$

$$ ( \sin \phi \sin \theta d\rho + \rho \cos\phi \sin \theta d\phi + \rho \sin \phi \cos \theta d\theta) \wedge (\cos \phi d\rho - \rho \sin \phi d\phi) $$

We notice that since $3$-forms are spanned exactly by $d\rho \wedge d\phi \wedge d\theta$, we can look at only the terms that have one of each of $d\rho, d\phi, d\theta$ in some order, as otherwise, because $df \wedge df = 0$, and we can alternate over the wedge product at the cost of a factor of $(-1)$ the other terms that repeat any single form vanish:

$$ =  \sin \phi \cos\theta d\rho \wedge  \rho \sin \phi \cos \theta d\theta \wedge ( - \rho \sin \phi d\phi) +  \rho \cos \phi \cos \theta d\phi \wedge \rho \sin \phi \cos \theta d\theta \wedge \cos \phi d\rho+ $$
$$( - \rho \sin \phi \sin \theta d\theta) \wedge  \sin \phi \sin \theta d\rho  \wedge ( - \rho \sin \phi d\phi) +  (- \rho \sin \phi \sin \theta d\theta) \wedge \rho \cos\phi \sin \theta d\phi \wedge  \cos \phi d\rho $$

Cleaning up with some algebra, this is equal to:

$$ - \rho^2 \sin^3 \phi \cos^2 \theta (d\rho \wedge d\theta \wedge d\phi) + \rho^2 \sin \phi \cos^2 \phi \cos^2 \theta (d\phi \wedge d\theta \wedge d\rho) +$$
$$ \rho^2 \sin^3 \phi \sin^2 \theta (d\theta \wedge d\rho \wedge d\phi) - \rho^2 \sin \phi \cos^2 \phi \sin^2 \theta (d\theta \wedge d\phi \wedge d\rho) $$

Factoring out a $\rho^2 \sin \phi$ from the entire sum, and using the alternating nature to swap $1$ forms to rewrite everything in terms of $d\rho \wedge d\phi \wedge d\theta$:

$$ \rho^2 \sin \phi(\sin^2 \phi \cos^2 \theta + \cos^2 \phi \cos^2 \theta + \sin^2 \phi \sin^2 \theta + \cos^2 \phi \sin^2 \theta)  (d\rho \wedge d\phi \wedge d\theta) $$

Regrouping, we use the fact that $\sin^2 + \cos^2 = 1$ twice to find this to be equal to:

$$  \rho^2 \sin \phi ( \sin^2 \phi(\cos^2 \theta + \sin^2 \theta)+ \cos^2 \phi(\cos^2 \theta + \sin^2 \theta) (d\rho \wedge d\phi \wedge d\theta) = $$

$$  \rho^2 \sin \phi ( \sin^2 \phi + \cos^2 \phi)  (d\rho \wedge d\phi \wedge d\theta) =  \rho^2 \sin\phi \: d\rho \wedge d\phi \wedge d\theta $$

as desired.



\end{proof}

\begin{problem}{Question 4}

Recall that the electrical and magnetic fields obey the following equations (Maxwell's equations):

$$ \begin{cases} \nabla \times E = -\frac{\partial B}{\partial t} & \nabla \times B = \frac{\partial E}{\partial t} \\ \nabla \cdot E = 0  & \nabla \cdot B = 0 \end{cases} $$

Corresponding to the vector fields $E, B$, we have the following forms:

$$ E = E_1 dx + E_2 dy + E_3 dz$$

$$ B = B_1 dy \wedge dz + B_2 dz \wedge dx + B_3 dx \wedge dy $$

Let $\mathbb{R}^4$ be flat space-time with coordinates $(x,y,z,t)$. Define $F$ to be the 2-form:

$$ F= E \wedge dt + B$$

Decide which of Maxwell's equations are equivalent to the condition $dF = 0$, and prove this fact.

\end{problem}

\begin{proof}[Solution]

Before we claim anything, we compute $F$ in terms of the wedge product of $1$-forms:

$$ F = E \wedge dt + B = (E_1 dx + E_2 dy + E_3 dz) \wedge dt + ( B_1 dy \wedge dz + B_2 dz \wedge dx + B_3 dx \wedge dy) = $$
$$E_1 dx \wedge dt + E_2 dy \wedge dt + E_3 dz \wedge dt + B_1 dy \wedge dz + B_2 dz \wedge dx + B_3 dx \wedge dy $$

Now, computing $dF$, and notating $\frac{\partial E_i}{\partial x} = (E_i)_x$, and same for $y,z$, and similarly for $B_i$ we see that:

$$ dF = dE_1 dx \wedge dt + dE_2 dy \wedge dt + dE_3 dz \wedge dt + dB_1 dy \wedge dz + dB_2 dz \wedge dx + dB_3 dx \wedge dy = $$

$$ (E_1)_y dy \wedge dx \wedge dt + (E_1)_z dz \wedge dx \wedge dt + (E_2)_x dx \wedge dy \wedge dt + (E_2)_z dz \wedge dy \wedge dt + (E_3)_x dx \wedge dz \wedge dt + (E_3)_y dy \wedge dz \wedge dt + $$
$$ (B_1)_x dx \wedge dy \wedge dz + (B_1)_t dt \wedge dy \wedge dz + (B_2)_y dy \wedge dz \wedge dx + (B_2)_t dt \wedge dz \wedge dx + (B_3)_z dz \wedge dx \wedge dy + (B_3)_t dt \wedge dx \wedge dy $$

where we have omitted writing any cases where we would have two of the same $1$-forms in the wedge product, because they vanish due to $df \wedge df = 0$.

Collecting like terms by the antcommutativity of the wedge product, we see this as equal to:

$$ [(B_1)_x + (B_2)_y + (B_3)_z] (dx \wedge dy \wedge dz) + [  -(E_1)_y + (E_2)_x dx + (B_3)_t) ] (dx \wedge dy \wedge dt) +$$
$$ [ -(E_2)_z + (E_3)_y + (B_1)_t ] (dy \wedge dz \wedge dt) + [ (E_1)_z - (E_3)_x + (B_2)_t ] (dz \wedge dx \wedge dt] $$

For this to vanish, we have the following equations then:

$$ \begin{cases} (B_1)_x + (B_2)_y + (B_3)_z = 0 \\  (B_1)_t  = (E_2)_z - (E_3)_y \\ (B_2)_t =  -(E_1)_z + (E_3)_x  \\ (B_3)_t = (E_1)_y - (E_2)_x \end{cases} $$

It should be clear, that the first equation corresponds to $\nabla \cdot B = 0$.

To see the final 3 equations, we compute $\nabla \times E$:

$$ \begin{vmatrix} x & y & z \\ \frac{\partial}{\partial x} & \frac{\partial}{\partial y} & \frac{\partial}{\partial z} \\ E_1 & E_2 & E_3 \end{vmatrix} = [(E_3)_y - (E_2)_z] x + [ (E_1)_z - (E_3)_x ] y + [(E_2)_x - (E_1)_y ] z $$

Thus, we see that the last three equations correspond exactly to:

$$ \frac{\partial}{\partial t} B = - \nabla \times E $$

Thus, we have that $dF = 0$ is equivalent to the following of Maxwell's equations:

$$ \begin{cases}  \nabla \times E = -\frac{\partial B}{\partial t} \\ \nabla \cdot B = 0 \end{cases} $$


\end{proof}

\begin{problem}{Question 5}

Let $\omega = dx^1 \wedge ... \wedge dx^n$ be the volume form and $X = \sum x^i \partial/ \partial x^i$ be the radial vector field on $\mathbb{R}^n$. Compute the contraction $i_X\omega$. 

\end{problem}

\begin{proof}[Solution]

First, we compute $i_X(dx^j)$:

$$ i_X(dx^j) = dx^j(X) = dx^j\left( \sum_{k} x^k \frac{\partial}{\partial x^k} \right) = x^j $$

Now, using the fact that $i_X$ is an antiderivation of degree $-1$, we proceed inductively. Denote $dx^1 \wedge ... \wedge dx^n = \bigwedge_{i} dx^i$, that is, in order.

We wish to show that for $n \geq 2$:

$$ i_X(\omega) =  \sum_{k=1}^n (-1)^{k-1} x^k \left(\bigwedge_{j\not=k} dx^j \right) $$

Technically, we see that for the special case $n=1$, then this sum still makes sense, as the wedge product is over nothing, and we have that as we proved for $\omega = x^1$ as a 1-form, $i_X(\omega) = x^1$.

Suppose $n = 2$.

Then, of course, we have by the action on the 1-form:

$$i_X (\omega) = i_X(dx^1 \wedge dx^2) = i_X(dx^1) \wedge dx^2 + (-1) dx^1 \wedge i_X(dx^2) = x^1 dx^2 - x^2 dx^1$$

which is exactly what we want in the case $n=2$.

Now, suppose that we have that:

$$ i_X(\omega) = i_X(dx^1 \wedge ... \wedge dx^l ) = \sum_k^l (-1)^{k-1} x^k \left(\bigwedge_{j\not=k} dx^j \right)$$

for all $l < n$.

Then, consider $\omega = dx^1 \wedge ... \wedge dx^n$. We have that:

$$ i_X(\omega) = i_X(dx^1 \wedge ... \wedge dx^n) = i_X(dx^1) \wedge \bigwedge_{j=2}^n dx^j  + (-1)dx^1 \wedge i_X\left( \bigwedge_{j=2}^n dx^j \right) $$

By the inductive step, since $i_X\left( \bigwedge_{j=2}^n dx^j \right)$ is the action on a $n-1$ form, this is equal to:

$$ x^1  \bigwedge_{j=2}^n dx^j - dx^1 \wedge  \sum_{l=2}^n (-1)^{l-2} x^l \left(\bigwedge_{i\not=l, i=2}^n dx^i \right)$$

where we notice that we have $(-1)^{l-2}$ because the first term is at $l=2$, so we need to do a mild bit of reindexing.

Bringing the $(-1)$ into the sum and the $dx^1$ into the wedge product on the left, and rewriting the first term a bit, we see:

$$ x^1 (-1)^{1-1} \bigwedge_{j=1, j \not =1 }^n dx^j + \sum_{l=2}^n (-1)^{l-1} x^l \left(\bigwedge_{i\not =l, i =1}^n dx^i \right) $$ 

We identify the first term as the $l=1$ term of the sum on the right, so we can combine this into a single sum:

$$ \sum_{l=1}^n (-1)^{l-1} x^l \left(\bigwedge_{j=1, j\not =l}^n dx^j\right) $$

as desired.

Thus, we see that

$$ i_X(\omega) = \sum_{l=1}^n (-1)^{l-1} x^l \left(\bigwedge_{j=1, j\not =l}^n dx^j\right) $$

\end{proof}



\begin{problem}{Question 6}

Let $U = (0, \infty) \times (0, 2\pi) \subseteq \mathbb{R}^2$. Define $F: U \to \mathbb{R}^2$ via $F(r, \theta) = (r \cos \theta, r \sin \theta)$. Decide whether $F$ is orientation-preserving or reversing as a diffeomorphism onto its image.

\end{problem}

\begin{proof}[Solution]

As per the definition, we take the standard orientations $(r,\theta)$ for $U$ and $(x,y)$ for $\mathbb{R}^2$. By definition then, we need to see what orientation equivalence class $F^*(dx^1 \wedge dx^2)$ lands in, where we see $dx^1 \wedge dx^2$ as a representative of the orientation class of counterclockwise orientation.

Well, by the commutativity with the wedge product and the commutativity with the differential, the pullback acts via the following:

$$ F^*(dx^1 \wedge dx^2) =  F^*(dx) \wedge F^*(dy) = d(F^*(x)) \wedge d(F^*(y)) = d(r \cos \theta) \wedge d(r \sin \theta)  =$$

$$ [ \cos \theta dr - r \sin \theta d\theta ] \wedge [ \sin \theta dr + r \cos \theta d\theta ]$$

Again, using the fact that $dr \wedge dr = 0 = d \theta \wedge d\theta$, we look only at terms that contain $dr \wedge d\theta, d\theta \wedge dr$ as those will be the only surviving terms:

$$ r \cos^2 \theta( dr \wedge d\theta) - r \sin^2 \theta d\theta \wedge dr = [ r\cos^2 \theta + r \sin^2 \theta] dr \wedge d\theta  = r dr \wedge d\theta$$

Since $r \in (0,\infty)$ we see that $F^*$ takes $dx \wedge dy$ to the orientation class of $[dr \wedge d\theta]$ and hence is orientation preserving relative to the choice of orientations for the manifolds $(U, [dr \wedge d\theta]), (\mathbb{R}^2, [dx \wedge dy])$.

I realize at this point, I could've used Proposition 21.11 and just computed the Jacobian determinant. Take the ordering of the coordinates as $x^1 = r, x^2 = \theta$, and the ordering in the codomain as $y^1 =x = r \cos \theta, y^2 = y = r \sin \theta$.

$$ \det\left[ \frac{\partial F^i}{\partial x^j} \right] = \begin{vmatrix} \cos \theta & -r\sin\theta \\ \sin \theta & r \cos \theta \end{vmatrix} = r \cos \theta^2 + r \sin^2 \theta = r$$

But again, since $ r > 0$, we have that this is always positive, hence orientation preserving with respect to $[dr \wedge d\theta]$ and $[dx \wedge dy]$.



\end{proof}

\end{document}