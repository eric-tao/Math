\documentclass[10pt]{article}
\setlength{\parskip}{0.25\baselineskip}
\usepackage[margin=1in]{geometry} 
\usepackage{amsmath,amsthm,amssymb, graphicx, multicol, array}
\usepackage[font=small,labelfont=bf]{caption}
\usepackage{float}
\usepackage{bbm}
\usepackage{dsfont}
\newcommand{\supp}{{\text{supp}}} 
\newcommand{\bv}{{\text{BV}}}
\newcommand{\ac}{{\text{AC}}}
\newcommand{\vol}{{\text{Vol}}}


\ExplSyntaxOn
\NewDocumentCommand{\cycle}{ O{\;} m }
 {
  (
  \alec_cycle:nn { #1 } { #2 }
  )
 }

\seq_new:N \l_alec_cycle_seq
\cs_new_protected:Npn \alec_cycle:nn #1 #2
 {
  \seq_set_split:Nnn \l_alec_cycle_seq { , } { #2 }
  \seq_use:Nn \l_alec_cycle_seq { #1 }
 }
\ExplSyntaxOff

\newenvironment{problem}[2][]{\begin{trivlist}
\item[\hskip \labelsep {\bfseries #1}\hskip \labelsep {\bfseries #2.}]}{\end{trivlist}}

\begin{document}
 
\title{Homework \#8}
\author{Eric Tao\\
Math 285: Homework \#8}
\maketitle

\begin{problem}{Question 1}

Let $G$ be a Lie group. Define the identity component $G_0 \subset G$ as the connected component that contains the identity element $e$. Let $\mu, i$ be the multiplication and inverse maps on $G$, respectively.

(a) For any $x \in G_0$, prove that $\mu(\{ x \} \times G_0 ) \subseteq G_0 $.

(b) Show that $i(G_0) \subset G_0$. 

(c) Show that $G_0$ is an open subset of $G$.

(d) Prove that $G_0$ is a Lie group.


\end{problem}

\begin{proof}[Solution]

(a)

First, we prove $\{ x \} \times G_0$ is connected. Suppose to the contrary, there exists $U, V$ open sets such that $U \cap V = \emptyset$ and $U \cup V = \{ x \} \times G_0 \subseteq G \times G$. Then, of course, we have that $U = \{ x \} \times \overline{U}, V = \{ x \} \times \overline{V}$, since it consists of points $(x,g)$ for any $g \in G_0$. Then, taking the projection in the second coordinate, we recover open sets $\overline{U}, \overline{V}$ such that $\overline{U} \cap \overline{V} = \emptyset$ and $\overline{U} \cup \overline{V} = G_0$. But, this contradicts the connectedness of $G_0$. Thus, $\{ x \} \times G_0$ is connected.

Now, from Proposition A.42 (numbering based off of appendix A from Canvas, not the book), we have that for $f$ a connected map, $X$ a connected domain, that $f(X)$ is connected. Since $\mu$ is a $C^\infty$ map on $G \times G$, hence continuous, and $\{ x \} \times G_0$, we have that  $\mu(\{ x \} \times G_0 )$ is connected. Furthermore, since $x \in G_0$ and $e\in G_0$, we find that $\mu(x, e) = x$ is contained within $\mu(\{ x \} \times G_0 )$. Thus, since $G_0$, by definition, is the largest connected subset of $G$ for all points in $G_0$ (Corollary A.45), we must have that  $\mu(\{ x \} \times G_0 ) \subseteq G_0$.

(b)

In the same style as part (a), we see that $i(G_0)$ is connected. Furthermore, by definition, $i(e) = e^{-1} = e$. Thus, $e \in i(G_0)$, and because $G_0$ by definition contains all connected sets containing $e$, we must have that $i(G_0) \subseteq G_0$.

(c)

First, we prove that if a topological space $S$ is locally connected, then the connected components of $S$ are open.

This should be clear by a combination of the definition with the local criterion for being open. Let $T \subset S$ be a connected component, and fix some point $t \in T$. Let $t \in U$ be any neighborhood containing $t$. By definition of being locally connected, there exists a connected neighborhood $V$ such that $t \in V$ and $V \subseteq U$. However, since $T$ is a connected component, we must have that $V \subseteq T$. Thus, we have found a neighborhood of $t$ completely contained within $T$. Since the choice of $t$ were arbitrary, we conclude that $T$ is open.

Since we may do this for each connected component, we conclude that every connected component is open.

Now, we wish to show that for an abstract Lie group $G$, $G$ is locally connected. First, we recall that $G$ is also a manifold. So fix a $g \in G$. Since $G$ is a manifold, we may find a chart $(U, \phi)$ such that $g \in U$ and such that $\phi$ is a homeomorphism to some $\mathbb{R}^n$, $n \in \mathbb{N}^+$. In particular then, looking at $\phi(U)$, we may find the connected component of $\phi(U) \subseteq \mathbb{R}^n$ that contains $\phi(g)$. Denote this component $V$. Then, because $\phi$ is a homeomorphism, $\phi^{-1}(V)$ is a connected open set that contains $g$, as requested. Since the choice of $g$ was arbitrary, this implies that $G$ is locally connected.

Then, by the earlier lemma, since $G_0$ is a connected component, and $G$ is locally connected, $G_0$ is open.

(d)

First, we note that by (a), (b), we have that $G_0$ is an abstract subgroup of $G$. Let $x, y \in G_0$. Then, by part (b), we have that $y^{-1} \in G_0$. Then, by part (a), we have that $\mu(x, y^{-1}) = xy^{-1} \in G_0$. Since the choice of $x, y$ was arbitrary, by the subgroup criterion, $G_0$ is an abstract subgroup, and hence an abstract group.

Now, by part (c), we have that since $G_0$ is an open subset of $G$, it is automatically a $C^\infty$ manifold via the subspace topology.

Hence, we need only show that $\mu, i$ restricted onto $G_0$ remain $C^\infty$ maps. However, because we may realize these as compositions of the following form, where $j: G_0 \to G$ is the inclusion map:

$$\begin{cases} \mu\big|_{G_0} & = \mu \circ (j, j):  G_0 \times G_0 \to G \times G \to G \\ i\big|_{G_0} & = i \circ j: G_0 \to G \to G \end{cases} $$

Since inclusion is certainly smooth, these are the composition of smooth maps, hence the restrictions are smooth.

Therefore, since $G_0$ is an abstract subgroup, with $C^\infty$ manifold structure as a subset of $G$, and $C^\infty$ multiplication and inverse maps, $G_0$ is a Lie group.


\end{proof}

\begin{problem}{Question 2}

Let $G$ be a Lie group with $\mu: G \times G \to G$ as its multiplication map. Let $l_a: G \to G$ and $r_b: G \to G$ denote left multiplication by $a$ and right multiplication by $b$ for $a,b \in G$, respectively.

Show that the differential of $\mu$ at a point $(a,b) \in G \times G$ has the form:

$$ \mu_{*, (a,b)} (X_a, Y_b) = (r_b)_{*}(X_a) + (l_a)_{*}(Y_b)$$

\end{problem}

\begin{proof}[Solution]

As always, let $c(t) = (c_1(t), c_2(t))$ where, denoting a derivative with respect to $t$ by $'$, we have the following:

$$ \begin{cases} c_1(0) = a & c_1'(0) = X_a \\ c_2(0) = b & c_2'(0) = Y_b\end{cases}$$

Then, using this curve to compute the differential, we have that via the product rule:

$$ \mu_{*, (a,b)} (X_a, Y_b) = \frac{d}{dt}\bigg|_{t = 0} (\mu \circ c)(t) = \frac{d}{dt}\bigg|_{t = 0} (c_1(t) c_2(t)) = c_1'(t)c_2(t) + c_1(t) c_2'(t) \big|_{t=0} =$$

$$ c_1'(0)c_2(0) + c_1(0) c_2'(0) = X_a b + a Y_b$$

Now, we may do the same procedure for $ (r_b)_{*}(X_a), (l_a)_{*}(Y_b)$, using the following curves:

$$ \begin{cases} d(0) = a & d'(0) = X_a \\ e(0) = b & e'(0) = Y_b \end{cases}$$

We find that: 

$$\begin{cases} (r_b)_{*}(X_a) = \frac{d}{dt}\bigg|_{t = 0} (r_b \circ d)(t) =  \frac{d}{dt}\bigg|_{t = 0} d(t) b = d'(0)b = X_a b \\  (l_a)_{*}(Y_b) = \frac{d}{dt}\bigg|_{t = 0} (l_a \circ e)(t) = \frac{d}{dt}\bigg|_{t = 0} a e(t) = ae'(0) = aY_b \end{cases} $$

Thus, we see that $\mu_{*, (a,b)} (X_a, Y_b)  =  X_a b + a Y_b =  (r_b)_{*}(X_a) +  (l_a)_{*}(Y_b)$ as desired.

\end{proof}

\begin{problem}{Question 3}
Let $G$ be a Lie group with $\mu: G \times G \to G$ as its multiplication map and $i: G \to G$ as its inverse map, and denote the identity element as $e$. Let $l_a: G \to G$ and $r_b: G \to G$ denote left multiplication by $a$ and right multiplication by $b$ for $a,b \in G$, respectively.

Show that the differential of $i$ at a point $a \in G$ has the form:

$$ i_{*, a}(Y_a) = -(r_{a^{-1}})_{*}(l_{a^{-1}})_{*} Y_a$$

\end{problem}

\begin{proof}[Solution]

Let $c$ be a curve such that $c(0) = a, c'(0) = Y_a$. We may then look at the following composition of maps:

$$ G \xrightarrow{(i,\text{id})} G \times G \xrightarrow{\mu} G $$

Firstly, we notice that:

$$  ( \mu \circ (i,\text{id}) )_{*, a}(Y_a) =  (\mu \circ (i,\text{id}) \circ c)'(0) =  \frac{d}{dt}\bigg|_{t=0} (\mu(c(t), c(t)^{-1})) =\frac{d}{dt}\bigg|_{t=0} c(t)c(t)^{-1} =  \frac{d}{dt}\bigg|_{t=0} e = \overline{0}  $$

where we denote $\overline{0}$ as the 0 vector.

But also, by the chain rule, we have that:

$$ ( \mu \circ (i,\text{id}) )_{*, a} =  \mu_{*, (a^{-1}, a)} \circ (i,\text{id}) )_{*, a} = \mu_{*, (a^{-1}, a)} \circ ( i_{*, a}\text{id}_{*, a}) )$$

and therefore:

$$ \mu_{*, (a^{-1}, a)} \circ ( i_{*, a}\text{id}_{*, a}) ) (Y_a) =  \mu_{*, (a^{-1}, a)} ( i_{*, a}(Y_a), Y_a)$$

By question 2, we have that:

$$  \mu_{*, (a^{-1}, a)}( i_{*, a}(Y_a), Y_a) = (r_a)_*  (i_{*, a}(Y_a)) + (l_{a^{-1}})_*( Y_a)$$

Thus, we have that:

$$ (r_a)_*  (i_{*, a}(Y_a)) + (l_{a^{-1}})_*( Y_a) = 0 \implies  (r_a)_*( i_{*, a}(Y_a)) ) = -(l_{a^{-1}})_*(Y_a) $$

Here, we notice that $ r_{a^{-1}} \circ r_a = \text{id}$. This should be easy to see as for any $g \in G$, we have that:

$$ r_{a^{-1}} \circ r_a(g) =  r_{a^{-1}}( r_a(g)) =  r_{a^{-1}} (ga) = gaa^{-1} = ge = g $$

Thus, let $Z_{a^{-1}}$ be an arbitrary tangent vector at $a^{-1}$ and via yet another curve $d(0) = a^{-1}, d'(0) = Z_{a^{-1}}$, we can see that:

$$((r_{a^{-1}})_{*, e} \circ (r_a)_{*, a^{-1}}) ( Z_{a^{-1}}) = \frac{d}{dt} \bigg|_{t=0} r_{a^{-1}} \circ r_{a} \circ d(t) =  \frac{d}{dt} \bigg|_{t=0} d(t) a a^{-1} =  \frac{d}{dt} \bigg|_{t=0} d(t) = d'(0) = Z_{a^{-1}} $$

where we've used the chain rule on $((r_{a^{-1}})_{*, e} \circ (r_a)_{*, a^{-1}})  = (r_{a^{-1}} \circ r_a)_{*, a^{-1}}$ 

Thus, since $Z_{a^{-1}}$ was arbitrary, we may conclude that $((r_{a^{-1}})_{*, e} \circ (r_a)_{*, a^{-1}})$ acts as identity.

Therefore, applying $(r_{a^{-1}})_{*, e}$ on the left to both side of the above differential equation we find that:

$$  (r_a)_*( i_{*, a}(Y_a)) ) = -(l_{a^{-1}})_*(Y_a) \implies (r_{a^{-1}})_{*, e}\circ  (r_a)_*( i_{*, a}(Y_a)) ) = (r_{a^{-1}})_{*, e}\circ -(l_{a^{-1}})_*(Y_a) \implies $$

$$ ( (r_{a^{-1}})_{*, e}\circ  (r_a)_{*, a^{-1}}) \circ i_{*, a}(Y_a) = - (r_{a^{-1}})_{*, e}(l_{a^{-1}})_*(Y_a) \implies \text{id}  \circ i_{*, a}(Y_a) = - (r_{a^{-1}})_{*, e}(l_{a^{-1}})_*(Y_a) \implies$$

$$  i_{*, a}(Y_a) = - (r_{a^{-1}})_{*, e}(l_{a^{-1}})_*(Y_a) $$

as desired.
\end{proof}

\begin{problem}{Question 4}

Define the unitary group as:

$$U(n) = \{ A \in GL_n(\mathbb{C}) : A^\dag A = I \}$$

where $A^\dag$ denotes the conjugate transpose of $A$ that is $[A^\dag]_{ij} = [\overline{a}_{ji} ]$.

Show that $U(n)$ is a regular submanifold of $GL_n(\mathbb{C})$ with dimension $n^2$.


\end{problem}

\begin{proof}[Solution]

In the same style as the orthogonal group, we first notice that for any $D \in GL_n(\mathbb{C})$, that:

$$ (D^\dag D)^\dag  = D^\dag D^{\dag^\dag} = D^\dag D $$

where we get the order reversing part from the properties of the transpose, and the fact that $ D^{\dag^\dag} = D$ as $\overline{\overline{a}} = a$ and $A^{T^T} = A$.

Thus, we have that matrices of the form $ D^\dag D$ are Hermitian.

Denote the set of Hermitian matrices as $H_n = \{ A \in GL_n(\mathbb{C}) : A = A^\dag \}$. We first notice that Hermitian matrices form a real vector space where we inherit the operations from $GL_n(\mathbb{R})$ and so we need only check closure under these operations. Clearly, the sum of two Hermitian is Hermitian  since the transpose and complex conjugation distribute over addition. The 0 matrix is clearly Hermitian, as the complex conjugate of 0 is itself. And for any Hermitian matrix $A$, of course $-A$ and $kA$ are Hermitian, since real scalars factor out from complex conjugation. Thus, these matrices form a real vector space.

In particular, we notice that because of the transpose, a Hermitian matrix is completely determined by its upper triangular entries. Moreover, examining the entries on the diagonal, since the transpose keeps these entries constant, the condition $\overline{a}_{ii} = a_{ii}$ forces $a_{ii}$ to be purely real. Thus, as a real vector space, we have that:

$$ \dim(H_n) = n + 2( 1 + ... + n-1) = n + n(n-1) = n^2$$ 

Then, in the same vein, we consider the map:

$$ F: GL_n(\mathbb{C}) \to H_n \text{ via } A \mapsto A^\dag A$$

and we wish to show that the differential is surjective, where we identify $T H_n \cong H_n$ as vector spaces:

$$F_{*, A}: T_A GL_n(\mathbb{C}) \to T_{f(A)}(H_n) \cong H_n $$

Using a curve $c(0) = A, c'(0) = X$ for $X \in T_A(GL_n(\mathbb{C})) \cong \mathbb{R}^{2(n \times n)}$, we see that:

$$ F_{*, A} (X) = \frac{d}{dt}\bigg|_{t=0} F(c(t)) = \frac{d}{dt}\bigg|_{t=0} c(t)^\dag c(t)  = c'(0)^\dag c(0) + c(0)^\dag c'(0) = X^\dag A + A^\dag X $$

where we justify the product rule by seeing that for $A = [a_{jk}(t) + b_{jk}(t)i]_{jk}, C = [c_{lm}(t) + d_{lm}(t)i]_{lm}$

$$(AC)_{pq} = \sum_{r = 1}^n A_{pr} C_{rq} = \sum_{r = 1}^n [a_{pr}(t) + b_{pr}(t)i]_{pr} [c_{rq}(t) + d_{rq}(t)i]_{rq}$$

And here, taking the derivative with respect to these scalar functions, we use the product rule to regroup as: $$\sum_{r = 1}^n [a_{pr}'(t) + b_{pr}'(t)i]_{pr} [c_{rq}(t) + d_{rq}(t)i]_{rq} +  [a_{pr}(t) + b_{pr}(t)i]_{pr} [c_{rq}'(t) + d_{rq}'(t)i]_{rq}$$

In any case, we see that for $F_{*, A}$ to be surjective on $U(n)$, we need that for an arbitrary $C \in H_n, A \in U(n)$ to find a $X \in M_n(\mathbb{C})$ such that:

 $$X^\dag A + A^\dag X = C$$

In the same fashion, we notice that $(X^\dag A)^\dag = A^\dag X$, so we need only show that there is an $X$ such that $A^\dag X =\frac{1}{2} C$. Since $A \in U(n)$, we have that:

$$ AA^\dag X = AC \implies X =\frac{1}{2} AC$$

Thus, the differential is surjective, and since the only property of the point $A$ we used was the defining property of $U(n)$,  every point of $U(n)$ is a regular point, hence a regular level set of $F$. By the regular level set theorem then, $U(n)$ is a regular submanifold of $GL_n(\mathbb{C})$, with real dimension $\dim(U(n)) = \dim(GL_n(\mathbb{C})) - \dim(H_n) = 2n^2 - n^2 = n^2$
\end{proof}

\begin{problem}{Question 5}
Let $\mathbb{H}$ denote the skew field of quaternions. Define the sympletic group as:

$$ Sp(n) = \{ A \in GL_n(\mathbb{H}) : A^\dag A = I \}$$

where $A^\dag$ denotes the quaternionic conjugate transpose of $A$. Show that $Sp(n)$ is a regular submanifold of $GL_n(\mathbb{H})$ and compute its dimension.


\end{problem}

\begin{proof}[Solution]

In the same fashion as problem 4, we start by showing that for $A, B \in GL_n(\mathbb{H})$, that $(AB)^\dag = B^\dag A^\dag$.

First, we prove that conjugation is an antihomomorphism.

%Let $p, q \in \mathbb{H}$. Then, we may express this as $p = p_0 + p_1i + p_2 j + p_3k, q = q_0 + q_1i + q_2 j + q_3k$.

%We see then that:

%$$pq = (p_0q_0 - p_1q_1 - p_2 q_2 - p_3 q_3) + i(p_0q_1 + p_1q_0 +p_2q_3 - p_3 q_2) + j(p_0q_2 - p_1q_3 + p_2 q_0 + p_3 q_1) + k (p_0 q_3 + p_1q_2 - p_2 q_1 + q_3 p_0) $$

%On the other hand, we look at $\overline{q} * \overline{p}$:

%$$(q_0 - q_1i - q_2 j - q_3k)(p_0 - p_1i - p_2 j - p_3k) = 

Since conjugation distributes over addition and we may factor (real) scalars, and $\mathbb{H}$ is simply a real vector space with respect to $1, i, j, k$, we need only prove this true for the basis.

It is clear that this is true for any combination with $1$, as $\overline{1} = 1$, so wlog, choosing $i$, $\overline{1i} = \overline{i} = \overline{1}\overline{i}$.

Similarly, this is true for any two of the same. WLOG choosing $i$, we have that $\overline{ii} = \overline{-1} = -1 = -i -i = \overline{i}\overline{i}$.

Thus, we need only show this for two different choices of $i,j,k$. First, choose $i,j$. We have that $\overline{ij} = \overline{k} = -k = ji = (-j)(-i) = \overline{j}\overline{i}$.

Without too much trouble, we can see that this is true for any cyclic choices of $i,j,k$, i.e. $(i,j), (j,k), (k,i)$. And, without too much convincing, this is true for the anti-cyclic permutations, where we introduce a negative: $\overline{ji} = \overline{-k} = k = ij = (-i)(-j) = \overline{i}\overline{j}$. Thus, due to the linear properties of the quaternion conjugation, we have that $\overline{pq} = \overline{q}\overline{p}$.

Next, we prove our desired result that $(AB)^\dag = B^\dag A^\dag$:

We look at $[(AB)^\dag]_{ij}$. By the definition of matrix multiplication, this is:

$$[(AB)^\dag]_{ij} = \overline{AB}_{ji} = \overline{\sum_{m=1}^n A_{jm}B_{mi} } = \sum_{m=1}^n \overline{A_{jm}B_{mi}} = \sum_{m=1}^n \overline{B}_{mi} \nobreakspace \overline{A}_{jm}$$

and on the other hand, we have that:

$$[B^\dag A^\dag]_{ij} = \sum_{m=1}^n B^\dag_{im} A^\dag_{mj} = \overline{B}_{mi} \overline{A}_{jm}$$

Therefore, going across entries, we conclude that $(AB)^\dag = B^\dag A^\dag$.

Now, we consider the set:

$$ H_n(\mathbb{H}) = \{ A \in GL_n(\mathbb{H}) : A = A^\dag \}$$

Clearly, we see that for any $A \in GL_n(\mathbb{H})$, that since $\overline{\overline{p}} = p$ for $p \in \mathbb{H}$, and $A^{T^T} = A$ for any matrix, that:

$$ (A^\dag A)^\dag = A^\dag (A^\dag)^\dag = A^\dag A $$

Defining the map:

$$ G: GL_n(\mathbb{H}) \to H_n(\mathbb{H}) $$

that sends a matrix $A \mapsto A^\dag A$, clearly, $Sp(n) = G^{-1}(I)$. So, we wish to show that this level set is actually regular.

To do so, we examine the differential:

$$ G_{*, A} T_A GL_n(\mathbb{H}) \to T_{G(A)}(H_n(\mathbb{H})) \cong H_n(\mathbb{H}) $$

Using a curve $c$ such that $c(0) = A, c'(0) = X$ for an arbitrary vector $X \in GL_n(\mathbb{H})$, we see that:

$$ G_{*, A}(X) = \frac{d}{dt} \bigg|_{t=0} F(c(t)) = \frac{d}{dt} \bigg|_{t=0} c(t)^\dag c(t) = c'(0)^\dag c(0) + c(0)^\dag c'(0)  = X^\dag A + A^\dag X$$

Again, this time without proof, we justify the product rule in the same way as in problem 4.

Again, for this to be surjective on the set $Sp(n)$, we would need that for an arbitrary $C \in H_n(\mathbb{H}), A \in Sp(n)$, we may find a $X \in M_n(\mathbb{H})$, where we recognize that as isomoprhic to the tangent space of $GL_n(\mathbb{H})$, such that:

$$ X^\dag A + A^\dag X = C$$

In the same fashion as question 4, because $(X^\dag A)^\dag = A^\dag X$, this is equivalent to showing that there exists $X$ such that $A^\dag X =\frac{1}{2} C$. Because $A \in Sp(n)$, multiplying on the left by $A$, we obtain on the left hand side:

$$ AA^\dag X = (A^\dag A)^\dag X = I^\dag X= IX = X $$

And therefore, we conclude that the desired $X$ is $\frac{1}{2} AC$.

Therefore, the differential is surjective at every point $A$. Since the only property we used about $A$ was that it lived in $Sp(n)$, this is true for all points in $Sp(n)$, and therefore we conclude that this is a regular level set of $F$, and hence a regular submanifold.

By the rank-nullity theorem, we compute the dimension as:

$$ \dim(Sp(n)) = \dim(GL_n(\mathbb{H})) - \dim(H_n(\mathbb{H})) = 4n^2 - (n + 4(1 + .... + n-1)) = 4n^2 - \left(n + 4 \frac{n(n-1)}{2}\right) $$
$$ = 4n^2 - (n + 2n^2 -2n) = 2n^2 +n $$

where $GL_n(\mathbb{H})$ has dimension $4n^2$ as it has $n^2$ entries and each entry is a $4$ dimensional real vector, and $H_n(\mathbb{H})$ follows the same argument as in problem 4, where the diagonal is strictly real, the strictly lower diagonal entries are determined exactly by the striclty upper diagonal entries.


\end{proof}



\begin{problem}{Question 6}
Let $X \in M_n(\mathbb{C})$. We call it skew-Hermitian if its complex transpose $X^\dag$ is equal to $-X$.

Let $V$ be the vector space of $n\times n$ skew-Hermitian matrices. Show that $\dim(V) = n^2$.

\end{problem}

\begin{proof}[Solution]

We use the same procedure as determining the dimension of the Hermitian matrices as done in Question 5.

We notice that because of the condition $X^\dag =- X$, that we have for off-diagonal entries: $[a_{ij}] = [-\overline{a}_{ji}]$. Therefore, we have that the strictly lower triangular entries are completely determined by the strictly upper triangular entries.

Furthermore, looking at the diagonal entries, since the transpose keeps these entries fixes, we have that for $a = x + yi$, that: 

$$x + yi = a_{ii} = -\overline{a}_{ii} = -x +yi \implies x = 0 $$

That is, the diagonal entries must be strictly imaginary.

Hence, in terms of dimensionality as a real vector space, we have $1 + ... + n-1$ free complex numbers in the strictly upper triangular entiries, and $n$ strictly imaginary numbers along the diagonal. Thus, the real dimension is:

$$ \dim(V) = n + 2(1 + ... + n-1) = n + 2 \frac{(n-1)(n-1+1)}{2} = n + n(n-1) = n^2 $$

as desired. 

\end{proof}


\begin{problem}{Question 7}

Show that the tangent space of $U(n)$ at the identity $I \in U(n)$ can be identified with the vector space of $n\times n$ skew-Hermitian matrices.

\end{problem}

\begin{proof}[Solution]

First, let $X$ be a tangent vector that lives in $T_I(U(n))$.

We know that there exists a curve $c: (-\epsilon, \epsilon) \to U(n)$ such that $c(0) = I, c'(0) = X$.

Because $c(t) \in U(n)$, this implies then that for all $t$, that for the map:

$$ F: GL_n(\mathbb{C}) \to GL_n(\mathbb{C}) \text{ via } F(A) = A^\dag A $$

that $ F(c(t)) = I$

Taking the derivative of the left hand side, evidently, we have that:

$$ \frac{d}{dt}\bigg|_{t=0} F(c(t)) = (F \circ c)_{*} \frac{d}{dt} \bigg|_{t=0} = F_{*, I} \circ c_{*, 0} \frac{d}{dt}\bigg|_{t=0} $$

where the first equality is simply the definition of the differential, viewing $\frac{d}{dt}\bigg|_{t=0}$ as a tangent vector in $T_0 (-\epsilon, \epsilon)$, and the second equality comes from the chain rule.

We see then that $ c_{*, 0} \frac{d}{dt}\bigg|_{t=0} = \frac{d}{dt} \bigg|_{t=0} c(t) = c'(0)$, again, by definition of the differential, so we have that:

$$  \frac{d}{dt}\bigg|_{t=0} F(c(t))  = F_{*, I} (c'(0)) = F_{*, I}(X) = X^\dag + X$$

where we use the differential $F_{*, A} = X^\dag A + A^\dag X$ computed in problem 4, with $A = I$.

However, the right hand side is simply:

$$\frac{d}{dt}\bigg|_{t=0} I = 0 $$

So, we have that:

$$ \frac{d}{dt}\bigg|_{t=0} \left(  F(c(t)) = I \right) \implies X^\dag + X = 0 \implies X^\dag = -X $$

Thus, as a subspace of $T GL_n(\mathbb{C})$, we have that $X \in TU(n)$ implies that $X^\dag = -X$, that is, $TU(n) \subseteq V$ where we identify this copy of the skew-Hermitian matrices living in $TGL_n(\mathbb{C}) \cong M_n(\mathbb{C})$. 

However, the dimensionality of $U(n)$ as a real vector space as computed in problem 4 is $n^2$, and the dimensionality of $V$ viewed as a real vector as computed in problem 6 is $n^2$. Therefore, since $TU(n), V$ have the same dimension as real vector spaces and $\dim(U(n)) = \dim(TU(n)) = \dim(V)$, we must have that $TU(n) = V$, and therefore, the tangent space of $U(n)$ can be identified with a copy of the skew-Hermitian matrices, with some identification of the tangent space of $GL_n(\mathbb{C}) \cong M_n(\mathbb{C})$.

\end{proof}

\end{document}