\documentclass[10pt]{article}
\setlength{\parskip}{0.25\baselineskip}
\usepackage[margin=1in]{geometry} 
\usepackage{amsmath,amsthm,amssymb, graphicx, multicol, array}
\usepackage[font=small,labelfont=bf]{caption}
\usepackage{float}
\usepackage{bbm}

\newcommand{\supp}{{\text{supp}}} 
\newcommand{\bv}{{\text{BV}}}
\newcommand{\ac}{{\text{AC}}}
\newcommand{\vol}{{\text{Vol}}}

\newenvironment{problem}[2][]{\begin{trivlist}
\item[\hskip \labelsep {\bfseries #1}\hskip \labelsep {\bfseries #2.}]}{\end{trivlist}}

\begin{document}
 
\title{Homework \#2}
\author{Eric Tao\\
Math 285: Homework \#2}
\maketitle

\begin{problem}{Question 1}

Let $\omega$ be the $1$-form $zdx - dz$ and let $X$ be the vector field $y \partial/\partial x + x \partial/\partial y$ on $\mathbb{R}^3$.

Compute $\omega(X)$ and $d\omega$. 

\end{problem}

\begin{proof}[Solution]

We recall that $\omega(X)$, in coordinates, is simply $\sum_{i} a_i b^i$, where $\omega = \sum_i a_i dx^i$, and $X = \sum_j b^j \frac{\partial}{\partial x^j} $

Thus, we have that:

$$\omega(X) = \sum_{i} a_i b^i = z * y + 0 * x + -1 * 0 = yz $$

In a similar fashion, recall that, by definition:

$$ d\omega = \sum_{i,j} \frac{\partial a_i}{\partial x^j} dx^j \wedge dx^i$$

Thus, we have that:

$$ d\omega = 1 dz \wedge dx = dz \wedge dx $$

since we notice that the only non-vanishing partial of $z dx$ is $\partial/\partial z$ and none of the partials of $-dz$ survive. 

\end{proof}

\begin{problem}{Question 2}

Suppose the standard coordinates on $\mathbb{R}^3$ are called $\rho, \phi, \theta$. If we have that:

$$ \begin{cases} x = \rho \sin \phi \cos \theta \\  y = \rho \sin \phi \sin \theta \\ z = \rho \cos \phi \end{cases} $$

Compute the following quantities in terms of $d\rho, d\phi, d\theta$: $dx, dy, dz, dx\wedge dy \wedge dz$.

\end{problem}

\begin{proof}[Solution]

Clearly, $x, y, z$ are $C^\infty$ functions on $\mathbb{R}$. Applying Proposition 4.3, which states that $df = \sum \partial f/ \partial x^i dx^i$, we see that:

$$ \begin{cases} dx = \sin \phi \cos \theta d\rho + \rho \cos \phi \cos \theta  d\phi - \rho \sin \phi \sin \theta d\theta \\ dy = \sin \phi \sin \theta d\rho + \rho \cos \phi \sin \theta  d\phi + \rho \sin \phi \cos \theta d\theta \\ dz = \cos \phi d\rho - \rho \sin \phi d\phi  \end{cases}$$

Now, we may compute the wedge product $dx \wedge dy \wedge dz$. We recall that odd degree multivectors vanish under the wedge product and that the wedge product distributes over addition, and since our 1-forms are exactly covector fields, hence covectors at each point, we need only consider elements that include some permutation of $d\rho \wedge d\phi \wedge d\theta$:

$$ dx \wedge dy \wedge dz = (\sin \phi \cos \theta d\rho) \wedge (\rho \sin \phi \cos \theta d\theta) \wedge ( - \rho \sin \phi d\phi) + ( \rho \cos \phi \cos \theta  d\phi ) \wedge (\rho \sin \phi \cos \theta d\theta) \wedge (\cos \phi d\rho) +  $$

$$ ( - \rho \sin \phi \sin \theta d\theta) \wedge (\sin \phi \sin \theta d\rho)  \wedge ( - \rho \sin \phi d\phi) +  ( - \rho \sin \phi \sin \theta d\theta) \wedge ( \rho \cos \phi \sin \theta  d\phi ) \wedge (\cos \phi d\rho) =  $$

$$  -\rho^2 \sin^3 \phi \cos^2 \theta (d\rho \wedge d\theta \wedge d\phi) + \rho^2 \sin\phi \cos^2 \phi \cos^2 \theta (d\phi \wedge d\theta \wedge d\rho) + $$
$$ \rho^2 \sin^3\phi \sin^2 \theta (d\theta \wedge d\rho \wedge d\phi) - \rho^2 \sin \phi \cos^2 \phi \sin^2 \theta (d\theta \wedge d\phi \wedge d\rho) $$ 

Rewriting everything to be of the form $d \rho \wedge d\phi \wedge d\theta$ using graded commutativity, and pulling out $\rho^2$:

$$\rho^2 (\sin^3 \phi \cos^2 \theta +  \sin\phi \cos^2 \phi \cos^2 \theta + \sin^3\phi \sin^2 \theta + \sin \phi \cos^2 \phi \sin^2 \theta) ( d \rho \wedge d\phi \wedge d\theta) $$

Looking at the first two and last two terms, we notice that:
$$\begin{cases} \sin^3 \phi \cos^2 \theta +  \sin\phi \cos^2 \phi \cos^2 \theta = \sin \phi \cos^2 \theta (\sin^2 \phi + \cos^2 \phi ) = \sin \phi \cos^2 \theta \\ \sin^3\phi \sin^2 \theta + \sin \phi \cos^2 \phi \sin^2 \theta  =  \sin \phi \sin^2 \theta(\sin^2 \phi + \cos^2 \phi) = \sin \phi \sin^2 \theta \end{cases}$$

So, in the end, we have that:

$$ dx \wedge dy \wedge dz = \rho^2 (\sin \phi \cos^2 \theta +  \sin \phi \sin^2 \theta) (d\rho \wedge d\phi \wedge d\theta) = \rho^2 \sin \phi  (d\rho \wedge d\phi \wedge d\theta) $$



\end{proof}

\begin{problem}{Question 3}

Let $V$ be a vector space of dimension 3 with basis $e_1, e_2, e_3$ and dual basis $\alpha^1, \alpha^2, \alpha^3$. For a $1$-covector $\alpha = \sum_{i=1}^3 a_i \alpha^i$ on $V$, we associate the vector $v_\alpha = \langle a_1, a_2, a_3 \rangle \in \mathbb{R}^3$. For a $2$-covector

$$ \gamma = c_1 \alpha^2 \wedge \alpha^3 + c_2 \alpha^3 \wedge \alpha^1 + c_3 \alpha^1 \wedge \alpha^2$$

we assoicate the vector $v_\gamma = \langle c_1, c_2, c_3 \rangle \in \mathbb{R}^3$.

Show that under this correspondence, the wedge product of $1$-covectors corresponds to the cross product of vectors in $\mathbb{R}^3$, that is:

$$ v_{\alpha \wedge \beta} = v_\alpha \times v_\beta $$

\end{problem}

\begin{proof}[Solution]

Recall that if we have identifications of $1$-covectors: $\alpha = \sum_i a_i dx^i$ and $\beta = \sum_j b_j dx^j$, then we have that:

$$ \alpha \wedge \beta = \sum_{i,j} (a_i b_j) dx^i \wedge dx^j$$

Writing this out in terms of coordinates, with respect to the dual basis, i.e. $dx^i = \alpha^i$, we have that:

$$\alpha \wedge \beta = a_1 b_2 \alpha^1 \wedge \alpha^2 + a_1 b_3 \alpha^1 \wedge \alpha^3 + a_2 b_1 \alpha^2\wedge \alpha^1 + a_2 b_3 \alpha^2 \wedge \alpha^3 + a_3 b_1 \alpha^3 \wedge \alpha^1 + a_3 b_2 \alpha^3 \wedge \alpha^2 =$$
$$ (a_2b_3 - a_3b_2) \alpha^2 \wedge \alpha^3 + (a_3b_1 - a_1b_3) \alpha^3 \wedge \alpha^1 + (a_1b_2  - b_1a_2) \alpha^1\wedge \alpha^2 $$

where we've used the fact that since $\alpha^i$ are covectors, $\alpha^i \wedge \alpha^i = 0$.

So, we have that:

$$ v_{\alpha\wedge\beta} = \langle a_2b_3 - a_3b_2, a_3b_1  - a_1b_3, a_1b_2 - b_1a_2 \rangle $$

In contrast, let's consider the cross product of $v_\alpha \times v_\beta = \langle a_1, a_2, a_3 \rangle \times \langle b_1, b_2, b_3 \rangle $. Using matrix notation:

$$v_\alpha \times v_\beta = \begin{vmatrix} \mathbf{i}& \mathbf{j}  & \mathbf{k} \\ a_1  &a_2 &a_3 \\ b_1 & b_2 & b_3 \end{vmatrix} =  (a_2b_3 - a_3b_2)i - (a_1 b_3 - a_3 b_1)j + (a_1b_2 - a_2b_1)k =$$
$$ \langle a_2b_3 - a_3b_2, a_3b_1 - a_1b_3, a_1b_2 - a_2b_1 \rangle $$

We notice these are the same, and we conclude that $ v_{\alpha \wedge \beta} = v_\alpha \times v_\beta $.


\end{proof}

\begin{problem}{Question 4}

Let $A = \oplus_{k=-\infty}^\infty A^k$ be a graded algebra over a field $K$, with $A^k = 0$ for $k  < 0$. Let $m \in \mathbb{Z}$.

Define a superderviation of $A$ with degree $m$ as a $K$-linear map $D: A \to A$ such that for all $k \in \mathbb{Z}$, we have that $D(A^k) \subset A^{k+m}$ and that for all $a \in A^k, b \in A^l$:

$$D(ab) = (Da) b + (-1)^{km}a Db $$

Let $D_1, D_2$ be superderivations of $A$ with degrees $m_1, m_2$ respectively. Define their commutator as:

$$ [D_1, D_2] = D_1 \circ D_2 - (-1)^{m_1m_2} D_2 \circ D_1 $$

Show that the commutator $[D_1, D_2]$ is a superderivation of degree $m_1 + m_2$.

\end{problem}

\begin{proof}[Solution]

Fix a $k \in \mathbb{Z}$, and suppose $x \in A^k$.

First, we wish to show that $[D_1,D_2](x) \in A^{k+m_1 + m_2}$.

It is enough to show that $D_1 \circ D_2(x), D_2 \circ D_1(x) \in A^{k+m_1 + m_2}$, because if that is true, then the sum and multiplication by scalars remains in $A^{k+m_1 + m_2}$ due to it being an algebra.

Well, because $D_1,D_2$ are superderivations, we have that $D_2(x) \in A^{k+m_2}$, so $D_1(D_2(x)) \in A^{k+m_2 + m_1}$. Similarly, $D_1(x) \in A^{k + m_1}$, so $D_2(D_1(x)) \in A^{k + m_1 + m_2}$ for the same reason.

Since algebras are closed under addition and scalar multiplication, this implies that $D_1 \circ D_2 - (-1)^{m_1m_2} D_2 \circ D_1 (x) \in A^{k+m_1 +m_2}$.

Now, we need to check the condition on products. Fix $k,l \in \mathbb{Z}$, and suppose that $x \in A^k, y \in A^l$.

Consider $[D_1,D_2](xy)$. We have that:

$$[D_1,D_2](xy) =   D_1 \circ D_2 - (-1)^{m_1m_2} D_2 \circ D_1 (xy) = D_1(D_2(xy)) - (-1)^{m_1 m_2} D_2(D_1(xy))$$

Because $D_1, D_2$ are superderivations, we have that:

$$ D_i(xy) = (D_i(x)) y + (-1)^{km_i}x (D_iy)$$

Therefore:

$$[D_1,D_2](xy) = D_1( (D_2(x)) y + (-1)^{km_2}x (D_2y)) - (-1)^{m_1 m_2} D_2((D_1(x)) y + (-1)^{km_1}x (D_1y)) =$$
$$ D_1(D_2(x)y) + (-1)^{km_2} D_1(xD_2(y)) -  (-1)^{m_1 m_2}[ D_2(D_1(x)y) + (-1)^{km_1} D_2(xD_1(y))] $$

Recalling that $D_2(x) \in A^{k+m_2}, D_1(x) \in A^{k+m_1}$, applying the fact that $D_1, D_2$ are superderivations again, we reduce to:

$$D_1 \circ D_2(x)y + (-1)^{(k+m_2) m_1} D_2(x) D_1(y) + (-1)^{km_2} [D_1(x) D_2(y) + (-1)^{km_1}x D_1 \circ D_2(y) ]  -  $$
$$(-1)^{m_1 m_2}[D_2 \circ D_1(x) y + (-1)^{(k+m_1)m_2} D_1(x) D_2(y) + (-1)^{km_1}[ D_2(x)D_1(y) + (-1)^{km_2} x D_2 \circ D_1(y) ]]  $$

First, we look at terms of the form $D_2(x) D_1(y)$. These are:

$$  (-1)^{(k+m_2) m_1} D_2(x) D_1(y) - (-1)^{m_1m_2}[ (-1)^{km_1} D_2(x) D_1(y)] = (-1)^{km_1 + m_1m_2}[ D_2(x)D_1(y) - D_2(x) D_1(y)] = 0 $$

Similarly, for $D_1(x)D_2(y)$, since $(-1)^{2p} = 1$ for $p \in \mathbb{Z}$:

$$  (-1)^{km_2} D_1(x) D_2(y) - (-1)^{m_1 m_2} (-1)^{(k+m_1)m_2} D_1(x) D_2(y) = (-1)^{km_2} [ D_1(x) D_2(y) - D_1(x) D_2(y) ] = 0$$

For terms with a $y$:

$$ D_1 \circ D_2(x)y - (-1)^{m_1 m_2}D_2 \circ D_1(x) y  = [D_1, D_2](x) y $$

And similarly, for terms with an $x$:

$$  (-1)^{km_2} (-1)^{km_1}x D_1 \circ D_2(y) - (-1)^{m_1 m_2} (-1)^{km_1} (-1)^{km_2} x D_2 \circ D_1(y) =  $$

$$ (-1)^{k(m_1 +m_2)} [ xD_1 \circ D_2(y) - (-1)^{m_1m_2} x D_2 \circ  D_1(y) ] = (-1)^{k(m_1 +m_2)}x [D_1, D_2](y)  $$

Thus, this horrible mess shows us that:

$$[D_1, D_2](xy) = [D_1, D_2](x) y +  (-1)^{k(m_1 +m_2)}x [D_1, D_2](y)$$

Therefore, $[D_1,D_2]$ satsifies the conditions to be a superderivation of degree $m_1 + m_2$.



\end{proof}

\begin{problem}{Question 5}

Consider the set $S = \mathbb{R} \setminus \{ 0 \} \cup \{ A, B \}$, the bug-eyed line or the line with two origins.

For $c,d \in \mathbb{R}$, define the following notation:

$$ \begin{cases} I_A(-c, d) = (-c, 0) \cup \{ A \} \cup (0, d) \\  I_B(-c, d) = (-c, 0) \cup \{ B \} \cup (0, d) \end{cases} $$

Define a topology on $S$ as follows: On $\mathbb{R} \setminus \{ 0 \}$, use the subspace topology from $\mathbb{R}$ with open intervals as a basis. At the point $A$, use the collection of sets $\{ I_A(-c, d) : c,d  > 0 \}$ as a basis, and analogously at $B$.

(a) Prove that the map $h: I_A(-c,d) \to (-c, d) \subseteq \mathbb{R}$ defined by:

$$ \begin{cases} h(x) = x & \text{ when } x \not = A \\ h(A) = 0 & \text{ else } \end{cases} $$

 is a homeomorphism.

(b) Show that $S$ is locally Euclidean, second countable, but not Hausdorff.

\end{problem}

\begin{proof}[Solution]

(a)

Take the map $h$ as defined in the statement above. We need only show that $h$ is a continous bijection that admits a continuous inverse.

We will show that $g: (-c,d) \to I_A(-c,d)$ defined by:

$$g(a) = \begin{cases} A & \text{ if } a= 0 \\ a & \text{ else } \end{cases}$$

is a left inverse and a right inverse.

First, consider the map $h \circ g: (-c,d) \to (-c,d)$.

Fix an $ a \in (-c,d)$. If $a = 0$, then we have that:

$$h \circ g(0) = h(g(0)) = h(A) = 0$$

Else, suppose $a \not = 0$. Then, by definition, we have that:

$$h \circ g(a) = h(g(a)) = h(a) = a$$

Thus, $g$ is a right inverse.

Similarly, looking at $g\circ h: I_A(-c,d) \to I_A(-c,d)$, fixing a $b \in I_A(-c,d)$, if $b = A$, then we have that:

$$ g \circ h(A) = g(h(A)) = g(0) = A$$

otherwise, for $b \not = A$, we have that:

$$ g \circ h(b) = g(h(b)) = g(b) = b $$

Thus, we have that $g$ acts as a left and right inverse, and thus $h$ is bijective, and $g$ is an inverse to $h$.

Now, we wish to show that $h, g$ is continuous. To do so, we need only show that pre-images of basis elements are taken to basis elements. This is because, working in our image space, suppose $U = \cup_{B \in \mathcal{B}} B$ for a collection of basis elements $\mathcal{B}$. If we have that $h^{-1}(B)$ is a basis element in our codomain for every $B$, then of course, $\cup_{B \in \mathcal{B}} h^{-1}(B)$, being a union of basis elements is an open set, and thus $h^{-1}(U)$ is open.

Then, it is enough to consider an open interval $(a,b) \subseteq (-c,d)$. If $0 \not \in (a,b)$, then $(a,b) \subseteq \mathbb{R} \setminus \{ 0 \}$. Since $S$ inherits the subspace topology on this set, then of course $(a,b)$ is a basis element of the topology on $S$. Furthermore, since $h$ acts via identity on $\mathbb{R} \setminus \{ 0 \}$, $h^{-1}((a,b)) = (a,b)$.

Now, suppose $0 \in (a,b)$. In the notation we have established then, write this interval as $(-a,b)$. Then, from the action of $h$, we see that $h^{-1}((-a,b)) = (-a,0) \cup A \cup (0,b)$. But, from the definiton of $I_A$, this is exactly $I_A(-a,b)$, and from the definition of the topology on $S$, this is exactly a basis element for neighborhoods of $A$.

Therefore, $h^{-1}(a,b)$ for any $a,b\in \mathbb{R}$ is taken to a basis element of $S$, and therefore $h$ is continuous.

In a similar fashion, we may do the same for $g: (-c,d) \to I_A(-c,d)$.

Take a basis element from $I_A(-c,d)$, and call it $C$. If $A \not \in C$, then of course $C$ comes from an open interval on $\mathbb{R} \setminus \{ 0 \}$, and thus $g^{-1}(C) = C$, as it acts via identity on $\mathbb{R} \setminus \{ 0 \}$.

Now, suppose $A \in C$. Then, being a basis element, $C = I_A(-a,b)$ for $-c \leq -a <  b \leq d$. Looking at the action of $g^{-1}(I_A(-a,b))$, we see that this is exactly:

$$ g^{-1}(I_A(-a,b)) = g^{-1}((-a, 0) \cup \{ A \} \cup (0, b)) =  g^{-1}((-a, 0)) \cup  g^{-1}(A)  \cup  g^{-1}(A)((0, b))=$$

$$ (-a,0) \cup \{ 0 \} \cup (0,b) = (-a,b)$$

Thus, for every basis element in  $I_A(-c,d)$, the inverse image under $g$ is a basis element of $(-c,d)$. Thus, $g$ is continuous.

Therefore, since $h$ is a continuous bijection that admits a continuous inverse, $h$ is a homeomorphism.

(b)

Without too much trouble, it should be clear that $S$ is locally Euclidean. Fix a $c,d \in \mathbb{R} : c,d > 0$. From part (a), we already have a chart from $I_A(-c,d)$ to a neighborhood of $\mathbb{R}$, an open interval, via $h$. It should be easy to see that swapping $B$ for $A$ everywhere, this also extends to a similar chart for $I_B(-c,d)$. Furthermore, on $S \setminus \{ A, B \}$, we see that we may take $f: S \setminus \{ A, B \} \to \mathbb{R}$ via $f(x) = x$, the identity, and the image is exactly $\mathbb{R} \setminus \{ 0 \}$ an open set. It should be clear that the identity is continuous. Thus, between these three charts, $S$ is locally Euclidean (of dimension 1).

Furthermore, $S$ is second countable. We may take our basis to be the union of:

1) Open intervals with rational endpoints in $\mathbb{R}$ such that either both endpoints are positive or both are negative:

$$ \{ (a,b) : a,b \in \mathbb{Q}, a \not = 0, ab > 0 \}$$

2) Open intervals of the form $I_A(-c,d)$ where $c,d > 0, c,d \in \mathbb{Q}$.

3) Open intervals of the form $I_B(-c,d)$ where $c,d > 0, c,d \in \mathbb{Q}$.

Using the fact that open intervals with rational endpoints are a countable basis for $\mathbb{R}$, we see that (1) generates the open sets for $\mathbb{R} \setminus \{ 0 \}$. Further, by the definition of the topology for $S$, (2) and (3) generate the neighborhoods for $A, B$ respectively, since for any $c,d \in \mathbb{R}$, we may take a sequence of rational numbers approaching $c, d$ from above and below, respectively.

Since each of these sets are countable, being at most $\mathbb{Q} \times \mathbb{Q}$, their union is also countable. Thus $S$ is second countable.

However, it should be clear that $S$ is not Hausdorff. Take the points $A, B$. From the definition of our topology, we already know that the neighborhoods of $A$ can be generated by $I_A(-c,d)$ and analogously for $B, I_B(-e,f)$. 

Fix any two neighborhoods $I_A(-c,d), I_B(-e,f)$. Pick any point:

$$p \in (\max\{-c,-e\}, \min\{d, f\}) \setminus \{ 0 \}\subseteq \mathbb{R}$$

Clearly, since $\max\{ -c, -e \} < p < \min\{ d,f \}$, we have that $p \in (-c,d) \setminus \{ 0 \}$ and that $p \in (-e,f) \setminus \{ 0 \}$. Thus, $p \in I_A(-c,d)$ and $p \in I_B(-e,f)$. Since this procedure may be done regardless of the choice of $c,d,e,f$, we can never find disjoint neighborhoods of $A,B$, and therefore $S$ is not Hausdorff.


\end{proof}

\begin{problem}{Question 6}

Define $S^2 = \{ (x,y,z) : x^2 + y^2 + z^2 = 1 \} \subseteq \mathbb{R}^3$, the unit sphere in 3-D.

Define the following charts:

$$\begin{cases} U_1 = \{ (x,y,z) \in S^2 : x > 0 \}, & \phi_1(x,y,z) = (y,z) \\  U_2 = \{ (x,y,z) \in S^2 : x < 0 \}, & \phi_2(x,y,z) = (y,z) \\  U_3 = \{ (x,y,z) \in S^2 : y > 0 \}, & \phi_3(x,y,z) = (x,z) \\  U_4 = \{ (x,y,z) \in S^2 : y <  0 \}, & \phi_4(x,y,z) = (x,z) \\  U_5 = \{ (x,y,z) \in S^2 : z > 0 \}, & \phi_5(x,y,z) = (x,y) \\  U_6 = \{ (x,y,z) \in S^2 : z < 0 \}, & \phi_6(x,y,z) = (x,y) \end{cases}$$

Describe the domain of $\phi_1 \circ \phi_4^{-1}$, and show that $\phi_1 \circ \phi_4^{-1}$ is a $C^\infty$ function on its domain. Do the same for $\phi_6 \circ \phi_1^{-1}$.

\end{problem}

\begin{proof}[Solution]

We recall that $\phi_1 \circ \phi_4^{-1}$ is a map from $\phi_4(U_1 \cap U_4)$. Since we know that $\phi_4$ acting on $U_4$ takes $(x,y,z) \mapsto (x,z)$, if we include the condition on $U_1$, where $x > 0$, this restricts us to the open half disk $\{ (x,z) \subseteq \mathbb{R}^2 : x^2 + z^2 < 1, x > 0 \}$. (Alternatively, relabelling the axes, this is the right half of the open disk).

On this domain, $\phi_4^{-1}$ takes $(x,z) \mapsto (x,-\sqrt{1 - x^2 - z^2}, z)$ since $y < 0$, and $\phi_1$ takes $(x,y,z) \mapsto (y,z)$. Therefore, we have that:

$$\phi_1 \circ \phi_4^{-1}: \phi_4(U_1 \cap U_4) \to \phi_1(U_1 \cap U_4) \text{ via } (x,z) \mapsto (-\sqrt{1 - x^2 - z^2}, z) $$

Clearly, the coordinate function $z$ is the identity, thus polynomial and $C^\infty$. Meanwhile, we notice that via power rule on $-(1 - x^2 - z^2)^{1/2}$, and the product rule on higher order derivatives, since we have that $y < 0, y= - \sqrt{1 - x^2 - z^2}$ via rearrangment of the equation for $S^2$, we see that $1 - x^2 - z^2$ does not vanish on this domain, and therefore is also $C^\infty$.

Doing the same thing for $\phi_6 \circ \phi_1^{-1}$, we see that the domain is the set  $\phi_1(U_1 \cap U_6)$. Of course, since $U_6$ restricts $U_1$ to additionally have $z < 0$, we see that this is exactly $\{ (y,z) \subseteq \mathbb{R} : y^2 + z^2 < 1, z < 0 \}$, a different half open disk. Relabelling axes again, we can visualize this as the lower half of the open disk.

On this domain, we see that $\phi_1^{-1}$ has the action of taking $(y,z) \mapsto (\sqrt{1 - y^2 - z^2}, y,z)$ and $\phi_6$ takes $(x,y,z) \mapsto (x,y)$.

Thus, we have that:

$$\phi_6 \circ \phi_1^{-1}: \phi_1(U_1 \cap U_6) \to \phi_6(U_1 \cap U_6) \text{ via } (y,z) \mapsto (\sqrt{1 - y^2 - z^2}, y)$$

Using the same argument, since on $U_1 \cap U_6$, $x > 0$, we must have that $1 - y^2 - z^2$ cannot vanish by rearranging $x = \sqrt{1 - y^2 - z^2}$, and therefore $\sqrt{1 - y^2 - z^2}$ is $C^\infty$ via the power rule and product rule. And, of course $y$ is polynomial, thus $C^\infty$. 

\end{proof}

\end{document}