\documentclass[10pt]{article}
\setlength{\parskip}{0.25\baselineskip}
\usepackage[margin=1in]{geometry} 
\usepackage{amsmath,amsthm,amssymb, graphicx, multicol, array}
\usepackage[font=small,labelfont=bf]{caption}

\newcommand{\supp}{{\text{supp}}} 
\newcommand{\bv}{{\text{BV}}}
\newcommand{\ac}{{\text{AC}}}

\newenvironment{problem}[2][Problem]{\begin{trivlist}
\item[\hskip \labelsep {\bfseries #1}\hskip \labelsep {\bfseries #2.}]}{\end{trivlist}}

\begin{document}
 
\title{Homework \#11}
\author{Eric Tao\\
Math 235: Homework \#11}
\maketitle
 
\section*{2.1}

\begin{problem}{6.3.6}

Assume that $g: [a,b] \to [c,d]$ and $f: [c,d] \to \mathbb{C}$ are continuous. Prove the following statements:

(a) If $f$ is Lipschitz and $g \in \ac[a,b]$, then $f \circ g \in \ac[a,b]$

(b) If $f \in \ac[c,d], g \in \ac[a,b]$ and $g$ monotone increasing on $[a,b]$, then $f \circ g \in \ac[a,b]$

(c) If $f \in \ac[c,d], g \in \ac[a,b]$, then

$$ f \circ g \in \ac[a,b] \iff f \circ g \in \bv[a,b] $$

\end{problem}
\begin{proof}[Solution]

(a)

Let $\epsilon > 0$ be given. 

Since $f$ is Lipschitz, we may find $K > 0$ such that $| f(x) - f(y) | \leq K | x - y | $.

Since $g$ is absolutely continuous, we may find a $\delta > 0$ such that for collections of nonoverlapping subintervals of $[a,b]$, that

$$ \Sigma_j (b_j - a_j) < \delta \implies \Sigma_j |g(b_j) - g(a_j)| < \frac{\epsilon}{K} $$

Well, take $[a_j,b_j]_j$ as a collection of countable, nonoverlapping subintervals of $[a,b]$ such that $\Sigma_j (b_j - a_j) < \delta$, and consider

$$ \Sigma | f \circ g (b_j) - f \circ g (a_j) | \leq \Sigma K |g(b_j) - g(a_j)  = K \Sigma | g(b_j) - g(a_j)| < K \frac{\epsilon}{K}  = \epsilon $$

Thus, $f \circ g \in \ac[a,b]$.

(b)

Let $\epsilon > 0$ be given.

Because $f$ is absolutely continuous, we may find $ \delta > 0$ such that, for $\{ [c_j,d_j] \}_j$ intervals in $[c,d]$, we have that

$$ \Sigma_j (d_j - c_j) < \delta \implies \Sigma_j |f(d_j) - f(c_j)| < \epsilon $$

Further, since $g$ is absolutely continuous, we may find a $\delta' > 0$ such that for $\{ [a_i, b_i] \}_i$ intervals in $[a,b]$, we have that

$$ \Sigma_i (b_i - a_i) < \delta' \implies \Sigma_i |g(b_i) - g(a_i)| < \delta $$

Now, take $\{ [a_i, b_i] \}_i$ intervals in $[a,b]$ such that $ \Sigma_i (b_i - a_i) < \delta'$. Since $g$ is monotone increasing, we notice that $\{ [g(a_i), g(b_i)] \}_i$ are actually intervals, non-overlapping since, due to the monotone increasing nature of $g$, they may only overlap on their endpoints. Further, from the $\delta'$ condition, we have that $\Sigma_i |g(b_i) - g(a_i)| < \delta$, which implies then that $\Sigma_i |f(g(b_i)) - f(g(a_i)) | < \epsilon$.

(c)

By Lemma 6.1.3, we know already that $h \in \ac[a,b] \implies h \in \bv[a,b]$. So, we need only prove that $f \circ g \in \bv[a,b] \implies f \circ g \in \ac[a,b]$. However, this is easy.

Let $Z \subseteq [a,b]$ be a set of measure 0. By corollary 6.3.2, $g(Z) \subseteq [c,d]$ is a set of measure 0. However, now we use the absolute continuity of $f$ as well, to see that $f(g(Z))$ is also a set of measure 0. Since the choice of $Z$ was arbitrary, we have that $|Z| = 0 \implies |f \circ g(Z)| = |f(g(Z))| = 0$. Then, by Banach-Zaretsky again, we have that $f \circ g \in \ac[a,b]$.


\end{proof}

\begin{problem}{6.3.10}

Suppose that $f: [a,b] \to \mathbb{C}$ is differentiable everywhere on $[a,b]$. Prove the following:

(a) $f \in \ac[a,b]$ if and only if $f \in \bv[a,b]$

(b) $f' = 0$ a.e. if and only if $f$ is constant on $[a,b]$.

\end{problem}
\begin{proof}[Solution]

(a)

We already have that $f \in \ac[a,b] \implies f \in \bv[a,b]$ by Lemma 6.1.3. So, now assume $f \in \bv[a,b]$.

By Corollary 5.4.3, since $f \in \bv[a,b]$, we have that $f' \in L^1[a,b]$. Then, by Corollary 6.3.3, since $f$ differentiable everywhere by hypothesis, we have that $f \in \ac[a,b]$.

(b)

Clearly, if $f$ is constant on $[a,b]$, then $f' = 0$ everywhere, stronger than almost everywhere.

Now, suppose $f' = 0$ almost everywhere. Clearly then, $f' \in L^1[a,b]$, because in particular, $\int_{[a,b]} f' = 0$. Therefore, we have that $f \in \ac[a,b]$ by 6.3.3 again. Further, by definition, since $f' = 0$ almost everywhere, $f$ is singular. Then, by 6.3.4, since $f$ is both singular and absolutely continuous, $f$ must actually be constant.

\end{proof}

\section*{2.2}

\begin{problem}{6.4.10}

Show that $f: [a,b] \to \mathbb{C}$ is Lipschitz if and only if $f \in \ac[a,b]$ and $f' \in L^\infty[a,b]$.

\end{problem}

\begin{proof}[Solution]

Firstly, suppose $f$ is Lipschitz. We have already that Lipschitz implies absolutely continuous by 6.1.3, which implies that $f'$ exists almost everywhere, by 6.1.5. Now, let $x$ be somewhere the dervative exists at. Then, we have that, for any $y \in [a,b], y \not = x$, by the definition of Lipschitz, there exists an $M > 0$ such that:

$$ | f(y) - f(x) | \leq M | x - y| \implies \frac{|f(y)- f(x)|}{|y-x|} \leq M $$

Now, if we view $y = x+h$, and then take the limit as $h \to 0$, this implies that $|f'(x)| \leq M$ as well. Since the existence of $M$ is independent of the point $x$, coming from the Lipschitz condition, we have then that on the $[a,b] \setminus Z, |Z|= 0$ where $f'$ is defined, that $|f' | \leq M \implies f' \in L^\infty[a,b]$.

Now, instead, suppose $f \in \ac[a,b]$ with $f' \in L^\infty[a,b]$. By the fundamental theorem of calculus (6.4.2), we have that $f' \in L^1$, and:

$$ f(x) - f(a) = \int_a^x f'(t) dt  \implies f(x) = f(a) + \int_a^x f'(t)dt$$

Now, consider the difference $| f(y) - f(x) |$ for $x,y \in [a,b]$. We have that:

$$ | f(y) - f(x) |  = | f(a) + \int_a^y f'(t) dt - f(a) - \int_a^x f'(t) dt | = | \int_x^y f'(t) dt |$$

Now, since we have that $f'$ is essentially bounded, suppose that $f' \leq \Vert f' \Vert_\infty$ almost everywhere. Then, we can say that on $[x,y]$, $f' \leq \Vert f' \Vert_\infty$ almost everywhere, so we have that:

$$ |f(y) - f(x)| =  | \int_x^y f'(t) dt | \leq | \int_x^y \Vert f' \Vert_\infty dt | = |x - y| \Vert f' \Vert_\infty $$

Thus, $f$ is Lipschitz, as we just take the Lipschitz constant as the uniform norm of $f'$.

\end{proof}

\begin{problem}{6.4.13}

Suppose that $f \in L^1(\mathbb{R})$ is such that $f'\in L^1(\mathbb{R})$ and $ f \in \ac[a,b]$ for every finite interval $[a,b]$. Show that $\lim_{|x| \to \infty} f(x) = 0 = \int_{-\infty}^\infty f'$. 

\end{problem}

\begin{proof}[Solution]

First, we wish to prove that $f$ is actually uniformly continuous. Let $\epsilon > 0$ be given. Because $f'$ is integrable, we have that there exists $\delta > 0$ such that for all measurable $E \subseteq \mathbb{R}$:

$$ |E| < \delta \implies \int_E |f| < \epsilon $$

Now, let $x,y \in \mathbb{R}$ such that $|x - y| < \delta$. WLOG, suppose $x < y$. Since $f \in \ac[x,y]$, by the fundamental theorem, we have that 

$$|f(y) - f(x)| = \left| \int_x^y f'(t) dt \right| \leq \int_x^y |f'(t)| dt < \epsilon $$

Thus, $f$ is uniformly continuous.

Now, suppose that $\lim_{x \to \infty} \not = 0$, where we tackle the positive infinity first. Then, fix any $\epsilon > 0$. We may find $\{ x_n \}_n \to \infty$ such that $|f(x_i)| > \epsilon$. In order to space these out, since we take $x_n \to \infty$, we may assume that $x_{n+1} - x_n > k$ for all $n$, as if it is not, we may always take a subsequence such that our points are sufficiently spaced out.

Now, from the uniform continuity, we have that there exists $\delta > 0$ such that if $| x - y | < \delta$, then $|f(x) - f(y)| < \epsilon/2$. Enforce that $\overline{\delta} = \min(\delta,1)$.

Consider the intervals of form $I_n = (x_n, x_n + \overline{\delta})$. By continuity, this implies that $|f| \geq \epsilon/2$ on these intervals. Then, we would have that 

$$\int_0^\infty |f| \geq \Sigma \int_{I_n} |f| \geq \Sigma \int_{I_n} \epsilon/2 = \Sigma \delta \epsilon/2 = \infty $$

since there are countably many of these intervals, and $\delta, \epsilon > 0$. 

But, this is a contradiction, thus $\lim_{x \to \infty} f(x) = 0$. The same argument works for $x \to -\infty$. So, we have that $\lim_{|x| \to \infty} f(x) = 0$.
 
Now, consider $f_n = f \chi_{[-n,n]}$. We notice, since $f \in \ac[-n,n]$ for each $n$, then $f_n \in \ac[-n,n]$. Then, we may apply the fundamental theorem to see that:

$$ f_n(n) - f_n(-n) = \int_{[a,b]} f_n'(t) dt = \int_{\mathbb{R}} f_n'(t)dt$$

where we use the construction of $f_n$ to argue that the integral of $f_n'$ is the same on $[-n,n]$ as it is on $\mathbb{R}$ since $f_n'$ is identically 0 outside of $[-n.n]$.

We have that, by construction, $f_n' \to f'$ pointwise a.e., and $|f_n'(x)| \leq f'$ a.e. 

Then, we have that

$$\int_{\mathbb{R}}f' = \lim_{n \to \infty} \int_{\mathbb{R}} f_n'(t) = \lim_{n \to \infty} f(n) - f(-n) = 0 $$

\end{proof}

\section*{2.3}

\begin{problem}{7.3.22}

Let $E$ be a measurable subset of $\mathbb{R}^d$, and fix a $1 \leq p < \infty$

(a) Suppose that $\Sigma f_n$ is absolutely convergent in $L^p(E)$, that is, $f_n \in L^p(E)$ for all $n$ and $\Sigma \Vert f_n \Vert_p < \infty$. Prove the following:

\begin{itemize}
\item the series $f(x) = \Sigma_{n=1}^\infty f_n(x)$ converges for almost every $x \in E$ \\
\item $f \in L^p(E)$\\
\item the series $f = \Sigma f_n$ converges in the $L^p$ norm, that is, $\lim_{N \to \infty} \Vert f - \Sigma_n^N f_n \Vert_p = 0$
\end{itemize}

(b) Use part (a) and theorem 1.2.8 to give another proof that $L^P(E)$ is complete with respect to $\Vert \cdot \Vert_p$.

(c) Show that if $\Sigma f_n$ is an absolutely convergent series in $L^1(E)$, then

$$\int_E \Sigma_{n=1}^\infty f_n = \Sigma_{n=1}^\infty \int_E f_n $$


\end{problem}
\begin{proof}[Solution]

(a)

First, we check the convergence of $\Sigma_{n=1}^\infty |f_n|$. Looking at the partial sums, we have that:

$$\Vert \Sigma_{n=1}^N |f_n| \Vert_p \leq \Sigma_{n=1}^N \Vert f_n \Vert_p \leq \Sigma \Vert f_n \Vert_p < \infty$$

Now, because $\Vert \Sigma_{n=1}^N |f_n| \Vert_p$, varying over $N$, is a monotone increasing set of numbers by the triangle inequality, bounded above by $\Sigma \Vert f_n \Vert_p < \infty$, we may apply the Monotone Convergence Theorem to state that $\Vert \Sigma_{n=1}^\infty |f_n| \Vert_p < \infty$. Then, $ \Sigma_{n=1}^\infty |f_n| < \infty$ almost everywhere by Lemma 4.1.8, since we would have that:

$$ \int_E \left(  \Sigma_{n=1}^\infty |f_n| \right)^p < \infty \implies  \int_E \Sigma_{n=1}^\infty |f_n| < \infty $$

Then, this implies that almost everywhere, $\Sigma_{n=1}^\infty |f_n(x)| < \infty$, and an absolutely convergent series of complex or real numbers is itself convergent. Thus,  $f(x) = \Sigma_{n=1}^\infty f_n(x)$ converges for almost every $x \in E$.

Further, via this process, it should be clear that $f \in L^p(E)$, as $|f| \leq \Sigma_{n=1}^\infty |f_n| < \infty$, so we can take both sides to the $p-th$ power, and, looking at the integrals, this remains true.

Lastly, we have that $f$ must converge in the $L^p$ norm, because we have the following chain of inequalities:

$$ \left|f -  \Sigma_{n=1}^N f_n\right|^p  \leq \left(|f| + \left|\Sigma_{n=1}^N f_n\right|\right)^p \leq \left(|f| + \Sigma_{n=1}^N |f_n|\right)^p \leq $$
$$ \left( \Sigma_{n=1}^\infty |f_n| + \Sigma_{n=1}^\infty |f_n|\right)^p  = (2 \Sigma_{n=1}^\infty |f_n| )^p < \infty $$

Therefore, we have that $|f - \Sigma_{n=1}^N f_n|^p \to 0$ almost everywhere, and that $\left|f -  \Sigma_{n=1}^N f_n\right|^p \leq (2 \Sigma_{n=1}^\infty |f_n| )^p < \infty$, so by the Dominated Convergence Theorem, we have that 

$$ \lim_{N \to \infty} \Vert f - \Sigma_n^N f_n \Vert_p =  \lim_{N \to \infty} \int_E \left|f -  \Sigma_{n=1}^N f_n\right|^p = \int_E 0 = 0$$

(b)

If we believe in part (a), theorem 1.2.8 states that if every absolutely convergence series in a metric space $X$ converges in $X$, then $X$ is complete. Part (a) just said that, for an arbitrary absolutely convergence series, it converges in $L^p(E)$, so we are done.

(c)

Without repeating the argument in (a), we see that if $\Sigma f_n$ is absolutely convergent in $L^1(E)$, then we have that $f = \Sigma f_n$ converges a.e., $f \in L^1(E)$, and that it converges in the $L^1$ norm. Then, by convergence in $L^1$, we have that:

$$ \lim_{N \to \infty} \int_E \Sigma_{n=1}^N f_n = \int_E f = \int_E \Sigma f_n $$

However, we know that because the $f_n$ are absolutely convergent, this means that they are in $L^1$ individually, so by the linearity of the integral, we have that:

$$ \lim_{N \to \infty} \int_E \Sigma_{n=1}^N f_n = \lim_{N \to \infty} \Sigma_{n=1}^N \int_E f_n $$

Lastly, since they are absolutely convergent, we have that:

$$ \Sigma \Vert f_n \Vert_1 = \Sigma_n \int_E |f_n| < \infty $$

which implies that $\int_E f_n$ is an absolutely convergent series, and thus convergent. Therefore, we may replace this as:

$$\Sigma_n \int_E f_n = \lim_{N \to \infty} \Sigma_n^N \int_E f_n =  \lim_{N \to \infty} \int_E \Sigma_{n=1}^N f_n = \int_E f = \int_E \Sigma f_n $$


\end{proof}


\begin{problem}{7.3.23}

Fix a $1 \leq p < \infty$. Given $f_n \in L^p(\mathbb{R}^d)$, prove that $f_n \to f$ in $L^p(\mathbb{R}^d)$ if and only if the following three conditions hold.

(a) $f_n \xrightarrow[]{m} f$

(b) For each $\epsilon > 0$, there exists a $\delta > 0$ such that for every measurable set $E \subseteq \mathbb{R}^d$ with $|E| < \delta$, we have that $\int_E |f_n|^p < \epsilon$ for every $n$.

(c) For each $\epsilon > 0$, there exists a measurable set $E \subseteq \mathbb{R}^d$ such that $|E| < \infty$ and $ \int_{E^c} |f_n|^p < \epsilon$ for every $n$.

\end{problem}
\begin{proof}[Solution]

First, assume (a)-(c) are true.

Let $\epsilon > 0$ be given.


It is fairly clear that if $f_n \to f$ in $L^p(\mathbb{R}^d)$ is true, (a)-(c) is true:

(a)

By Theorem 7.3.4, if they converge in the $L^p$ norm, we automatically have that $f_n \xrightarrow[]{m} f$.

(b)

First, we assume that the $f_n$ are simple functions with compact support. Then, because $f_n \in L^p(\mathbb{R}^d$, we can look at $max(\{ f_n(x): x \in \text{Supp}(f_n )\}) < \infty$. This must be finite because, if not, $\Vert f_n \Vert_p = \infty$. Further, we may talk in particular about the max over all such $f_n$. This must be finite, because if not,  

Ok, you know, I don't know how this works.

\end{proof}

 

\end{document}