\documentclass[10pt]{article}
\setlength{\parskip}{0.25\baselineskip}
\usepackage[margin=1in]{geometry} 
\usepackage{amsmath,amsthm,amssymb, graphicx, multicol, array}
\usepackage[font=small,labelfont=bf]{caption}

\newcommand{\supp}{{\text{supp}}} 
\newcommand{\bv}{{\text{BV}}}
\newcommand{\ac}{{\text{AC}}}

\newenvironment{problem}[2][Problem]{\begin{trivlist}
\item[\hskip \labelsep {\bfseries #1}\hskip \labelsep {\bfseries #2.}]}{\end{trivlist}}

\begin{document}
 
\title{Homework \#10}
\author{Eric Tao\\
Math 235: Homework \#10}
\maketitle
 
\section*{2.1}

\begin{problem}{5.4.6}

Assume that $f: [a,b] to \mathbb{R}$ is continuous, and $D^+ f \geq 0$ on $(a,b)$. Prove that $f$ is monotone increasing on $[a,b]$.

\end{problem}
\begin{proof}[Solution]

First, suppose $D^+f \geq \delta > 0$, and we have $x, y \in (a,b)$ such that $x < y$. Then, since $f$ is a continuous function on a closed and bounded interval $[x,y]$ it attains a maximum on that interval. Suppose $x_0$ be a point on $(x,y)$ such that $f(x_0)$ is a maximum. Then, we have, for $t > x_0$:

$$ \frac{f(t) - f(x_0)}{t - x_0} \leq 0 $$

due to being a maximum. Then, since this is true for any $t > x_0$, this implies that:

$$ D^+f(x_0) =\limsup_{h \to 0^+} \frac{f(x_0 + h) - f(x_0)}{h} \leq 0 $$

But, by hypothesis, $D^+ f \geq \delta$ on $(a,b) \supset [x,y]$, and we have a contradiction. Thus, this means that $x_0 \not \in (x,y)$, and therefore, we may only have a maximum at $x$ or $y$ itself. But, because of the rightwards limit on $D^+$, we may make the same argument for $x$. Therefore, $f(y)$ is a maximum on $[x,y]$, and thus $f(x) \leq f(y)$. Since the choice of $x,y \in (a,b)$ was arbitrary, this means that we are monotone increasing on all of $(a,b)$, and due to continuity, this remains true on $[a,b]$.

Now, suppose we have  $D^+f \geq 0$. Fix some $\delta < 0$, and define the function $g(x) = f(x) + \delta x$. This is a continuous function on $[a,b]$, being the sum of two continuous functions, and further, we have that:

$$ D^+ g(x) = \limsup_{h \to 0^+} \frac{ g(x + h) - g(x)}{h} =   \limsup_{h \to 0^+} \frac{ f(x + h) + \delta(x + h)- f(x) - \delta x}{h} = \limsup_{h \to 0^+} \frac{ f(x + h) - f(x)}{h} + \delta = D^+f + \delta $$

Since we have that $D^+f \geq 0, \delta > 0$, we have that $D^+g > 0$. Then, we have that $g$ is monotone increasing, by above. Then, take $x, y \in [a,b]$ such that $x < y$. We have that:

$$ g(x) \leq g(y) \implies f(x) + \delta x \leq f(y)  + \delta y \implies f(y) - f(x) \geq \delta (x -y) $$

However, the choice of $\delta > 0$ was arbitrary. So, we take a sequence of $\delta \to 0$ and retrive that $f(y) - f(x) \geq 0$. Thus, $f$ is monotone increasing on $[a,b]$.

\end{proof}

\begin{problem}{5.4.8}

Let $\phi$ be the Cantor-Lebesgue function on $[0,1]$. Extend $\phi$ onto all of $\mathbb{R}$ by setting $\phi(x) = \phi(0) = 0$ for $x < 0$ and $\phi(x) = \phi(1) = 1$ for $x > 1$. Let $\{ [a_n,b_n] \}_n$ be an enumeration of all subintervals of $[0,1]$ such that $a_n,b_n$ are rational endpoints in $[0,1]$ with $a_n < b_n$. For each $n \in \mathbb{N}$, set:

$$ f_n(x) = 2^{-n} \phi\left( \frac{x - a_n}{b_n - a_n} \right)$$

Observe that $f_n$ is monotone increasing on $\mathbb{R}$ and has uniform norm $\Vert f_n \Vert_u = 2^{-n}$. Prove the following:

(a) The series $f = \Sigma f_n$ converges uniformly on $[0,1]$.

(b) $f$ is continuous and monotone increasing on $[0,1]$.

(c) $f$ is strictly increasing on $[0,1]$.

(d) $f$ is singular on $[0,1]$, that is, $f'(x)$ exists for almost every $x \in [0,1]$ and $f' = 0$ almost everywhere.

\end{problem}
\begin{proof}[Solution]

(a)

Let $\epsilon > 0$ be given. We notice, by the shape of the $f_n$, that because $\phi$ is bounded between 0 and 1, that $f_n$ is bounded between $0$ and $2^{-n}$. Then, take any point $x \in [0,1]$, and choose $k$ such that $2^{-k} < \epsilon$. If we look at partial sums, then we notice:

$$ f(x) - \Sigma_{i=1}^M f_i(x) = \Sigma_{i=M+1}^\infty f_i(x) \leq \Sigma_{i=M+1}^\infty 2^{-i} = 2^{-M} $$

Thus, if we choose $M = k$, then we have that the difference from $f$ to the partial sum $\Sigma_i^k f_i$ can be no more than $2^{-k} < \epsilon$. Since this is true regardless of the point $x$, this implies that this is uniform convergence.

(b)

We recall that $\phi$ is continuous, therefore, since $f_n$ merely multiplies it by a constant, and shifts the window on where $f_n$ is increasing, $f_n$ is continuous as well. Then, since we've proved in part (a) that the convergence to $f$ is uniform, we must have that $f$ is continuous, since the uniform convergence of continuous functions is continuous. Further, because each $f_n$ is monotone increasing, the sum of monotone increasing, non-negative functions must also be monotone.

(c)

Let $0 \leq x < y \leq 1$. We may find two rational points $p,q$ such that $0 \leq x < p < q < y \leq 1$. Since these are rational numbers, it has some enumeration in the subintervals with rational endpoints $\{ [a_i, b_i ] \}$ and corresponds with a $f_i = 2^{-i} \phi\left( \frac{x - a_i}{b_i - a_i} \right)$. In particular, we notice that $f_i(p) = f_i(a_i) = 0, f_i(q) = f_i(b_i) = 2^{-i}$. Then, if we consider the series $\Sigma f_n(y), \Sigma f_n(x)$, looking term by term, because each of the $f_n$ are monotone, non-negative, and because at least $f_i(y) = f_i(q) > f_i(p) > f_i(x)$, we have that $\Sigma f_n(y) > \Sigma f_n(x)$. Since this can be done with any choice of $x,y$, we have then that $f$ is actually strictly increasing.

(d)

Fixing an $x \in [0,1]$, due to the fact that we are bounded above on each $f_n$ by $\Vert f_n \Vert_u = 2^{-n}$, we are actually bounded above on $f$ by $\Sigma_{n=1}^\infty 2^{-n} = 1$. Further, because of the fact that the $f_n$ are non-negative, we have that the partial sums are monotone increasing. Thus, by the monotone convergence theorem, we have that the series $f = \Sigma f_n$ converges for every $x \in [0,1]$. Then, by lemma 5.4.4, we have that $f$ is differentiable almost everywhere, and:

$$ f'(x) = \Sigma f_n'(x) $$

almost everywhere. However, we know from working with the Cantor-Lebesgue function, that this function has 0 derviative almost everywhere on $[0,1]$, and on the extension to the full real line, it still has 0 derivative almost everywhere. Then, we can see that, for each $f_n$, there is a $Z_n$ such that $|Z_n| = 0$, and that $f_n$ has non-0 derivative. Then, if we look at $[0,1] \setminus \cup_n Z_n$, on this set, by definition, $f_n' = 0$ for all $n$. Then, on that set, we have that:

$$f'(x) = \Sigma f_n'(x) = \Sigma 0 = 0$$

and because $|\cup_n Z_n| = 0$, this is almost everywhere.

\end{proof}

\section*{2.2}

\begin{problem}{5.5.17}

Given a locally integrable function $f$ on $\mathbb{R}^d$, define a non-centered maximal function by:

$$M^*f(x) = \sup \left\{ \frac{1}{|B|} \int_B |f| : B \text{ is any open ball that contains } x \right\} $$

Prove that $Mf \leq M^*f \leq 2^d Mf$.

\end{problem}

\begin{proof}[Solution]

Clearly, since $Mf$ is defined as

$$Mf(x) =  \sup_{ h > 0} \frac{1}{|B_h(x)|} \int_{B_h(x)} |f(t)| dt $$

that is, the supremum over only balls centered on $x$ and we are defining $M^*f$ over every ball containing $x$, which includes balls centered on $x$, this implies, by the properties of the supremum, that $Mf \leq M^*f$. So, we need only prove that $M^*f \leq 2^dMf$

Suppose we have a ball $B$ with center $c$, radius $r$, such that $x \in B$. Let $z \in B$. We claim that $| z - x| < 2r$. We can see this via the triangle inequality:

$$ | z- x| \leq |z - c| + | c - x| \leq r + r = 2r $$

Therefore, $z$ is contained within a ball of radius $2r$ around $x$. Since the choice of $z$ was arbitrary, this implies that all of $B$ is contained within this ball, which we will call $B'$. We also recall, that from 2.3.15, about linear changes of variable, since this is merely a translation composed with a dilation by 2 of $B$, that we have that $|B'| = |L(B)| = |2I \cdot T (B) | = |\det(2I \cdot T)| |B|$, where we use the trick about looking at the ball in a $\mathbb{R}^{d+1}$ space to view a translation as a linear transformation.

Here, we notice that the determinant of a translation is $1$, and the determinant of a dilation by 2 in every coordinate is $2^d$. Thus, we have that $|B'| = 2^d|B|$.

Then, looking at the integrand of the maximal functions, we have that:

$$ \frac{1}{|B|} \int_B |f| \leq \frac{2^d}{|B'|} \int_{B'} |f| dt $$

because the fractions are equal, but $B \subseteq B'$ and $|f|$ is non-negative, so $\int_B |f| \leq \int_{B'} |f|$.

But, then we have that, by the definition of $Mf$, that since $B'$ is a ball centered on $x$:

$$ \frac{2^d}{|B'|} \int_{B'} |f| dt = 2^d \frac{1}{|B'|} \int_{B'} |f| dt \leq 2^d Mf $$

Since we may do this for every ball $B$ that contains $x$, this extends to the supremum. Thus, we have that $M^*f \leq 2^d Mf$
\end{proof}

\begin{problem}{5.5.19}

Let $A$ be any subset of $\mathbb{R}^d$ with $|A|_e > 0$. Define the density of $A$ at a point $x \in \mathbb{R}^d$ to be:

$$ D_A(x) = \lim_{r \to 0} \frac{|A \cap B_r(x)|_e}{|B_r(x)|} $$

whenever this limits exists. Prove the following:

(a) $D_A(x) = 1$ for almost every $x \in A$.

(b) $A$ is measurable if and only if $D_A(x) = 0$ for almost every $x \in A$.

Additionally, exhibit a measurable set $E$ and a point $x$ such that $D_E(x)$ does not exist, and given $0 < \alpha < 1$, exhibit a measurable set $E$ and a point $x$ such that $D_E(x) = \alpha$.

\end{problem}

\begin{proof}[Solution]

(a)

I'm really not quite sure how to prove this in the general case. This is clear for a measurable set $E$, since then we may take $f = \chi_E$, locally integrable, so by applying the Lebesgue Differentiation Theorem, we find that:

$$\lim_{h \to 0} \frac{1}{|B_h(x)|} \int_{B_h(x)} f(t) dt = f(x) $$

However, we notice that $ \int_{B_h(x)} f(t) dt =  \int_{B_h(x)} \chi_E(t) dt = |B_h(x) \cap \chi_E|$, so we get that:

$$  \lim_{h \to 0} \frac{ |B_h(x) \cap \chi_E|}{|B_h(x)|} = \chi_E(x) $$

for almost every $x \in \mathbb{R}^d$. In particular then, this means that restricting to $A$, this is 1 for almost every $x \in A$.

(b)

The forward direction is clear, from another application of the LDT, and noticing that $\chi_A(x) = 0$ for $x \not \in A$. I'm not sure how to attack the reverse direction.

An easy example for $\alpha \in (0,1)$ in $\mathbb{R}^2$. Take a point $x = (0,0)$, and take $E$ to be defined in radial coordinates, as $E = \{ (r,\theta) : 0 \leq \theta < 2\pi\alpha \}$. Clearly, for any ball centered on the origin, we cut out exactly $2 \pi\alpha/2\pi$ of the ball.

I do not see an easy example for when $D_E(x)$ does not exist.


\end{proof}

\section*{2.3}

\begin{problem}{6.1.9}

Prove that $f \in \text{AC}[a,b]$ if and only if, for every $\epsilon > 0$, there exists a $\delta > 0$ such that for every finite collection of nonoverlapping subintervals $\{ [a_j,b_j] \}_j$ of $[a,b]$, we have that:

$$ \Sigma_{j=1}^N (b_j - a_j) < \delta \implies \Sigma_{j=1}^N |f(b_j) - f(a_j)| < \epsilon $$


\end{problem}
\begin{proof}[Solution]

It is clear that if $f \in \ac[a,b]$, then the statement holds, because we recall that we define absolutely continuous as, for every $\epsilon > 0$, there exists $\delta > 0$ such that for either finite or countably infinite non overlapping collections of subintervals,

$$ \Sigma_{j=1}^N (b_j - a_j) < \delta \implies \Sigma_{j=1}^N |f(b_j) - f(a_j)| < \epsilon $$

So, it is already true by definition.

Now, instead, suppose we only know that the $\epsilon-\delta$ criteria holds for finitely many collections of subintervals. Then, we wish that this holds for countably infinite collections of subintervals.

Let $\epsilon > 0$ be given. Choose $\delta > 0$ such that, for every finite collection of intervals, we have that

$$ \Sigma_{j=1}^N (b_j - a_j) < \delta \implies \Sigma_{j=1}^N |f(b_j) - f(a_j)| < \epsilon/2 $$

Let $\{ [x_j,y_j] \}_j$ be a countably infinite collection of nonoverlapping subintervals such that $[a_j,b_j] \subseteq [a,b]$ for all $j$, and such that

$$ \Sigma_{j=1}^\infty (y_j - x_j) < \delta $$

Then, we look at a sequence of finite collection of subintervals, that is, $\{ x_j,y_j \}_j^M$. In particular, we have that:

$$ \Sigma_{k=1}^M (y_j - x_j) \leq \Sigma_{j=1}^\infty (y_j - x_j) < \delta $$

because of the fact $y_j - x_j \geq 0$. Then, we have that, by hypothesis:

$$  \Sigma_{j=1}^M |f(y_j) - f(x_j)| < \epsilon/2 $$

But, this is true for every $N$, since they are all finite. Then, taking the limit as $N \to \infty$, we have that:

$$ \Sigma_{j=1}^\infty = \lim_{M \to \infty} \Sigma_{j=1}^M  |f(y_j) - f(x_j)| < \epsilon/2 < \epsilon $$

Thus, $f \in \ac[a,b]$. 

\end{proof}

\begin{problem}{6.1.10}

(a) Prove that $\ac[a,b]$ is a closed subspace of $\bv[a,b]$ with respect to the norm $\Vert f \Vert_\bv$ defined by 5.2.26. That is, show that if $f_n \in \ac[a,b]$, $f \in \bv[a,b]$, and $\Vert f - f_n \Vert_\bv \to 0$, then $f \in \ac[a,b]$.

(b) Exhibit functions $f_n, f$ such that $f_n \in \ac[a,b]$ and $f_n$ converges uniformly to $f \in \bv[a,b]$, but $f \not \in \ac[a,b]$. Thus the uniform limit of absolutely continuous functions need not be absolutely continuous.

\end{problem}
\begin{proof}[Solution]

(a)

Let $\epsilon  > 0$ be given. First, since $\Vert f_n - f \Vert_{\bv} \to 0$, we may pick $N$ such that $\Vert f_m - f \Vert_{\bv} < \epsilon/2$ for every $m > N$. Choose $n$ such that $n$ is the smallest such $m$ that works. Since $f_n \in \ac[a,b]$, we may choose $\delta$ such that for $\{ [a_j, b_j] \}_{j=1}^M$:

$$ \Sigma_j^M b_j - a_j < \delta \implies \Sigma_j^M | f_n(b_j) - f_n(a_j) | < \epsilon/2 $$

 Then, consider the sum:

$$ \Sigma_j^M | f(b_j) - f(a_j) | = \Sigma_j^M | f(b_j) - f_n(b_j) + f_n(b_j) - f_n(a_j) + f_n(a_j) - f(a_j)| \leq $$
$$\Sigma_j^M | [f(b_j) - f_n(b_j)] - [ f(a_j) - f_n(a_j)]| + \Sigma^M | f_n(b_j) - f_n(a_j)| =\Sigma_j^M | (f- f_n)(b_j) - (f - f_n)(a_j)]| + \Sigma_j^M | f_n(b_j) - f_n(a_j)|  $$

Now, since $\Vert f - f_n \Vert_\bv < \epsilon/2$, we have that, in particular, $V[f - f_n; a,b] < \epsilon/2$. Then, since  $\{ [a_j, b_j] \}_{j=1}^M$ are non-overlapping, we may extend this to a partition on $[a,b]$ by including every $a_j, b_j$ with $a,b$, that is, if we have that $a_1 < b_1 < a_2 < b_2 < ... < a_M < b_M $, then we can take the partition:

$$ \Gamma = \{ a = x_0 < a_1 = x_1 < b_1 = x_2 < ... < b_M = x_{2M}  < b = x_{2M + 1} \}$$

This is a proper partition on $[a,b]$, and we have that:

$$ \Sigma_j^M |(f-f_n)(b_j) - (f-f_n)(a_j) | \leq \Sigma_{i=0}^{2M} | (f - f_n)(x_{i+1}) - (f - f_n)(x_{i})| \leq \Vert f - f_n \Vert_\bv < \epsilon/2$$

because every subinterval $\{ [a_j, b_j] \}_{j=1}^M$ is contained within the parition, and because $\Vert f - f_n \Vert_\bv = V[f - f_n;a,b] + \Vert f - f_n \Vert_u$, then since they are all non-negative, we have that $ \Sigma_{i=0}^{2M} | (f - f_n)(x_{i+1}) - (f - f_n)(x_{i})| \leq  V[f - f_n;a,b] \leq \Vert f - f_n \Vert_\bv < \epsilon /2$.

Further, by the choice of $\delta$, we have that $ \Sigma_j^M | f_n(b_j) - f_n(a_j)|  < \epsilon/2$

Thus, we have that with this choice of $\delta$, that:

$$  \Sigma_j^M | f(b_j) - f(a_j) | \leq \Sigma_j^M | (f- f_n)(b_j) - (f - f_n)(a_j)]| + \Sigma_j^M | f_n(b_j) - f_n(a_j)|  < \epsilon/2 + \epsilon/2 = \epsilon $$

By the last problem, 6.1.9, we have that showing this works for finite subintervals is enough to show that $f \in \ac[a,b]$.

(b)

Consier the functions that iterate to the Cantor-Lebesgue function, $\phi$. That is, suppose $C_1 = [0,1] \setminus (1/3,2/3)$, and $C_n$ defined iteratively by removing middle thirds, and $\phi_1$ being linear on $C_1$ and constantly $2^{-1}$ on $(1/3,2/3)$, and defining $\phi_n$ iteratively.

From the book, we know that $\phi_n \to \phi$ uniformly, since it can only differ at most by $2^{-n}$ regardless of the choice of $x$. Further, we see that $f_n \in \ac[a,b]$ for all $n$. This is because fix an $n$. Then, the measure of the construction of the Cantor set on which $\phi_n$ is linear is exactly $(2/3)^n$. Then, we know that the slope on those segments is exactly $(3/2)^n$, since it must range from $0$ to $1$. Then, let $\epsilon > 0$ be given. Choose $\delta$ such that $\delta < (2/3)^n \epsilon$. Then, consider any set of countable non-overlapping intervals  $\{ [a_j, b_j] \}_{j=1}$ such that $[a_j,b_j] \subseteq [a,b]$ and $\Sigma_j b_j - a_j < \delta$. Then, since $\phi_n$ is linear only on the complement of the iterations of the Cantor set, we have that:

$$ \Sigma f(b_j) - f(a_j) \leq \Sigma (3/2)^n (b_j - a_j) \leq (3/2)^n \delta  < \epsilon $$

Thus, for each $n$, $\phi_n \in \ac[a,b]$. Further, since $\phi$ is monotone, we know that $V[\phi;a,b] = 1$ and thus $\phi \in \bv[a,b]$. However, from example 6.1.2 in the book, $\phi$ is not in $\ac[a,b]$. Thus, the uniform limit of absolutely continuous functions need not be absolutely continuous.



\end{proof}

 

\end{document}