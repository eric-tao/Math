\documentclass[10pt]{article}
\setlength{\parskip}{0.25\baselineskip}
\usepackage[margin=1in]{geometry} 
\usepackage{amsmath,amsthm,amssymb, graphicx, multicol, array}
\usepackage[font=small,labelfont=bf]{caption}

\newcommand{\supp}{{\text{supp}}} 
\newcommand{\bv}{{\text{BV}}}

\newenvironment{problem}[2][Problem]{\begin{trivlist}
\item[\hskip \labelsep {\bfseries #1}\hskip \labelsep {\bfseries #2.}]}{\end{trivlist}}

\begin{document}
 
\title{Homework \#8}
\author{Eric Tao\\
Math 235: Homework \#8}
\maketitle
 
\section*{2.1}

\begin{problem}{5.2.18}

Suppose that $f: [a,b] \to \mathbb{C}$. Show that there exists partitions $\Gamma_k$ of $[a,b]$ such that $\Gamma_{k+1}$ is a refinement of $\Gamma_k$ for each $k$, and $S_{\Gamma_k} \nearrow V[f; a,b]$ as $k \to \infty$.
\end{problem}
\begin{proof}[Solution]

First, we wish to show that for any partition $\Gamma_k$ and refinement $\Gamma_{k+1}$, that $S_{\Gamma_k} \leq S_{\Gamma_{k+1}}$. 

Let $ S_{\Gamma_k} = \{ a = x_0 < ... < x_i = b \}$ and $S_{\Gamma_{k+1}} = \{ a = y_0 < ... < y_j = b \}$ be a refinement, where $i < j$ and for every $0 \leq i' \leq i$, there exists a $j'$ such that $x_{i'} = y_{j'}$.

Look at one pair of $x_{i'}, x_{i' + 1}$. If, in the refinement, we have that $x_{i'} = y_{j'}$ and $x_{i'+1} = y_{j'+1}$, then we have that $|f(x_{i'+1}) - f(x_{i'})| = |f(y_{j'+1}) - f(y_{j'})|$. Else, suppose not. Then, we have that $x_{i'} = y_{j'}$ and $x_{i'+1} = y_{j'+n}$ for some n. Then, by liberal usage of the triangle inequality, we have that:

$$ |f(x_{i'+1}) - f(x_{i'})| = |f(y_{j'+n}) - f(y_{j'})| = \left|f(y_{j'+n}) - f(y_{j'}) + \Sigma_{k =1}^{n-1} (f(y_{j'+k}) - f(y_{j' +k}) \right| = $$
$$ \left|\Sigma_{k =1}^{n} (f(y_{j'+k}) - f(y_{j' + (k-1)}) \right| \leq \Sigma_{k=1}^n \left| f(y_{j'+k}) - f(y_{j' + (k-1)}) \right| $$

Since we may do this for every $0 \leq i' \leq i$, that means that $S_{\Gamma_k} \leq S_{\Gamma_{k+1}}$.

First, assume $V[f;a,b] < \infty$. Now, since $V[f; a,b]$ is the supremum of $S_\Gamma$ over every partition $\Gamma$, we may construct a sequence $\Gamma_k$ of partitions such that $V[f; a,b] - S_{\Gamma_k} < 1/k$.

In particular now, define a new sequence of partitions as such. Let $\Gamma_1' = \Gamma_1$. Then, take $\Gamma_i' = \Gamma_{i-1}' \cup \Gamma_{i}$, where we understand the union operation as meaning to take every point in $\Gamma_{i-1}', \Gamma_i$ and create a partition with all points. We notice that for each $i$, $\Gamma_i'$ is a refinement of both $\Gamma_{i-1}', \Gamma_{i}$. Then, we have that $\Gamma_{i-1}' \leq \Gamma_i'$ from the work we did above, and further, we know that $V[f;a,b] - 1/k \leq S_{\Gamma_i'} \leq V[f;a,b]$ by the choice of the $\Gamma_i$'s. Thus, we have an increasing sequence of refinements that converges to $V[f;a,b]$.

The unbounded case is clear, instead of taking  $V[f; a,b] - S_{\Gamma_k} < 1/k$, we simply take $S_{\Gamma_k} > k$ for each $k \geq 1$, and proceed in the same way.

%Now, assume there does not exist any sequence of refinements such that $S_{\Gamma_k} \nearrow V[f; a,b]$. Since we've proved that refinements are monotone increasing, we can drop the $\nearrow$ and just assume $\to$. First, assume $V[f; a,b] = \infty$. Then, for every sequence of 
 


\end{proof}

\begin{problem}{5.2.21}

Assume that $E \subseteq \mathbb{R}$ is measurable, and suppose that $f: E \to \mathbb{R}$ is Lipschitz on the set $E$, that is, there exists a $K \geq 0$ such that:

$$ | f(x) - f(y) | \leq K|x- y|  \text{ for all } x,y \in E$$

Prove that $|f(A)|_e \leq K|A|_e$, for any $A \subseteq E$.

\end{problem}
\begin{proof}[Solution]

Let $\{ Q_k \}_k$ be a collection of boxes such that $A \subseteq \cup_k Q_k$. Let's look at one specific box, $Q_i$. Since $A \subseteq E$, we can take $d_i = \sup(\{ x-y : x,y \in E \cap Q_i \})$, where we notice $d_i \leq \text{Vol}(Q_i)$ Consider the image of $f(E \cap Q_i)$. Since $f$ is Lipschitz, and $Q_i \cap E$ an intersection of measurable sets, the image is measurable. In particular, we notice that, for $x,y \in E \cap Q_i$, we have:

$$ |f(x) - f(y)| \leq K|x-y| \leq K d_i $$ 

Then, if we fix an $x$, that means $f(E \cap Q_i)$ can be contained within an interval of length $K d_i$. We may repeat this process for each $Q_i$. We notice, since $Q_k$ covers $A$, then so must $E \cap Q_k$. So, we have that 

$$|\cup_k f(E \cap Q_k)|_e \leq \Sigma_k (K d_k) \leq K \Sigma_k (d_k) \leq K \Sigma_k \text{Vol}(Q_k)$$

Since we can do this for any cover by boxes $Q_k$ of $A$, $f(A) \subseteq \cup_k f(E \cap Q_k)$ for every collection of boxes, and via the properties of the infimum, we have that:

$$|f(A)|_e \leq K |A|_e$$

\end{proof}

\begin{problem}{5.2.22}
Fix $a,b > 0$ and define:

$$ f(x) = \begin{cases} |x|^a \sin (|x|^{-b}), & x \not = 0 \\ 0, & x = 0\\ \end{cases} $$

Prove the following:

(a) $f \in \bv[-1,1] \iff a > b$

(b) If $a = b$ then $f \in C^\alpha[-1,1]$ with exponent $\alpha = \frac{b}{b+1}$.

(c) $ C^\alpha[-1,1]$ is not contained in $\bv[-1,1]$ for any $0 < \alpha < 1$.
\end{problem}

\begin{proof}[Solution]

(a)

First, we notice that $f$ is symmetric across $x = 0$, and so we restrict ourselves to looking on $[0,1]$, and we may drop the absolute values. Computing $f'$ on $(0,1]$, we find that 

$$f' = ax^{a-1} \sin(x^{-b}) + x^{a} \cos(x^{-b}) -b x^{-b -1} =  a x^{a-1}\sin(x^{-b}) - b  x^{a- b-1}\cos(x^{-b}) $$

Now, we wish to check if this function is in $L^1[0,1]$. We see that, via the triangle inequality, and the fact that $|\sin(y)|, |\cos(y)| \leq 1$ for all $y$:

$$\int_0^1 |a x^{a-1}\sin(x^{-b}) - b  x^{a- b-1}\cos(x^{-b})| \leq \int_0^1 |a x^{a-1}\sin(x^{-b})| + \int_0^1 | b  x^{a- b-1}\cos(x^{-b})| \leq $$
$$ \int_0^1 |a x^{a-1}| + \int_0^1 | b  x^{a- b-1}| = \int_0^1 a x^{a-1} + \int_0^1 b  x^{a- b-1} = $$

We notice that if $a = b$, then the integral on the right diverges, since the integral becomes $\int_0^1 b x^{-1} = b \ln(x)\bigg\rvert_0^1$ which diverges. So, here, we take the case $a \not = b$:

$$ x^a \bigg\rvert_0^1 + \frac{b}{a-b}x^{a-b} \bigg\rvert_0^1 = 1 + \frac{b}{a-b}x^{a-b} \bigg\rvert_0^1 $$ % = \frac{a}{a-b}$$

Here, we notice that if $a < b$, that the remaining integrand goes to infinity at 0, but if we have that $a > b$, then:

$$ 1 + \frac{b}{a-b}x^{a-b} = 1 + \frac{b}{a-b} = \frac{a}{a-b}  < \infty $$.

So, we have then that if $a> b$, then $\Vert f' \Vert_1 < \infty$, and thus, by 5.2.9, $f \in \bv[0,1]$.

Now, we consider a partition with form $\Gamma_k = \{ 1 > (2/\pi)^{1/b} > ... > (2/k\pi)^{1/b} > 0 \}$, where we take $k \geq 4$. Let's compute $S_{\Gamma_k}$. 

$$ S_{\Gamma_k} = |\sin(1) - (2/\pi)^{a/b}\sin(\pi/2)| + | (2/\pi)^{a/b}\sin(\pi/2) -  (2/2\pi)^{a/b}\sin(2\pi/2)| + ... + | (2/k\pi)^{a/b}\sin(k\pi/2) - 0| \leq$$
$$ \Sigma_{i=1}^{k-1}  | (2/i\pi)^{a/b}\sin(i\pi/2) -  (2/(i+1)\pi)^{a/b}\sin((i+1)\pi/2)|$$

where we've omitted the first and last term. We notice, that $\sin(i\pi/2)$ is 0 whenever $i$ is even. Then, we can rewrite this as:

$$ \Sigma_{i=1}^{k-1}  | (2/i\pi)^{a/b}\sin(i\pi/2) -  (2/(i+1)\pi)^{a/b}\sin((i+1)\pi/2)| = 2 \Sigma_{i=1}^{\lfloor k/2 \rfloor -1} (2/(2i+1)\pi)^{a/b}$$

Because we count each odd $ (2/i\pi)$ twice, once with $i-1$, and once with $i + 1$, and we drop the $\sin$ and absolute values, because $\sin$ takes on $\pm 1$. We also ignore $2i - 1 = 1$, because it's only counted once, due to the $\sin(1)$ term.

Here, we consider the sum $2\Sigma_{i=1}^{\lfloor k/2 \rfloor -1} (2/(2i+1)\pi)^{a/b} = 2(2/\pi)^{a/b} \Sigma_{i=1}^{\lfloor k/2 \rfloor -1} (1/(2i + 1))^{a/b}$. We recognize this as some constant times the sum of odd reciprocals. In particular, we know that as $k \to \infty$, this sum diverges so long as $a/b \leq 1$. Thus, since we have found a partition that diverges, $V[f;a,b]$ must diverge as well, since we can always union this sequence of partitions into any other partition. Therefore, for $f \in bv[a,b]$, $a/b > 1$, and thus, $a > b$.

Therefore, we have a biconditional.

(b)

Suppose $a = b$, then $f = x^b sin(x^{-b})$ on $(0,1]$. Again, we restrict ourselves to looking on $(0,1]$ due to symmetry, as if it is true here, then it is true on all of $[0,1]$

First, suppose $0 < x < y \leq 1$, define $h = y - x$, and then consider the case where $h \leq x^b+1$.

We have, via the Mean Value Theorem, that because $f$ is differentiable on $(0,1]$, that $|f(x) - f(y)| = |f'(t)|h$ for some $x < t < b$. From part (a), we computed the derivative as:

$$ f'(x) = bx^{b-1} \sin(x^{-b}) - b x^{-1} \cos(x^{-b}) = bx^{-1}(  x^{b}\sin(x^{-b}) - \cos(x^{-b})) $$.

We notice, that because $\sin,\cos$ are bounded by $\pm1$, and since $ t \in [0,1]$, we have that $t^b \in (0,1)$, we may take the estimate:

$$ |f'(t)| =   |bt^{-1}|| t^{b}\sin(t^{-b}) - \cos(t^{-b})| \leq |b/t|(| t^b \sin(t^{-b}) | + | \cos(t^{-b})|) \leq 2b/t $$.

Then, we have that $|f(x) - f(y)| = |f'(t)|h \leq 2bh/t$. Now, we have that $x < t < y$, so therefore, since $b+1 > 0$, we have that $x^{b+1} \geq t^{b+1}$, and by our case, this implies that $t^{b+1} > h \implies  t> h^{1/(b+1)}$. Since this is a lower bound for $t$, this is an upper bound for the fraction $2b/t$, so we have that:

$$ |f(x) - f(y)| \leq 2bh/t \leq   2bh/h^{1/(b+1)} = 2b h^{1 - 1/(b+1)}= 2b h^{b /b+1} $$

Since $2b$ is a constant, we have $b/b+1$ as a H\"older exponent in this case.

Now, suppose $h > x^{b+1}$.

If we look at $|f(y) - f(x)|$, we have that:

$$ |f(y) - f(x)| \leq |f(y)| + |f(x)| \leq | y^b \sin (y^{-b})|  +  |x^b \sin (x^{-b})| \leq y^b + x^b$$

Now, from our case, we already have that because $x^{b+1} < h$, since $b > 0 \implies b/b+1 > 0$, we may take both sides to the $b/b+1$-th power, and obtain that $(x^{b+1})^{b/b+1} < h^{b/b+1} \implies x^b < h^\alpha$.

On the other hand, we look at $y^b/h^\alpha$. In particular, since $0 < b/b+1 < 1, 0 < h < 1$, we have that $y^b/h^\alpha \leq y^b/h^b = (y/y-x)^b$. We notice here that because $h > x^{b+1}$, that instead, if we view this as fixing a $y$, $h$ can be no less than some constant multiple of $y$, $Cy$, as otherwise, $x$ cannot get too close to $y$ without making $h \leq x^{b+1}$. Then, we have that:

$$ |f(y) - f(x)| \leq  y^b + x^b \leq h^\alpha + C^b h^\alpha = (1 + C^b) h^\alpha$$

Now, to finish, we just take our H\"older constant to be the max of $(1 + C^b), 2b$ and we are done.

(c)

Fix an $0 < \alpha < 1$. Then, since $\alpha = b/b+1 = 1 - 1/b+1$, we have that $1/(1 + b) = 1 - \alpha \implies b+1 = 1/(1-\alpha) \implies b = \alpha/(1-\alpha)$. By our choice of $\alpha$, $b > 0$. Then, from part (a), (b), we may find a function $f_b$ defined as above with this choice of $b$ such that it belongs to $C^\alpha[-1,1]$ but does not belong to $\bv[-1,1]$.



\end{proof}

\begin{problem}{5.2.23}

(a) Suppose that $\{ f_n \}_{n \in \mathbb{N}}$ is a sequence of complex-valued functions $f_n: [a,b] \to \mathbb{C}$ and that $f_n \to f$ pointwise on $[a,b]$. Prove that:

$$ V[f;a,b] \leq \liminf_{n \to \infty} V[f_n; a,b]$$

(b) Exhibit functions $f_n, f$ such that $f_n \in \bv[a,b]$ for each $n \in \mathbb{N}$ and $f_n \to f$ pointwise, but $f \not \in \bv[a,b]$.
\end{problem}

\begin{proof}[Solution]

(a)

Let $\Gamma$ be a partition on $[a,b]$. Then, by 4.2.18, Fatou's lemma for series, we can say that 

$$S_\Gamma = \Sigma_{j=1}^n | f(x_j) - f(x_{j-1}) | =  \Sigma_{j=1}^n \liminf_{n \to \infty} | f_n(x_j) - f_n(x_{j-1}) | \leq $$

$$ \liminf_{n \to \infty}\Sigma_{j=1}^n | f_n(x_j) - f_n(x_{j-1}) | = \liminf_{n \to \infty} S_\Gamma[f_n; a,b]$$

Since this is true for an arbitrary partition, this is true for every partition. Then, since $V$ is the sup over all $\Gamma$ of $S_\Gamma$, this implies that:

$$ V[f;a,b] \leq \liminf_{n \to \infty} V[f_n; a,b]$$

(b)

Consider the sequence of functions 

$$f_n = \begin{cases} 0, & \text{ if } x \in [a,a+1/n) \\ 1/(x-a), & \text{ if  } x \in [a+1/n,b] \end{cases}$$

It is clear that this function has bounded variation, because for any $f_n$, it is monotone increasing on $[a,a+1/n]$ and monotone decreasing on $[a+1/n,b]$, so it has total variation exactly equal to $n + (n-1/(b-a)) = 2n-1/(b-a)$, thus $f_n \in \bv[a,b]$ for all $n \geq 1$. However, this converges to $1/(x-a)$, which is not bounded, and thus is not in $\bv[a,b]$.

\end{proof}

\begin{problem}{5.2.26}
Prove the following:

(a) $\Vert f \Vert = V[f;a,b]$ defines a seminorm on $\bv[a,b]$ and

$$ \Vert f \Vert_{\bv} = V[f;a,b] + \Vert f \Vert_u : f \in \bv[a,b]$$

is a norm on $\bv[a,b]$.

(b) $\bv[a,b]$ is a Banach space with respect to $\Vert \cdot \Vert_{\bv}$.

(c) $ \Vert f \Vert_{\bv'}  = V[f;a,b] + |f(a)|$ defines an equivalent norm for $\bv[a,b]$. That is, it is a norm, and there exists $C_1,C_2 > 0$ such that:

$$ C_1 \Vert f \Vert_{\bv} \leq \Vert f \Vert_{\bv'} \leq C_2 \Vert f \Vert_{\bv} : f \in \bv[a,b]$$

\end{problem}
\begin{proof}[Solution]

(a)

Clearly, we have that $V[f;a,b] \geq 0$ for any $f \in \bv[a,b]$, because it is the supremum of non-negative numbers. Then, we need only check for the triangle inequality, and factoring scalars.

Let $f,g \in \bv[a,b]$, and fix a partition $\Gamma = \{ a  = x_0 < ... < x_n = b \}$. We notice, by the triangle inequality on the complex numbers, we have that, for each $(x_i, x_{i+1})$:

$$ |f+g(x_{i+1}) - f+g(x_i)| = |f(x_{i+1}) + g(x_{i+1}) - f(x_i) - g(x_0)| \leq |f(x_{i+1}) - f(x_i)| +  |g(x_{i+1}) - g(x_i)| $$

Since this is true for every interval in the partition, this implies then that $S_{\Gamma}^{f+g} \leq S_{\Gamma}^f + S_{\Gamma}^g$, where we use $S_{\Gamma}^f$ to denote the sum for the function $f$. Then, since the variation is simply the supremum over all partitions, and this holds for every partition, we have that:

$$ \Vert f + g \Vert = V[f+g; a,b] \leq V[f; a,b] + V[g;a,b] = \Vert f \Vert + \Vert g \Vert$$.

Now, let $k \in \mathbb{R}$. Consider now $\Vert kf \Vert$. Again, looking at any partition $\Gamma$, we see that:

$$|kf(x_{i+1}) - kf(x_i)| = |k||f(x_{i+1}) - f(x_i)|$$

Since this is true for each interval in our partition, it implies that $S_{\Gamma}^{kf} = |k| S_{\Gamma}^f$. Again, via the properties of the supremum, this implies then that $\Vert kf \Vert = |k| \Vert f \Vert$.

Now, we look at $ \Vert f \Vert_{\bv} = V[f;a,b] + \Vert f \Vert_u : f \in \bv[a,b]$. Because of the fact that we have shown that $V[f;a,b]$ is a seminorm on $\bv[a,b]$ and that we already know that $\Vert f \Vert_u$ is a norm, we know that this is already a seminorm. Then, it suffices to show that $\Vert f \Vert_{\bv} = 0 \implies f = 0$. Since both portions are non-negative, this implies, in particular, $\Vert f \Vert_u = 0$. But, because this is a norm, this implies that $f = 0$, and we are done. Thus, this is a norm.

(b)

Suppose we have a Cauchy sequence of functions $f_n \in \bv[a,b]$, that is, such that $\Vert f_m - f_n \Vert_{\bv} \to 0$ as $m,n \to \infty$. By the definition of $\Vert \cdot \Vert_u$, for this to go to $0$, we must have that $\Vert f_m - f_n \Vert_u \to 0$ as well, that is, it must be Cauchy with respect to the uniform norm.  Then, fix any $x \in [a,b]$, and look at $|f_m(x) - f_n(x)|$. In particular, we have that, for an $\epsilon > 0$ given, there must be $N$ such that for all $m,n > N$, $|f_m(x) - f_n(x)| \leq \Vert f_m - f_n \Vert_u < \epsilon$, by the properties of the supremum. Then, this means that $f_n(x)$ is a sequence of Cauchy real numbers, and thus convergent. Then, define $f(x) = \lim_{n \to \infty} f_n(x)$, that is, the point-wise convergence of the sequence.

Now, we claim that if $f_n$ is Cauchy, then it is convergent to $f$, and that $f \in \bv[a,b]$. Firstly, we see that $f$ must be bounded, because from the fact that $f_n \to f$ in the uniform norm, let $\epsilon > 0$, we can see that $\Vert f - f_n \Vert_u < \epsilon $ for at least some $n$. Then,  by the reverse triangle inequality, we have that $|\Vert f \Vert_u  - \Vert f_n \Vert_u|< \epsilon  \implies   -\epsilon < \Vert f \Vert_u  - \Vert f_n \Vert_u < \epsilon  \implies  -\epsilon < \Vert f_n \Vert_u - \epsilon < \Vert f \Vert_u < \Vert f_n \Vert_u + \epsilon \implies \Vert f \Vert_u < \Vert f_n \Vert_u + \epsilon < \infty$.

Now, we wish that $f$ to be of bounded variation. Because the $f_n$ are Cauchy in $\Vert \cdot \Vert_{\bv}$, we have that they must be Cauchy as well in $\Vert \cdot \Vert$, that is, in their variation, since both the uniform norm and the seminorm must go to 0. But, this then implies that the sequence of $\Vert f_n \Vert$ under the seminorm is bounded. Then, if that's bounded, we have from problem 5.2.23 part (a), that:

$$ V[f;a,b] \leq \liminf_{n \to \infty} V[f_n; a,b] < \infty$$

Thus, $f  \in \bv[a,b]$. Then, it is clear from the triangle inequality and from the seminorm properties that $f_n \to f$ in the seminorm as well, and thus $f_n \to f$ in the full norm.

(c)

First, we look at the case $f(a) \geq 0$. Then, using the Jordan decomposition on $f = g - h$ for $g,h$ monotone increasing, and the seminorm properties to see that $V[f;a,b] \leq V[g;a,b] + V[h;a,b]$, we conclude that $f(a) \leq \Vert f \Vert_u \leq f(a) + V[f;a,b]$, since to maximize $|f|$, we would need $V[h;a,b] = 0$. We can actually see that this argument works for $f(a) < 0$, where instead of taking the positive distance, we take $V[g;a,b] = 0$ to maximize $|f|$. So, we actually have that $|f(a)| \leq \Vert f \Vert_u \leq |f(a)| + V[f;a,b]$.

Then, we take $C_1 = 1, C_2 = 2$. 

From $|f(a)| \leq \Vert f \Vert_u$, we can add $V[f;a,b]$ to both sides to obtain:

$$ \Vert f \Vert_{\bv'} =  V[f;a,b] + |f(a)| \leq V[f;a,b] +   \Vert f \Vert_u  =  \Vert f \Vert_{\bv}$$ so we have that $C_1 \Vert f \Vert_{\bv'} = \Vert f \Vert_{\bv'} \leq \Vert f \Vert_{\bv}$

Further, we have that from the other side, we obtain:

$$ \Vert f \Vert_u \leq |f(a)| + V[f;a,b] \implies  V[f;a,b] + \Vert f \Vert_u \leq |f(a)| + 2V[f;a,b] $$

so we can see that:

$$C_2 \Vert f \Vert_{\bv'} =  2|f(a)| + 2V[f;a,b] \geq  |f(a)| + 2V[f;a,b] \geq  V[f;a,b] + \Vert f \Vert_u = \Vert f \Vert_{\bv} $$.

Thus, these norms are equivalent. If you really want the other inclusion, we can reverse the inclusions by dividing via the constants.
\end{proof}

\section*{2.2}


\begin{problem}{5.3.5}
Assume that $E \subseteq \mathbb{R}^d$ satisfies that $0 < |E|_e < \infty$, and let $\mathcal{B}$ be a Vitali covering of $E$. Given an $\epsilon > 0$, prove that there exist a countable collection of balls $B_k \in \mathcal{B}$ such that

$$\left| E \setminus \cup_k B_k \right|_e = 0 \text{ and } \Sigma_k |B_k| < |E|_e + \epsilon $$

\end{problem}
\begin{proof}[Solution]

We first proceed in the same way as the proof of 5.3.3.

Let $U \supseteq E$ be an open set such that $|U| < |E|_e + \epsilon$. Call $\mathcal{B}'$ the restriction of $\mathcal{B}$ such that for all $B \in \mathcal{B}'$, $B \subseteq U$. Since these were closed sets, and we live in an open set surrounding $U$, we must still have a Vitali cover, as we just shrink ourselves to the case where the ball has radius less than the open ball around each point.

Fix any $B_1 \in \mathcal{B}$ and proceed inductively, picking disjoint balls as follows. Suppose we have picked $n$ balls. Then, if $|E \setminus B_1 \cup ... \cup B_n|_e = 0$ we are done. Otherwise, pick a point in $E \setminus B_1 \cup ... \cup B_n$. Since this has positive measure, we can find an open set around it $U' \setminus B_1 \cup ... \cup B_n$, with set difference of measure less than $\epsilon$. Then, we pick $B_{n+1}$ such that it contains $x$, disjoint from the other $B_1,...B_n$, and, defining

$$s_n = \sup\{ \text{radius}(B) : B \in \mathcal{B}, B \cap B_i, 1 \leq i < n \}$$

such that $\text{radius}(B_{n+1})$. We continue this process, stopping only if $E \setminus B_1 \cup ... \cup B_N|_e = 0$, otherwise obtaining a countable collection of disjoint balls. From the argument of 5.3.3, we have that:

$$ \Sigma_{k=1}^\infty |B_k| = | \cup_k B_k | \leq |U| < |E|_e + \epsilon $$

Now, take a point $x \in E \setminus \cup_k^\infty B_k$. Fix a $m$. By necessity, $x$ must also be in $x \in E \setminus \cup_{k=1}^m B_k$. Then, by the argument in 5.3.3, for some $i > m$, it belongs to some $B_i^*$, where this is a closed ball with the same center as $B_i$ but $\text{radius}(B_i^*)  = 5 \text{radius}(B_i)$. Then, we have that:

$$ \left| E \setminus \cup_k^\infty B_k \right|_e \leq \left| E \setminus \cup_k^m B_k \right|_e \leq \Sigma_{k=m+1}^\infty|B_k^*|  = 5^d \Sigma_{k=m+1}^\infty |B_k|$$

However, since $\Sigma_{k=1}^\infty |B_k| < \infty$, we must have that $ \Sigma_{k=m+1}^\infty |B_k|$ can be picked arbitrarily small, i.e. we can find $m_n$ such that $ \Sigma_{k=m_k+1}^\infty |B_k| < 1/k$. Since the choice of $m$ was arbitrary, we can pick this sequence of $m_n$'s which implies then that:

$$ \left| E \setminus \cup_k^\infty B_k \right|_e \leq  \Sigma_{k=m_k+1}^\infty |B_k| < 1/k $$

for every $k$. Then, we must have that $\left| E \setminus \cup_k B_k \right|_e = 0$.

\end{proof}

 

\end{document}