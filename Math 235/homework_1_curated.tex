\documentclass[10pt]{article}
\setlength{\parskip}{0.25\baselineskip}
\usepackage[margin=1in]{geometry} 
\usepackage{amsmath,amsthm,amssymb, graphicx, multicol, array}
 
\newenvironment{problem}[2][Problem]{\begin{trivlist}
\item[\hskip \labelsep {\bfseries #1}\hskip \labelsep {\bfseries #2.}]}{\end{trivlist}}

\begin{document}
 
\title{Homework \#1}
\author{Eric Tao\\
Math 235: Homework \#1}
\maketitle
 
\section*{2.1}

\begin{problem}{1.1.20}
Assume ${\{x_n\}_{n\in\mathbb{N}}}$ is a Cauchy sequence in a metric space X, and there exists a subsequence ${\{x_{n_k}\}_{k\in\mathbb{N}}}$  that converges to $x \in X$. Prove that $x_n \rightarrow x$.
\end{problem}

\begin{proof}[Proof]
Let $ \epsilon > 0$ be given.

Since ${\{x_n\}}$ is Cauchy, we can choose $N$ such that for all $ m,n > N$, $d(x_m,x_n) < \frac{\epsilon}{2}$.

Similarly, since $x_{n_k} \rightarrow x$, there exists $N_k$ such that for all $n_k > N_k$, $d(x_{n_k},x) < \frac{\epsilon}{2}$. In particular, we may choose $N_k$ such that $N_k > N$

Let $m > N_k$. 

Then, by the triangle inequality, we have $d(x, x_m) \leq d(x,x_{n_k}) + d(x_m,x_{n_k})$, where $n_k > N_k$, as above. Since $n_k > N_k$, we have that $d(x,x_{n_k}) < \frac{\epsilon}{2}$, by the convergence of the subsequence.
Similarly, since the entire sequence is Cauchy, and $m,n_k > N_k > N$, we have that $d(x_m,x_{n_k}) < \frac{\epsilon}{2}$.

Thus, $d(x, x_m) \leq d(x,x_{n_k}) + d(x_m,x_{n_k}) <  \frac{\epsilon}{2} +  \frac{\epsilon}{2} = \epsilon $ 

\end{proof}

\begin{problem}{1.3.8}
Let $g \in C_0(\mathbb{R})$ be any function that does not belong to $C_c(\mathbb{R})$. For each integer $n > 0$, define a compactly supported approximation to $g$ by setting $g_n(x) = g(x)$ for $|x| \leq n$ and $g_n(x) = 0$ for $|x| > n+1$, and let $g_n$ be linear on $[n,n+1]$ and $[-n-1, -n]$. Show that $\{g_n\}_{n \in \mathbb{N}}$ is Cauchy in $C_c(\mathbb{R})$ with respect to the uniform norm, but it does not converge uniformly to any function in $C_c(\mathbb{R})$. Conclude that $C_c(\mathbb{R})$ is not complete with respect to $\Vert \cdot \Vert_u$ and is not a closed subset of $C_0(\mathbb{R})$
\end{problem}

\begin{proof}[Proof]
First, we begin by examining $\Vert g_m - g_n \Vert_u$, where, WLOG, we take $m > n$. We notice then, by the triangle inequality, we have that, for all $x \in \mathbb{R}$:
$$|g_m(x) - g_n(x)| = |g_m(x) + (-g_n(x))| \leq |g_m(x)| + |g_n(x)|$$.

We also notice that because $g_m$ and $g_n$ only differ on the interval $[n,m+1]$ and $[-m-1, -n]$, we need only consider $x \in [n,m+1] \cup [-m-1,-n]$ or, $ n \leq |x| \leq m+1$  as $|g_m - g_n| = 0$ for all $x$ outside of those two intervals.


Then, we proceed as follows. Let $\epsilon > 0$ be given. Since $g \in C_0(\mathbb{R})$, we may choose $k \in \mathbb{N}$ such that for all $x > k$, $|g(x)| < \frac{\epsilon}{6}$.

Now, choose $m > n > k$. From the remarks above, we see the following, where we define $I =  [n,m+1] \cup [-m-1,-n]$ for bookkeeping:
$$ \Vert g_m - g_n \Vert_u = \sup_{x \in I} |g_m(x) - g_n(x)| \leq \sup_{x \in I} (|g_m(x)| + |g_n(x)|)$$.

 However, since $m,n > k$, we have that both $|g_m(x)| < \frac{\epsilon}{6}$ and $|g_n(x)| < \frac{\epsilon}{6}$. Thus:

$$ \Vert g_m - g_n \Vert_u \leq \sup_{x \in I} (|g_m(x)| + |g_n(x)|) \leq \sup_{x \in I} (\frac{\epsilon}{6} + \frac{\epsilon}{6}) = \frac{\epsilon}{3} < \epsilon $$. Thus,  $\{g_n\}_{n \in \mathbb{N}}$ is Cauchy in $C_c(\mathbb{R})$. However, it cannot be convergent to any function in $C_c(\mathbb{R})$ because it is convergent to $g$:

Let $\epsilon > 0$ be given. Since $g \in C_0(\mathbb{R})$, we may choose $k \in \mathbb{N}$ such that for all $x > k$, $|g(x)| < \frac{\epsilon}{6}$. Choose any $n > k$. Then, we have:
$$ \Vert g - g_n \Vert_u = \sup_{x \in \mathbb{R}} |g - g_n| = \sup_{ |x| > n } (|g|  + | g_n|) \leq \frac{\epsilon}{6} + \frac{\epsilon}{6} = \frac{\epsilon}{3} < \epsilon $$

However, by construction, $g$ does not belong to $C_c(\mathbb{R})$, and belongs to the superset $C_0(\mathbb{R})$. Since limits are unique in normed metric spaces, the limit point does not belong in $C_c(\mathbb{R})$, not all Cauchy sequences are complete, and $C_c(\mathbb{R})$ does not contain all of its limit points and is therefore not closed.
\end{proof}

\begin{problem}{1.4.4}
Prove the following statements:

(a) If $f$ is H\"{o}lder continuous on an interval $I$ for some exponent $\alpha > 0$, then $f$ is uniformly continuous on $I$.

(b) If $f$ is H\"{o}lder continuous on an interval $I$ for some exponent $\alpha > 1$, then $f$ is constant on $I$.

(c) The function $f(x) = |x|^{\frac{1}{2}}$ is  H\"{o}lder continuous on $[-1,1]$ for exponents $0<\alpha\leq1/2$ but not for any exponent $\alpha >  1/2$.

(d) The function $g$ defined by $g(x) = -1/\ln x$ for $x > 0$ and $g(0) = 0$ is uniformly continuous on $[0,1/2]$, but it is not H\"{o}lder continous for any exponent $\alpha > 0$.
\end{problem}

\begin{proof}[Proof]

(a)

Let $\epsilon > 0$ be given. Because $f$ is  H\"{o}lder continuous on $I$ for some $\alpha_0 > 0$, fix a $K_0 > 0$ such that $|f(b) - f(a)| \leq K_0|b - a|^{\alpha_0} = K_0 * d(a,b)^{\alpha_0}$.  Choose $\delta> 0$ such that $\delta = (\epsilon/K_0)^{1/\alpha_0}$. Then, for $d(a,b) < \delta$, our  H\"{o}lder continuity equation becomes:

$$ d(f(a),f(b)) = |f(b) - f(a)|  \leq K_0|b - a|^{\alpha_0} = K_0 * d(a,b)^{\alpha_0} < K_0 * [(\epsilon/K_0)^{1/\alpha_0}]^{\alpha_0} = K_0 * \epsilon/K_0 = \epsilon$$.

Thus, $f$ is uniformly continuous on $I$.

(b)

First, fix some point $x_0 \in I$ and assume $f$ is H\"{o}lder continuous with exponent $\alpha_1 > 1$. We may reexpress $\alpha_1 = 1 + \alpha_1'$, with $\alpha_1' > 0$. Then, for any other point $h \in I$, we have that, for some $K_1 > 0$:

$$ |f(x_0) - f(h)| \leq K_1 |x_0 - h|^{\alpha_1' + 1} $$

Rearranging, we have:

$$ \frac{|f(x_0) - f(h)|}{|x_0 - h|} \leq K_1 |x_0 - h|^{\alpha_1'} $$

Taking the limit as both sides of $h \rightarrow x_0$, we find that:

$$\lim_{h \rightarrow x_0} \frac{|f(x_0) - f(h)|}{|x_0 - h|} \leq \lim_{h \rightarrow x_0} K_1 |x_0 - h|^{\alpha_1'} = 0 $$

We recognize the left hand side as the derivative of $f$ at $x_0$, and so we find that at an arbitrary $x_0 \in I$, $f'$ exists and is equal to 0. In particular, since the choice of $x_0$ was arbitrary, we see that $f'$ exists on $I$ and is identically $0$. Then, by the mean value theorem, we have that:

$$ \forall a,b \in I, \frac{f(b) - f(a)}{b-a} = f'(c) = 0  \implies f(b) - f(a) = 0 \implies f(a) = f(b)$$

Thus, $f$ is constant.

(c)

By a similar argument to the proof of 1.4.3 above, we will look at $f(x) = |x|^{\frac{1}{2}}$ on $[0,1]$ and claim by symmetry, that this extends to $[-1,1]$. First, let's consider the following quantity, for $\alpha \in (0,1]$ and $a,b \in [0,1]$, with, wlog, $b > a$:

$$ \frac{|f(b) - f(a)|}{|b-a|^{\alpha}} = \frac{|\sqrt{b} - \sqrt{a}|}{|\sqrt{b} - \sqrt{a}|^\alpha|\sqrt{a} +\sqrt{b}|^\alpha} = \frac{|\sqrt{b} - \sqrt{a}|^\alpha}{|\sqrt{a} +\sqrt{b}|^\alpha} * |\sqrt{b} - \sqrt{a}| ^{1-2\alpha} = \Big| 1 - \frac{2\sqrt{a}}{\sqrt{a} + \sqrt{b}}\Big|^{\alpha}  * |\sqrt{b} - \sqrt{a}| ^{1-2\alpha} $$

We notice that for the quantity, $\frac{2\sqrt{a}}{\sqrt{a} + \sqrt{b}}$, it is non-negative, and since $b > a$, $\sqrt{b}+ \sqrt{a} > 2\sqrt{a}$ we have that $0 \leq \frac{2\sqrt{a}}{\sqrt{a} + \sqrt{b}} \leq \frac{2\sqrt{a}}{2\sqrt{a}} = 1$. Then:

$$ \Big| 1 - \frac{2\sqrt{a}}{\sqrt{a} + \sqrt{b}}\Big|^{\alpha} \leq |1|^\alpha  = 1 $$

Now, we look at the second quantity $K =|\sqrt{b} - \sqrt{a}| ^{1-2\alpha}$. Here, we split into three cases: $1-2\alpha < 0$,$1-2\alpha > 0$,$1-2\alpha = 0$.

For $1-2\alpha<0$, or, $\alpha > 1/2$, we have that this quantity is unbounded, as it looks like $1/|\sqrt{b} - \sqrt{a}| ^{2\alpha - 1}$ where $2\alpha - 1$ is positive. Suppose, for example, we fix $ a = 0$. Then, for the quantity  $1/\sqrt{b} ^{2\alpha - 1}$, as $b \rightarrow 0$, this value is unbounded. Then, there can not exist any constant to make $f$  H\"{o}lder continuous.

Now, suppose $1-2\alpha = 0 \implies \alpha = 1/2$. Then we have $K = 1$, and, in particular, $1$ is a bound.

Finally, suppose $1 - 2\alpha > 0 \implies 0 < \alpha < 1/2$. We notice that $\sqrt{b} - \sqrt{a}$ is bounded above by $1$, so we have $1$ to a positive exponent, which is itself $1$. Then, as in the second case, $1$ is a bound.

From these cases, we conclude that $f$ is  H\"{o}lder continuous on $[0,1]$ and thus $[-1,1]$ with exponent $\alpha$ when $\alpha \in (0,1/2]$ and not for any $\alpha > 1/2$.

(d)

Firstly, we see that because $-\ln x$ is continuous on $(0,0.5]$, we have that $-1/\ln x$ is continuous on $(0,0.5]$. Further, we see that because as $x \rightarrow 0$, $\ln x \rightarrow \infty$, so $-1/\ln x \rightarrow 0$, so $g$ is continuous all of $[0,0.5]$. Then, because $f$ is continous, $[0,0.5]$ a closed, bounded subset of $\mathbb{R}$ and thus compact, and we live in a metric space, we have that $f$ is uniformly continuous by the Heine-Cantor Theorem.

However, consider the quantity for $\alpha \in (0,1]$ and $b \in [0,1]$:

$$ \frac{|f(b) - f(0)|}{|b-0|^{\alpha}} = \frac{|-1/\ln b|}{(b)^{\alpha}} =  \frac{1/b^{\alpha}}{-\ln b}$$

We notice that as $b \rightarrow 0$, $1/b^{\alpha} \rightarrow \infty$ and $-\ln b \rightarrow \infty$, $1/x, -\ln x$ differentiable on $(0,0.5)$, we satisfy the conditions of L'H\^{o}pital's rule.

Then, we have that:

$$ \lim_{b \rightarrow 0} \frac{1/b^{\alpha}}{-\ln b} =  \lim_{b \rightarrow 0} \frac{-\alpha/b^{\alpha + 1}}{-1/b} = \lim_{b \rightarrow 0} \frac{\alpha}{b^{\alpha}} \rightarrow \infty $$

Then, by  L'H\^{o}pital's rule, the original quantity is unbounded. Since the choice of $\alpha$ did not affect the quantity, we conclude that $g$ cannot be  H\"{o}lder continuous for any $\alpha \in (0,1]$ on $[0,0.5]$.

\end{proof}

\section*{2.2}

\begin{problem}{2.1.29}
Prove that a countable union of sets that each have exterior measure zero has exterior measure zero. That is, if $Z_k \subseteq \mathbb{R}^d$ and $|Z_k|_e = 0$ for each $k \in \mathbb{N}$, then $|\cup_k Z_k|_e = 0$
\end{problem}

\begin{proof}[Proof]
Here, we apply Theorem 2.1.13 from Heil, the countable subadditivity of exterior measures, by that, we have:

$$ |\cup_k Z_k|_e \leq \Sigma_{k=1}^{\infty} |Z_k|_e$$

Since $|Z_k|_e = 0$ for all $k$, then:

$$ |\cup_k Z_k|_e \leq \Sigma_{k=1}^{\infty} |Z_k|_e = \Sigma_{k=1}^{\infty} 0 = 0 $$ Since by the definition of exterior measure, the measure of a set is non-negative as the volume of boxes are non-negative, we have that $|\cup_k Z_k|_e \leq 0 \rightarrow |\cup_k Z_k|_e = 0$
\end{proof}

\begin{problem}{2.1.32}
Show that if $f: \mathbb{R} \rightarrow \mathbb{R}$ is continuous, then its graph

$$ \Gamma_f = \{ (x, f(x)) | x \in \mathbb{R}\} \subseteq \mathbb{R}^2$$

has measure zero, i.e., $|\Gamma_f|_e = 0$.
\end{problem}

\begin{proof}[Proof]
Let $\epsilon > 0$ be given. Let $\{ q_n | n \in \mathbb{N} \}$ be an enumeration of the rationals.

Because $f$ is continous, in particular, continuous at each point $q_n$, there exists a $\delta > 0$ such that $ |x - q_n| < \delta \rightarrow |f(x) - f(q_n)| < \epsilon / 2^{(n+1)}$. Set $0 < w_n < \text{max}(\delta,1/2)$ and define the interval $I_n = [q_n - w_n, q_n + w_n]$. Then, construct the box:
$$Q_n = I_n \times [f(q_n) - \epsilon / 2^{n+1}, f(q_n) + \epsilon / 2^{n+1}]$$

Since the rationals are dense, these intervals $I_n$ cover $\mathbb{R}$. Also, because of the continuity of $f$ as above, we see that $\Gamma_f$ in the interval $I_n$ is completely contained within $Q_n$. We see that for any $x_n$ in $I_n$, $|q_n - x_n| \leq w_n$, so that:
$$|f(q_n) - f(x_n)| <  \epsilon / 2^{n+1} \rightarrow f(q_n) - \epsilon / 2^{n+1} <  f(x_n) < f(q_n) + \epsilon / 2^{n+1}$$

Further, due to our condition on $w_n$, we have that:

$$\text{vol}(Q_n) = ( [q_n + w_n] - [q_n - w_n]) * ([ f(q_n) + \epsilon / 2^{n+1}] - [ f(q_n) - \epsilon / 2^{n+1}]) = 2w_n * 2\epsilon/2^{n+1} < \epsilon/2^n $$

Thus, we have that the collection of $\{ Q_n \}$ for $n \in \mathbb{N}$ has volume $\Sigma_{n=1}^{\infty} \epsilon/2^n = \epsilon$ and since we can construct a collection of boxes with the sum of their volumes as arbitrarily small, the outer measure $|\Gamma_f|_e = 0$.

\end{proof}

\begin{problem}{2.1.35}
Find the exterior measures of the following sets.

(a) $L = \{(x,x) | 0 \leq x \leq 1 \}$ the diagonal of the unit square in $\mathbb{R}^2$.

(b) An arbitrary line segment, ray, or line in $\mathbb{R}^2$
\end{problem}

\begin{proof}[Proof]
First, we will prove part (a).

Define a parametrization of the line $L$ via $f: [0,1] \rightarrow L$ where $f(t) = t(1,1) + (0,0)$. We claim this has exterior measure 0. Before we prove that, let's first show that this line L is differentiable and thus continuous.

$$ f'(t_0) = \lim_{h\rightarrow 0} \frac{f(t_0+h) - f(t_0)}{h} =  \lim_{h\rightarrow 0} \frac{(t_0+h)(1,1) + (0,0) - [t_0(1,1) + (0,0)] }{h}  = \lim_{h\rightarrow 0} \frac{h(1,1)}{h} = (1,1) $$.

Since this is true irrespective of the point $t_0$, the derivative exists for all points on the domain. Therefore, $f$ is also continuous on its domain.

Let $\epsilon > 0$ be given. 

Let $\{ q_n \}$ be an enumeration of $\mathbb{Q} \cap [0,1]$. Then, use the same construction and argument as 2.1.32 above.

Part (b) follows in a similar fashion. Here, we note that for a line segment $S$, we have a parametrization that looks like  $f: [0,1] \rightarrow \mathbb{R}^2$ where $f(t) = t(x_1,y_1) + (x_0,y_0)$, where $(x_0,y_0),(x_1,y_1)$ are the endpoints of the line segment. For a ray $R$, we have a parametrization of form $g: [0, \infty) \rightarrow \mathbb{R}^2$ where $g(t) = t(x'_1, y'_1) + (x'_0, y'_0)$ for $(x'_0,y'_0)$ the fixed starting point and $(x'_1,y'_1)$ any other point on the ray. Finally, for a line $L$, we have a parametrization of form $h: \mathbb{R} \rightarrow \mathbb{R}^2$ where $h(t) = t(x''_1, y''_1) + (x''_0, y''_0)$ for any two fixed points $(x''_1, y''_1),(x''_0, y''_0)$.

The same argument applies. The functions $f,g,h$ are differentiable via the same calculation, therefore continuous. Then, we may apply a construction as 2.1.32 above, and therefore the exterior measure is 0.
\end{proof}


\begin{problem}{2.1.39}
Given a set $E \subseteq \mathbb{R}^d$, show that $|E|_e = 0$ if and only if there exist countably many boxes $Q_k$ such that $\Sigma \text{vol}(Q_k) < \infty$ and each point $x \in E$ belongs to infinitely many $Q_k$.
\end{problem}

\begin{proof}[Proof]
First, suppose $|E|_e = 0$. Then, for each $k \in \mathbb{N}$, there must exists some collection of countably many boxes $Q_k = \{ Q_{k,j} \}$ such that $\Sigma_j \text{vol}(Q_{k,j}) < 2^{-k}$ due to the properties of the infimum. Now, consider the collection of collection of boxes $\{  Q_{k} \} = \{ Q_{k,j} \} $ with $k,j \in \mathbb{N}$. This collection is countably infinite, because there exists a bijection from $\mathbb{N} \times \mathbb{N} \rightarrow \mathbb{N}$. Further, $\Sigma_{k} \Sigma_{j} \text{vol}(Q_{k,j}) \leq \Sigma_k 2^{-k} = 1$. Lastly, since each $Q_k$ is a cover of $E$, for any point $e \in E$, and for each $k$, there exists a box $\{ Q_{k,j} \}$ such that $e \in \{Q_{k,j}\}$.

Now, suppose we have a set $E \subseteq \mathbb{R}^d$, and that there exists countably many boxes $Q_k$, $\Sigma \text{vol}(Q_k) < \infty$, and, for each point $x \in E$, $x$ belongs to infinitely many $Q_k$. Let $\epsilon > 0$ be given. Construct a collection of boxes $\{ Q_j \}$ as follows. First, choose any $Q_1$ such that $\text{vol}(Q_1) < \epsilon/2$. Then, choose any $Q_2$ such that $\text{vol}(Q_2) < \epsilon/2^2$ and $Q_2 \cap (E \setminus Q_1)$ is non-empty. More generally, construct $Q_i$ such that $\text{vol}(Q_i) < \epsilon/2^k$ and $Q_i \cap (E \setminus \cup_{n=1}^{i-1}Q_n)$ is non-empty, where this iterative process may or may not terminate.

Let's justify our construction first. By hypothesis, $x \in E$ belongs to infinitely many $Q_k$, i.e. we have a collection of countably infinitely many $Q_k$. However, we also have that $\Sigma \text{vol}(Q_k) < \infty$, so then this implies that for some $k_0 > K$ and $\epsilon > 0$, that $\text{vol}(Q_k) < \epsilon$ for all $k > k_0$ as otherwise, we would have that the infimum is non-0, and then our lower bound on the sum of the volumes would be $\inf_k{Q_k} \times \#\{Q_k\} = \infty$. Further, we can refine this and say that for any point $x$, that we may find a small enough $Q_k$ that contains $x$ for approximately the same reason. Suppose not, then for every $Q_j$ that contains some $x_0$, we have that $\inf_j \{Q_j\} > 0$. But, there are infinitely many of such $Q_j$, so then we have that $\Sigma_j (Q_j) \geq \inf_j\{Q_j\} \times \#\{Q_j\} = \infty$, a contradiction. Finally, since we know that we have only countably many boxes, we are certain that this sequence either terminates, or becomes a countably infinite subset of $Q_k$.

Now, we have that $\Sigma_i \text{vol}(Q_i) \leq \Sigma_i \epsilon/2^i = \epsilon$. Since we can find a sub-cover of arbitrarily small volume of $E$, it follows that $|E|_e = 0$.
\end{proof}

\section*{2.3}

\begin{problem}{2.2.32}
Show that if $A$ and $B$ are any measurable subsets of $\mathbb{R}^d$, then

$$ | A \cup B | + | A \cap B | = |A| + |B| $$.
\end{problem}

\begin{proof}[Proof]
We proceed here by using Carath\'eodory's criterion:

Since $A$ is measurable, we have that $|B| = |A \cap B| + | A \setminus B |$, where we drop the exterior measure since we know that $B$ is measurable, $A \cap B$ is an intersection of measurable sets, thus measurable, and $A \setminus B$ is measurable because  $A \setminus B = A \cap B^c$, and $A, B^c$ are measurable sets.

Similarly, we also have that $|A| = |B \cap A| + |B \setminus A|$ due to the measurability of $B$.

So, we have that $|A| + |B| = 2 * |A \cap B | + |A \setminus B | + |B \setminus A|$. Now, we consider the sum $|A \cap B| + |A \setminus B| + |B \setminus A|$. These are measurable sets, but moreover, they must be disjoint. Then, via countable additivity, we have that:

 $$|A| + |B| = |A \cap B | + [| A \cap B | + |A \setminus B | + |B \setminus A|] = |A \cap B| + | [ (A \cap B) \cup (A \setminus B) \cup (B \setminus A)] | = | A \cup B | + | A \cap B | $$.
\end{proof}

\begin{problem}{2.2.33}
Assume that $\{E_n\}_{n \in \mathbb{N}}$ is a sequence of measurable subsets of $\mathbb{R}^d$ such that $|E_m \cap E_n| = 0$ whenever $m \not = n$. Prove that $|\cup E_n| = \Sigma | E_n | $
\end{problem}

\begin{proof}[Proof]

Argue by induction on $n$ the number of distinct sets in the sequence.

From 2.2.32, we have that $|E_1| + |E_2| = |E_1 \cup E_2 | + | E_1 \cap E_2 | =  |E_1 \cup E_2 |$.

Now, suppose we have that $ |\cup_k E_k | = \Sigma_k|E_k| $ for $ k = 1,...m$. By 2.2.32 again, we have then that: $|E_{m+1}| + \Sigma_{k=1}^m |E_k| = |E_{m+1}| + |\cup_{k=1}^{m} E_k| = |\cup_{k=1}^{m+1} E_k | + | E_{m+1} \cap (\cup_{k=1}^{m} E_k)|$.

But we see that $E_{m+1} \cap (\cup_{k=1}^{m} E_k) = \cup_{k=1}^m (E_{m+1} \cap E_k)$. But, by countable subadditivity and by hypothesis, we have that:

$$|\cup_{k=1}^m (E_{m+1} \cap E_k)| \leq \Sigma_{k=1}^m (E_{m+1} \cap E_k) = 0 \implies |\cup_{k=1}^m (E_{m+1} \cap E_k)| = 0$$.

So, we have that $|E_{m+1}| + |\cup_{k=1}^{m} E_k| = |\cup_{k=1}^{m+1} E_k | + | E_{m+1} \cap (\cup_{k=1}^{m} E_k)| = |\cup_{k=1}^{m+1} E_k | $, completing our inductive hypothesis.

\end{proof}

\begin{problem}{2.2.34}
Let $S_r = \{ x \in \mathbb{R}^d | \Vert x \Vert = r \}$ be the sphere of radius $r$ in $\mathbb{R}^d$ centered at the origin. Prove that $|S_r|  = 0$.
\end{problem}

\begin{proof}[Proof]
First, we remark that $S_r$ is measurable because $S_r$ is closed. This can be relatively easily seen by the fact that $S_r^c$ is the union of the open ball centered on the origin of radius r, and the compliment of the closed ball with the same radius and center.

Now, we consider the following construction. Let $A_r$ be the closed ball of radius $r$, and let $B_r$ be the open ball of radius $r$, both centered at the origin. We claim that $|A_r| = |B_r|$, where we see that these are measurable already due to being closed and open, respectively.

Now, let $\delta >0$, and consider $B_{r(1+\delta)}$. Obviously, we have the following inclusions:

$$ B_r \subseteq |A_r| \subseteq |B_{r(1+\delta)}$$

Then, we have via subadditivity:

$$|B_r| \leq |A_r| \leq |B_{r(1+\delta)}|$$

Even though we do not have a clear link between the volume of a ball and its measure yet, consider the following: Suppose we have a cover of the (open) ball of radius $r$ boxes. Suppose you then stretch each box by $(1 + \delta)$ in each dimension, that is, we take $\Pi_{i=1}^d [a_i,b_i] \mapsto \Pi_{i=1}^d [(1+\delta)a_i,(1+\delta)b_i]$. This is a covering of the ball of radius $r(1 + \delta)$, as, we can identify any point in $B_{r(1+\delta)}$ as coming from a point $B_{r}$ via $f: B_{r} \rightarrow B_{r(1+\delta)}$ via $\vec{x} \mapsto \vec{x}/\Vert x \Vert * (1+ \delta) * \Vert x \Vert$, so then we can find the box that it was covered by down in $B_{r}$, and then it must be covered in the expanded box.

Then, that means, for any $\{ Q_k \}$ that covers $B_r$, we have that $\{ (1+ \delta)Q_k \}$ covers $B_{r(1+\delta)}$. Further, we notice that by the definition of the volume of a box, we have that $\text{vol}([1+\delta]Q_k) = \Pi(b_i(1+\delta) - a_i(1+\delta)) = (1+ \delta)^d \Pi(b_i -a_i) = (1+\delta)^d \text{vol}(Q_k)$ where $d$ denotes the dimension of our space. Since this is true for every covering of $B_r$, and because $|B_{r(1+\delta)}|$ is the infimum of all such coverings of $B_{r(1+\delta)}$, we have that $|B_{r(1+\delta)} \leq (1+ \delta)^d |B_r|$.

Then, we have that

$$|B_r| \leq |A_r| \leq  |B_{r(1+\delta)}| \leq (1+\delta)^d |B_r|$$.

But, $\delta$ can be arbitrarily small, and as $\delta \to 0$, $(1+ \delta)^d \to 0$ as $d \geq 1$.

Thus, we have that $|B_r| = |A_r|$.

Now, then, since we have countable additivity for Lebesgue measurable sets, and we know that $A_r = B_r \cup S_r$, then we have that $|A_r| = |B_r| + |S_r|$, and thus, $|S_r| = 0$.

\end{proof}

\end{document}