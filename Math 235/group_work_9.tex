\documentclass[10pt]{article}
\usepackage{graphicx}
\usepackage{pst-node,pst-tree,pstricks}
\usepackage{amssymb,amsmath}
\usepackage{hyperref}
\usepackage{amsmath,amsthm,amssymb, graphicx, multicol, array}

\newenvironment{problem}[2][Problem]{\begin{trivlist}
\item[\hskip \labelsep {\bfseries #1}\hskip \labelsep {\bfseries #2.}]}{\end{trivlist}}

% environments shortcuts
\newcommand{\beq}{\begin{equation}}
\newcommand{\eeq}{\end{equation}}
\newcommand{\beqa}{\begin{eqnarray}}
\newcommand{\eeqa}{\end{eqnarray}}
\newcommand{\beqas}{\begin{eqnarray*}}
\newcommand{\eeqas}{\end{eqnarray*}}

\newcommand{\bit}{\begin{itemize}}
\newcommand{\eit}{\end{itemize}}
\newcommand{\bits}{\begin{itemize*}}
\newcommand{\eits}{\end{itemize*}}
\newenvironment{enumerate*}{\begin{enumerate}
    \setlength{\topsep}{0ex}
    \setlength{\parskip}{0ex}
    \setlength{\partopsep}{-1ex}
    \setlength{\itemsep}{0pt}
    \setlength{\parsep}{0ex}}
{\end{enumerate}}

\newcommand{\benum}{\begin{enumerate*}}
\newcommand{\eenum}{\end{enumerate*}}
%\newcommand{\benums}{\begin{enumerate*}}
%\newcommand{\eenums}{\end{enumerate*}}
\newcommand{\mybullet}{$\bullet$}

% math mode commands

\newcommand{\fracpartial}[2]{\frac{\partial #1}{\partial  #2}}
\newcommand{\rrr}{{\mathbb R}}
\newcommand{\bigOO}{{\cal O}}
\newcommand{\dataset}{{\cal D}}

\newcommand{\X}{\mathbf{X}}
\newcommand{\calB}{\mathcal{B}}
\newcommand{\calF}{\mathcal{F}}
\newcommand{\calG}{\mathcal{G}}
\newcommand{\calN}{\mathcal{N}}
\newcommand{\calT}{\mathcal{T}}
\newcommand{\calH}{\mathcal{H}}
\newcommand{\vol}{\text{Vol}}

\newcommand{\trace}{\operatorname{trace}}
\newcommand{\diag}{\operatorname{diag}}
\newcommand{\sign}{\operatorname{sgn}}
\newcommand{\onevector}{{\mathbf 1}}
\newcommand{\bbone}[1]{{\mathbf 1}_{[#1]}}

\newcommand {\argmax}[2]{\mbox{\raisebox{-1.7ex}{$\stackrel{\textstyle{\rm #1}}{\scriptstyle #2}$}}\,}  % to replace with the amsmath construction

\newlength{\picwi}
\newcommand{\backskip}{\hspace{-2.5em}} % how much to skip back for an empty item?

% Set up some colors
\definecolor{myblue}{rgb}{0.14,0.11,0.49}
\definecolor{myred}{rgb}{0.74,0.1,0.05}
\definecolor{mygreen}{rgb}{0.,0.52,0.32}
\definecolor{myyellow}{rgb}{0.96,0.92,0.13}
\definecolor{myorange}{rgb}{0.7,0.41,0.1}
\definecolor{mypurple}{rgb}{0.51,0.02,.8}
\definecolor{mygray}{rgb}{0.6,0.6,0.6}

\newcommand{\myblue}[1]{\textcolor{myblue}{#1}}
\newcommand{\myred}[1]{\textcolor{myred}{#1}}
\newcommand{\mygreen}[1]{\textcolor{mygreen}{#1}}
\newcommand{\myorange}[1]{\textcolor{myorange}{#1}}
\newcommand{\myyellow}[1]{\textcolor{myellow}{#1}}
\newcommand{\mypurple}[1]{\textcolor{mypurple}{#1}}
\newcommand{\mygray}[1]{\textcolor{mygray}{#1}}


% Stlyle stuff
% notes are for students , \notes with \mmp{} are for me

\newcommand{\comment}[1]{}
\newcommand{\mmp}[1]{\emph MMP: {#1}}
\newcommand{\mydef}[1]{\myred{\bf {#1}}}
\newcommand{\myemph}[1]{\mygreen{ {#1}}}
\newcommand{\mycode}[1]{\myblue{\tt {#1}}}
\newcommand{\myexe}[1]{{\small \mypurple{Exercise} {#1}}}

\newcommand{\reading}[2]{{\small \myemph{{\bf Reading} CRLS:} {#1}, \myemph{Python APPB4AWD} {#2}}}


\begin{document}
\title{Math 235: Groupwork \#9}
\author{Eric}
\maketitle

\begin{problem}{1}

Fix a $1 \leq p < \infty$. Prove that the set $S$ of simple functions is dense in $L^p(\mathbb{R})$.

\end{problem}

\begin{proof}[Solution]

Let $f \in L^p(R)$. By Corollary 3.2.15, there exists simple functions $\phi_n$ such that $\phi_n(x) \to f(x)$ for each $x \in \mathbb{R}$ and $|\phi_n(x)| \leq |f(x)|$ for all $x \in \mathbb{R}, n \in \mathbb{N}$.

Then, we have that $f - \phi_n \to 0$ pointwise everywhere. Further, by an application of the triangle inequality, we have that, for all $x$:

$$|f(x) - \phi_n(x)| \leq |f(x)| + |\phi_n(x)| \leq 2|f(x)| $$

Then, since these are non-negative real numbers, we use the monotonicity of $x^p$ on $[0,\infty)$ to conclude that:

$$ | f(x) - \phi_n(x)|^p \leq 2^p |f(x)|^p $$

We notice that, because $f \in L^p(R)$, we have that $2^p |f|^p$ is integrable, as we notice $\int_{\mathbb{R}} 2^p |f|^p = 2^p \left(\left(\int_{\mathbb{R}} |f|^p\right)^{1/p}\right)^p = 2^p \Vert f \Vert_p^p < \infty$

Thus, by the Dominated Convergence Theorem, we have that:

$$ \lim_{n\to\infty} \int_\mathbb{R} |f - \phi_n|^p = 0 \implies \lim_{n\to\infty}  \left(\int_\mathbb{R} |f - \phi_n|^p \right)^{1/p}= 0$$

Since if the integral goes to $0$, then taking the $1/p$ power will also go to $0$.

Thus, we have shown that for an arbitrary function $f \in L^p(\mathbb{R})$, that we can find simple functions $\phi_n$ such that $\phi_n \to f$ in the $L^p$-norm. Thus, the set of simple functions is dense in $L^p(\mathbb{R})$.

\end{proof}


\end{document}