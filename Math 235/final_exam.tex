\documentclass[10pt]{article}
\setlength{\parskip}{0.25\baselineskip}
\usepackage[margin=1in]{geometry} 
\usepackage{amsmath,amsthm,amssymb, graphicx, multicol, array}
\usepackage[font=small,labelfont=bf]{caption}

\newcommand{\supp}{{\text{supp}}} 
\newcommand{\bv}{{\text{BV}}}
\newcommand{\ac}{{\text{AC}}}

\newenvironment{problem}[2][Problem]{\begin{trivlist}
\item[\hskip \labelsep {\bfseries #1}\hskip \labelsep {\bfseries #2.}]}{\end{trivlist}}

\begin{document}
 
\title{}
\author{Eric Tao\\
Math 235: Final exam}
\maketitle

\begin{problem}{1}

Let $E = \{ (x, x^3) : x \in \mathbb{R} \} \subset \mathbb{R}^2$. Using only the definition, prove that $E$ is Lebesgue measurable, and find its measure $|E|$.

\end{problem}
\begin{proof}[Solution]

Let $\epsilon > 0$ be given. First, we notice that we may slice up $E = \cup E_n$, where $E_n = \{ (x, f(x)) : x \in [-n,n] \}$. Then, we notice that since this is the graph of a continuous function $f(x) = x^3$, for each $E_n$, since $[-n,n]$ is compact, we may find $\delta$ such that $|x - y| < \delta \implies |f(x) - f(y)| < \epsilon/(4n * 2^k)$. Dividing up $[-n,n]$ into equal length intervals $[a_l,b_l] : b_l - a_l < \delta$, we consider:

$$ C = \cup_k [a_l,b_l] \times [M_n - \frac{\epsilon}{4n 2^k}, M_n + \frac{\epsilon}{4n 2^k}] $$

where $M_n = \sup_{x \in [-n,n]} |f(x)|$. Then, we have that $E_n \subseteq C$, by construction. Further, by monotonicity, we have that:

$$ |E_n|_e \leq \Sigma_l (b_l - a_l)  \frac{\epsilon}{2n 2^k} = 2n\frac{\epsilon}{2n 2^k} =\frac{\epsilon}{2^k} $$

Then, we use subadditivity to have that:

$$ |E|_e \leq \Sigma |E_n|_e  = \Sigma \frac{\epsilon}{2^k}  = \epsilon $$ 

Since $\epsilon$ was arbitrary, this tells us that $|E|_e = 0$, which implies that $|E| = 0$, measurable.

\end{proof}

\begin{problem}{2}

Let $T = \{ (x,y) \in \mathbb{R}^2 : 0 \leq y \leq x < \infty \}$.

Evaluate $\lim_{n \to \infty} \iint_T e^{-(x + \frac{1}{n} + \frac{n+1}{n} y)} (dxdy)$ justifying your answer.

\end{problem}
\begin{proof}[Solution]

First, we look at the integrand  $e^{-(x + \frac{1}{n} + \frac{n+1}{n} y)}$ for a fixed $n$ and claim this is measurable. Well, we see that $ x + \frac{1}{n} + \frac{n+1}{n} y$ is polynomial in $x,y$, therefore continuous, and we know that the exponential function is continuous. Therefore, this function is continuous, being the composition of two continous functions. Thus, by 3.1.6, this function is measurable. 

Further, we notice that we can rewrite the integral as $\int_{[0,\infty] \times [0, \infty]} \chi_{T} e^{-(x + \frac{1}{n} + \frac{n+1}{n} y)}$. Then, on this domain, we may apply Tonelli's theorem, and instead, evaluate the interated integral:

$$ \int_{[0,1]} \left( \int_{[0,1]} \chi_{T} e^{-(x + \frac{1}{n}+ \frac{n+1}{n} y)} dx\right) dy = e^{-\frac{1}{n}}  \int_{[0,1]} \left( \int_{[0,1]} \chi_{T} e^{-(x + \frac{n+1}{n} y)} dx\right) dy $$

Computing $\int_{[0,1]} \chi_{T} e^{-(x + \frac{n+1}{n} y)} dx$ first, we see that:

$$ \int_{[0,1]} \chi_{T} e^{-(x  + \frac{n+1}{n} y)} dy = \int_{[y,\infty]} e^{-(x + \frac{n+1}{n} y)} dx = -  e^{-(x + \frac{n+1}{n} y)} \Big|_y^\infty =  e^{-( \frac{2n+1}{n} y)}$$

where we use the fact that $\lim_{k \to \infty} e^{-k} = 0$. 

Now, evaluating the second integral:

$$\int_{[0,1]} \chi_T e^{-( \frac{2n+1}{n} y)} dy = \int_{[0,\infty]} e^{-( \frac{2n+1}{n} y)} dy = -\frac{n}{2n+1}  e^{-( \frac{2n+1}{n} y)} \Big|_0^\infty = \frac{n}{2n+1}$$

So, we have that by Tonelli's, that:

$$ \iint_T e^{-(x + \frac{1}{n} + \frac{n+1}{n} y)} (dxdy) = \frac{n}{2n+1}$$

Thus, we have that:

$$\lim_{n \to \infty} \iint_T e^{-(x + \frac{1}{n} + \frac{n+1}{n} y)} (dxdy) = \lim_{n \to \infty} \frac{n}{2n+1} = \frac{1}{2} $$



\end{proof}

\begin{problem}{3}

Assume that $f \in \ac[0,1]$ and that there is a function $g$, continuous on $[0,1]$ such that $f' = g$ a.e.  Show that $f$ is differentiable everywhere on $[0,1]$ and that $f' = g$ for all $x \in [0,1]$. Show by example that absolute continuity is necessary.

\end{problem}

\begin{proof}[Solution]

By 5.2.8, since $g$ is continuous, we can define:

$$ G(x) = \int_a^x g(t) dt $$

such that $G$ is Lipschitz and $G' = g$ everywhere. Then, by 6.1.3, we have that $G$ is absolutely continous. Then, consider the difference $f - G$. This is a difference of absolutely continous functions, thus absolutely continuous. Further, looking at its derivative $(f - G)'$, we have that:

$$(f - G)' = f' - G' = f' - g' = 0 $$ 

almost everywhere. Now, $f - G$ is an absolutely continuous function and singular, therefore, $f - G = a$ for some constant $a$.

Then, we have that $f = G + a$ everywhere, and since $G' = g$ everywhere, $f' = (G + a)' = G' = g$ everywhere.

Counterexample for $f \not \in \ac[0,1]$. Choose $f = \phi$, the Cantor-Lebesgue function. We have that $f' = 0$ almost everywhere, so we may pick $g = 0$. However, $f$ is not differentiable everywhere.

\end{proof}

\begin{problem}{4}

Let $f \in L^1(\mathbb{R})$ and define $F(x) = \int_\mathbb{R} f(t) \frac{\sin tx}{t} dt$.

(a) Prove that $F$ is differentiable on $\mathbb{R}$ and find $F'$.

(b) Is $F$ absolutely continuous on each compact subinterval of $\mathbb{R}$? Justify your answer.

\end{problem}

\begin{proof}[Solution]

(a)

Try to compute this directly: $F' = \lim_{h \to 0} F(x+h) - F(x) / h$

Condering just the interior, we have that:

$$\frac{ F(x+h) - F(x)}{h} = \frac{\int_\mathbb{R}  f(t) \frac{\sin t(x+h)}{t} dt - \int_\mathbb{R} f(t) \frac{\sin tx}{t} dt}{h} =  \int_\mathbb{R}  f(t) \frac{\sin t(x+h) - \sin tx}{ht} dt$$

Here, we see that $\lim_{h \to 0}  f(t) \frac{\sin t(x+h) - \sin tx}{t}/{h} = f(t)/t \lim_{h \to 0} \frac{\sin t(x+h) - \sin tx}{th} = f(t) \cos(tx)$ everywhere, by the definition of the derivative of $(\sin)' = \cos$. But, we also have that $|cos| \leq 1$ everywhere. Then, by the mean value theorem, since the difference of sines over $h$ is simply a secant line, it must be equal to $cos'$ somewhere, so bounded above by $1$. Then, we have that $| f(t) \frac{\sin t(x+h) - \sin tx}{ht}| \leq f(t) $, since we pull the $t$ as part of the secant line. But, by hypothesis, $f$ is integrable, so we have that by Dominated Convergence Theorem:

$$ \lim_{h \to 0}  \int_\mathbb{R}  f(t) \frac{\sin t(x+h) - \sin tx}{ht} dt =  \int_\mathbb{R}  \lim_{h \to 0} \left(  f(t) \frac{\sin t(x+h) - \sin tx}{ht} dt \right) =  \int_\mathbb{R} f(t) cos(tx) dt $$

(b)

Since we see that $F$ is differentiable, and that $F'$ is integrable, because $|f(t) cos(tx)| \leq |f(t)|$, then $F$ must be absolutely continuous for any arbitrary interval $[a,b]$ by corollary 6.3.3

\end{proof}

\begin{problem}{5}

Let $f \in L^p(\mathbb{R})$, where $1 \leq p < \infty$. For $\alpha > 0$, define:

$$ E_\alpha(f) = \{ x \in \mathbb{R} : |f(x)| > \alpha \} $$

(a) Show that $E_\alpha$ has finite Lebesgue measure.

(b) Use (a) to show that every $f \in L^p(\mathbb{R})$, $1 \leq p \leq 2$ can be decomposed as $f_1 + f_2$ where $f_1 \in L^1(\mathbb{R})$ and $f_2 \in L^2(\mathbb{R})$.

\end{problem}
\begin{proof}[Solution]

(a) Fix an $\alpha > 0$. Suppose not, and $|E_\alpha| = \infty$. Then consider $\int_{E_\alpha} f \leq \int_{\mathbb{R}} f$. In particular, look at $\int_{E_\alpha} |f|$. Then, we have that:

$$ \int_{E_\alpha} |f| \geq \int_{E_\alpha} \alpha = \infty$$

This implies then that at least one of $\int f^+, \int f^-$ divergse on $E_\alpha$. If both diverge, then the integral does not exist, a contradiction. If at least one diverges, wlog, $\int f^+$, then we would have that $\int f  = \infty$, a contradiction since $f \in L^p$ implies $\int f < \infty$. Thus, the set must have finite Lebesgue measure.

(b)

Ran out of time, sorry!


\end{proof}


 

\end{document}