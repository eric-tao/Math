\documentclass[10pt]{article}
\usepackage{graphicx}
\usepackage{pst-node,pst-tree,pstricks}
\usepackage{amssymb,amsmath}
\usepackage{hyperref}
\usepackage{pst-node}
\usepackage{mathtools}
\usepackage{amsthm}

% environments shortcuts
\newcommand{\beq}{\begin{equation}}
\newcommand{\eeq}{\end{equation}}
\newcommand{\beqa}{\begin{eqnarray}}
\newcommand{\eeqa}{\end{eqnarray}}
\newcommand{\beqas}{\begin{eqnarray*}}
\newcommand{\eeqas}{\end{eqnarray*}}
\newcommand{\codim}{\text{codim}}

\newcommand{\bit}{\begin{itemize}}
\newcommand{\eit}{\end{itemize}}
\newcommand{\bits}{\begin{itemize*}}
\newcommand{\eits}{\end{itemize*}}
\newenvironment{enumerate*}{\begin{enumerate}
    \setlength{\topsep}{0ex}
    \setlength{\parskip}{0ex}
    \setlength{\partopsep}{-1ex}
    \setlength{\itemsep}{0pt}
    \setlength{\parsep}{0ex}}
{\end{enumerate}}

\newcommand{\benum}{\begin{enumerate*}}
\newcommand{\eenum}{\end{enumerate*}}
%\newcommand{\benums}{\begin{enumerate*}}
%\newcommand{\eenums}{\end{enumerate*}}
\newcommand{\mybullet}{$\bullet$}

% math mode commands

\newcommand{\fracpartial}[2]{\frac{\partial #1}{\partial  #2}}
\newcommand{\rrr}{{\mathbb R}}
\newcommand{\bigOO}{{\cal O}}
\newcommand{\dataset}{{\cal D}}

\newcommand{\X}{\mathbf{X}}
\newcommand{\calB}{\mathcal{B}}
\newcommand{\calF}{\mathcal{F}}
\newcommand{\calG}{\mathcal{G}}
\newcommand{\calN}{\mathcal{N}}
\newcommand{\calT}{\mathcal{T}}
\newcommand{\calH}{\mathcal{H}}
\newcommand{\ind}{\text{Ind}}
\newcommand{\res}{\text{Res}}
\newcommand{\vol}{\text{Vol}}

\newcommand{\trace}{\operatorname{trace}}
\newcommand{\diag}{\operatorname{diag}}
\newcommand{\sign}{\operatorname{sgn}}
\newcommand{\onevector}{{\mathbf 1}}
\newcommand{\bbone}[1]{{\mathbf 1}_{[#1]}}

\newcommand {\argmax}[2]{\mbox{\raisebox{-1.7ex}{$\stackrel{\textstyle{\rm #1}}{\scriptstyle #2}$}}\,}  % to replace with the amsmath construction

\newlength{\picwi}
\newcommand{\backskip}{\hspace{-2.5em}} % how much to skip back for an empty item?

% Set up some colors
\definecolor{myblue}{rgb}{0.14,0.11,0.49}
\definecolor{myred}{rgb}{0.74,0.1,0.05}
\definecolor{mygreen}{rgb}{0.,0.52,0.32}
\definecolor{myyellow}{rgb}{0.96,0.92,0.13}
\definecolor{myorange}{rgb}{0.7,0.41,0.1}
\definecolor{mypurple}{rgb}{0.51,0.02,.8}
\definecolor{mygray}{rgb}{0.6,0.6,0.6}

\newcommand{\myblue}[1]{\textcolor{myblue}{#1}}
\newcommand{\myred}[1]{\textcolor{myred}{#1}}
\newcommand{\mygreen}[1]{\textcolor{mygreen}{#1}}
\newcommand{\myorange}[1]{\textcolor{myorange}{#1}}
\newcommand{\myyellow}[1]{\textcolor{myellow}{#1}}
\newcommand{\mypurple}[1]{\textcolor{mypurple}{#1}}
\newcommand{\mygray}[1]{\textcolor{mygray}{#1}}
\newtheorem{definition}{Definition}[section]
\newtheorem{theorem}{Theorem}[section]
\newtheorem{proposition}{Proposition}[section]
\newtheorem{corollary}{Corollary}[section]

% Stlyle stuff
% notes are for students , \notes with \mmp{} are for me

\newcommand{\comment}[1]{}
\newcommand{\mmp}[1]{\emph MMP: {#1}}
\newcommand{\mydef}[1]{\myred{\bf {#1}}}
\newcommand{\myemph}[1]{\mygreen{ {#1}}}
\newcommand{\mycode}[1]{\myblue{\tt {#1}}}
\newcommand{\myexe}[1]{{\small \mypurple{Exercise} {#1}}}

\newcommand{\reading}[2]{{\small \myemph{{\bf Reading} CRLS:} {#1}, \myemph{Python APPB4AWD} {#2}}}


\begin{document}
\begin{Large}
\centerline{Math 285}
\centerline{Lecture Notes}  % lecture number here
\centerline{\bf }       % lecture title here
\centerline{}      %date here
\end{Large}

\section{September 6th}

We will start with a review of calculus, recast in into differential forms.

\begin{definition}
Let $U \in \mathbb{R}^n$ be an open set. For a function $f: U \to \mathbb{R}$, we say $f \in C^k_p$ at a point $p$ if all partial derivatives of $f$ with order $\leq k$ exist and are continuous at $k$.
\end{definition}

Example: $C^0(\mathbb{R})$ describes functions that are at least continuous over the real numbers.
In our setting, we will usually concern ourselves with functions that belong to $C^{\infty}$, where $C^\infty = \cap_{i=0}^\infty C^{i}$

\begin{definition}
Let $U \subseteq \mathbb{R}^n$, and let $f: U \to \mathbb{R}$. We call $f$ analytic at a point $p \in U$ if it agrees with its Taylor’s series at $p$ in some neighborhood of $p$.
\end{definition}

We notice that because taking derivatives is linear, that is, we can differentiate term by term, that if $f$ is analytic, then $f \in C^\infty$. However, the converse need not be true:

Consider:

$$ f = \begin{cases} e^{-1/x} & \text{ when } x > 0  \\ 0 & \text{ else } \end{cases} $$

Without too much work, we see that this function is continuous. Moreover, the derivative of $e^{-1/x}$ is equal to $x^{-2}e^{-1/x} = x^{-2}f$. Taking the limit as $x \to 0$, and using L’H\^opital’s rule where necessary, we can see this goes to 0. Alternatively, we can look at:

$$ f’(0) = \lim_{x \to 0} \frac{f(x) - f(0)}{x}$$

with reasonable usage of L’H\^opital’s.

The upshot is that, inductively, we may show that $f^{(k)}(0) = 0$, and thus, at the point $x = 0$, the Taylor series for $f$ is identically 0. However, in no neighborhood of 0, is $f(x)$ identically 0. Thus, $f$ is not analytic. However, via computation, we see that $f \in C^{\infty}$. So, $C^\infty \nRightarrow$ analytic.

Another way to see this concept, is if we think about Taylor’s Series up to $k$-th order. This is just a Taylor series truncated at the $k$-th term, with a remainder term $R_{k+1}$. Then, in such a view, $f$ is analytic at a point $p \iff \lim_{k \to \infty} R_k = 0$.

\begin{definition}
Let $U \in \mathbb{R}^n$ be a set, and $p \in U$. We call $U$ star-shaped with respect to $p$ if, for all $q \in U$, that the line segment $\overline{pq} \subset U$.
\end{definition}

This motivates the hypotheses for Taylor’s Theorem with a remainder term:

\begin{theorem}
Let $U \subset \mathbb{R}^n$ be a star-shaped open set with respect to a point $p \in U$. Let $f: U \to \mathbb{R}$. If $f \in C^\infty$, then there exist $g_1,...,g_n \in C^\infty$ such that

$$ f(x) = f(p) + \sum_{i=1}^n g_i(x) (x^i - p^i) \text{ with } g_i(p) = \frac{\partial f}{\partial x^i}(p) $$
\end{theorem}

Proof: Let $y \in U$, and let $x \in \overline{py} \subseteq U$. Taking a parametrization of $\overline{py}: x(t) = p + t(y-p) $ where the $i$-th component is given by $x^i(t) = p^i + t(y^i - p^i)$.

Now, consider $f(y) - f(p) = f(x(1)) - f(x(0))$. Using the fundamental theorem:

$$ f(x(1)) - f(x(0)) = \int_0^1 \frac{d}{dt} f(x(t)) dt  = \int_0^1 \sum \frac{\partial f}{\partial x^i} (x(t)) \frac{dx^i}{dt} dt = $$

$$ \sum_i  \int_0^1 \frac{\partial f}{\partial x^i} (p + t(y-p)) (y^i - p^i) dt =   \sum_i  \left(\int_0^1 \frac{\partial f}{\partial x^i} (p + t(y-p))dt \right) (y^i - p^i) $$

Thus, we identify: $$g_i(y) = \int_0^1 \frac{\partial f}{\partial x^i} (p + t(y-p))dt$$

It should be clear that:

$$g_i(p) =  \int_0^1 \frac{\partial f}{\partial x^i} (p + t(p-p))dt = \int_0^1 \frac{\partial f}{\partial x^i}(p)dt =  \frac{\partial f}{\partial x^i}(p) $$

as $ \frac{\partial f}{\partial x^i}(p)$ is not a function of $t$.

Further, by an application of the dominated convergence theorem:

$$ \frac{\partial}{\partial y^j} g_i(y) = \frac{\partial}{\partial y^j} \left( \int_0^1 \frac{\partial f}{\partial x^i} (p + t(p-p))dt\right) = $$
$$ \int_0^1 \frac{\partial}{\partial y^j} \frac{\partial f}{\partial x^i} (p + t(p-p))dt $$

which exists and is continuous because $f \in C^\infty$. Thus, $g_i \in C^\infty$

\section{September 11th}

We want to reformulate the concept of a tangent vector in a coordinate-free way, because we should not need to immerse our manifold in an ambient Euclidean space.

Note that we will use parentheses for a point, and angle brackets for a vectors.

Recall that for a surface traced out in $\mathbb{R}^3$ by some function $M: f(x^1, x^2, x^3) = 0$, we can say that the tangent space is:

$$ T_p(M)  = \{ v_p \in T_p(\mathbb{R}^3) : \nabla f(p) \cdot v_p = 0 \} $$

But this depends on the space we’re immersed in.

To move towards a coordinate independent description, we instead look at the directional derivative.

\begin{definition}
If $v_p \in T_p(U)$ and $f \in C^\infty(U)$, then the directional derivative of $f$ in the direction of $v_p$ at the point $p$ is denoted by $D_{v_p} f$.

Explicitly, we can describe this as a cross section $f(p + tv)$, and thus:

$$ D_{v_p} = \frac{d}{dt} \bigg|_{t=0} = \sum_i \frac{\partial }{\partial x^i}\bigg|_{x = (p + tv_p)} \frac{dx^i}{dt} \bigg|_{t=0} = \sum_i \frac{\partial }{\partial x^i}(p) v^i= \sum_i v_i \frac{\partial}{\partial x^i} \bigg|_{p}  $$ 
\end{definition}

This leads to the concept of the germs of a function:

\begin{definition}
Let $(f, U)$ denote a $C^\infty$ function and its domain: $f: U \to \mathbb{R}$. Fix a $p \in U$.

We say that $(f, U) \sim (g, V)$ if there exists $W \subset U \cap V$ such that $p \in W$, and that restricted to $W$, $f = g$. We denote these equivalence classes as $[(f,U)]$ and call these the germs of functions.

Further, we denote the set of equivalence classes at $p$ as $C_p^\infty$.

\end{definition}

With some work, we can show that due to our equivalence being on some neighborhood of $p$, and the derivative being a local characteristic, that we may apply the directional derivative as:

$$ D_{v_p}: C^\infty_p \to \mathbb{R}$$

Without too much trouble, we can see that there is a algebra of germs over $\mathbb{R}$ with the following operations:

$$ \begin{cases} [(f,U)] + [(g,V)] = [(f+g, U \cap V)] \\ [(f,U)] * [(g,V)] = [(f*g, U \cap V)] \\ \lambda[(f,U)] = [(\lambda f, U )] \end{cases} $$

\begin{proposition}
Let $D_{v_p} : C_p^\infty \to \mathbb{R}$.

(i) $D_{v_p}$ is $\mathbb{R}$-linear.

(ii) $D_{v_p}$ follows a Leibniz rule, that is: $D_{v_p}(fg) = D_{v_p}(f) g(p) + f(p) D_{v_p}(g) $


\end{proposition}

\begin{definition}
Let $D: C^\infty_p \to \mathbb{R}$. If $D$ satisfies (i) and (ii) from Proposition 2.1, then we call it a derivation at $p$, or equivalently, a point-derivation of $C^\infty_p$. 
\end{definition}

\begin{definition}
We denote the set of point-derivations of $C_p^\infty$ as $\mathcal{D}_p(\mathbb{R}^n)$.
\end{definition}

Note that $\mathcal{D}_p(\mathbb{R}^n)$ is closed under addition and scalar multiplication, but not under multiplication. Thus, this forms a vector space, but not an algebra.

So, now we can recast our tangent space.

\begin{theorem}

The map defined by: 

$$ \phi: T_p(\mathbb{R}^n) \to \mathcal{D}_p(\mathbb{R}) $$

such that $v_p \mapsto D_{v_p}$

is a linear isomorphism of vector spaces.

\end{theorem}

\begin{proof}

Suppose $D_{v_p}$ is the 0 operator. By definition:

$$ \varphi(v_p) = D_{v_p} = \sum_{i} v^i \frac{\partial}{\partial x^i} \bigg|_{p} $$

Since this is true for all functions, it is in particular true for the function $f = x^j$. Of course then:

$$ D_{v_p} (f) = \sum_{i} v^i \frac{\partial}{\partial x^i} \bigg|_{p} (f) = v^j_p = 0 $$

Since the choice of $x^j$ was arbitrary, this may be performed for each $x^j$. Thus, $v^i_p = 0$ for all $i$, and thus $v_p = 0$.

Now, let $D \in \mathcal{D}_p(\mathbb{R}^n)$ be an arbitrary point-derivation.

Define $D_{v_p} = \sum v^i \frac{\partial}{\partial x^i} \bigg|_p$ where $v^j =  D(x^j)$.

We claim that for an arbitrary $f \in C^\infty_p$, that $D f = D_{v_p} f$.

Using Theorem 1.1 (Taylor’s theorem with Remainder), we expand $f$ as:

$$ f(x) = f(p) + \sum g_i(x) (x^i - p^i) : g_i(p) =  \frac{\partial f}{\partial x^i}(p) $$

So, computing:

$$ D(f)  = D(f(p) + \sum g_i(x) (x^i - p^i)) = 0 + \sum D(g_i(x) (x^i - p^i)) = $$

$$\sum D(g_i) (p^i - p^i) + g_i(p) D(x^i - p^i) = \sum \frac{\partial f}{\partial x^i}(p) D(x^i) $$

We notice that this is exactly the same form as $\sum v^i \frac{\partial}{\partial x^i} \bigg|_p$ due to our identification of $v^i = D(x^i)$. Thus, $D = D_{v_p}$ on all $f$.
\end{proof}

Note: we will notate in the future as $e_{i,p} = \frac{\partial}{\partial x^i} \bigg|_p$ from now on. 

Because we have a bijection, we can establish the following definition:

\begin{definition}
$T_p(U)$ is the set of point derivations of $C^\infty_p$.
\end{definition}

\begin{definition}
Let $X: U \to \coprod_{p \in U} T_p (U)$. If $X_p \in T_p(U)$, we call such a function a vector field, where we use $\coprod$ to remind ourselves that the tangent spaces are disjoint.
\end{definition}

Unpacking the definition a little bit, if $X_p \in T_p(U)$ then:

$$ X_p = \sum a^i(p) \frac{\partial}{\partial x^i} \bigg|_p$$

Denoting $a^i: U \to \mathbb{R}$ as $p \mapsto a^i(p)$, then we have that:

$$ X = \sum a^i \frac{\partial}{\partial x^i}$$

\begin{definition}
Let $X$ be a vector field as above. We say that $X \in C^\infty$ if each $a^i: U \to \mathbb{R}$ is $C^\infty$.
\end{definition}

Notation: The set of $C^\infty$ vector spaces on $U$ is an $\mathbb{R}$-vector space. We denote this by $\mathcal{X}(U)$.



\end{document}