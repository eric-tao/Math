\documentclass[10pt]{article}
\usepackage{graphicx}
\usepackage{pst-node,pst-tree,pstricks}
\usepackage{amssymb,amsmath}
\usepackage{hyperref}
\usepackage{pst-node}
\usepackage{mathtools}

% environments shortcuts
\newcommand{\beq}{\begin{equation}}
\newcommand{\eeq}{\end{equation}}
\newcommand{\beqa}{\begin{eqnarray}}
\newcommand{\eeqa}{\end{eqnarray}}
\newcommand{\beqas}{\begin{eqnarray*}}
\newcommand{\eeqas}{\end{eqnarray*}}
\newcommand{\codim}{\text{codim}}

\newcommand{\bit}{\begin{itemize}}
\newcommand{\eit}{\end{itemize}}
\newcommand{\bits}{\begin{itemize*}}
\newcommand{\eits}{\end{itemize*}}
\newenvironment{enumerate*}{\begin{enumerate}
    \setlength{\topsep}{0ex}
    \setlength{\parskip}{0ex}
    \setlength{\partopsep}{-1ex}
    \setlength{\itemsep}{0pt}
    \setlength{\parsep}{0ex}}
{\end{enumerate}}

\newcommand{\benum}{\begin{enumerate*}}
\newcommand{\eenum}{\end{enumerate*}}
%\newcommand{\benums}{\begin{enumerate*}}
%\newcommand{\eenums}{\end{enumerate*}}
\newcommand{\mybullet}{$\bullet$}

% math mode commands

\newcommand{\fracpartial}[2]{\frac{\partial #1}{\partial  #2}}
\newcommand{\rrr}{{\mathbb R}}
\newcommand{\bigOO}{{\cal O}}
\newcommand{\dataset}{{\cal D}}

\newcommand{\X}{\mathbf{X}}
\newcommand{\calB}{\mathcal{B}}
\newcommand{\calF}{\mathcal{F}}
\newcommand{\calG}{\mathcal{G}}
\newcommand{\calN}{\mathcal{N}}
\newcommand{\calT}{\mathcal{T}}
\newcommand{\calH}{\mathcal{H}}
\newcommand{\ind}{\text{Ind}}
\newcommand{\res}{\text{Res}}
\newcommand{\vol}{\text{Vol}}

\newcommand{\trace}{\operatorname{trace}}
\newcommand{\diag}{\operatorname{diag}}
\newcommand{\sign}{\operatorname{sgn}}
\newcommand{\onevector}{{\mathbf 1}}
\newcommand{\bbone}[1]{{\mathbf 1}_{[#1]}}

\newcommand {\argmax}[2]{\mbox{\raisebox{-1.7ex}{$\stackrel{\textstyle{\rm #1}}{\scriptstyle #2}$}}\,}  % to replace with the amsmath construction

\newlength{\picwi}
\newcommand{\backskip}{\hspace{-2.5em}} % how much to skip back for an empty item?

% Set up some colors
\definecolor{myblue}{rgb}{0.14,0.11,0.49}
\definecolor{myred}{rgb}{0.74,0.1,0.05}
\definecolor{mygreen}{rgb}{0.,0.52,0.32}
\definecolor{myyellow}{rgb}{0.96,0.92,0.13}
\definecolor{myorange}{rgb}{0.7,0.41,0.1}
\definecolor{mypurple}{rgb}{0.51,0.02,.8}
\definecolor{mygray}{rgb}{0.6,0.6,0.6}

\newcommand{\myblue}[1]{\textcolor{myblue}{#1}}
\newcommand{\myred}[1]{\textcolor{myred}{#1}}
\newcommand{\mygreen}[1]{\textcolor{mygreen}{#1}}
\newcommand{\myorange}[1]{\textcolor{myorange}{#1}}
\newcommand{\myyellow}[1]{\textcolor{myellow}{#1}}
\newcommand{\mypurple}[1]{\textcolor{mypurple}{#1}}
\newcommand{\mygray}[1]{\textcolor{mygray}{#1}}


% Stlyle stuff
% notes are for students , \notes with \mmp{} are for me

\newcommand{\comment}[1]{}
\newcommand{\mmp}[1]{\emph MMP: {#1}}
\newcommand{\mydef}[1]{\myred{\bf {#1}}}
\newcommand{\myemph}[1]{\mygreen{ {#1}}}
\newcommand{\mycode}[1]{\myblue{\tt {#1}}}
\newcommand{\myexe}[1]{{\small \mypurple{Exercise} {#1}}}

\newcommand{\reading}[2]{{\small \myemph{{\bf Reading} CRLS:} {#1}, \myemph{Python APPB4AWD} {#2}}}


\begin{document}
\begin{Large}
\centerline{Math 233}
\centerline{Lecture Notes}  % lecture number here
\centerline{\bf }       % lecture title here
\centerline{}      %date here
\end{Large}


\vspace{2em}
\section*{Jan 18th}

First, settle some notation:

Disks:

Let $ a \in \mathbb{C}$, and $r > 0$. Denote the open disk of radius $r$, centered at $a$ as:

$$ D(a,r) = \{ z \in \mathbb{C} : | z - a | < r \}$$

Similarly, denote the closed disk as:

$$ \overline{D}(a,r) =  \{ z \in \mathbb{C} : | z - a | \leq r \}$$

Connected sets and components:

Let $X$ be a topological space. Call $E \subseteq X$ disconnected when there exist non-empty subsets $A,B \subset E$ such that $A\cup B = E$ and $\overline{A} \cap B = A \cap \overline{B} = \emptyset$. We say that $A,B$ is said to separate $E$.

We call $E \subseteq X$ if it does not admit a separation into subsets.

Now, suppose $E \subseteq X, x_0 \in E$. Then, if we have $A \subset E$ such that $x_0 \in A$, and $A$ connected, then $\cup A$ is connected. Moreover, since we have the (potentially uncountable) union over all connected sets, this must be the largest such connected sets that includes $x_0$. Call this maximal connected set the component of $E$ that contains $x_0$. Call the collection of such connected subsets of $E$ over all $x_0$ the connected components of $E$. It should be clear that the set $E$ must be the disjoint union of the connected components.

Now, let $E \subseteq \mathbb{C}$ be an open set. Let $a \in E$. Because $E$ is open, there exists $r > 0$ such that $D(a,r) \subseteq E$. Then, $D(a,r)$ is in the connected component containing $a$. Thus, the components of $E$ are open.

Note: we will use $\Omega$ to denote open sets in $\mathbb{C}$.

Call an open, connected subset of $\mathbb{C}$ a region.

Derivatives:

Suppose that $\Omega \subseteq \mathbb{C}$ is an open set, and we have $f: \Omega \to \mathbb{C}$. Let $z \in \Omega$. Then, if it exists, we define the derivative of $f$ at $z$ as:

$$f’(z_0) = \lim_{z \to z_0} \frac{f(z) - f(z_0)}{z - z_0}$$

That is, for the limit to exists, we must have that:

$$  \lim_{z \to z_0}  \left| \frac{f(z) - f(z_0)}{z - z_0} - f’(z_0) \right|  = 0 $$

We note that in the $\mathbb{R}^2$ sense of derivatives, complex differentiable implies real differentiable, but not the converse, because the extra structure from the Cauchy-Riemann equations.

Theorem:

If $f’(z_0)$ exists, then $f$ is continuous at $z_0$.

Let $\Omega \subset \mathbb{C}$ be an open set, $f: \Omega \to \mathbb{C}$. If $f$ is differentiable on all of $\Omega$, then we call $f$ holomorphic (or analytic).

We denote the set of all holomorphic functions on $\Omega$ by $\mathcal{H}(\Omega)$.

Let $f,g \in \mathcal{H}(\Omega)$. Then $f+g, fg \in \mathcal{H}(\Omega)$ and $f/g \in \mathcal{H}(\Omega)$ if $0 \not \in g(\Omega)$. The normal rules hold: product rule, quotient rule, etc.

Chain rule:

Suppose that $g \in \mathcal{H}(\Omega)$, $g(\Omega) \subset \Omega_1$, and $f \in \mathcal{H}(\Omega_1)$. Then we claim that $h = f \circ g \in \mathcal{H}(\Omega)$ and $h’ = f’(g(z_0)) * g’(z_0)$

Proof:

Let $z_0 \in \Omega, w_0 = g(z_0) \in \Omega_1$. Since $f’(w_0)$ exists, because $f$ is holomorphic, define $\phi: \Omega_1 \to \mathbb{C}$ as:

$$ \phi(w) = \begin{cases}
\frac{f(w) - f(w_0)}{w - w_0} & \text{ if } w \in \Omega \setminus \{ w_0 \} \\ f’(w_0) & \text{ if } w = w_0 \\ \end{cases}$$

We see that by the definition of the derivative, that $\phi$ is continuous on $\Omega_1$, and we have that $f(w) - f(w_0) = \phi(w) (w - w_0)$. However, we have that $w_0 = g(z_0), w = g(z)$, so we have that:

$$ \frac{f(g(z)) - f(g(z_0))}{z - z_0} = \phi(g(z)) \frac{g(z) - g(z_0)}{z - z_0} $$

If we take the limit of both sides as $z \to z_0$, we see that the right hand side is exactly $\phi(g(z_0)) g’(z_0)$. Further, the left hand side is exactly the definition of the derivative of $h$ at $z_0$. Hence, $h’(z_0)$ exists, and equals $f’(g(z_0)) g’(z_0)$.

Power rule:

By direct computation, we have that $\frac{d}{dz}(z^n) = n z^{n-1}$. This holds for all $n \in \mathbb{Z}$.

Definition:

Let $f: \mathbb{C} \to \mathbb{C}$. If $f$ is holomorphic, then we call $f$ entire. Example: $1, z, z^2$ are entire.

Functions representable by Power Series:

Consider a power series $\Sigma_{n=0}^\infty c_n (z-a_n)^n$.

By the root test, we can say this series has radius of convergence $R$, where $\frac{1}{R} = \limsup_{n \to \infty} | c_n|^{1/n}$. In particular, the series converges absolutely for all $z \in D(a,R)$. Further, the series diverges for all $z \not \in \overline{D}(a,R)$. Now, for $0 \leq r < R$, the power series converges uniformly via the Weierstrauss M-test.

Now, let $f: \Omega \to \mathbb{C}$. We say that $f$ is representable by power series on $\Omega$ provided that for any disk $D(a,r) \subseteq \Omega$, $f$ may be represented by a power series $f(z) = \Sigma_{n=0}^\infty c_n (z-a)^n, z \in D(a,r)$.

Theorem:

Suppose that $f$ is representable by a power series on $\Omega$. Then $f$ is holomorphic on $\Omega$, that is, $f \in \mathcal{H}(\Omega)$. Moreover, if $D(a,r) \subset \Omega$, and $f(z) = \Sigma_{n=0}^\infty c_n (z-a)^n$ on this disk, then for any $z \in D(a,r)$, $f’(z) = \Sigma_{n=1}^\infty nc_n(z-a)^n$. We notice that this is also a power series; as such, $f’$ is also representable by a power series (and, thus, holomorphic).

Proof omitted, long computation, refer to lecture notes or Rudin.

If $f$ is representable by power series on $\Omega$, then $f’$ is also representable, and thus holomorphic. Moreover, $f^{(n)}$ is holomorphic. By term-by-term differentiation, we have that:

$$ f^{(k)} = \Sigma_{n=k}^\infty n(n-1)...(n-k+1) c_n (z-a)^{n-k}$$

Further, we have that at $z = a$, that $f^{(k)}(a) = k! c_k \implies c_k = f^{(k)}(a)/k!$. So we can find every coefficient by successively taking derivatives, and, in particular, equivalent to a Taylor series.

\section*{Jan 23rd}

Suppose that $\Omega \subseteq \mathbb{C}$ is open, and $f$ is a complex-valued function on $\Omega$. We say that $f$ is representable by a power series in $\Omega$ if, for every disk $D(a,r) \subseteq \Omega$, $f$ has a power series representation in $\Omega$ of form:

$$ f(z) = \sum_{n=1}^\infty c_n (z - a)^n $$

Note that, of course, the power series representation is dependent on the choice of disk.

From last time, we saw that if $f$ is analytic, then $f$ is in fact holomorphic.

General question: Can we generate functions representable by power series?

Theorem:

Suppose $(X,\mu)$ is a complex (or finite, positive) measure space. Let $\varphi: X \to \mathbb{C}$ be a bounded, complex, measurable function. Let $\Omega \subseteq \mathbb{C}$ be an open set disjoint from $\varphi(X)$. Then the function

$$f(z) = \int_X \frac{d\mu(x)}{\varphi(x) - z} $$

defined on $z \in \Omega$ is analytic on $\Omega$.

Example:

Let $g: [-1,1] \to \mathbb{R}$, continuous. Then, this can be used as a measure. Let $\Omega = \mathbb{C} \setminus [-1,1]$. Then:

$$ f(z) = \int_{-1}^1 \frac{ g(x)}{x - z} dx $$

is analytic on $\Omega$.

Proof:

%Suppose that $D(a,r) \subset \Omega$. For any $z \in D(a,r)$, we notice that $|\varphi(x) - z|

First, fix a point $z \in \Omega$. Because $\Omega$ is open, we may find a $\delta > 0$ such that $D(z,\delta) \subset \Omega$. Because $\Omega \subset \varphi(X)^c$, this implies that for any $x \in X$, $| \varphi(x) - z| > \delta \implies \frac{1}{| \varphi(x) - z|}  \leq \frac{1}{\delta}$. Thus, $\frac{1}{ \varphi(x) - z} $ is bounded above. Thus, because this function is bounded, measurable (being the absolute value of a difference of measurable functions, we have that:

$$\int_X \frac{d\mu(x)}{\varphi(x) - z} $$

exists as a complex value for every $z$.

Now, let $D(a,r) \subset \Omega$. For any $z \in D(a,r)$, we have that:

$$ f(z) = \int_X \frac{d\mu(x)}{\varphi(x) - z}  = \int_X \frac{d\mu(x)}{(\varphi(x) - a) - (z - a)} =  \int_X \frac{1}{\varphi(x) - a} * \frac{d\mu(x)}{1 - \frac{z - a}{\varphi(x) - a}} $$

Now, we recall, $|z - a| < | \varphi(x) - a|$ due to $z \in D(a,r)$ and $\varphi(x)$ being outside of $\Omega$. Thus 

$$ \left|\frac{z - a}{\varphi(x) - a}\right| < 1 $$

so we may expand this as a geometric series. Rewriting our integrand then:

$$\int_X \frac{1}{\varphi(x) - a} * \frac{d\mu(x)}{1 - \frac{z - a}{\varphi(x) - a}} =  \int_X \frac{d\mu(x)}{\varphi(x) - a} * \sum_{n=0}^\infty \left( \frac{z - a}{\varphi(x) - a} \right)^n = \int_X \sum_{n=0}^\infty \frac{(z-a)^n}{(\varphi(x) - a)^{n+1}} d\mu(x) $$

Assuming we may interchange the integral, this is equal to:

$$\sum_{n=0}^\infty \left( \int_X \frac{(z-a)^n}{(\varphi(x) - a)^{n+1}} d\mu(x) \right)$$

We notice here, that $(z-a)^n$ is independent of the integral. So, if we define $c_n = \int_X \frac{d\mu(x)}{\varphi(x) - a)^{n+1}}$, we identify this as:

$$ \sum_{n=0}^\infty c_n (z-a)^n $$.

Now, we come back to proving that this integral may be interchanged.

Claim:

The series $\sum_{n=0}^\infty \frac{(z-a)^n}{(\varphi(x) - a)^{n+1}}$ converges uniformly in $x$.

Since $z \in D(a,r)$, there exists a $\rho > 0$ such that $|z-a| < \rho < r$, so we have that:

$$ \left| \frac{(z-a)^n}{(\varphi(x) - a)^{n+1}} \right| = \frac{|z-a|^n}{|\varphi(x) - a|^{n+1}} \leq \frac{\rho^n}{r^{n+1}}$$

But, we notice that because $\rho < r$, we see that $\sum_{n=0}^\infty \frac{\rho^n}{r^{n+1}} < \infty$.

Thus, identifying these as $M_n$ and applying the Weierstrauss M-test, we have that the series converges uniformly.

And finally, for a uniformly convergent series, we may interchange an integral and sum.

The exponential function:

We define the function $exp(z)$ as:

$$ exp(z) = \sum_{n=0}^\infty \frac{z^n}{n!} $$

We notice by the ratio test, this series converges absolutely for all $z \in \mathbb{C}$. Thus, this function has a radius of convergence of $R = \infty$.

We also notice that because this function is analytic, we may consider its derivative.

$$\frac{d}{dz} exp(z) = \sum_{n=0}^\infty \frac{d}{dz} \frac{z^n}{n!} = \sum_{n=1}^\infty \frac{z^{n-1}}{(n-1)! } = \sum_{n=0}^\infty \frac{z^n}{n!}  = exp(z)$$

We will, going forward, denote $e^z = exp(z)$.

Now, let $a,b \in \mathbb{C}$.

Consider $e^{a+b}$:

$$ e^{a+b} = \sum_{n=0}^\infty \frac{(a+b)^n}{n!} = \sum_n \frac{1}{n!} \sum_k \frac{n!}{k!(n-k)!} a^k b^{n-k} = \sum_n \sum_k \frac{a^k}{n!} \frac{b^{n-k}}{(n-k)!} $$  

Now, if we rewrite the inner sum, we may rewrite as:

$$  \sum_k \frac{a^k}{n!} \frac{b^{n-k}}{(n-k)!}  = \sum_{n-k = l, k \geq 0, l \geq 0} \frac{a^k}{n!} \frac{b^l}{l!}$$ 

We identify this as summing over every $(k,l) \in \mathbb{N}^2$, where we take $\mathbb{N}$ to include $0$. Then, we claim that this is actually equal to:

$$\sum_{k=0}^\infty \sum_{l=0}^\infty  \frac{a^k}{k!} \frac{b^l}{l!} = \sum_{k=0}^\infty\frac{a^k}{k!} \sum_{l=0}^\infty  \frac{b^l}{l!} = e^a e^b$$

where we justify the rearranging of the terms due to the absolute convergence of the exponential function.

Note that if $\theta\in \mathbb{R}$, then

$$ e^{i\theta} = \sum_{n=0}^\infty \frac{(i\theta)^n}{n!} = \sum_{n=0}^\infty \frac{(i\theta)^{2n}}{(2n)!}  + \sum_{n=0}^\infty \frac{(i\theta)^{2n+1}}{(2n+1)!} = \sum_{n=0}^\infty (-1)^n\frac{(\theta)^{2n}}{(2n)!}  + i\sum_{n=0}^\infty (-1)^n\frac{(\theta)^{2n+1}}{(2n+1)!} = \cos(\theta) + i \sin(\theta)$$

Thus, if we have $z \in \mathbb{C}$, we can rewrite as $z = x + iy, x,y \in \mathbb{R}$, and:

$$ e^z = e^x e^{iy} = e^x (\cos(y) + i \sin(y)) $$

In particular:

$$ |e^z| = | e^x (\cos(y) + i \sin(y)) | = |e^x| |\cos(y) + i \sin(y)| = e^x = e^{\Re(z)}$$

Because for any $y \in \mathbb{R}$,  $\cos(y) + i \sin(y)$ lies on the unit circle, so $|\cos(y) + i \sin(y)| = 1$ and $e^x > 0$ for all $x \in \mathbb{R}$, so $|e^x| = e^x$.

We notice then, that $e^z \not = 0$ for any $z$, since $e^x > 0$. 

We also notice that $e^z$ cannot be one to one, because from the expression $e^x (\cos(y) + i \sin(y)) $, we see that $e^{z} = e^{z + 2\pi i}$

Conversely, suppose that $e^z = e^w$, for $z,w \in \mathbb{C}$. Then, we must have that $z = w + 2\pi i n$ for some $n \in \mathbb{Z}$.

Thus, $z \to e^z$ is one to one on the horizontal strip $-\pi < y \leq \pi$

\section*{Jan 25th}

A curve in a topological space $X$ is a continuous map $\gamma: [\alpha,\beta] \to X$. If $\gamma(\alpha) = \gamma(\beta)$, we call the curve closed.

A path is a piecewise differentiable curve in $\mathbb{C}$. That is, we have a partition $\alpha = s_0 < s_1 <.... < s_n = \beta$ such that $\gamma’(t)$ exists on every subinterval $(s_{i-1}, s_i)$. Note that the path refers to the map. On the other hand, we will use $\gamma^*$ to refer to the image. We will call a path closed if the curve is closed.

Integration over paths:

Let $\gamma$ be a path on $\mathbb{C}$, and suppose that $f$ is a complex-valued function, continuous on $\gamma^*$. We define the integral of $f$ over $\gamma$ as the complex number:

$$ \int_\gamma f(z) dz \coloneqq \int_\alpha^\beta f(\gamma(t)) \gamma’(t) dt$$

where we remark that even though $\gamma’(t)$ may not exist at our corner points, they represent a set of measure 0. Further, the value of this integral is invariant under differentiable changes of parameter. Example: suppose $t = \phi(s)$, where $\phi: [\alpha’, \beta’] \to [\alpha, \beta]$, and $\phi \in C^1([\alpha’, \beta’])$. Then, via the chain rule, we would have that:

$$\int_\alpha^\beta f(\gamma(t)) \gamma’(t) dt = \int_{\alpha’}^{\beta’} f(\gamma(\varphi(s)) \gamma’(\varphi(s)) \varphi’(s) ds =  \int_{\alpha’}^{\beta’} f(\gamma_1(s)) \gamma_1’(s) ds $$

We see that it is not too difficult to concatenate paths. If we have two paths, $\gamma_1, \gamma_2$, with intervals, $[\alpha_1,\beta_1], [\alpha_2, \beta_2]$, then if $\beta_1 = \alpha_2$, then we may form the path $\gamma$ on the interval $[\alpha_1, \beta_2$, such that $\gamma(t) = \gamma_1$ if $t \in [\alpha_1, \beta_1]$ and $\gamma(t) = \gamma_2$ if $t \in [\alpha_2, \beta_2]$. In this case, we write:

$$ \int_\gamma f(z) dz  = \int_{\gamma_1} f(z) dz + \int_{\gamma_2} f(z) dz $$

Since we may always translate and scale, without loss of generality, going forward, we will usually take paths on $[0,1]$. 

Let $\gamma: [0,1] \to \mathbb{C}$. Define the path opposite to $\gamma$ as $\gamma_1: [0,1] \to \mathbb{C}$ such that $\gamma_1(t) = \gamma(1-t)$. It should be clear that $\gamma_1^* = \gamma^*$. Further, we have that:

$$ \int_{\gamma_1} f(z) dz = \int_0^1 f(\gamma_1(t)) \gamma_1’(t) dt = \int_1^0 f(\gamma(1-t)) \left(-\gamma’(1-t)\right) dt =  $$
$$  - \int_1^0 f(\gamma(s)) \gamma’(s)(-ds) = - \int_0^1 f(\gamma(s)) \gamma’(s) ds $$

where we use the fact that, by the chain rule:

$$ \gamma’(t) = -\gamma’(1-t)$$

Now, we recall that in general:

$$ \left| \int f(z) dz \right| \leq \int | f(z) | dz $$

In a path integral context, we have that:

$$ \left| \int_\gamma f(z) dz \right|  = \left| \int_\alpha^\beta f(\gamma(t)) \gamma’(t) dt \right| \leq $$
$$ \int_\alpha^\beta |f(\gamma(t))| |\gamma’(t)| dt \leq \int_\alpha^\beta \max_{z \in \gamma^*} |f(z)| |\gamma’(t)| dt =  \max_{z \in \gamma^*} |f(z)| L(\gamma)  $$

That is, the integral is bounded above by the maximum value on the path times the length of the path.

Note: annoyingly, we will sometimes denote the directed line segment in the complex plane for $a,b \in \mathbb{C}$ from $a$ to $b$ as $[a,b]$. We should know from context if this is a path, real interval, etc.

Let $(a,b,c)$ be an positively-oriented, ordered triple in $\mathbb{C}$, and define $\Delta$ as the triangle with vertices $a,b,c$ together with their interior points. We define the ordered boundary $\partial \Delta$ as the closed path traversing the triangle, which is $[a,b], [b,c], [c,a]$.

Then, we would have that:

$$ \int_{\partial \Delta} f(z) dz = \int_{[a,b]} f(z)dz +  \int_{[b,c]} f(z)dz  +  \int_{[c,a]} f(z)dz $$

Theorem:

Let $\gamma$ be a closed path, and let $\Omega = \mathbb{C} \setminus \gamma^*$

For any $z \in \Omega$, define the function:

$$\text{Ind}_\gamma(z) \coloneqq \frac{1}{2\pi i} \int \frac{d\zeta}{\zeta - z}$$

Then the $\text{Ind}_\gamma(z)$ is integer-valued, and is constant on each component of $\Omega$. In particular, it takes on $0$ on the unbounded component of $\Omega$.

We will see later that the index corresponds to the winding number of the component.

Remark:

We know that $\gamma^*$ is compact, that is, is contained in some disk $\overline{D}(0,R)$. Consider $\mathbb{C} \setminus \overline{D}(0,R)$. This must be connected, so it may only lie in a single component of $\Omega$. Then, for every other component of $\Omega$, they are contained in this disk, and therefore bounded as well.

Proof:

We see by the definition of the index, that it must be analytic (by the generating theorem for power series representable functions), and therefore holomorphic, and therefore cts. Thus, if we can prove that the index is integer-valued, then is must be constant on each connected component of $\Omega$. Moreover, if $z$ belongs to the unbounded component of $\Omega$, we notice that we may always choose $z$ such that $|\zeta - z| > M$ for all $\zeta \in \gamma, M \in \mathbb{N}$. So, if it actually is integer-valued, then the integral must be 0.

Now, fix some $z \in \Omega$. Assume that $\gamma: [\alpha, \beta] \to \gamma^*$. Then:

$$\text{Ind}_\gamma(z) \coloneqq \frac{1}{2\pi i} \int \frac{d\zeta}{\zeta - z} = \frac{1}{2\pi i} \int_\alpha^\beta \frac{\gamma’(s) ds}{\gamma(s)- z}$$

Here, we define the related function $\varphi: [\alpha,\beta] \to \mathbb{C}$ via

$$ \varphi(t) = \int_\alpha^t \frac{\gamma’(s) ds}{\gamma(s)- z} $$

Then, by the fundamental theorem of calculus, we have that:

$$ \varphi’(t) = \frac{\gamma’(t)}{\gamma(t)- z}  \implies \gamma’(t) - \varphi’(t)\left( \gamma(t) - z\right) = 0 \implies \frac{d}{dt} \left[ (\gamma(t) - z) e^{-\varphi(t)} = 0 \right]$$ 

This implies that $(\gamma(t) - z) e^{-\varphi(t)}$ is a constant on $[\alpha, \beta]$

Thus, we have that:

$$ (\gamma(\alpha) - z) e^{-\varphi(\alpha)} = (\gamma(\beta) - z) e^{-\varphi(\beta)} \implies e^{-\varphi(\alpha)} = e^{-\varphi(\beta)} $$

Since we may pick $z \not = 0$, and because $\gamma$ is closed, $\gamma(\alpha) = \gamma(\beta)$.

But $\varphi(\alpha) = 0$ from the definition of $\varphi$. So, we have that $\varphi(\beta) = 2\pi i k$ for some $k \in \mathbb{Z}$.

Thus, we have that:

$$ \text{Ind}_\gamma(z) = \frac{1}{2\pi i} \varphi(\beta) = m $$

\section*{Jan 30th}

Theorem [10.11]

If $\gamma$ is a positively oriented circle with center $a$ and radius $r$, then $\text{Ind}_\gamma(z) = 0 $ if $|z - a| > r$ and $1$ if not, where we take the index to be the complement of our path.

Proof:

Take $z = a$, since we can choose any point to apply the index to. Choose the parametrization $\gamma: [0, 2\pi]$ that sends $t \to a + r e^it$

Then, we compute:

$$\text{Ind}_\gamma(z) \coloneqq \frac{1}{2\pi i} \int \frac{d\zeta}{\zeta - a} = \frac{1}{2\pi i} \int \frac{rie^{it}}{re^{it}} = \frac{1}{2\pi} \int_0^{2\pi} dt = 1 $$

The Local Cauchy Theorem [10.12]:

Suppose a function $f \in \mathcal{H}(\Omega)$, such that $f’$ is continuous. Then:

$$\int_\gamma f’(z) dz = 0$$

for every closed path $\gamma$. 

Proof:

Let $\gamma$ be parametrized by $[\alpha,\beta]$. Then, we can rewrite this as:

$$ \int_\gamma f’(z)dz = \int_\alpha^\beta f’(\gamma(t)) \gamma’(t) dt = f(\gamma(\beta)) - f(\gamma(\alpha)) = 0$$

because this is a closed path, $\gamma(\beta) = \gamma(\alpha)$

Corollary:

Since $z^n$ is the derivative of $\frac{z^{n+1}}{n+1}$ for all $n \in \mathbb{Z}$ except $n = -1$, then we have that:


$\int_\gamma z^n dz = 0$ for every closed path $\gamma$, so long as $n \geq 0, n \in \mathbb{N}$

Further, if $\gamma$ is a closed path such that $0 \not \in \gamma^*$, then this also vanishes for $n \in \mathbb{Z}, n \leq -2$ 

Theorem [Cauchy’s Theorem for Triangles]

Suppose $\Delta$ is a closed triangle in a plane open set $\Omega$. Let $p \in \Omega$, $f$ be a continuous function on $\Omega$, and $f \in \mathcal{H}(\Omega \setminus \{ p \})$. Then:

$$ \int_{\partial \Delta} f dz = 0 $$

Proof:

Case 1:

$ p \not \in \Delta$

Call $J = \int_{\partial \Delta} f(z) dz $. Define $a’$ as the midpoint of $\overline{BC}$, and $b’,c’$ in the same way, and connect $a’,b’,c’$ to subdivide our triangle into 4 smaller triangles. Then, we may rewrite $J$ as:

$$J = \sum_{j=1}^f \int_{\partial \Delta_j} f(z) dz$$

Choose $\Delta_1$ as one of the subdivided triangles such that:

$$ \left| \int_{\partial\Delta_1} f(z) dz \right| \geq \left| \frac{J}{4} \right| $$

Of course, $\Delta \supset \Delta_1$. But, we may repeat this process. Subdivide further $\Delta_1$ into 4 component triangles with the correct orientation to preserve the original integral, and take $\Delta_2$ as having at least $\left| \frac{J}{4^2} \right|$.

Then, we construct a chain:

$$ \Delta \supset \Delta_1 \supset ... \supset \Delta_n \supset \Delta_{n+1} \supset ... $$

with 

$$ |J| \leq 4^n \left| \int_{\partial \Delta_n} f(z) dz \right| $$

But, we notice, looking at the lengths, that 

$$ | \partial \Delta_n | = 2^{-n} | \partial \Delta| = 2^{-n} L $$

Since this is a nested set of compact sets, we expect that $\cap_{n=1}^\infty \Delta_n = \{ z_0 \}$ for some unique point. But, we must have that $z_0 \in \Delta$ with $f$ differentiable at $z_0$. 

Taking $\epsilon > 0$, for $r > 0$, we have that:

$$| f(z) - f(z_0) - f’(z_0)(z - z_0) | \leq \epsilon |z - z_0| $$ so long as $| z - z_0 | < r$. In particular, we may choose $n$ such that $| z - z_0| < r$ for all $z \in \Delta_n$ , since we can always choose a triangle small enough.

We claim that:

$$\int_{\partial \Delta_n} f(z) dz = \int_{\partial \Delta_n} (f(z) - f(z_0) - f’(z_0)(z - z_0) = -f(z_0) \int_{\partial \Delta_n} dz - f’(z_0) \int_{\partial \Delta_n} zdz + f’(z_0) z_0 \int_{\partial \Delta_n} dz$$

Taking the absolute values, we notice that:

$$ \left|  \int_{\partial \Delta_n}  f(z) dz \right | \leq \int_{\partial \Delta_n} |(f(z) - f(z_0) - f’(z_0)(z - z_0)| \leq \epsilon  \int_{\partial \Delta_n} | z- z_0| dz \leq \epsilon 2^{-n} L | \partial \Delta_n|  = \epsilon (2^{-n}L)^2 $$

But, we recall that $|J| \leq 4^n \left| \int_{\partial \Delta_n} f(z) dz \right|  \leq \epsilon L^2 $. Since $\epsilon$ was arbitrary, this implies $|J| < 0$. 

Now, suppose $p \in \Delta$. 

First, suppose that $p$ is one of the vertices of our triangle, $a = p$. Take $x, y$ such that $x \in \overline{pc}$, $y \in \overline{pb}$, with $x, y$ “close” to $p$, and then subdivide such that we draw $\overline{xp}, \overline{xy}$. Clearly, the subdivided triangle with parts without $p$ have integral due to the first part of the proof. But we may subdivide such that the triangle has arbitrarily small area. So this must converge to 0.

Now, suppose $p$ is an interior point. Then, we just connect $a,b,c$ to $p$, and apply what we discovered when $p$ is a vertex.

Theorem [ 10.14 Cauchy’s Theorem on a convex set ]:

Suppose $\Omega$ is a convex open set, $p \in \Omega$, $f$ continuous on $\Omega$, and $f \in \mathcal{H}(\Omega \setminus \{ p \})$. Then, $f = F’$ for some $F \in \mathcal{H}(\Omega)$. Hence, $\int_\gamma f(z) dz = 0$ for every closed path $\gamma$ on $\Omega$. 

Proof:

Fix an $a \in \Omega$, and let $z \in \Omega$. Consider:

$$ f(z) = \int_{[a,z]} f(z) dz$$

If $z_0 \in \Omega$, then we may consider the triangle with $\{ a, z, z_0 \}$, which is contained within $\Omega$ by convexity. But, then, we have that the integral over the entire triangle must vanish. So we have that:

$$ \int_{[a,z]} f(z) dz + \int_{[z,z_0]} f(\zeta) d\zeta + \int_{[z_0, a]} f(\zeta) d\zeta = 0$$

But we recognize the first integral as $F(z)$, and the last integral as $-F(z_0)$ by orientation. So we have that:

$$ F(z) - F(z_0) = \int_{[z_0,z]} f(\zeta) d\zeta $$

Now, assuming $z \not = z_0$, we have that:

$$ \frac{f(z) - f(z_0)}{z - z_0} - f(z_0) = \frac{1}{z - z_0} \int_{[z_0,z]} (f(\zeta) - f(z_0)) d\zeta$$

Taking the absolute value of both sides, we use the continuity of $f$ at $z_0$. 

We have that for $\epsilon > 0$, we can find $\delta > 0$ such that $| \zeta - z_0| < \delta \implies | f(\zeta) - f(z_0) | < \epsilon | $

So, if $| z - z_0 | < \delta$, we have that:

$$\left| \frac{F(z) - F(z_0)}{z - z_0} - f(z_0) \right| \leq \frac{1}{z - z_0} \epsilon | z - z_0|  = \epsilon $$

Thus, we have that $F’(z_0) = f(z_0)$, and that

$$ \int_\gamma f(z) dz = \int_\gamma F’(z) dz = 0$$

\section*{Feb 1st}

Theorem [10.15, Cauchy’s formula for a Convex Set]

Suppose $\gamma$ is a closed path in a convex, open set $\Omega$. If $z \in \Omega$, and $z \not \in \gamma^*$, then:

$$f(z) \text{Ind}_\gamma(z) = \frac{1}{2\pi i} \int_\gamma \frac{f(\zeta)}{\zeta - z} d\zeta $$

Proof:

Fix a $z \in \Omega$. Define $$g(\zeta) = \begin{cases} \frac{f(\zeta) - f(z)}{\zeta - z} & \text{ if } \zeta \in \Omega, \zeta \not = z \\ f’(z) & \text{else} \end{cases}$$

We remark that of course, $g$ is holomorphic on $\Omega \setminus \{ z \}$ because $f$ is holomorphic everywhere. 

Then, by the Cauchy theorem on a convex set, we have that:

$$  0 = \int_\gamma g(\zeta) d\zeta = \int_\gamma \frac{f(\zeta) - f(z)}{\zeta - z}  d\zeta \implies f(z) \text{Ind}_\gamma(z) = \frac{1}{2\pi i} \int_\gamma \frac{f(\zeta)}{\zeta - z} d\zeta $$

Theorem: [10.16]

For every open set $\Omega$ in the plane, every $f \in \mathcal{H}(\Omega)$ is representable by a power series in $\Omega$. 

Recall: [10.7] If $\mu$ is a finite, complex measure on a measurable set $X$, $\varphi$ is a complex, measurable function on $X$, $\Omega$ an open set in the plane which is disjoint from $\varphi(X)$, and $$ f(z) = \int_X \frac{d\mu(\zeta)}{\varphi(\zeta) - z}$$, when $f$ is representable by a power series.

Proof:

Let $f \in \mathcal{H}(\Omega)$. Pick $ a \in \Omega$, and $R$ such that $D(a, R) \subset \Omega$. Let $\gamma$ be positively oriented circle of center $a$, and radius $r < R$. 

Then, $D(a,R)$ is a convex set, so by 10.15, we have that $$f(z) = \int_\gamma \frac{f(\zeta)}{\zeta - z} d\zeta = \int_\alpha^\beta \frac{f(\gamma(t))}{\gamma(t) - z} \gamma’(t) dt $$

Here, we recognize that $$ f(\gamma(t)) \gamma’(t)dt$$ defines a complex valued measure, so so by 10.7, we have that $f(z)$ is representable by a power series (and further, it is unique). That is, there exist unique $\{ c_n \} \subset \mathbb{C}$ such that $f(z) = \sum_{n=0}^\infty c_n (z -a)^n$ for all $z \in D(a,r)$. But we picked $r < R$, arbitrarily, so we can do this as $r \to R$, and this actually applies on the entire $D(a,R)$. Since we can do this for any $a$, we have a representation for every disk in $\Omega$. 

Corollary:

$$f \in \calH(\Omega) \implies f’ \in \calH(\Omega)$$
 
 That is, if $f$ is holomorphic, so is its derivative.
 
Theorem [10.17, Morera’s Theorem]

Suppose $f$ is a continuous, complex-valued function on an open set $\Omega$ such that $\int_{\partial \Delta} f(z) dz = 0$ for every closed triangle $\Delta \subset \Omega$. Then, $f \in \calH(\Omega)$.

Proof:

Let $V$ be a convex, open set in $\Omega$. Then, we may construct $F \in \calH(V)$ by fixing a $z_0 \in \Omega$, and for all $z \in \Omega$, we define $F(z) = \int_{[z_0, z]} f(\zeta) d\zeta$.

This implies that since $F’ = f$, that $f \in \calH(V)$.

Since we can do this for each convex open set $V$, and we can patch $\Omega$ via convex open sets, this is true on $\Omega$.

We wish to investigate the power series representation a bit more closely.

Theorem [10.18]

Suppose that $\Omega \subseteq \mathbb{C}$ is a region, $f \in \calH(\Omega)$, and define $Z(f) = \{ a \in \Omega : f(a) = 0 \}$, that is, the set of zeros.

Then, either $Z(f) = \Omega$, or $Z(f)$ has no limit point in $\Omega$. In the latter case, there exists a correspondence between each $a \in Z(f)$ and a unique positive integer $m = m(a)$ such that $f(z) = (z-a)^m f(z)$ where $g \in \calH(\Omega)$ and $g(\Omega) \not = 0$. Further, $Z(f)$ is at most countable. We call the integer $m(a)$ the order of the zero $a$.

Proof:

Let $A$ be the set of all limit points of $Z(f)$. Clearly then, $A \subset Z(f)$. Fix an $a \in Z(f)$ and choose $r > 0$ such that $D(a,r) \subset \Omega$.

Because $f$ is holomorphic, it has a power series representation:

$$f(z) = \sum_{n=0}^\infty c_n (z-a)^n$$ where $z \in D(a,r)$. 

Suppose $c_n = 0$ for all $n$. Then, we have that $D(a,r) \subset A$, since it is 0 everywhere. Then, we have that $a \in \mathring{A}$

Otherwise, there exists a smallest positive integer $m$ such that $c_m \not = 0$. Define $$g(z) = \begin{cases} (z-a)^{-m} f(z) & z \in \Omega \setminus \{a\} \\ c_m & \text{ if } z = a \end{cases}$$

Certainly, $g \in \calH(\Omega \setminus \{ a \})$. In fact, we may write $g$ as:

$$ g(z) = \sum_{k=0}^\infty c_{m+k} (z-a)^k$$ where $z \in D(a,r)$, by examining the power series of $f$. Since $g$ is representable by a power series, then $g \in \calH(D(a,r)) \implies g \in \calH(\Omega)$, where we extend to all of $\Omega$. Lastly, it should be clear that $g(a) \not = 0$. 

Then, we can conclude $a$ is an isolated point. Then, if $a \in A$, we are back in case 1, and $A$ must be open, which implies that $B = \Omega \setminus A$ is open, which implies that $\Omega = A \cup B$. But since $\Omega$ is connected, then either $A = \Omega$ or $B = \Omega$, and either $Z(f) = \Omega$ or $Z(f)$ has no limit points, and has at most finitely many points in each compact subset of $\Omega$. Thus, over all of $\Omega$, $Z(f)$ is at most countable, since $\Omega \subset \mathbb{C}$ is $\sigma$-finite.

Corollary:

If $f, g \in \calH(\Omega)$, for $\Omega$ a region, $f(z) = g(z)$ for all $z$ in some set which has a limit point in $\Omega$, then $f(z) = g(z)$ on all of $\Omega$. (easy, consider the function $f - g$.)

Definition:

If $a \in \Omega$, and $f \in \calH(\Omega \setminus \{ a \})$, then $f$ is said to have an isolated singularity at $a$. If $f$ can be extended to be defined at $a$ such that the extended function $\tilde{f}$ is holomorphic on all of $\Omega$, then we call the singularity removable.

Theorem:

Suppose $f \in \calH(\Omega \setminus \{ a \})$, and $f$ is bounded on $D’(a,r)$ for some $r > 0$, where we denote $D’$ as the punctured disk, $\{ z : 0 < |z-a| < r \}$. Then $f$ has a removable singularity at $a$. 

Proof:

Let $h(a) = 0$, $h(z) = (z-a)^2 f(z)$, for $z \in \Omega \setminus \{ a \}$. 

Then, we notice that $$ \frac{h(z) - h(a)}{z-a}= (z-a) f(z)$$ so the absolute value:

$$  \left| \frac{h(z) - h(a)}{z-a} \right| \leq |z-a| |f(z)| \leq |z-a| M$$ because $f$ is bounded on $D’$. Taking the limit as $z \to a$, we find that $h’(a) = 0$.

So $h$ is holomorphic on all of $D(a,r)$, so we have a power series representation:

$$h(z) = \sum_{n=2}^\infty c_n(z-a)^n = (z-a)^2 \sum_{n=1}^\infty c_{n+1} (z-a)^n = (z-a)^2 f(z)$$ where we use the fact that $f(a) = 0$, and $h’(a) = 0$, and therefore, we have that $$ f(z) = \sum_{n=0}^\infty c_{n+2} (z-a)^n$$, and $f(a) = c_2$.

\section*{Feb 6th}

Theorem [10.21]:

Let $a \in \Omega$, $f \in \calH(\Omega \setminus \{ a \})$. One of the following cases must occur:

a) $f$ has a removable singularity at $a$.

b) There are complex numbers $c_1,...,c_m$ for a positive integer $m$, $c_m \not = 0$ such that $f(z) - \sum_{k=1}^m \frac{c_m}{(z-a)^k}$ has a removable singularity at $a$.

c) If $r > 0$, and $D(a,r) \subset \Omega$, then $f(D’(a,r))$ is dense in the plane.

Remark/Definition: 

In case (b), $f$ is said to have a pole of order $m$ at $a$. The function $\sum_{k=1}^m c_k (z - a)^{-k}$ is called the principal part of $f$ at $a$. We also note that as $z \to a$, $| f(z) | \to \infty$.

In case (c), $f$ is said to have an essential singularity at $a$. That is, for any $w \in \mathbb{C}$, there exists a sequence for any $r > 0$, $\{ z_n \}$ such that $z_n \in D(a,r)$ and as $z_n \to a$, $f( z_n) \to w$. 

Proof. Suppose (c) is not true. Then, we have that $r > 0, \delta > 0$, we may find a $w \in \mathbb{C}$ such that

$$ | f(z)  - w | > \delta$$ for all $z \in D’(a,r)$.

Define $g(z) = \frac{1}{f(z) - w}$, on $z \in D’(a,r)$.

Clearly, by our condition then, since the denominator cannot vanish, we have that $g \in \calH(D’(a,r))$, and $|g| =  \frac{1}{|f(z) - w|} < \frac{1}{\delta}$. But, this means that we are bounded on the punctured disk, so $g$ has a removable singularity, and $g \in \calH(D(a,r))$.

If $g(a) = z_0 \not = 0$, then we may conclude $f$ has a removable singularity as well, since we would have that

$$ g(z) = \frac{1}{f(z) - w} \implies f(z) = w + \frac{1}{g(z)}$$ so if $g$ is bounded, then $f$ is bounded as well.

Now, suppose $g$ has a zero of order $m \geq 1$ at $a$

Then, we can rewrite as::

$$g(z) = (z-a)^m g_l(z) $$ for $z \in D(a,r), g_l \in \calH(D)$, and $g_l(a) \not = 0$.

Define $h = \frac{1}{g_l}$ on $D$. We of course have that $h \in \calH(D)$, and we have that:

$$ g(z) = \frac{1}{f(z) - w} = (z-a)^m g_l(z) \implies f(z) - w = (z-a)^{-m} \frac{1}{g_l(z)} =  (z-a)^{-m} h(z)$$.

But, since $h$ is holomorphic, thus analytic, we have that:

$$h(z) = \sum_{n=0}^\infty b_n (z-a)^n$$ with $b_0 \not = 0$.

But this is exactly the case in (b), where we identify the first $1,...,m$ terms as the principal part.

Theorem:

If $f(z) = \sum_{n=0}^\infty c_n (z-a)^n$ for $z \in D(a,r)$, and if $0 < r < R$, then we have that:

$$ \sum_{n=0}^\infty |c_n|^2 r^{2n} = \frac{1}{2\pi} \int_{-\pi}^\pi | f(a + re^{i\theta})|^2 d\theta $$

Proof:

Suppose $0 < r < R$. Then, since $f$ is analytic on this disk, we can see that:

$$f(a + re^{i\theta}) = \sum_{n=0}^\infty c_n (a + re^{i\theta} - a)^n = \sum_{n=0}^\infty c_n r^n e^{in\theta}$$

which converges uniformly on $[-\pi, \pi]$. But, we have that $\left\{ \frac{1}{\sqrt{2\pi}} e^{i \pi \theta } \right\}_{k=-\infty}^\infty$ is a orthonormal basis for $L^2[-\pi,\pi]$. So, we have that for all $n \geq 0$, that $c_n r^n = \frac{1}{2\pi} \int_{-\pi}^\pi f(a + re^{i\theta}) e^{-in \theta} d\theta$ and via Parseval’s theorem, we get that:

$$ \sum_{n=0}^\infty | c_n|^2 r^{2n} = \frac{1}{2\pi} \int_{-\pi}^\pi |f(a + r e^{i\theta})|^2 d\theta$$

Theorem [10.23, Liouville’s Theorem]:

Every bounded, entire function is constant.

Proof:

Let $f$ be an entire function. Suppose $f$ is bounded. Then, we have that for some $M > 0$, that $f(z) < M$ for all $z \in \mathbb{C}$.

Since $f$ is entire, we can write a power series around $0$:

$$f(z) = \sum_{n=0}^\infty c_n z^n$$

for $r > 0$.

But, by the last theorem, we have that:

$$\sum_{n=0} |c_n|^2 r^{2n} = \frac{1}{2\pi} \int_{-\pi}^\pi |f(re^{i\theta})|^2 d\theta \leq M^2 $$

because of the boundedness.

But, this implies that, since $r$ can be taken to be arbitrarily large, that for all $n > 0$, $c_n = 0$. Thus, $f(z) = c_0$, a constant.

Theorem [10.24 Maximum Modulus Theorem]:

Suppose $\Omega$ is a region, $f \in \calH(\Omega)$, and $\overline{D}(a,r) \subseteq \Omega$. Then:

$$f(a) \leq \max_{\theta} | f(a + r^{i\theta})|$$

with equality if and only if $f$ is constant.

Consequently, $|f|$ has no local maximum at any interior point of $\Omega$, unless $f$ is constant.

Proof:

We wish to show that $|f(a)| \leq \max_{\theta} | f(a + r^{i\theta})|$.

Assume not. Then, we have that:

$$|f(a + r e^{i\theta}| \leq \max_{\theta} | f(a + r^{i\theta})| < |f(a)|$$ for all $\theta \in [-\pi,\pi]$.

Since $f$ is holomorphic, $f$ is analytic. thus, we have that, for a power series centered at $a$:

$$\sum_{n=0}^\infty |c_n|^2 r^{2n} \leq |f(a)|^2 = |c_0|^2  \implies \sum_{n=1}^\infty |c_n|^2 r^{2n} < 0 \implies c_n = 0 \text{  }\forall n \geq 1$$

Thus, $f(z) = f(a)$ on $\Omega$. Thus, we have a contradiction, and we cannot have that $f(a) > \max_{\theta} | f(a + r^{i\theta})| < |f(a)|$.

Corollary:

Same hypotheses, but if $f$ also has no zero in $D(a,r)$:

$$|f(a)| \geq \min_{\theta} | f(a + r^{i\theta})|$$

Proof:

If $f$ has a zero on $a + r^{i\theta}$, then the result is immediate. Now, suppose $f$ has no zero. We may find a region $\Omega_0$ that contains $\overline{D}(a,r)$ such that $f \not = 0$ on $\Omega_0$. Then, define $g = \frac{1}{z}$, on $\Omega_0$ and apply 10.24 on $g$.

Small application:

If $p(t)$ is a polynomial with complex coefficients, then $p(z) = 0$ has at least one solution.

Proof:

Suppose $p(z) \not = 0$ for all $z \in \mathbb{C}$. Consider then $g = \frac{1}{p}$. Clearly, since polynomials are entire, then $g$ must be entire. Further, we see that since $\lim_{|z| \to \infty} |p(z)| \to \infty$, that $\lim_{|z| \to 0} |g(z)| \to 0$. In particular, we see that $g$ must be bounded on $\overline{D}(0,R)$. Thus, $g$ is bounded, entire, and thus constant. But this is impossible. Thus, $p$ has at least one root.

\section*{Feb 8th}

Fundamental Theorem of Algebra:

If $n$ is a positive integer, and

$$P(z)  = z^n + a_{n-1} z^{n-1} + .... + a_1 z + a_0 $$

where $a_i \in \mathbb{C}$, then $P$ has precisely $n$ zeros in the plane, counted with multiplicity.

Proof:

First, we want to show that $P$ has at least one zero, and go from there.

First, suppose that $P(z) \not = 0$ for all $z$. Then, $f(z) = \frac{1}{P(z)} $ is entire. Choose $R> 1$ such that $R > \sum_{i=0}^{n-1} |a_i| $. Since $f$ is entire, it must be continuous everywhere too. Then, on the compact set $\overline{D}(0,R)$, it must further be bounded. 

Now, when $|z| > R$, we have that:

$$|P(z)| = | z^n + ... + a_0 | \geq |z|^n - |a_{n-1}| |z|^{n-1} - .... - |a_0| \geq $$
$$|z|^n - \left( \sum_{i=0}^{n-1} |a_i| \right) |z|^{n-1}  = |z|^{n-1} \left( |z| - \sum_{i=0}^{n-1} |a_i| \right)$$

where we use the fact that $|z| > R > 1$ to say that $|z|^k \leq |z|^{n-1}$ for all $k \leq n-1$.

But, then, since $|z| \geq R$, we have that:

$$ |z|^{n-1} \left( |z| - \sum_{i=0}^{n-1} |a_i| \right) \geq R^{n-1} \left( R - \sum_{i=0}^{n-1} |a_i| \right)$$

which, for a choice of $P$, is a constant on the lower bound. Call this constant $C$.

Then, we have that $f$ is bounded on the disk, due to continuity and compactness, and since $P \geq C$ off the disk, then $f \leq \frac{1}{C}$. Thus, $f$ is bounded on all of $\mathbb{C}$ then, taking the max of these two bounds.

But, this is impossible, because by Liouville’s, we have that $f$ must be constant. But, this is a contradiction, as if $f$ is constant, then $P$ is constant, which is true only if its degree is $0$.

Thus, $P$ has at least one zero.

Then, by the root theorem, we can write $P(z)$ as:

$$P(z) = (z - \alpha) g(z)$$

which, by degree arguments, $g$ must have degree $n-1$. We may iterate this $n$ times, as at that point $g(z)$ would have degree $0$. Thus, we have exactly $n$ zeros.

Corollary:

Any $n \times n$ complex-valued matrix $A$ has an eigenvalue.

This should be clear, it just means that we always have solutions to the characteristic polynomial working in the complex numbers.

Theorem [10.26 Cauchy Estimate’s]:

Let $f \in \calH(D(a,R))$ and $|f(z)| \leq M$ for all $z \in D(a,R)$. Then we have that:

$$| f^{(n)}(a)| \leq \frac{n!M}{R^n}$$

for each $n$.

Proof:

Since $f$ is holomorphic on $D(a,r)$, $f$ has a power series expansion.  Choose any $r < R$. Then, we have that:

$$ \sum_{n=0}^\infty |a_n|^2 r^{2n} = \frac{1}{2\pi} \int_0^{2\pi} | f(a + re^{i\theta}) |^2 d\theta $$

Then, we have that:

$$ |a_n|^2 r^{2n} \leq \frac{1}{2\pi} \int_0^{2\pi} | f(a + re^{i\theta}) |^2 d\theta$$

since of course, if the sum is equal, a specfic term must be at most the whole thing, since these are non-negative.

But, the integral is bounded, so we have that:

$$ \frac{1}{2\pi} \int_0^{2\pi} | f(a + re^{i\theta}) |^2 d\theta \leq \frac{1}{2\pi} \int_0^{2\pi} M^2 d\theta  = M^2$$

Hence, we have that $|a_n|^2 r^{2n} \leq M^2 \implies |a_n| \leq \frac{M}{r^n} $

However, we also recall that:

$$ a_n = \frac{f^{(n)}(a)}{n!}$$

Thus, we have that:

$$ \left| \frac{f^{(n)}(a)}{n!} \right| \leq frac{M}{r^n} \implies |f^{(n)}(a)| \leq \frac{n! M}{r^n} $$

And we get the desired result by taking $r \to R$.

Definition [10.27]

A sequence of functions $\{ f_j \}$ on $\Omega$ is said to converge to $f$ uniformly on compact subsets of $\Omega$ if, to every compact proper subset $K \subset \Omega$ and to every $\epsilon > 0$, there corresponds an $N$, dependent on $\epsilon$ and $K$, such that $|f_j - f| < \epsilon$ for all $ z \in K$ so long as $j > K$. 

Note: $f_j$ need not converge uniformly to $f$ on all of $\Omega$.

Example: Take $\Omega = D(0,1)$, and take $f_i(z) = z^i$. This converges uniformly on compact subsets of $\Omega$, but not uniformly on $\Omega$. Conceptually, of course, we see that a compact subset of $\Omega$ has a minimum distance to $\Omega^c$, and so we have an upper bound to $|z|$ on the compact set, and therefore a lower bound to the speed of convergence of $f_j$.

Theorem:

Suppose $f_j \in \calH(\Omega)$, and $f_j \to f$ uniformly on compact subsets of $\Omega$. Then, we have that $f \in \calH(\Omega)$, and $f’_j \to f’$ uniformly on compact subsets of $\Omega$.

Proof:

Let $\Delta$ be a closed triangle in $\Omega$, which we remark is compact.

Then, we have that:

$$\int_{\partial \Delta} f(z) dz = \int_{\partial \Delta} \left[ \lim_{j \to \infty} f_j(z) \right] dz$$

Because the convergence is uniform on compact sets, like $\Delta$, we may interchange the limit and integral:

$$\int_{\partial \Delta} \left[ \lim_{j \to \infty} f_j(z) \right] dz = \lim_{j \to \infty} \int_{\partial \Delta} f_j(z) dz  = 0 $$

because $f_j(z)$ is holomorphic, each integral is 0. Hence, since the choice of $\Delta$ was arbitrary, this is true for every triangle, so by Morera’s, $f$ is holomorphic as well.

Now, let $K$ be an arbitrary compact subset of $\Omega$. We claim that we may define $E \supset K$ such that $E$ is compact, and for some $r > 0$, $\overline{D}(a,r) \subset E$ for every $a \in K$.

Now, fix some $z \in K$. By Cauchy’s estimate for $n=1$, we have that:

$$|f’(a)| \leq \frac{M}{r}$$

Then, applying this to the function $ f_j - f$ over the disk $\overline{D}(z,r)$, we have that:

$$| f_j’ - f’| \leq \frac{1}{r} \Vert f_j - f \Vert_{\overline{D}(z,r)} \leq \frac{1}{r}  \Vert f_j - f \Vert_{E} $$.

Therefore, given $\epsilon > 0$, there is an $N$ such that $\Vert f_j - f \Vert_e  < r\epsilon$ whenever $j > N$. Then, if $z \in K, j \geq N$, since we can always create such a disk, we have the same bound, and thus uniform convergence on all of $K$.

Corollary:

Under the same hypothesis, $f^{(n)}_j \to f^{(n)}$ uniformly on every compact set $K \subseteq \Omega$ for every positive integer $n$. 

This should be clear. Just apply the previous theorem to $f’, f_j’$.

Open Mapping Theorem:

Objective: We wish to show that if $\Omega$ is a region, and $f \in \calH(\Omega)$, then $f(\Omega)$ is either a region, or a single point.

Lemma [10.29]:

If $f \in \calH(\Omega)$, and we define $g: \Omega \times \Omega \to \mathbb{C}$ via:

$$ g(z,w) = \begin{cases} \frac{f(z) - f(w)}{z  - w} & \text{ if } w \not = z \\ f’(z) & \text{ if } w = z \end{cases} $$

Then, $g$ is continuous on $\Omega \times \Omega$. 

Proof:

Clearly, we have that on $z \not = w$, that $g$ is continuous, since $f(z) - f(w)$ is continuous, and $z - w$ is non-0, continuous. Thus, we need only consider the points on $z = w$.

Fix some $(a,a) \in \Omega \times \Omega$.

Let $\epsilon > 0$. Since $f’$ is holomorphic, and thus continuous, we may choose $\delta > 0$ such that $| f’(\zeta) - f’(a) | < \epsilon$ for $\zeta \in D(a,\delta)$

Let $z,w \in D(a,\delta)$. since $D(a, \delta)$ is convex, we have that:

$$ f(z) - f(w) = \int_{[w,z]} f’(\zeta) d\zeta = (z-w) \int_0^1 f’((1-t)w + tz) dt $$

for the parametrization $\zeta(t) = (1-t)w + tz$ over $t \in [0,1]$, and we pull out $(z-w)$ because $d\zeta = (z-w) dt$.

So, if we have that $z \not = w$, we get that $$g(z,w) = \frac{f(z) - f(w)}{z-w} = \int_0^1 f’((1-t)w + tz) dt $$

We notice that this equation also holds if $z = w$. So, we then see that:

$$ | g(z,w) - g(a,a) | = \left| \int_0^1 f’((1-t)w + tz) dt - f’(a) \right| = \left|  \int_0^1 f’((1-t)w + tz)- f’(a) dt \right| \leq $$ 
$$ \int_0^1 | f’((1-t)w + tz)- f’(a)| dt  \leq \int_0^1 \epsilon dt = \epsilon$$

due to the continuity of $f’$, and because $d((1-t)w + tz, a) < \delta$.

Thus, we have continuity at any $(a,a) \in \Omega \times \Omega$.

Theorem [10.30]:

Suppose $\varphi \in \calH(\Omega)$, $z_0 \in \Omega$, and $\varphi’(z_0) \not = 0$. Then, $\Omega$ contains a neighborhood $V$ of $z_0$ such that the following hold:

(a) $\varphi$ is one-to-one on $V$

(b) $W = \varphi(V)$ is open

(c) If $\psi: W \to V$ is defined by $\psi(\varphi(z)) = z$, then $\psi \in \calH(W)$

Thus, $\varphi: V \to W$ has a holomorphic inverse. 

We will pick up on the proof next time.

\section*{Feb 13th}

Proof:

Recall that the function $g$ defined on $\Omega \times \Omega$ via:

$$ g(z,w) = \begin{cases} \frac{\varphi(z) - \varphi(w)}{z - w} & \text{ for } z \not = w \\ \varphi’(z) & \text{ else } \end{cases} $$

is continuous. 

Thus, we may find a neighborhood $V$ of $z_0$ such that:

$$|g(z_1,z_2) - g(z_0,z_0) | < \frac{1}{2}  | \varphi’(z_0) | $$

due to continuity, for all $z_1,z_2 \in V$.

In particular then, we would have that:

$$\left|\frac{\varphi(z_1) - \varphi(z_2)}{z_1- z_2}  - \varphi’(z_0) \right| < \frac{1}{2}  | \varphi’(z_0) | $$

Hence, we would have that, via the reverse triangle inequality:

$$| \varphi(z_1) - \varphi(z_2)| \geq | \varphi’(z_0)| | z_1 - z_2| $$

which holds on $z_1 \not = z_2$, from the original formulation. But, we see that this expression also holds on $z_1 = z_2$.

It follows from this that $\varphi$ is one-to-one, as suppose not. Then, we have $z_1 \not = z_2$, but $\varphi(z_1) = \varphi(z_2)$, and we can see that this inequality would not hold, as $\varphi’(z_0) \not = 0$.

We also notice that if we instead restrict on $z_1 = z_2$, that we can also make the statement:

$$| \varphi’(z) | \geq \frac{1}{2} | \varphi’(z_0) | $$ for $z \in V$.

Now, we wish to show that $\varphi(V)$ is open. 

Let $a \in V$, and choose $r > 0$ such that $\overline{D}(a,r) \subset V$.

Claim: $D(\varphi(a), c) \subset \varphi(V)$.

By the inequality $ | \varphi(z_1) - \varphi(z_2)| \geq | \varphi’(z_0)| | z_1 - z_2| $, we have that:

$$ | \varphi(a + re^{i\theta}) - \varphi(a)| \geq | \varphi’(z_0)| r > 2c $$

since $r$ is some fixed constant, $|\varphi’(z_0)|$, we can find some $c > 0$ such that this is true.

Now, consider a $\lambda$ such that $| \lambda - \varphi(a) | < c$.

Because $| \lambda - \varphi(a) | < c$, and $|\lambda - \varphi(a + r e^{i\theta}) > c$, by the corollary to the maximum modulus theorem, $\lambda - \varphi(z)$ must have a 0 on $D(a,r)$ as suppose not. Then, the minimum of the function has to occur on the boundary. But, we have that $| \lambda -\varphi(a) | < c$, and $\lambda - \varphi(a)$ does not belong to the boundary. Thus, $\lambda - \varphi(z)$ attains a 0. 

Thus, $\lambda \in D(\varphi(a), c)$ for all $\lambda$, and thus $D(\varphi(a), c) \subset \varphi(V)$. Since this can be done for all $a$, we must have that $\varphi(V)$ is open. 

Now, let $W = \varphi(V)$, and consider the inverse map $\psi: W \to V$. Since $\phi$ is one-to-one, onto, and $W$ is open, and inverse map exists. 

Now, we wish to show that $\psi \in \calH(W)$. Fix some $w_1 \in W$. Then, we may find a $z_1 \in V$ such that $\varphi(z_1) = w_1$.

Now, let $w \in W$, with corresponding point $z = \psi(w)$. Then, we have that:

$$ \frac{\psi(w) - \psi(w_1)}{w - w_1} = \frac{z - z_1}{\varphi(z) - \varphi(z_1)}$$

Again, from $ | \varphi(z_1) - \varphi(z_2)| \geq | \varphi’(z_0)| | z_1 - z_2| $, we have then that as $\varphi(z) \to \varphi(z_1)$, we have that $z \to z_1$. Equivalently, in terms of $w$, we have that as $w \to w_1$, that $z \to z_1$. 

Then, we have that:

$$\lim_{w \to w_1} \frac{\psi(w) - \psi(w_1)}{w - w_1} = \lim{z \to z_1} \frac{z - z_1}{\varphi(z) - \varphi(z_1)} = \frac{1}{\varphi’(z_1)}$$

Since $\varphi’(z_0) \not = 0$ everywhere, this is well-defined everywhere, and thus $\psi$ is holomorphic on $W$.

Definition [10.31]:

For $m = 1,2,3...$, we denote the $m$-th power function $z \to z^m$ by $\pi_m$. 

Note that of course, $\pi_m’(z) = m z^{m-1} = m\pi_{m-1}$, and if $z \not = 0$, then $\pi_m$ must be an open map by 10.30. But also, at $z = 0$, we have that $\pi_m(D(0,r)) = D(0,r^m)$. Thus, $\pi_m$ is an open map everywhere.

Theorem [10.32]:

Suppose $\Omega$ is a region, $f \in \calH(\Omega)$, $f$ not constant, $z_0 \in \Omega$, and $w_0 = f(z_0)$. Let $m$ be the order of the zero of $f - w_0$ at $z_0$.

Then, there exists a neighborhood $V \subset \Omega$ of $z_0$, with $\varphi \in \calH(V)$, such that:

(a) $f(z) = w_0 + [\varphi(z)]^m$ for all $z \in V$

(b) $\varphi’$ has no 0 on $V$, and $\varphi$ is an invertible mapping of $V$ onto a disk $D(0,r)$.

Proof:

Since $f(z)  - w_0$ has a zero at $z_0$, by 10.18, there is an $h \in \calH(\Omega)$ such that:

$$ f(z) - w_0 = (z - z_0)^m h(z) $$

with $h(z_0) \not = 0$.

Thus, we can find a disk $D(z_0, s)$ such that $h(z) \not =0$ on all of the disk, due to continuity.

In particular, we have that $\frac{h’(z)}{h(z)}$ is holomorphic on $D(z_0,s)$. Since $D(z_0, s)$ is convex, $h’/h$ has an antiderivative. Call $\psi \in \calH(D(z_0,s))$ such that:

$$\psi’ = \frac{h’}{h} \implies h’(z) - h(z) \psi’(z) = 0 \implies h’(z) e^{-\psi(z)} - h(z) \psi’(z)e^{-\psi(z)}  = 0 \implies \frac{d}{dz} \left(h(z) e^{-\psi(z)}\right) = 0$$

Then, since the derivative is 0 on a disk, it must be constant on the disk, so we can say that:

$$h(z) e^{-\psi(z)} = C \implies h(z) = C e^{\psi(z)} $$

We can rewrite $c = e^{c_0}$, and get that:

$$h(z) = e^{c_0+ \psi(z)}$$

replacing $\psi(z)$ with $c_0 + \psi(z)$, we can choose $h(z) = e^{\psi(z)}$.

Thus, we have that:

$$ f(z) - w_0 = (z - z_0)^m h(z) = (z - z_0)^m e^{\psi(z)} = \left((z - z_0) e^{\psi(z)/m}\right)^m$$

So we see that we can identify $\varphi(z) = (z - z_0) e^{\psi(z)/m}$. We notice that $\varphi$ is holomorphic on our disk, and we have that $f(z) - w_0 = \varphi^m(z)$. Notice that:

$$\varphi’(z) = e^{\psi(z)/m} + (z - z_0) \frac{\psi’(z)}{m} e^{\psi(z)/m}$$

so specifically, at $z_0$, we have that $\varphi’(z_0) =  e^{\psi(z_0)/m} \not =0$.

Thus, we have that, by the inverse function theorem [10.30], when restricted to some neighborhood $V$, must map onto, 1:1 to a disk, since we can always find a disk in a neighborhood of the image of $z_0$ under $\varphi$, which is $\varphi(z_0) = (z_0 - z_0) e^{\psi(z_0)/m} = 0$. 

Theorem [10.33]:

Suppose $\Omega$ is a region, $f\in \calH(\Omega)$, and $f$ is one-to-one on $\Omega$. Then, $f’(z) \not = 0$ for every $z \in \Omega$, and the inverse of $f$ is holomorphic.

Proof: Suppose $f’(z_0) = 0$ for some $z_0 \in \Omega$. Let $w_0 = f(z_0)$. Then, we have that:

$$ f(z) - w_0 = f’(z_0) (z - z_0) + .... = (z - z_0)^m h(z) $$.

where we have that $m > 1$ and $h(z_0) \not = 0$ by 10.18. We notice the first term vanishes since the derivative is 0. The previous theorem tells us that we can rewrite $f = w_0 + \varphi^m$ on some neighborhood $V$ of $z_0$. We can assume $V \subset \Omega$. Then, $f$ is $m$ to one on $V \setminus \{ z_0 \}$, a contradiction. Thus we have that $f’(z) \not = 0$ on all of $\Omega$. 

Calculus of Paths, Chains, and Cycles:

Suppose we have $\gamma_1,\gamma_2,...,\gamma_n$ be paths in the plane, and let $K = \cup \gamma_1^*$. 

Each $\gamma_k$ gives rise to a linear functional $\tilde{\gamma_k}$ on $C(k)$ given by:

$$\tilde{\gamma_k} = \int_{\gamma_k^*} f(z) dz$$

Define $\tilde{\Gamma} = \tilde{\gamma_1} + .... + \tilde{\gamma_n}$.

Then, we can introduce a formal sum $\Gamma = \gamma_1 + ... + \gamma_n$, and we may define:

$$\int_{\Gamma} f(z) dz = \tilde{\Gamma}(f)$$

where explicitly, this is just the sum of integrals.

Call $\Gamma$, the formal sum of paths, a chain. Call a chain a cycle if each $\gamma_i$ is a closed path.

\section*{Feb 15th}

Now, if each $\gamma_k$ is a path in a specific open set $\Omega$, then we say that $\Gamma$ is a chain in $\Omega$.

And, of course, $\Gamma^* = \cup_{i=1}^n \gamma_i^*$

Now, suppose $\Gamma$ is a cycle, with $\alpha \not \in \Gamma^*$.

We define then that:

$$\ind_\Gamma \alpha = \sum_i \ind_{\gamma_i}^n \alpha$$

Of course, if every $\gamma_k$ is replaced by its opposite, that is, the path traversed backwards, we denote the chain by $-\Gamma$, and:

$$ \int_{-\Gamma} f(z) dz = - \int_{\Gamma} f(z) dz $$

In particular then, this implies that:

$$\ind_\Gamma \alpha = - \int_{-\Gamma} \alpha$$

Chains may be added and subtracted in the obvious way.

In particular, the statement $\Gamma = \Gamma_1 + \Gamma_2$ is equivalent to saying:

$$\int_\Gamma f(z) dz = \int_{\Gamma_1} f(z) dz + \int_{\Gamma_2} f(z) dz$$

Finally, we notice that of course, there may be multiple ways to express a specific chain in terms of paths. Then, we say that if:

$$ \gamma_1 + ... + \gamma_m = \eta_1 + ... + \eta_n$$

then, this is equivalent on saying:

$$\sum_{i=1}^m \int_{\gamma_i} f(z) dz = \sum_{j=1}^n \int_{\eta_i} f(z) dz $$

Theorem [10.35 Cauchy’s Theorem]:

Let $f \in \calH(\Omega)$, where $\Omega$ is an open set in the complex plane. Let $\Gamma$ be a cycle in $\Omega$ such that:

$$ \ind_\Gamma \alpha = 0 $$

for all $\alpha \not \in \Omega$.

Then, we have that for $z \in \Omega \setminus \Gamma^*$:

$$ f(z) \ind_\Gamma(z) = \frac{1}{2\pi i} \int_\Gamma \frac{f(w)}{w - z} dw $$

We call this the Cauchy Integral Formula. Further, we have that:

$$\int_\Gamma f(z) dz = 0$$

If $\Gamma_0, \Gamma_1$ are cycles in $\Omega$ such that, for every $\alpha \not \in \Omega$:

$$\ind_{\Gamma_0} \alpha = \ind_{\Gamma_1} \alpha$$

then, we have that

$$\int_{\Gamma_0} f(z)dz = \int_{\Gamma_1} f(z) dz$$

Proof:

Recall that the function $g(z,w): \Omega \times \Omega \to \mathbb{C}$ given by:

$$ g(z,w) = \begin{cases} \frac{f(z) - f(w)}{z  - w} & \text{ if } w \not = z \\ f’(z) & \text{ if } w = z \end{cases} $$

is continuous on $\Omega \times \Omega$. Moreover, if we take compact subsets $K \subset \Omega \times \Omega$, $g$ is uniformly continuous. 

In particular, if $z_n \to z$ in $\Omega$, then $g(z_n,w) \to g(z,w)$ uniformly for all $w \in \Gamma^*$, since we may always take the compact subset $\{ (z,w) | z \in (z_i)_{i=1}^\infty, w \in \Gamma^* \}$.

Hence, the function:

$$ h(z) = \int_\Gamma g(z,w) dw$$ is continuous in $\Omega$. 

We wish to show that $h(z) = 0$ on all of $\Omega$. 

If $z \in \Omega \setminus \Gamma^*, h = 0$, then we would have that, by the definition of $g$, that:

$$ \frac{1}{2\pi i} \int_\Gamma \frac{f(z) - f(w)}{z - w} dw = 0$$

But, reexpressing the left hand side, we would have that:

$$\frac{1}{2\pi i} \int_\Gamma \frac{f(z) - f(w)}{z - w} dw = \frac{1}{2\pi i} \left(\int_\Gamma \frac{f(z)}{z - w} dw - \int_\Gamma \frac{f(w)}{z - w} dw\right) = - f(z) \ind_\Gamma z - \frac{1}{2\pi i} \int_\Gamma \frac{f(w)}{z - w} dw $$

Setting this equal to 0, we have that:

$$- f(z) \ind_\Gamma z - \frac{1}{2\pi i} \int_\Gamma \frac{f(w)}{z - w} dw = 0 \implies f(z) \ind_\Gamma z = \frac{1}{2\pi i} \int_\Gamma \frac{f(w)}{w - z} dw = 0$$

So, in summary, if $h(z) = 0$ everywhere, then we are done.

First, we wish to show that $h \in \calH(\Omega)$.

First, fix a $w \in \Omega$, we have that the function that sends $z \to g(z,w)$ is holomorphic, since the singularity at $z = w$ is removable. 

Now, let $\Delta \subset \Omega$ be a closed triangle. We have that:

$$ \int_{\partial \Delta} h(z) dz = \int_{\partial \Delta} \left( \int_\Gamma g(z,w) dw \right)dz  = \int_\Gamma \int_{\partial \Delta} g(z,w) dz dw = 0 $$

By Fubini’s which applies due to the uniform continuity of $g$ on $\Delta \times \Gamma$, a compact subset, and because $g$ is holomorphic, so $ \int_{\partial \Delta} g(z,w) dz = 0$.

Let $V$ be the open set $\{ z \in \mathbb{C} \setminus \Gamma^* : \ind_\Gamma z = 0 \}$. This must be open, because each region in $\mathbb{C} \setminus \Gamma ^*$ is open, and this is merely the union of valid regions. 

It should be clear that $\Omega^c \subseteq V$. Thus, $V \supset \mathbb{C} \setminus \Omega$.

Define $h_1$ on $V$ via:

$$h_1(z) =  \frac{1}{2\pi i} \int_\Gamma \frac{f(w)}{w - z} dw $$

We have that $h_1$ is analytic, thus $h_1 \in \calH(V)$.

We claim that $h = h_1$ on $\Omega \cap V$.

In fact, we have that, for $z \in \Omega \cap V$, that:

$$h(z) = \int_\Gamma g(z,w) dw = \int_\Gamma \frac{f(w) - f(z)}{w - z} dw = \int_\Gamma \frac{f(w)}{w - z} dw - f(z) \int_\Gamma \frac{1}{w - z} dw = \int_\Gamma \frac{f(w)}{w - z} dw = h_1(z) $$

because $\int_\Gamma \frac{1}{w - z} dw = \ind_\Gamma z = 0$ because $z \in V$. Thus, we have that they coincide on their intersection, so we can define a holomorphic function on all of $\mathbb{C}$ such that $H \in \calH(\Omega \cup V) = \calH(\mathbb{C})$, such that $H = h$ on $\Omega$, and $H = h_1$ on $V$.

Now, $V$ contains the unbounded component of $\mathbb{C} \setminus V^*$. On this component, we have that:

$$H(z) = h_1(z) = \int_\Gamma \frac{f(w)}{w - z} dw \to 0$$

as $| z| \to 0 $. Since $f(w)$ is on a compact set, therefore bounded, but $|w - z| \to \infty$ as $|z| \to \infty$.

Thus, by Liouville’s, $H$ is constant, and since this constant goes to 0 as $|z| \to \infty$, we must have that $H = 0$. This completes the first assertion. 

For the second assertion, fix a $z_0 \in \Omega \setminus \Gamma^*$, and let:

$$f_1(z) = f(z)(z - z_0)$$

Clearly, $f_1 \in \calH(\Omega)$, $f_1(z_0) = 0$. So, we have then that:

$$ 0 = f_1(z_0) \ind_\Gamma z = \frac{1}{2\pi i} \int_\Gamma \frac{f_1(z)}{z - z_0} dz $$

But, $f_1(z) = f(z) (z - z_0)$, so we have that:

$$ \int_\Gamma f(z) dz = 0$$

Finally, suppose that we have $\Gamma_1, \Gamma_2$, such that $\ind_{\Gamma_1} \alpha = \ind_{\Gamma_2} \alpha$ for all $\alpha \in \mathbb{C} \setminus \Omega$.

Consider $\Gamma = \Gamma_1 - \Gamma_0$. Clearly, we still have that $\ind_\Gamma \alpha = 0$ for the same $\alpha$.

Then,, we have that:

$$ 0 = \int_\Gamma f(z) dz = \int_{\Gamma_1} f(z) dz + \int_{-\Gamma_2} f(z) dz =  \int_{\Gamma_1} f(z) dz - \int_{\Gamma_2} f(z) dz$$

Hence:

$$ \int_{\Gamma_1} f(z) dz  = \int_{\Gamma_2} f(z) dz$$

Preview: we will look at homotopy and index calculations next.

\section*{Feb 22nd}

Theorem:

Let $\gamma_0, \gamma_1$ be two closed paths in $\mathbb{C}$ both with the parameter interval $[0,1]$. Let $\alpha \in \mathbb{C}$ such that

$$| \gamma_0(t) - \gamma_1(t) | < | \alpha - \gamma_1(t) | $$

for all $t \in [0,1]$.

Then, 

$$ \ind_{\gamma_0}(\alpha) = \ind_{\gamma_1}(\alpha) $$

Remark, we notice essentially that this means, as we travel along the paths at the same “rate”, that the paths are closer to each other than they are to $\alpha$, so the winding is the same.

Proof:

Clearly, from the above inequality, $\alpha \not \in \gamma_1^*$. But also, we must have that $\alpha \not \in \gamma_0^1$, since then we would have equality.

Thus, we notice that the index of $\alpha$ with respect to $\gamma_0, \gamma_1$ is well-defined.

Define the path $\eta: [0,1] \to \mathbb{C}$ via

$$\eta(t) = \frac{\gamma_0(t) - \alpha}{\gamma_1(t) - \alpha} $$

Note that for any $t$, that $| \eta(t)  - 1 | < 1 $. This should be clear:

$$ \eta(t) - 1 = \frac{\gamma_0(t) - \alpha}{\gamma_1(t) - \alpha} - \frac{\gamma_1(t) - \alpha}{\gamma_1(t) - \alpha} = \frac{\gamma_0(t) - \gamma_1(t)}{\gamma_1(t) - \alpha} $$

which has modulus less than 1 by hypothesis.

Thus, we have that $\eta^* \subset D(1,1)$. Therefore, we must have that $0$ belongs to the unbounded component of $\eta$, since $0 \in \mathbb{C} \setminus D(1,1)$. Thus, $\ind_\eta(0) = 0$.

Thus, we have that:

$$ \frac{1}{2\pi i} \int_{\eta} \frac{dz}{z} = 0 \implies  \int_{\eta} \frac{dz}{z} = 0 \implies \int_{0}^1 \frac{\eta’(t) dt}{\eta(t)} = 0$$

Now, computing $\eta’(t)$ and substituting:

$$  \int_{0}^1 \frac{\eta’(t) dt}{\eta(t)} = 0 \implies  \int_{0}^1 \frac{\gamma_0’(t)}{\gamma_0(t) - \alpha } - \frac{\gamma_1’(t)}{\gamma_1(t) - \alpha } dt = 0 \implies \int_0^1 \frac{\gamma_0’(t)}{\gamma_0(t) - \alpha }  =\int_0^1 \frac{\gamma_1’(t)}{\gamma_1t) - \alpha }  $$ 

Then, by the definition of the index, this implies that $\ind_{\gamma_0}(\alpha) = \ind_{\gamma_1}(\alpha)$.

Now, let’s talk about homotopy.

Let $\gamma_0, \gamma_1$ be two closed curves on a topological space $X$, with the same parameter interval, which we will standardize as $[0,1]$.

We call $\gamma_0, \gamma_1$ $X$-homotopic if there exists a continuous function $H: [0,1] \times [0,1] \to X$ such that $H(s,0) = \gamma_0(s), H(s,1) = \gamma_1(s), H(0,t) = H(1,t)$. 

Intuitively, we think of this as saying that we may continuously deform $\gamma_0 \to \gamma_1$.

If $\gamma_0$ is homotopic to the constant curve $\gamma_1(t) = \alpha$, then we say that $\gamma_0$ is null-homotopic.

If $X$ is connected, and every curve in $X$ is null-homotopic, then we call $X$ simply connected.

Example, every convex region in the complex plane is simply connected.

This should be clear, let $\alpha$ be any point, let $\gamma$ be any curve, define $H(s,t) = (1 - t) \gamma(s) + t \alpha$.

Objective:

If $\Gamma_0, \Gamma_1$ are $\Omega$-homotopic closed paths on a region $\Omega \in \mathbb{C}$, then:

$$\ind_{\Gamma_0} \alpha = \ind_{\Gamma_1} \alpha$$ for every $\alpha \not \in \Omega$.

This would mean then, if $f \in \calH(\Omega)$, that $\int_{\Gamma_0} f(z) dz = \int_{\Gamma_1} f(z)dz $.

Theorem:

If $\Gamma_0, \Gamma_1$ are $\Omega$-homotopic closed paths in a region $\Omega \subset \mathbb{C}$. If $\alpha \not \in \Omega$, then we have that $\ind_{\Gamma_0} \alpha = \ind_{\Gamma_1} \alpha$.

Proof sketch:

We know that there exists a continuous map $H: [0,1] \times [0,1] \to \Omega$ such that $H(s,0) = \gamma_0(s), H(s,1) = \gamma_1(s), H(0,t) = H(1,t)$.

Since $H$ has a compact domain, $H$ is actually uniformly continuous. But also, the continuous function $H(s,t) - \alpha$ is also uniformly continuous. Further, since $H$’s domain is compact, and $H$ is continuous, $H$’s image is compact as well. Then, there exists $\epsilon > 0$ such that $|H(s,t) - \alpha| > 2\epsilon$ for all $s,t$.

From the uniform continuity, there exists $\delta > 0$ such that:

$$ |H(s,t) - H(s’,t’)| < \epsilon $$ when $\Vert (s,t) - (s’,t’) \Vert < \delta$.

Introduce a grid on $[0,1] \times [0,1]$ such that the diagonal of each square in the grid is less than $\delta$.

Now, take any two points in the same subsquare in the grid. Then, we notice that 

$$|H(s,t) - H(s’,t’)| < \epsilon < 2\epsilon < |H(s,t) - \alpha|  $$

and by the last lemma, we try to patch this together. However, there’s a technical issue here: for each $t_0$, $H(s,t_0)$ need not be a path (i.e. differentiable at all but finitely many points.).

The solution here is to instead make them piecewise linear, that is, take equally spaced $1/n$ points on the unit square at a fixed $t$, and interpolate. Then, we patch like this anyway.

Corollary:

If $\Gamma_0$ is a closed path in $\Omega$, null-homotopic, then:

$$\int_{\Gamma_0} f(z) dz = 0$$ for all $f \in \calH(\Omega)$.

The Calculus of Residues:

First, we want to define a meromorphic function.

Let $\Omega$ be a region. We call a function $f$ meromorphic in $\Omega$ if there exists a (potentially empty) set $A \subset \Omega$ such that the following conditions hold:

(1) $A$ has no limit points in $\Omega$

(2) $f$ is holomorphic on $\Omega \setminus A$

(3) $f$ has a pole at each point in $A$.

Examples:

Rational functions of polynomials are meromorphic.

For a holomorphic, non-constant function $\varphi(z)$, $\frac{1}{\varphi(z)}$ is meromorphic, since the zeros of $\varphi(z)$ have no limit points, and the fraction may only have poles at these zeros. In particular, the order of the pole of $\frac{1}{\varphi(z)}$ is the same as the order of the zero of $\varphi(z)$ at each point $a$.

Recall: we say that $f$ has a pole at a point $a$ if there exists an $m \in \mathbb{N}$ and constants $c_1,...,c_m \in \mathbb{C}$ such that:

$$ f(z) - \sum_{n=1}^m c_n (z - a)^n $$ 

has a removable singularity at $a$. In particular, we call the residue of $f$ at $a$:

$$ \text{Res}(f,a) = c_1$$

Why do we care? Well, note that if $\Gamma$ is any cycle in $\Omega$, $a \not \in \Gamma^*$, then:

$$\int_\Gamma \sum_{n=1}^m c_n (z - a)^n  = \int_\Gamma c_1 (z-a)^{-1} = 2 \pi i c_1 \ind_\Gamma(a)  = 2\pi i \text{Res}(f,a) \ind_\Gamma(a)$$

\section*{Feb 23rd}

Definition:

Let $f$ be a complex valued function. We call $a \in \mathbb{C}$ a pole of $f$ if $f$ has a singularity at $a$ and there exists a positive integer $m$, with constants $c_1,...,c_m, c_m \not = 0$ such that:

$$ f - \sum_{k=1}^m c_k (z-a)^{-k} $$

has a removable singularity.

Define $Q(z) = \sum_{k=1}^m c_k (z-a)^{-k}$. We denote this as the principal part of $f$ at $a$, and we call $c_1$ the residue of $f$ at $a$, denoted $\res(f,a)$

Why do we call this the principal part? Well, if $f - Q$ has a removable singularity, it is holomorphic, and thus we can write a power series:

$$f = Q + \sum_{k=0}^\infty d_k (z-a)^k$$, which is valid on a punctured disk $D’(a,r)$.

Now, if $\Gamma$ is a cycle on the domain of $f$, then we have that:

$$ \int_\Gamma Q = \int_\Gamma  \sum_{k=1}^m c_k (z-a)^{-k} dz = \int_\Gamma c_1 (z -a)^{-1} dz  = 2 \pi i \ind_\Gamma(a) \res(f,a)$$

And we can move the $2 \pi i$ to the other side if we want to.

Now, we can extend this to the concept of meromorphic functions and the residue theorem:

Theorem [Residue Theorem]:

Suppose that $f$ is meromorphic on $\Omega$, with poles at $A \subset \Omega$. Let $\Gamma$ be a cycle in $\Omega \setminus A$ such that $\ind_\Gamma(\alpha) = 0$ for all $\alpha \in \Omega^c$. 

Then, we have that:

$$\frac{1}{2\pi i} \int_\Gamma f(z) dz = \sum_{a \in A} \res(f,a) \cdot \ind_\Gamma(a)$$

Remark: it turns out, from the proof of the theorem, that although $A$ may be countably infinite, the sum over $a \in A$ must be a finite sum.

Proof:

Let $B  =   \{ a \in A : \ind_\Gamma(a) \not = 0 \}$. We claim that $B$ has only finitely many points. Note that $B$ is bounded, as points in the unbounded component of $\mathbb{C} \setminus \Gamma^*$ have index 0. 

Now, suppose $B$ is infinite. Then, because $B$ is bounded, infinite, we can produce a sequence of points $\{ b_n \}$, and due to compactness, this must have a convergent subsequence, hence, a limit point. Call this limit point $z_0$. However, by hypothesis, $A$ has no limit points, thus $z_0 \not \in \Omega$. But then, $\ind_\Gamma(z_0) = 0$. However, the set $\{ z \in \mathbb{C} \setminus \Gamma^* : \ind_\Gamma(z) = 0 \}$ is open. Then, there should be an $r > 0 $ such that for all $z \in D(z_0,r)$, $\ind_\Gamma(z) = 0$. However, we’ve found a convergent subsequence $b_{n_k} \to z_0$, and $\ind_\Gamma(b_{n_k}) \not = 0$. Contradiction.

Thus, $B$ must be finite.

Now, define a new set $\Omega’ = \Omega \setminus (A \setminus B)$. Then $\Omega’ $ is open, and $\int_\Gamma(\alpha) = 0$  still for $\alpha \in \mathbb{C} \setminus \Omega’$. 

Let $\alpha_1,...,\alpha_n$ be the elements of $B$. For each $j = 1,...,n$, let $Q_j(z)$ be the principal part of $f$ at the point $\alpha_j$. Then, $g(z) = f(z) - \sum_{k=1}^n Q_k(z)$ is holomorphic on $\Omega’$, since it must have removable singularities at each $\alpha_j$. 

Now, by Cauchy’s theorem, for $g \in \calH(\Omega’)$, we have that:

$$ \int_\Gamma f(z) dz = \int_\Gamma g(z) + \sum_{k=1}^n Q_k(z) dz = \int_\Gamma g(z) + \sum_{k=1}^n \int_\Gamma Q_k(z) dz =$$
$$ \sum_{k=1}^n \int_\Gamma Q_k(z) dz =2 \pi i  \sum_{j=1}^n \res(f,\alpha_j) \cdot \ind_\Gamma(\alpha_j)$$

But, this is exactly 

$$2 \pi i  \sum_{j=1}^n \res(f,\alpha_j) \cdot \ind_\Gamma(\alpha_j) = 2 \pi i  \sum_{a \in B} \res(f,a) \cdot \ind_\Gamma(a) =   2 \pi i  \sum_{a \in A} \res(f,a) \cdot \ind_\Gamma(a)$$

since the index is 0 for $a \in A \setminus B$.

Counting the Zeros of a Holomorphic Function

Theorem:

Suppose $\gamma$ is a closed path in a region $\Omega$ such that $\ind_\gamma(\alpha) = 0$ for all $\alpha \in \mathbb{C} \setminus \Omega$. Suppose further that $\ind_\gamma(\alpha) = 0, 1$ for every $\alpha \in \Omega \setminus \gamma^*$.

Define $\Omega_1 = \{ \alpha : \ind_\gamma(\alpha) = 1 \}$. For every $f \in \calH(\Omega)$, denote the zeros of $f$ in $\Omega_1$ as $N_f$, counting multiplicities. Suppose that $f$ has no zeros in $\gamma^*$.

Then, we have that $N_f = \frac{1}{2\pi i} \int_\Gamma \frac{f’(z)}{f(z)} dz = \ind_\Gamma(0)$, where $\Gamma(t) = f(\gamma(t))$

(Rouche’s Theorem) Further, if $g \in \calH(\Omega)$, and $| f(z) - g(z) | < |f(z)|$ for all $z \in \gamma^*$, then $N_f = N_g$.

Proof:

Let $A = \{ a \in \Omega : f(a) = 0 \}$. Then, of course, $A \cap \gamma^*$ is empty, since $f$ has no zeros on $\gamma^*$. Further, on $f’/f$, the singularities are precisely the points of $A$. Let $a \in A$, and let $m_a$ be the order of the zero of $f$ at $a$. Then, by definition, we have that $f(z) = (z-a)^{m_a} g(z)$, for some $g \in \calH(\Omega)$. Then:

$$\frac{f’(z)}{f(z)} = \frac{f’(z)}{(z-a)^{m_a} g(z)} $$

and thus $f’/f$ has a pole of order at most $m_a$ at $a$. In fact, by the product rule, we have that:

$$ \frac{f’(z)}{(z-a)^{m_a} g(z)} = \frac{m(z-a)^{m_a-1}g(z) + (z-a)^{m_a} g’(z)}{(z-a)^{m_a} g(z)}   = \frac{m_a}{z-a} + \frac{g’(z)}{g(z)} $$

where we notice, that because $g(z)$ cannot vanish on a neighborhood of $a$, $g’/g$ must be holomorphic on such a neighborhood. Thus, $f’/f$ has a simple pole at $a$, with $m_a$ as the residue. 

Thus, by the residue theorem, we have that:

$$\frac{1}{2\pi i} \int_\gamma  \frac{f’(z)}{f(z)} dz = \sum_{a \in A} \res(\frac{f’}{f}, a), a) \cdot \ind_\gamma(a) = \sum_{a \in (A\cap \Omega_1} \res(\frac{f’}{f}, a), a) \cdot \ind_\gamma(a)=$$
$$\sum_{a \in (A\cap \Omega_1} \res(\frac{f’}{f}, a), a) = \sum_{a \in (A\cap \Omega_1} m_a = N_f$$

Now, on the other hand, we have that:

$$\frac{1}{2\pi i}\int_\gamma  \frac{f’(z)}{f(z)} dz = \frac{1}{2\pi i}\int_a^b  \frac{f’(\gamma(t))}{f(\gamma(t))} dt = \frac{1}{2\pi i}\int_a^b  \frac{\Gamma’(t))}{\Gamma(t) - 0} dt = \ind_\Gamma(0) $$

Now, for Rouche’s Theorem:

Suppose that we have $g$ holomorphic, and $| f(z) - g(z) | < |f(z)|$ for all $z \in \gamma^*$. Define $\Gamma_0(t) = g(\gamma(t))$, for $a \leq t \leq b$. Then, by our inequality, we must have that:

$$ | \Gamma(t) - \Gamma_0(t) | < | \Gamma(t) |  = |\Gamma(t) - 0|$$

for all $t$ in our parameter interval. But, by our theorem, we have then that $\ind_{\Gamma}(0) = \ind_{\Gamma_0}(0)$, and thus, $N_f = N_g$.

\section*{Feb 27th}

Harmonic Functions:

Firstly, let’s talk about the Cauchy-Riemann equations.

Let $f: \Omega \to \mathbb{C}$ be a complex valued function, and let $f(z) = f(x + iy) = u(x,y) + iv(x,y)$, that is, we imagine this as a transformation from $\mathbb{R}^2 \supset \Omega \to \mathbb{R}^2$.

Thus, we make the identification $z = x + iy = (x,y) = \begin{pmatrix} x \\ y \end{pmatrix}$

Fix a $z_0 = (x_0, y_0)$, and assume $f$ is differentiable as a transformation at $z_0$. That is, we have that:

$$ u_x, u_y, v_x, v_y $$ exist at $z_0$, and that the following holds:

$$ \lim_{(\Delta x, \Delta y) \to (0,0)} \frac{1}{\sqrt{x^2 + y^2}} \begin{pmatrix} u(x_0 + \Delta x, y_0 + \Delta y) \\ v(x_0 + \Delta x, y_0 + \Delta y) \end{pmatrix} - \begin{pmatrix} u(x_0, y_0) \\ v(x_0, y_0) \end{pmatrix}  - $$
$$\begin{pmatrix} u_x(x_0,y_0) & u_y(x_0, y_0) \\ v_x(x_0, y_0) & v_y(x_0, y_0) \end{pmatrix} \begin{pmatrix} \Delta x \\ \Delta y \end{pmatrix} = \begin{pmatrix} 0 \\0 \end{pmatrix} $$

But, this is equivalent to:

$$ \lim_{\Delta z \to 0 }\frac{1}{\Delta z} \{(u(x_0 + \Delta x, y_0 + \Delta y) + i v(x_0 + \Delta x, y_0 + \Delta y) ) - (u(x_0, y_0) + i v(x_0, y_0)) - $$
$$[(u_x(x_0, y_0) \Delta x + u_y(x_0, y_0) \Delta y) + i (v_x(x_0, y_0) \Delta x + v_y(x_0,y_0) \Delta y)] \}= 0$$

Introducing 
$$\partial = \frac{1}{2} \left(\frac{\partial}{\partial x} - i \frac{\partial}{\partial y}\right)$$
$$\overline{\partial} = \frac{1}{2} \left(\frac{\partial}{\partial x} + i \frac{\partial}{\partial y}\right)$$, we have then that:

$$\lim_{\Delta z \to 0 }\frac{1}{\Delta z} [ f(z_0 + \Delta z) - f(z_0) - \partial f(z_0) \Delta z - \overline{\partial} f(z_0) \overline{\Delta z}] $$

which is equivalent then to:

$$ \lim_{\Delta z \to 0 } \left[ \frac{ f(z_0 + \Delta z)}{\Delta z} - \partial f(z_0) - \overline{\partial} f(z_0) \frac{\overline{\Delta z}}{\Delta z} \right] = 0 $$

We claim that $f$ is differentiable then if and only if $\overline{\partial} f(z_0) = 0$. 

Well, we notice that if $\Delta z = \Delta x$, then $ \frac{\overline{\Delta z}}{\Delta z} = 1$ and if $\Delta z = i \Delta y$, then $ \frac{\overline{\Delta z}}{\Delta z} = -1$.

Proof:

Suppose $f’(z_0)$ exists. Then, we have that the limit:

$$ \lim_{\Delta z \to 0 } \overline{\partial} f(z_0) \frac{\overline{\Delta z}}{\Delta z} = 0$$

exists, since we have that each piece of the limit above exists. But, since $\overline{\partial} f(z_0)$ is a constant, we must have that

$$\overline{\partial} f(z_0) \lim_{\Delta z \to 0 }  \frac{\overline{\Delta z}}{\Delta z} = 0$$

But by our remark, that limit does not exist unless $\overline{\partial} f(z_0) = 0$.

Now, suppose $\overline{\partial} f(z_0) = 0$. Then, by the big limit, we have that:

$$ \lim_{\Delta z \to 0 } \left[ \frac{ f(z_0 + \Delta z)}{\Delta z} - \partial f(z_0)  \right]  = 0 \implies f’(z_0) = \partial f(z_0) $$

Thus, $f’(z_0)$ exists, and takes on the value $\partial f(z_0)$. 

Theorem:

Suppose that $f: \Omega \to \mathbb{C}$, $f(x,y) = u(x,y) + i v(x,y)$, and that $f$ is a differentiable transformation from $\Omega \to \mathbb{R}^2$. Then we have that $f \in \calH(\Omega) \iff \overline{\partial}f(z) = 0$. Moreover, if this holds, then we have that $f’(z) = \partial f(z)$ for all $z  \in \Omega$. 

In particular, if $\overline{\partial}f(z) = 0$, then we have:

$$\overline{\partial}f(z) = 0 \implies  \left(\frac{\partial}{\partial x} + i \frac{\partial}{\partial y}\right) (u + iv) = 0 \implies $$

$$ (u_x - v_y) + i (u_y + v_x) = 0 \implies u_x = v_y, u_y = -v_x $$

We call these the Cauchy-Riemann equation, that is, either a functional that is 0, or first-order differentiable functions that are equal.

Further, if $f \in \calH(\Omega)$ looking at $f’(z) = \partial f(z)$:

$$ \partial f(z) = \frac{1}{2} \left[ (u_x + v_y)  + i (v_x - u_y) \right] = u_x + i v_x =  v_y - i u_y $$

Now, since $f$ is holomorphic, so is $f’$. Thus, we have that $u_x, u_y, v_x, v_y$ must be continuous on $\Omega$ as well. Repeating this argument on $f’$, we may say that $u,v$ have continuous partials of second order as well. By induction then, we have that $u,v \in C^{\infty} (\Omega)$. 

Let’s now talk about the Laplacian (on $\mathbb{R}^2)$:

This is the second order differential operator:

$$\Delta = \frac{\partial^2}{\partial x^2} +  \frac{\partial^2}{\partial y^2}$$

We notice that, in terms of $\partial, \overline{\partial}$ that we defined earlier, that we have:

$$ \Delta = 4 \partial \overline{\partial} = 4 \overline{\partial} \partial $$

Well, recall that if $f \in \calH(\Omega)$, then $\overline{\partial} f = 0$. Thus, we would have that $\Delta f = 0$ if $f$ were holomorphic. Then, we have that:

$$ \Delta (u + i v) = \Delta u + i \Delta v \implies \Delta u = 0, \Delta v = 0$$

This motivates the following definition:

Definition:

Suppose that $f \in C^2(\Omega)$. We call $f$ harmonic on $\Omega$ if $\Delta f = 0$ on $\Omega$.

Then, we can say that $f$ is holomorphic if, for $f = u + iv$, that $u,v$ are both harmonic. 

Objective:

We want to show that if $u(x,y)$ is a real-valued harmonic on a region $\Omega \subset \mathbb{R}^2$, then $u$ is locally (say, a neighborhood of $\Omega$) the real part of a holomorphic function.

Let’s first set some notation. Let $U$ now denote $D(0,1) = \{ z : |z| < 1 \}$, that is, the open unit disk. Let $T$ denote $\overline{D}(0,1) = \{ z : |z| = 1 \}$, that is, the unit circle. 

The Poisson kernel:

Let $ r \in \mathbb{R}, 0 \leq r < 1$. Define the function $P_r: \mathbb{R} \to \mathbb{R}$:

$$P_r(t) = \sum_{n = -\infty}^\infty r^{|n|} e^{i n t} = 1 + \sum_{n=1}^\infty r^n e^{i n t} + \sum_{n=1}^{\infty} r^{n} e^{-int} = 1 + \sum_{n=1}^\infty r^n(e^{int} + e^{-int}) = 1 + 2\Re \left( \sum_{n=1}^\infty r^n e^{int} \right) =$$

$$ 1 + 2\Re \left( \frac{r e^{it}}{1 - r e^{it}} \right) = \Re \left( 1 + 2 \frac{r e^{it}}{1 - r e^{it}} \right) = \Re \left(  \frac{1 + re^{it}}{1 - re^{it}} \right) = \frac{1 - r^2}{1 + r^2 - 2r\cos(t)}$$

where we get the last expression by analyzing $$ \frac{1 + re^{it}}{1 - re^{it}} \frac{1 - re^{-it}}{1 - re^{-it}}$$ and its real part.

Now, suppose $z = r e^{i \theta} \in U$, that is, $ 0 \leq r < 1$. Then, we have that:

$$ P_r(\theta -t) = \Re \left( \frac{1 + re^{i(\theta - t)}}{1 - re^{i(\theta - t)}} \right) = \Re \left( \frac{e^{it} + re^{i\theta}}{e^{it}- re^{i\theta}} \right) =  \Re \left( \frac{e^{it} + z}{e^{it}- z} \right)$$

Looking at that expression, we see that:

$$ \frac{e^{it} + z}{e^{it}- z} \frac{e^{-it} - \overline{z}}{e^{-it}- \overline{z} } = \frac{1 - |z|^2 + (ze^{it} - \overline{z} e^{it})}{|e^{it} - z|^2}$$

and we identify $ze^{it} - \overline{z} e^{it}$ as the difference of conjugates, and thus purely imaginary.

Thus, 

$$ P_r(\theta -t) = \frac{1 - |z|^2}{|e^{it} - z|^2}$$

which is sometimes denoted as $P(z,e^{it})$.

Ok, why do we care?

Well, recall

$$P_r(t) = \frac{1 - r^2}{1 + r^2 - 2r\cos(t)}$$

(i) $1 - r^2 > 0$ since $0 \leq r < 1$. Further, $1 + r^2 - 2r\cos(t) \geq 1 + r^2 - 2r  = (1-r)^2 > 0$. Thus, we have that $P_r(t)$ is strictly positive. 

(ii) We see that $P_r(t)$ is maximized when the denominator is minimized, i.e when $\cos(t) = 1 \iff t = 0$. Thus, we have that:

$$ P_r(t) \leq P_r(0) = \frac{1-r^2}{(1-r)^2} = \frac{1+r}{1-r}$$

(iii) On $[0,\pi]$, it should be clear that because $\cos$ is decreasing on that interval, $1 + r^2 - 2r\cos(t)$ is increasing, and so $P_r(t)$ is decreasing. We can see then that the minimum is at $t = \pi$, which has value $P_r(\pi) = \frac{1 - r}{1+r}$.

(iv) Since $\cos(t)$ is even, $P_r(t)$ must be an even function. Also periodic of $2 \pi$, for the same reason.

(v) If we fix a $0 < \delta \leq \pi$, then $P_r(\delta) = \frac{1 - r^2}{1 + r^2 - 2r \cos(\delta)}$. If we take the limit as $r \to 1$, then $P_r(t) \to 0$. 

(vi) $\frac{1}{2\pi} \int_0^{2\pi} P_r(t) dt = 1$.

Proof:

$$\frac{1}{2\pi} \int_0^{2\pi} P_r(t) dt = \frac{1}{2\pi} \int_0^{2\pi} \left( \sum_{n=-\infty}^\infty r^{|n|} e^{int} \right) dt$$

By the Weierstrauss M-test, we may interchange the sum and integral, and we notice that when $n \not = 0$, the integral is 0. Thus, this is exactly:

$$ \frac{1}{2\pi} \int_0^{2\pi} 1 = 1$$

\section*{March 1st}

Recall from last time, our discussion on the Poisson kernel, for $0 \leq r < 1$:

$$P_r(t) = \sum_{n=-\infty}^\infty r^{|n|} e^{int} = \frac{1 - r^2}{1 + r^2 - 2r \cos(t)} = \Re \left(\frac{e^{-it} + r}{e^{-it} - r}\right)$$

Further, we had if $z = re^{i\theta}$, that:

$$P_r(\theta - t) = \frac{1 - r^2}{1 + r^2 - 2r \cos(\theta - t)} = \Re \left(\frac{e^{it} + z}{e^{it} - z}\right) = \frac{1 - |z|^2}{|e^{it} - z|^2}$$

We also denote sometimes, $P_r(\theta - t) = P(z,e^{it})$. Also recall the 7 properties above. 

So now, we are able to talk about a Poisson transform. Let $f \in L^1(T)$, that is, $\int_{T} |f(z)| dz < \infty$.

We define the function $P(f)$ on $U$ by:

$$P(f) (re^{i\theta}) = \frac{1}{2\pi} \int_0^{2\pi} P_r(\theta - t) f(e^{it}) dt  = \frac{1}{2\pi} \int_0^{2\pi} P_r(z, e^{it} ) f(e^{it}) dt = $$

$$ \frac{1}{2\pi} \int_0^{2\pi} \frac{1 - |z|^2}{|e^{it} - z|^2} f(e^{it}) dt $$ 

where each of these are the same, due to properties we have on $P_r( \theta - t)$.

But, does this integral exist, and where?

Well, we have that $\frac{1 - |z|^2}{|e^{it} - z|^2} $ is continuous with respect to $t$, and since $U$ is contained within a compact set, we can say that it is bounded on $U$. Further, $f$ is $L^1$, thus, the integral is bounded above by the maximum of the first expression times the $L^1$ integral of $f$, and the expression is $L^1(T)$.

Let’s now analyze the first equation a bit more:

$$P(f) (re^{i\theta}) = \frac{1}{2\pi} \int_0^{2\pi} P_r(\theta - t) f(e^{it}) dt  = \frac{1}{2\pi} \int_0^{2\pi} \frac{1 - r^2}{1 + r^2 - 2r \cos(\theta - t)} f(e^{it}) dt$$

If we have that $f$ is real valued, then we have that:

$$ P(f) (re^{i\theta}) = \frac{1}{2\pi} \int_0^{2\pi} \Re \left(\frac{e^{it} + z}{e^{it} - z} f(e^{it})\right)dt = \frac{1}{2\pi} \Re \left(   \int_0^{2\pi} \frac{e^{it} + z}{e^{it} - z} f(e^{it}) dt \right)$$

Doing a change of coordinates $\zeta = e^{it}$ we find:

$$\frac{1}{2\pi} \Re \left( \frac{1}{i}  \int_0^{2\pi} \frac{\zeta + z}{\zeta - z} f(\zeta) \frac{d\zeta}{\zeta} \right)$$

which, we notice by our theorem, the integrand is an analytic, and thus holomorphic function in $z$. Thus, $P(f)$ is a harmonic function in $U$.

Now, $P$ is a linear map of functions from $L^1(T)$ to functions on $U$, due to the linearity of the integral. 

Thus, if $f$ is now complex valued, we can then separate it into real and imaginary parts $f = f_1 + i f_2$, for real valued functions $f_1, f_2$. Using the linearity, we find:

$$P(f) = P(f_1 + if_2) = P(f_1) + i P(f_2)$$

Then, we have that $P(f_1), P(f_2)$ are harmonic functions. And, since the sum of harmonic functions are still harmonic, even with a complex factor, $P(f)$ is harmonic.

Theorem:

If $f \in L^1(t)$, then $P(f)$ is a harmonic function on $U$. 

Proof: as above

Theorem:

Suppose $f \in L^1(T)$, and further, $f \in C(T)$. That is, suppose $f$ is actually continuous on the unit circle. Define the function $Hf$ on $\overline{U}$ via:

$$ Hf(z) = \begin{cases} Pf(z) & \text{ if } z \in U \\ f(z) & \text{ if } z \in T \end{cases}$$

Then, $Hf$ is actually continuous on $\overline{U}$. 

Proof:

Note that, for any $z \in U$, we have that

$$Hf(z) = \int_0^{2\pi} Pf(z, e^{it}) f(e^{it}) dt \implies |Hf(z)| \leq \frac{1}{2\pi} \int_0^{2\pi} Pf(z,e^{it}) |f(e^{it})| dt \leq$$

$$\frac{1}{2\pi} \int_0^{2\pi} Pf(z,e^{it}) \Vert |f| \Vert_\infty dt  = \Vert f \Vert_\infty$$

where $\Vert f \Vert_\infty$ is the maximum of $f$ over $T$.

Moreover, if $z \in T$, then $| Hf(z) | = |f(z)| \leq \Vert f \Vert_\infty$, by definition.

Hence, over the closed unit disk, we have that:

$$ \Vert Hf \Vert_\infty \leq \Vert f \Vert_\infty $$

Fix a $k \in \mathbb{Z}$, and compute $H(e^{ikt})$. Well, suppose $z \in U$:

$$ H(e^{ikt})(z) = P(e^{ikt})(z) = \frac{1}{2 \pi } \int_0^{2\pi} P_r(\theta - t) e^{ikt} dt =   $$

$$ \frac{1}{2 \pi } \int_0^{2\pi} \sum_{-\infty}^\infty r^{|n|} e^{in(\theta - t)} e^{ikt} dt = \frac{1}{2 \pi } \int_0^{2\pi} \sum_{-\infty}^\infty r^{|n|} e^{in\theta} e^{it(n-k)} dt $$

Again, by the Weierstrauss M-test, we may interchange the sum and integral:

$$ \frac{1}{2\pi}  \sum_{-\infty}^\infty \left( r^{|n|} e^{in\theta} \int_0^{2\pi} e^{it(n-k)} dt  \right)$$

But, the integral vanishes unless $n-k = 0$ and if $ n=k$, the integral takes on $2\pi$ so we have that:

$$ H(e^{ikt})(z) = r^{|k|} e^{ik\theta}$$

moreover, on $z \in T$, then $H(e^{ikt})(z) = e^{ikt}$ by definition. Hence, we have that

$$ H(e^{ikt})(z) = r^{|k|} e^{ik\theta}$$

on all of $\overline{U}$. 

Then, we have that (with some manipulation on $r^{|k|} e^{ik\theta}$ for $k < 0$):

$$ H(e^{ikt})(z) = \begin{cases} z^k & \text{ if } k \geq 0 \\ \overline{z}^{|k|} & \text{ if } z < 0 \end{cases} $$

Thus, this is continuous on $\overline{U}$ and harmonic on $U$. 

Now, suppose that $f$ is a trigonometric polynomial on $T$, that is:

$$ f(e^{it}) = \sum_{k=-N}^N c_k e^{ikt} $$

By linearity, then, $H(f) = \sum_{k=-N}^N c_kH(e^{ikt})$, and since each piece is continuous, this is a sum of continous functions, and thus the whole thing is continuous.

Now, by the Stone-Weierstrauss Theorem:

Suppose that $f \in C(T)$. Then, there exists a sequence of trigonometric polynomials $P_n \to f$ uniformly. Then, of course:

$$ \Vert Hf - HP_n \Vert_{\overline{U}} = \Vert H(f - P_n)  \Vert_{\overline{U}}  \leq \Vert f - P_n \Vert_T \to 0$$

Hence, $HP_n \to Hf$ uniformly on $\overline{U}$, and thus, since each $HP_n$ is continuous, so must be $Hf$. 

Great, so now what? Turns out that if $f \in C(T)$, then $Hf$ is a function in $\overline{U}$ that solves the Dirichlet problem (at least on the closed unit disk):

Given a function $f$ on $T$, find a harmonic function $u$ defined on $U$ that coincides with $f$ on the boundary $T$. 

Slightly more explicitly, there is a continuous function $u \in C(\overline{U})$ such that $u$ is harmonic on $U$ and $u$ on $T$ coincides with $f$. 

Theorem [Uniqueness]:

Let $u \in C(\overline{U})$ be real-valued. Suppose that $u$ is harmonic on $U$. Then, on $U$, $u$ is always realized as the Poisson transform of its restriction to $T$, that is:

$$ u(z) = \frac{1}{2\pi} \int_0^{2\pi} P(z,e^{it}) u(e^{it}) dt $$

Moreover, from the above, we can prove that it is the real part of a holomorphic function:

$$ f(z) = \frac{1}{2\pi} \int_0^{2\pi} \frac{e^{it} + z}{e^{it} - z} u(e^{it}) dt$$

Proof:

We define $u_1(z) = \frac{1}{2\pi} \int_0^{2\pi} P(z,e^{it}) u(e^{it}) dt$. Then, of course since the Poisson transform of $u$ is harmonic, $u_1$ is harmonic. Further, we have that on $T$, $u_1 = u$.

Consider the quantity $v = u_1 - u$ on $\overline{U}$. Of course, since $u, u_1$ are harmonic, continuous, $v$ is harmonic, continuous. Further, $v = 0$ on $T$.

We claim that actually, $v = 0$ on $U$ as well. Suppose not. Then, there exists $z_0 \in U$ such that $v(z_0) \not = 0$.

Since $u, u_1$ are real valued, $v$ is real-valued as well. Then, we may assume that $v(z_0) > 0$, since otherwise, we consider $-v$, another harmonic function with the same properties.

Let $v(z_0) = 2 \epsilon$. Define an auxiliary function on $\overline{U}$:

$$ g(z) = v(z) + \epsilon |z|^2 $$

Clearly, on $T$, $g(z) = \epsilon$. On the other hand, we have:

$$ g(z_0) = v(z_0) + \epsilon |z_0|^2 = 2\epsilon + \epsilon |z_0|^2 > \epsilon$$

Certainly, this is greater than $\epsilon$, continuous. Since $g$ is continuous on a compact set, it attains a maximum, and by the inequality above, it must be an interior maximum.

Let $z_1$ be an interior maximum. Consider the Laplacian of $g$:

$$ \Delta (g(z_1)) = \Delta (v + \epsilon |z|^2)(z_1) = 4 \epsilon  > 0$$

However, due to the properties of being a maximum, we see that:

$$ \Delta (g(z_1)) = \frac{\partial^2 g}{\partial x^2}(z_1) + \frac{\partial^2 g}{\partial y^2}(z_1)  \leq 0$$

Where each component must be separately non-positive. Contradiction.

\section*{March 6th}

So, from last time, we’ve solved the Dirichlet problem on the unit disk, that is, given a continuous function $f$ on the unit circle $T$, there is a unique function $u$, continuous on the closed unit disk, harmonic, and agrees with $f$ on $T$. Specifically, $u = P(f)(z)$ that is, the integral of the Poisson kernel of $f$. If $f$ is real valued, then so is $u$, and $u$ is actually the real part of a holomorphic function.

However, we notice that since every disk is merely a scaling and translation of the unit disk, then this should extend to any disk. 

Let $F$ be a continuous function on the boundary of an arbitrary disk, $D(a,r), r > 0$. Then, there exists a unique function $u \in C(\overline{D}(a,r))$ such that $u$ is harmonic on $D(a,r)$, and $u = F$ on the boundary. 

Proof:

Define $F_0(e^{it}) = F(a + Re^{it})$. Clearly, $F_0$ on $T$ matches $F$ on the boundary of $D(a,r)$. Then, we know that there exists $u_0$ on $\overline{U}$, harmonic, and agrees with $F_0$ on $T$. That is:

$$u_0(se^{i\theta}) = \frac{1}{2\pi} \int_0^{2\pi} \frac{1 - s^2}{1 + s^2 - 2s \cos(\theta - t)} F_0(e^{it}) dt $$

Then, of course, just define $u(a + z) = u_0(\frac{z}{r})$. Thus, we have that:

$$u(a + Re^{i\theta}) = u_0(\frac{R}{r} e^{i\theta}) = \frac{1}{2\pi} \int_0^{2\pi} \frac{r^2 - R^2}{r^2 + R^2 - 2rR \cos(\theta - t)} F(a + re^{it}) dt $$

Thus, we’ve found the unique function that works, unique because $u$ was unique. 

Upshot here:

The holomorphic function of which $u$ is the real part on $D(a,r)$ is unique up to a purely imaginary constant. 

Proof:

Suppose $f_1, f_2$ holomorphic functions on $D(a,r)$ such that $\Re(f_1) = \Re(f_2) = u$.

Well, $\Psi = f_1 - f_2$ is certainly holomorphic, with 0 real part. But, by Cauchy-Riemann then, since the real part is $0$, the imaginary part may only be a constant. 

Wait what. Fulton claims open mapping theorem. If $\Psi$ is non-constant on the open disk, then $\Psi$ must be an open set, or a single point. Since it must be a subset of the imaginary axis, it cannot be an open set, and therefore, it must be a single point. Then, $\Psi$ is constant in the imaginary part. 

Since every harmonic function $u$ is the real part of a holomorphic function, and the real and imaginary parts of a holomorphic function is $C^\infty$, then we have that $u \in C^\infty$ on some open set $\Omega$ that it’s harmonic on. In layman’s terms: if you have second order derivatives in complex-valued functions, you actually have derivatives of all orders. 

Now, take $u$ to be harmonic on $D(a,R)$, continuous on the closure, and $u \in C(\overline{D}(a,R))$. Then, from what we’ve shown, we have:

$$u(a + re^{i\theta}) = \frac{1}{2\pi} \int_0^{2\pi} \frac{R^2 - r^2}{r^2 + R^2 - 2rR \cos(\theta - t)} u(a + Re^{it}) dt $$

Well, let’s evaluate at $r = 0$:

$$u(a) = \frac{1}{2\pi} \int_0^{2\pi}  u(a + Re^{it}) dt $$

So, what did we show? The value of the harmonic function in the middle of the disk is the average of the value of the harmonic function over the boundary. Kinda neat actually. We realize this is like a mean value theorem.

Theorem:

If $u$ is harmonic on an open set $\Omega$, and if $\overline{D}(a,R) \subset \Omega$, then $$u(a) = \frac{1}{2\pi} \int_0^{2\pi} u(a + Re^{i\theta}) d\theta$$

that is, the average value of the function $u$ on the boundary. 

Turns out the converse is also true, but we need some more.

Now, let’s look at the integrand a bit more. For $0 \leq r < R$, we have that:

$$\frac{R-r}{R+r} \leq \frac{R^2 - r^2}{r^2 + R^2 - 2rR \cos(\theta - t)}  \leq \frac{R + r}{R - r}$$

since of course, $-1 \leq \cos \leq 1$.

This leads to Harnack’s Theorem:

Theorem:

Let $\{ u_n \}_{n=1}^\infty$ be a sequence of harmonic functions in a region in $\Omega$.

(a) If $u_n \to u$ uniformly on compact sets in $\Omega$, then $u$ is harmonic.

(b) Suppose that $u_1 \leq u_2 \leq ... \leq u_n \leq ... $ in $\Omega$. Then, one of the following occurs:

(i) $u_n$ converges uniformly on compact sets to a harmonic function $u$ in $\Omega$.

(ii) $u_n \to \infty$ for all $z \in \Omega$

Proof:

(a)

So, of course, since $u_n$ are harmonic, thus cts, we have that $u$ must be continuous as well, since continuity is a local property, we may restrict to a small enough compact subset of $\Omega$ for any $a \in \Omega$.

Since harmonicity is a local property, let $\overline{D}(a,R)$ be any closed disk in $\Omega$. Then, we have that, for each $n$, that:

$$u_n(a + re^{i\theta}) = \frac{1}{2\pi} \int_0^{2\pi}  \frac{R^2 - r^2}{r^2 + R^2 - 2rR \cos(\theta - t)} u_n(a + Re^{it}) dt$$

Taking the limit of both sides as $n \to \infty$:

$$u(a + re^{i\theta}) = \lim_{n \to \infty} u_n(a + re^{i\theta}) = \lim_{n\to\infty} \frac{1}{2\pi} \int_0^{2\pi}  \frac{R^2 - r^2}{r^2 + R^2 - 2rR \cos(\theta - t)} u_n(a + Re^{it}) dt = $$ 
$$\frac{1}{2\pi} \int_0^{2\pi}  \frac{R^2 - r^2}{r^2 + R^2 - 2rR \cos(\theta - t)} u(a + Re^{it}) dt $$

where, we notice that $u_n \to u$ uniformly on the compact set $\partial D(a,R)$, and these are positive functions, so we may interchange the limit and integral. 

Thus, $u$ is harmonic, and since we can do this for any point $a \in \Omega$, it must be harmonic for all of $\Omega$. 

(b)

First, since we can do the substitution, $\{ u_n - u_1 \}$, we can assume that $u_1 = 0$, and $u_n \geq 0$ for all $n$, that is, non-negative. 

Let $a \in \Omega$, and choose $R >0$ such that $\overline{D}(a,R) \subset \Omega$. Since the $u_n$ are harmonic, we have the same equation:

$$u_n(a + re^{i\theta}) = \frac{1}{2\pi} \int_0^{2\pi}  \frac{R^2 - r^2}{r^2 + R^2 - 2rR \cos(\theta - t)} u_n(a + Re^{it}) dt$$

By the following inequality:

$$\frac{R-r}{R+r} \leq \frac{R^2 - r^2}{r^2 + R^2 - 2rR \cos(\theta - t)}  \leq \frac{R + r}{R - r}$$

we notice then that, because $0 \leq u_n$ for all $n$:

$$\frac{1}{2\pi} \int_0^{2\pi}  \frac{R - r}{R + r}  u_n(a + Re^{it}) dt \leq u_n(a + re^{i\theta}) \leq \frac{1}{2\pi} \int_0^{2\pi}  \frac{R + r}{R - r}  u_n(a + Re^{it}) dt$$

so:

$$\frac{1}{2\pi} \frac{R - r}{R + r} \int_0^{2\pi} u_n(a + Re^{it}) dt \leq u_n(a + re^{i\theta}) \leq \frac{1}{2\pi} \frac{R + r}{R - r} \int_0^{2\pi}   u_n(a + Re^{it}) dt$$

Here, we use the fact that the integral of $u(a) = \frac{1}{2\pi} \int_0^{2\pi} u(a + Re^{i\theta}) d\theta$ for $u$ a harmonic function.

So:

$$ \frac{R - r}{R + r} u_n(a) \leq u_n(a + re^{i\theta}) \leq \frac{R + r}{R - r} u_n(a) $$

What does this mean? Well, analyze via the following: Consider $\Omega_1 = \{ a \in \Omega : \{ u_n(a) \} \text{ is bounded } \}$, and $\Omega_2 =  \{ a \in \Omega : \{ u_n(a) \} \text{ is unbounded } \}$.

Suppose $a \in \Omega_1$. Then, we have that $D(a,R) \subset \Omega_1$, due to our inequality, which says $u_n(z) \leq \frac{R + r}{R - r} u_n(a) < \infty$, and thus $\Omega_1$ is open, since we can always find open balls around each point.

Conversely, we see that if $a \in \Omega_2$, then $D(a,R) \subset \Omega_2$, and thus is open for the same reason, just using the lower estimate.

Then, because $\Omega$ is a region, and thus connected, we must have that either $\Omega = \Omega_1$ or $\Omega = \Omega_2$. Of course, if $\Omega = \Omega_2$, thus $u_n \to \infty$, and we are done. 

\section*{March 8th} 

Continuing with Harnack’s Theorem:

So, suppose $\Omega = \Omega_1$. Then, there exists a function $u$ such that $u_n \to u$ pointwise on all of $\Omega$. But, we can restrict ourselves to compact sets of $\Omega$. In such a way, if $a \in \Omega$, then choose $R> 0$ such that $\overline{D}(a,r) \subset \Omega$. Since $\Omega$ is open, we may also find $R_1 > R$ such that we have $overline{D}(a,r) \subset \overline{D}(a,R_1) \subset \Omega$.

For each $0 \leq r < R$, and any $\theta$, we have that:

$$u_n(a + re^{i\theta}) = \frac{1}{2\pi} \int_0^{2\pi}  \frac{R^2 - r^2}{r^2 + R^2 - 2rR \cos(\theta - t)} u_n(a + Re^{it}) dt$$

Notice first that $$ u_n(a + Re^{i\theta}) \leq \frac{R_1 + R}{R_1 - R} u_n(a) < \infty $$

since we can use our estimate again on $\overline{D}(a,R_1)$, and find an upper bound that works for the inner circle. Then, we can use this to take limits of both sides, via a monotone convergence theorem to interchange limits:

$$ u(a + re^{i\theta}) = \frac{1}{2\pi} \int_0^{2\pi}  \frac{R^2 - r^2}{r^2 + R^2 - 2rR \cos(\theta - t)} \left( \lim_{n\to\infty}u_n(a + Re^{it})\right) dt =$$ $$ \frac{1}{2\pi} \int_0^{2\pi}  \frac{R^2 - r^2}{r^2 + R^2 - 2rR \cos(\theta - t)} u(a + Re^{it})dt$$

We notice, that because $u_n \to u$, and $u_n$ are measurable, bounded, we have that $u$ must be a bounded, measurable function. A bounded, measurable function implies that it is $L^1$, and the Poisson integral of a $L^1$ function is harmonic. Since we can do this for all points of $\Omega$, we have that $u$ is harmonic (and thus continuous) on $\Omega$. 

Now, we need only verify that $u_n \to u$ uniformly on compact subsets. Let $K$ be a compact subset, and let $\epsilon > 0$ be given. 

For each $n$, define $K_n = \{ z \in K : u(z) - u_n(z) \geq \epsilon \}$. It is clear that $K_n \supseteq K_{n+1}$. Further, we notice that $u(z) - u_n(z) \geq \epsilon$ is a closed set, and thus the preimage is closed. Thus, $K_n$ is compact, being a closed subset of a compact set. Finally, we have that $\cap_n K_n = \emptyset$, because $\lim_{n\to\infty} (u(z) - u_n(z)) = 0$. 

Since $K_n$ are compact, and their intersection is empty, by the finite intersection property, there exists a $N$ such that $\cap_{n=1}^N K_n = \empty$. Hence, we have that for $M \geq N+1$, we have that for all $m > M$, $u(z) - u_m(z) < \epsilon$. 

Mean Value Property:

Let $\Omega$ be an open set in $\mathbb{R}^2$, and let $f$ be a continuous function on $\Omega$. Suppose that $\overline{D}(a,r) \subset \Omega$. 

Define:

$$ M^r f(a) = \frac{1}{2\pi} \int_0^{2\pi} f(a + re^{i\theta}) d\theta $$

that is, the average value of the function $f$ over the circle of radius $r$ centered at $a$, $|z-a| = r$.

We’ve already shown that if $u$ is harmonic, then we have that:

$$M^r u(a) = u(a)$$ for any $r > 0$ such that $D(a,r) \subset \Omega$. What about the converse?

Definition:

Let $u$ be a continuous function on $\Omega$. Fix some $z \in \Omega$. If there exists a sequence of positive numbers $\{ r_n \}_{n=1}^\infty$ such that $r_n \to \infty$ such that $M^{r_n} u(z) = u(z)$ for all $n$ for every $z$, then we say that $u$ satisfies the Mean Value Property. Remark: we notice that the sequence of $r_n$ depends on the choice of $z$.

Theorem:

$u$ is harmonic on $\Omega$ if and only if $u$ satisfies the Mean Value Property.

Proof:

We’ve already proved that if $u$ is harmonic, then $M^{r} u(a) = u(a)$ for any $r$ such that the disk lies within $\Omega$. Thus, we only need to prove the other direction.

Suppose that $u$ satisfies the Mean Value Property on $\Omega$. We see then, that because we can split $u$ into real and imaginary components, that each of those must satisfy the Mean Value Property on $\Omega$ independently. Thus, without loss of generality, we may assume $u$ is a real-valued function. 

It suffices to prove that if $\overline{D}(a,R) \subset \Omega$, then $u$ is harmonic on $D(a,R)$, since then we can construct such disks for any point $a \in \Omega$.

Define the function $v$ on $D(a,R)$ via:

$$v(a + re^{i\theta}) = P(u) = \frac{1}{2\pi} \int_0^{2\pi}  \frac{R^2 - r^2}{r^2 + R^2 - 2rR \cos(\theta - t)} u(a + Re^{it}) dt $$

By our work on the Poisson kernel, $v$ is harmonic on $D(a,r)$. 

If we set $v(a + Re^{it}) = u(a + Re^{it})$, then $v$ is continuous on the closed disk and harmonic on the interior.

Consider $w = v - u$ defined on the closed disk. We know that $w$ is continuous, since $u,v$ continuous, and attains $0$ on the boundary. Further, since $v$ is harmonic, it satisfies the MVP, and $u$ satisfies by hypothesis, thus $w$ satisfies the MVP. 

Suppose there exists $z_0 \in \overline{D}(a,r)$ such that $w(z_0) \not = 0$. Since we can always replace $w’ = u - v$, we may assume that $w(z_0) > 0$. Clearly, since $w$ is 0 on the boundary, $z_0 \in D(a,r)$. Define $M = \max_{z \in \overline{D}(a,r)} w(z)$, which exists and is finite because this is a compact set.

Define $E = \{ z \in   \overline{D}(a,r) : w(z) = M \}$. Clearly, $E$ is closed, since $w$ continuous, and $\{ M \}$ is closed, so $w^{-1}(M)$ is closed. Let $s = \max_{z \in E} |z - a| $. Because $E$ is compact, $|z-a|$ attains a maximum, so we can find $z_1 \in E$ with $w(z_1) = s$. Let $\{ r_n \}$ be a sequence that corresponds to the point $z_1$ such that $w$ satisfies the MVP. In particular, choose an $r_n$ small enough such that $\overline{D}(z_1, r_n) \subset D(a,r)$.

Now, we have that 

$$w(z_1) = \frac{1}{2\pi} \int_0^{2\pi} w(z_1 + e^{i\theta}) d\theta $$

We notice, since $z_1$ is the furthest point from $a$ on $E$, we must have that

$$  \frac{1}{2\pi} \int_0^{2\pi} w(z_1 + e^{i\theta}) d\theta  <  \frac{1}{2\pi} 2\pi M = M $$, a contradiction.

\section*{March 13th}

The Schwartz Reflection Principle:

First, some notation:

Define $\Pi^+$ as the upper half plane, and $\Pi^-$ as the lower half plane.

Theorem:

Let $L$ be an open interval on the $x$ axis. Let $\Omega^+ \subset \Pi^+$ be a region, such that for every $t \in L$, there exists an $r > 0$ such that $D(a,r) \cap \Pi^+ \subset \Omega^+$. Let $\Omega^-$ be the reflection of $\Omega^+$ with respect to the $x$-axis, that is, viewing the plane as the complex plane:

$$ \Omega^- = \{ \overline{z} : z \in \Omega^+ \} $$

Suppose that $f = u + i v \in \mathcal{H}(\Omega^+)$, that is, a holomorphic function on $\Omega^+$. Further, suppose that:

$$ \lim_{z \to t} v(z) = 0 $$ for every $t \in L$. 

Let $\Omega = \Omega^+ \cup \Omega^- \cup L$. Then there exists a function $F \in \mathcal{H}(\Omega)$ such that $F = f$ on $\Omega^+$, and that $F(\overline{z}) = \overline{F(z)}$ for $\overline{z} \in \Omega$. 

Proof:

Let $f, \Omega^+$ be given. We extend $v$ to all of $\Omega$ by letting $v(t) = 0$ on $t \in L$ and $v(z) = -v(\overline{z})$ for all $z \in \Omega^-$. Because of our limit condition, since $v$ is a continuous function on $\Omega^+$, this must be a continuous function on $\Omega$.

This extension of $v$ must satisfy the Mean Value Property for all $z \in \Omega$, since $v$ is harmonic on $\Omega^+$, due to how we extend to $\Omega^-$, since the values are the same, with a negative, it must still satisfy the mean value property on $\Omega^-$. And finally, on $L$, this should be clear, since any disk around a point on $L$ has positive and negative values on the upper and lower semicircle, i.e. they cancel, and $v(t) = 0$. Thus, $v$ satisifes the MVP everywhere. Thus, $v$ is actually harmonic on all of $\Omega$.

This implies that we may realize $v$ as, locally, the imaginary part of a holomorphic function on $\Omega$.

Fix some $t \in L$. Choose an open disk $D_t$ centered on $t$ such that $\overline{D}_t \subset \Omega$ and a function $f_t \in \mathcal{H}(\overline{D}_t)$ such that $\Im f_t = v$, that is, a holomorphic function locally around $t$.

We recall that $f_t$ is unique up to adding a real constant. So, since $\Im f = v$ on the top half, we can choose the real constant such that $f = f_t$ on $D_t \cap \Pi^+$

In particular, since we can patch along $L$, with many neighborhoods, we can actually claim that $f = f_t$ for all $t$, since we can find neighborhoods close enough together such that they overlap on the upper half plane, and thus since they agree on the intersection, they agree everywhere. But this implies that $f_t = f$  on $V = \cup_{t \in L} D_t$. Call this function $F \in \calH(V)$.

Note that because $f = f_t$ on $D_t \cap \Pi^+$ for each $t$, we have that $f_t$ must be real on $L$ as well, due to continuity. Thus, since we have that $f^{(n)}(t)$ is purely real for $t \in L, n \in \mathbb{N}$, we can look at a power series:

$$ f_t(z) = \sum_{n=0}^\infty c_n (z-t)^n$$ with $c_n$ purely real.

Thus, we have that $f_t(\overline{z}) = \overline{f_t(z)}$.

Then, we extend $F$ onto all of $\Omega$ via:

$$F(z) = \begin{cases} f(z) & \text{ if } z \in \Omega^+ \\ f_t(z) & \text{ if } z \in D_t \\ \overline{f(\overline{z})} & \text{ if } z \in \Omega^- \end{cases} $$

First, is this well defined? Of course, due to our choice of $f_t = f$. Further, we know that $F$ is already holomorphic on $V$, and $\Omega^+ \setminus V$. So it remains to just show that $F$ is holomorphic on $\Omega^- \setminus V$.

Well, take $a \in \Omega^-$, and choose $r > 0$ such that $D(a,r) \subset \Omega^-$.

In particular, we know that $\overline{a} \in \Omega^+$, so $F$ is holomorphic there. Thus, we have a power series:

$$ f(z) = \sum_{n=0}^\infty d_n (z - \overline{a})^n $$ for all $D(\overline{a},r)$.

But, by definition:

$$ F(z) = \overline{f(\overline{z})} = \overline{ \sum_{n=0}^\infty d_n (z - \overline{a})^n} = \sum_{n=0} \overline{d_n} (z -a )^n $$

Since the original function converges, so must this, and thus $F$ is analytic and thus holomorphic on a neighborhood for every point in $\Omega^-$.

Corollary:

Thus, under the same hypotheses as above, for all $t \in L$, $\lim_{z \to t} u(z)$ exists.

Preview: we can use this to prove Picard’s Little Theorem:

Theorem:

Let $f$ is a non-constant, entire function. Then $f(\mathbb{C}) = \mathbb{C}$ or, for some $a \in \mathbb{C}$, $\mathbb{C} \setminus \{ a \}$.  Further, for every $z \in f(\mathbb{C})$, $f^{-1}(z)$ is infinite. 

Boundary Behaviors of Holomorphic Functions:

Now, for any function $u$ defined on the unit disk $U$, and $0 \leq r < 1$, define $u_r$ on the unit circle:

$$u_r(e^{i\theta}) = u(r e^{i\theta})$$

That is, specifically the points on the circle of radius $r$.

We’ve already seen that under the Poisson integral, if we define:

$$\calH(f)(z) = \begin{cases} P(f)(z) & \text{ if } z \in U \\ f(z) & \text{ if } z \in T \end{cases}$$

This is a continuous, and thus uniformly continuous.

Now, let’s talk about this in a $L^p$ context:

Theorem:

Let $1 \leq p \leq \infty$, and suppose $f \in L^p(T) \implies f \in L^1(T)$ , that is $\int_T |f|^p d\theta < \infty$ . Now, let $u$ be the Poisson kernel of $f$, $u = P[f]$. Then $u$ is harmonic on $U$, and $\Vert u_r \Vert_p \leq \Vert f \Vert_p$ for all $0 \leq r < 1$. Moreover, if $1 \leq p < \infty$, then:

$$ \lim_{r \to 1} \Vert u_r - f \Vert_p = 0$$

Proof:

Unfun, computational. See lecture notes.

Maximum Modulus Theorem:

Recall the theorem:

Let $\Omega$ be an open set, $f \in \calH(\Omega)$, and $\overline{D}(a,r) \subset \Omega$. Then, we have that:

$$|f(a) | \leq \max_{0 \leq \theta \leq 2\pi} | f(a + re^{i\theta})| $$

Furthermore, we have equality if and only if $f$ is constant on $\Omega$, alternatively, if equality holds, then $f$ is constant on $\overline{D}(a,r)$. 

Corollary:

Suppose $\Omega$ is a region, and $f \in \calH(\Omega)$. Further, suppose $|f|$ has a local maximum on $\Omega$. Then, $f$ must be constant.

Proof:

Suppose $|f|$ attains a local maximum at $a$. Choose a neighborhood such that $\overline{D}(a,r)$ is at most $|f(a)|$. But, by the maximum modulus principle, we have that $|f(a)| \leq \max \overline{D}(a,r)$. Thus, we have equality, and that $f$ is constant on the disk. Then, consider the holomorphic function $g = f - c$, such that $c$ is the constant that $f$ attains on the disk. Thus, $g= 0$ on the disk, and thus must be 0 on all of $\Omega$.

Corollary:

Let $\Omega$ be a region, bounded, $f$ a continuous function on $\overline{\Omega}$, and holomorphic on $\Omega$. Then we have that:

$$ |f(z)| \leq \Vert f \Vert_{\partial \Omega} $$

If equality holds for some $z$ then $f$ is actually constant.

Proof:

Same idea, just use the fact that $\overline{\Omega}$ is compact, so $|f|$ attains a maximum there.

Consequences on the unit disk:

Let $H^\infty = \{ f \in \calH(U) : f$ is bounded $\}$.

Schwartz Lemma:

Suppose that $f \in H^\infty, \Vert f \Vert_\infty \leq 1, f(0) = 0$. Then:

(i) $| f’(0)| \leq 1$

(ii) $ |f(z)| \leq |z| $ for all $z \in U$.

If we have equality in either case, then there exists $\lambda \in \mathbb{C}, |\lambda| = 1$ such that $f(z) = \lambda z$.

Proof:

Define the function $g$ on $U$ such that:

$$g(z) = \begin{cases} \frac{f(z)}{z} & \text{ if } z \in U \setminus \{ 0 \} \\ f’(0) & \text{ if } z= 0 \end{cases} $$

We notice that since $f(0) = 0$, $f(z) /z$ has a removable singularity at $0$, and looking at the power series, takes on the value $f’(0)$, as above. Thus, $g$ is actually holomorphic on $U$.

Fix any $z \in U$. Then, for any $r > |z|$, we have that:

$$ |g(z)| \leq \max_{\partial D(0,r)} |g| = \max_{\partial D(0,r)} \frac{f(\zeta)}{\zeta} \leq \frac{1}{r} $$ 

Taking the limit as $r \to 1$, we find that $|g(z)| \leq 1$, and we are done.

Now, suppose there exists $z \in U$ such that $|f(z)| = |z|$. Thus, at $z$, $g(z) = 1$. Thus, since $|g| \leq 1$, this is a local maximum, and thus constant, with $g(z) = \lambda$, $|\lambda| = 1$. Then, $f(z)/z = \lambda \implies f(z) = \lambda z$. Same argument applies for $f’(0) =1$.

We want to try to generalize this:

Let $\alpha, \beta \in \mathbb{C}$ such that $|\alpha| <1, |\beta| < 1$. Among every function $f \in H^\infty$, with $\Vert f \Vert_\infty \leq 1, f(\alpha) = \beta$, how large can $|f’(\alpha)|$ potentially be?

We answer this question by defining 

$$\varphi_\alpha(z) = \frac{z - \alpha}{1 - \overline{\alpha} z} $$

This is holomorphic everywhere except potentially $z = \frac{1}{\overline{\alpha}}$ which is outside the unit disk since $|\alpha| < 1$.

Lemma:

$ \varphi_\alpha$ is a one to one holomorphic function from $U$ onto $U$, with inverse $\varphi_{-\alpha}$. Further, $\varphi_\alpha$ maps $\alpha \to 0$, and $\varphi_\alpha’(0) = 1 - |\alpha|^2, \varphi’(\alpha) = \frac{1}{1 - |\alpha|^2}$. So, we use this to use the Schwartz Lemma.

\section*{March 15th}

Again, Maximum Modulus Theorem and consequences.

Recall the following results:

Suppose $\Omega$ is a region, and $f \in \calH(\Omega)$. Further, suppose $|f|$ has a local maximum on $\Omega$. Then, $f$ must be constant.

and

Let $\Omega$ be a region, bounded, $f$ a continuous function on $\overline{\Omega}$, and holomorphic on $\Omega$. Then we have that:

$$ |f(z)| \leq \Vert f \Vert_{\partial \Omega} $$

If equality holds for some $z$ then $f$ is actually constant.

Also:

$$ \Vert f \Vert_{\overline{\Omega}} = \Vert f \Vert_{\partial \Omega} $$

Also recall:

Let $H^\infty(\Omega) = \{ f \in \calH(\Omega) : f$ is bounded $\}$.

By convention, we denote $H^\infty(U) = H^\infty$. These are examples of Hardy spaces.

Recall the Schwarz lemma:

Suppose that $f \in H^\infty, \Vert f \Vert_\infty \leq 1, f(0) = 0$. Then:

(i) $| f’(0)| \leq 1$

(ii) $ |f(z)| \leq |z| $ for all $z \in U$.

If we have equality in either case, then there exists $\lambda \in \mathbb{C}, |\lambda| = 1$ such that $f(z) = \lambda z$.

Finally, recall the last question:

Let $\alpha, \beta \in \mathbb{C}$ such that $|\alpha| <1, |\beta| < 1$. Among every function $f \in H^\infty$, with $\Vert f \Vert_\infty \leq 1, f(\alpha) = \beta$, how large can $|f’(\alpha)|$ potentially be?

We answer this question by defining 

$$\varphi_\alpha(z) = \frac{z - \alpha}{1 - \overline{\alpha} z} $$

This is holomorphic everywhere except potentially $z = \frac{1}{\overline{\alpha}}$ which is outside the unit disk since $|\alpha| < 1$.

Lemma:

$ \varphi_\alpha$ is a one to one holomorphic function from $U$ onto $U$, with inverse $\varphi_{-\alpha}$. Further, $\varphi_\alpha$ maps $\alpha \to 0$, and $\varphi_\alpha’(0) = 1 - |\alpha|^2, \varphi’(\alpha) = \frac{1}{1 - |\alpha|^2}$. 

Proof:

It should be clear that if we actually compute $\varphi’(z)$, we can find those values for the derivative. It is not hard either to show that $\varphi_{-\alpha}(\varphi_\alpha(z)) = z$, from unfun algebra. But, because $\varphi_\alpha$ has a left inverse, it must be one-to-one.

Lastly, if we consider:

$$\varphi_\alpha(e^{i\theta}) = \frac{e^{i\theta} - \alpha}{1 - \overline{\alpha}e^{i\theta}} = e^{i\theta} \frac{1 - \alpha e^{-i\theta}}{1 - \overline{\alpha} e^{i\theta}}$$

Since we recognize the fraction as $\overline{z}/z$, this is the product of two modulus 1 values, and hence lies on the unit circle $T$. Thus, we have that:

$$\varphi_\alpha(T)= T$$, that is, $\varphi_\alpha$ defines a bijection from $T \to T$.

Now, since our function is one-to-one, since $\varphi_\alpha(U)$ is a holomorphic function, this image is a connected open set, disjoint from $\varphi_{\alpha}(T)$. Hence, either $\varphi_\alpha(U) \subset U$ or $\varphi_\alpha(U) \cap \overline{U} = \emptyset$. Since $\varphi_\alpha(\alpha) = 0$, we conclude that $\varphi_\alpha(U) \subset U$. Then, since we can apply this to $\varphi_{-\alpha}$, we see that:

$$\varphi_\alpha(\varphi_{-\alpha}(U)) = \varphi_\alpha(U) \implies U \subset \varphi_\alpha(U) \implies U = \varphi_\alpha(U)$$

Ok, back to the question:

Since for $f$ constant, $f’(z) = 0$ for all $z$, including $\alpha$, we may assume $f$ is non-constant. Then, we know that $f(U) \subset U, f(U) \cap T = \emptyset$, as otherwise, it would be an interior maximum.

Introduce the function:

$$g  = \varphi_\beta \circ f \circ \varphi_{-\alpha}$$

By the properties we have discussed, we have that $\Vert g \Vert_\infty \leq 1$ on $\overline{U}$.

It should be clear that $g(0) = 0$. We may apply the Schwarz lemma to $g$, which says that:

$$g’(0) \leq 1$$

So, we have that:

$$g’(z) = \varphi_\beta’(f(\varphi_{-\alpha})(z)) \cdot f’(\varphi_{-\alpha}(z)) \cdot \varphi_{-\alpha}’(z)$$ so

$$g’(0) =  \varphi_\beta’(f(\varphi_{-\alpha})(0)) \cdot f’(\varphi_{-\alpha}(0)) \cdot \varphi_{-\alpha}’(0)$$

which implies that:

$$ |\varphi_\beta’(0) | \cdot |f’(0)| \cdot |\varphi_{-\alpha}’(0)| \leq 1$$

But, by the properties of $\varphi$, we have then that:

$$| f’(0) |  \leq \frac{1 - |\beta|^2}{1 - |\alpha|^2}$$

Now, if equality holds, we have that the composite map is a rotation, that is:

$$g(z)  = \varphi_\beta \circ f \circ \varphi_{-\alpha}(z) = e^{i\theta} z$$ for some $\theta \in [0,2\pi)$. Composing with the inverses of $\varphi$, we see that:

$$ f = \varphi_{-\beta} \circ r_\theta \circ \varphi_\alpha$$ where $r_\theta$ sends $z \to e^{i\theta} z$.

Theorem:

Suppose that $f$ is a one-to-one holomorphic map of $U$ onto $U$, and $f(0) = 0$. Then, $f(z) = e^{i\theta}z$ for some $\theta$.

Proof:

Since $f$ is one-to-one, we know that $f’(z) \not = 0$, for all $z \in U$. Hence, we have a global $f^{-1}$, holomorphic, since $f$ has local holomorphic inverses everywhere, due to the non-vanishing of the derivative, since we patch things together, $f^{-1}$ has to agree with the local holomorphic inverses on each neighborhood, so we have that there’s a neighborhood at each point where $f^{-1}$ is holomorphic.

Define $g = f^{-1}$. Since we have that $g(f(z)) = z$, we have that:

$$g’(0) \cdot f’(0) = 1 $$

by taking the derivative of both sides. The Schwarz lemma says that $|f’(z)| \leq 1, |g’(z)| \leq 1$, therefore, $g’(0) = f’(0) = 1$ and therefore, $f$ is a rotation.

Corollary:

Suppose that $f$ is a one-to-one holomorphic map form $U$ onto $U$, and that $f(\alpha) = 0$. Then, $f = r_\theta \circ \varphi_\alpha$.

Proof:

Consider $g = f \circ \varphi_{-\alpha}$. This is a function such that $g(0) = 0$. Thus, $g(z) = r_\theta(z)$. Composing by $\varphi_{\alpha}$ on the right, we are done.

Phragmen-Lindelof methods:

Recall the setup for the Max. Modulus Theorem:

Let $\Omega$ be a bounded region, $f$ holomorphic on $\Omega$, and continuous on $\overline{\Omega}$, then:

$$ \Vert f \Vert_{\overline{\Omega}} = \Vert f \Vert_{\partial \Omega} $$

non-example for unbounded region:

Let $\Omega = \{ z = x + iy : -\pi/2 \leq y \leq \pi/2 \}$, and consider $f(z) = e^{e^{z}}$. Clearly, on $y = \pm\pi/2$, $f(z) = e^{\pm i e^{x}}$, so on this line $|f(z)| = 1$. However, of course, on the real line, $e^{e^{z}}$ is unbounded. So, the boundedness of the region seems to be a necessary one. 

Theorem:

Let $a,b \in \mathbb{R}$, and define $\Omega = \{ z = x + yi : a < x < b \}$. Let $f \in \calH(\Omega)$, and bounded on $\overline{\Omega}$, $f \not = 0$.

For each $x \in [a,b]$, define $M(x) = \sup_{y} | f(x + iy)| $.

Then, for any $x \in [a,b]$, we have that:

$$ M(x) \leq M(a)^{b-x/b-a} \cdot M(b)^{x-a/b-a}$$

In particular, we can conclude that:

$$M(x) \leq \max\{ M(a), M(b) \} $$ for all $x$, and thus:

$$\Vert f \Vert_{\overline{\Omega}} = \Vert f \Vert_{\partial \Omega}$$

But, we notice that the original inequality is equivalent to:

$$ M((1-t)a + tb) \leq M(a)^{1-t} \cdot M(b)^{t} $$

via a rescaling. In particular, if $M(a), M(b) > 0$, then we have that $\log(M)$ is a convex function.

\section*{March 27th}

Recall the example above with $e^{e^{z}}$. Now:

Theorem:

Let $\Omega = \{ z = x + iy : -\pi/2 < y < \pi/2 \}$. Suppose we have that $f \in C(\overline{\Omega}), f \in \calH(\Omega)$, and that there exists $\alpha < 1, A > 0$ such that:

$$ |f(z)| \leq e^{Ae^{\alpha |x|}} $$ for all $z \in \Omega$. Suppose also that $|f(x \pm \pi/2 i)| \leq 1$ for all $x$. Then, we have that $|f(z)| \leq 1$ for all $z \in \Omega$. 

Proof (by Phragmen-Lindelof methods):

Choose a $\beta > 0$ such that $\alpha < \beta  < 1$. For any $\epsilon > 0$, let $h_\epsilon(z) = e^{-\epsilon(e^{\beta z} + e^{-\beta z} )}$. For any $z = x + iy \in \overline{\Omega}$, we have that:

$$\Re(e^{\beta z} + e^{-\beta z} ) = \Re(e^{\beta x}( \cos(\beta y) + i \sin(\beta y)) + e^{-\beta x}( \cos(\beta y) - i \sin(\beta y))) = \cos(\beta y) (e^{\beta x} + e^{-\beta x})$$ 

Now, recall that $ -\beta \pi/2 \leq \beta y \leq \beta \pi/2 $.

Then, we have the estimate that $\cos \beta y \geq \cos \beta \pi/2$, so:

$$ \cos(\beta y) (e^{\beta x} + e^{-\beta x}) \geq \cos(\beta \frac{\pi}{2}) (e^{\beta x} + e^{-\beta x})$$

We notice that $\cos(\beta \frac{\pi}{2})$ is a positive constant with respect to $x$, and since $(e^{\beta x} + e^{-\beta x}) > 0$, the whole thing is positive. For the future, define $\delta = \cos(\beta \frac{\pi}{2})$.

Thus, we have that:
 
$$|h_\epsilon(z)| = \left| e^{-\epsilon(e^{\beta z} + e^{-\beta z} )} \right| = e^{-\epsilon \Re(e^{\beta z} + e^{-\beta z})} \leq e^{-\epsilon \delta(e^{\beta x} + e^{-\beta x})} < e^{0} = 1$$

Hence, we have that $| f h_\epsilon | \leq 1$ on $\partial \Omega$. 

But, we have that:

$$ |f(z) h_\epsilon(z)| \leq e^{Ae^{\alpha |x|}} e^{-\epsilon \delta(e^{\beta x} + e^{-\beta x})} = e^{Ae^{\alpha |x|}-\epsilon \delta(e^{\beta x} + e^{-\beta x})} $$

%Now, taking the estimate that:

%$$ e^{\beta z} + e^{-\beta z} \geq e^{\beta| z|}$$

%we see that:

%$$|f(z) h_\epsilon(z)|  \leq e^{Ae^{\alpha |x|}-\epsilon \delta(e^{\beta z} + e^{-\beta z})}  \leq e^{Ae^{\alpha |x|}-\epsilon \delta e^{\beta |z|} } $$

Analyzing this quantity, since $\beta > \alpha$, we have that as $x \to \infty$, $Ae^{\alpha |x|} < \epsilon \delta(e^{\beta x}$. So, the thing goes to 0. Similarly, using the $-x$ side, we can make the same argument for $x \to -\infty$.

Now,  back to our picture, this means then that we can choose an $R > 0$ such that as long as $|x| > R$, $|f(z) h_\epsilon(z)|  \leq 1$ on $\Omega$, because our quantity goes to 0 as $ |x| \to \infty$. 

Now, restrict our attention to $\Omega \cap \{ x + yi : -R \leq x \leq R \}$, that is, the bounded remainder. By what we’ve done, $|f h_\epsilon| \leq 1$ on the left and right. But further, we have that this is still true on the top and bottom by hypothesis. Thus, we may apply the maximum modulus principle to conclude that $|f h_\epsilon| \leq 1$ on the bounded portion. Thus, $|f h_\epsilon| \leq 1$ on all of $\Omega$. 

Since this is independent of $\epsilon > 0$, we can take $\epsilon \to 0$, and we are done. 

What’s the upshot? The idea is that if we can find a perturbation (like $h_\epsilon$), and say something about $f h_\epsilon$ for any $\epsilon$, if we can reduce the perturbation to 0, we can say that this holds for $f$ as well. 

Approximation of Rational Functions:

Let $f(z)$ be entire. Then, of course, $f$ is analytic, and thus has a Maclaurin series:

$$f(z) = \sum_{k=0}^\infty c_k z^k$$

which converges absolutely and uniformly on compact sets. 

In particular, if:

$$P_n(z) = \sum_{k=0}^n z^k $$

then of course, $P_n \to f$ uniformly on compact sets. 

We wish to prove a similar result if $f$ is not entire, but merely holomorphic on an open set. It turns out that this is possible, but we may not be able to use polynomials. We need rational functions. (It turns out if your region is simply connected, then polynomials ar enough)

Runge’s Theorem:

Suppose that $\Omega$ is an open set, and that $f \in \calH(\Omega)$ is holomorphic. Then, there exists a sequence of rational functions $\{ R_n \}$, with no poles in $\Omega$, such that $R_n \to f$ uniformly on compact subsets of $\Omega$. 

Corollary: If $\Omega$ is simply connected, then $R_n$ can be polynomials, not just rational functions. 

Before we prove this, we need to introduce the Riemann sphere:

We adjoin a point at infinity to the complex numbers $\{ \infty \}$ as follows:

Let $S^2 = \mathbb{C} \cup \{ \infty \}$, that is, the normal 2-sphere in $\mathbb{R}^3$ as follows. For each $r > 0$, we take:

$$D’(\infty,r) = \{ z \in \mathbb{C} : |z| > r \}$$

and then declare the open neighborhoods of $\infty$ as exactly $\{ \infty \} \cup D’(\infty,r)$. Now, on all of the sphere, we declare $V\subset S^2$ is open if $V$ is a union of collections of disks. 

We can show that this forms a topology, that this topology is compact, and that this is homeomorphic to the unit sphere in $\mathbb{R}^3$. 

Explicitly, we can say the homeomorphism has the following form:

$$\varphi: S^2 \to \{ (x,y,z) : x^2 + y^2 + z^2 = 1\}$$

via $\varphi(\infty)  = (0,0,1)$, and otherwise, using the projection between $(0,0,1)$ and the complex plane identified by the $x-y$ plane. Explicitly:

$$\varphi(r e^{i\theta}) =\left(\frac{2r}{r^2 + 1} \cos \theta,\frac{2r}{r^2 + 1} \sin \theta, \frac{1-r^2}{r^2 + 1}\right)$$

Now, what does it mean to be holomorphic at $\infty$?

If $f$ is holomorphic on $D’(\infty, r)$, then we say that $f$ has a singularity at $\infty$. Note that of course, the map that takes:

$$ z \to f\left(\frac{1}{z}\right)$$ is holomorphic on $D’$, because of course, this maps $D’(\infty,r) \to D’(0,1/r)$.

Then, we say that the singularity that $f$ has at $\infty$ is the type of singularity that $f(1/z)$ has at $z = 0$. 

Cases:

If $f$ is bounded and holomorphic on $D’(\infty)$, then $f(1/z)$ is bounded holomorphic on $D’(0)$. Then, of course, this singularity is removable at 0 for $f(1/z)$, so we say that $f$ has a removable singularity at $\infty$.

Similarly, if $f(1/z)$ has a pole of order $m$ at $z=0$, then we say that $f(z)$ has a pole of order $m$ at $\infty$. 

Further, suppose that the principal part of $f(1/z)$ at $z = 0$ is:

$$ \frac{c_1}{z} + \frac{c_2}{z^2}  + ... + \frac{c_m}{z^m} $$

Then, we say that the principal part of $f$ at $\infty$ is $c_1 z + .... + c_m z^m$.

Finally, if $f(1/z)$ has an essential singularity at $0$, then we say that $f(z)$ has an essential singularity at $\infty$. 

Example:

Suppose $f$ an entire function, non-polynomial. Then, of course we have that $f(z) = \sum_{k=0}^\infty c_k z^k$. Thus, we have that $$f(1/z) = \sum_{k=0}^\infty c_k \frac{1}{z^k}$$ and thus, has an essential singularity at $0$, which means $f$ has one at $\infty$. 

Connectedness considerations:

Suppose that $\Omega$ is a set in $\mathbb{C}$. We note that $S^{2}  \setminus \mathbb{C}$ may be connected, but $\mathbb{C} \setminus \Omega$ might not be.

Example: Take $\Omega = \{ z = x + iy : -1 < x < 1 \}$.

\section*{March 29th}

Theorem [Approximation of Open Set via Compact Sets]:

Let $\Omega \subset \mathbb{C}$ be open. Then, there exists a sequence of compact sets $\{ K_n \}_{n \in \mathbb{N}}$ such that the following hold:

1) $K_n \subset K_{n+1}^o$, that is, the interior of $K_{n+1}$

2) $\cup_{n=1}^\infty K_n = \Omega$

3) Each component of $S^2 \setminus K_n$ contains a component of $S^2 \setminus \Omega$. 

Intuitively, we can understand 3) as meaning that for each hole in $K_n$, it must contain at least one hole in $\Omega$. i.e., the hole in $K_n$ cannot be filled in by $\Omega$. 

Proof:

For $n \in \mathbb{N}$, let $$V_n = D’(\infty, n) \cup \left( \cup_{a \in \mathbb{C} \setminus \Omega} D(a,1/n) \right) $$

Now, let $K_n = \mathbb{C} \setminus V_n$. Clearly, since $V_n$ is the union of a collection of open sets, $V_n$ itself must be open. Thus, $K_n$ is closed. Further, by construction, $K_n$ must be bounded, because it is contained within the compliment of $D’(\infty, n)$, so $K_n \subset \overline{D}(0,n)$, thus $K_n$ is compact. 

Now, without much work, we can see that $V_{n+1} \subset V_n$, with strict inclusion. Thus, we have that $K_n \subset K_{n+1}$. 

Now, suppose $z \in K_n$. Consider the disk $D(z, 1/n - 1/(n+1))$. We notice that by construction of $K_n$, we have that $|z| \leq n$. Hence, we have that $D(z, 1/n - 1/(n+1)) \subset \overline{D}(0,n+1)$, because $1/n - 1/(n+1) < 1$, and therefore, $D(z, 1/n - 1/(n+1)) \cap D’(\infty, n+1) = \emptyset$.

Further, for any $a \in \mathbb{C} \setminus \Omega$, we have that $D(z, 1/n - 1/(n+1)) \cap D(a, 1/n+1) = \emptyset$, because suppose not. Then, we have that, for some $w$ in the intersection, that $d(z,w) < 1/n - 1/(n+1)$, $d(w,a) < 1/n+1$, and by the triangle inequality, we would have that:

$$ d(z,a) \leq d(w,a) + d(z,w) < 1/n $$

a contradiction, since this would imply that $z \in D(a, 1/n)$, and thus in $V_n = K_n^c$. 

Thus, we have that since $D(z,1/n - 1/n+1) \cap D(a, 1/n+1) = \emptyset$ for all $a$, and $D(z,1/n - 1/n+1) \cap D’(\infty, n+1) = \emptyset$, thus, since $V_n$ is simply the union of all of these things, this implies that $D(z, 1/n - 1/n+1) \cap V_{n+1} = \emptyset$. Since $V_{n+1}, K_{n+1}$ partition $\mathbb{C}$, we must have that this is contained within $K_{n+1}$, and thus, $z \in K_{n+1}^o$. Therefore, $K_n \subset K_{n+1}^o$. 

Now, of course, we should be able to see that for all $n$, $\Omega^c \subset V_n$, and therefore, $K_n \subset \Omega$ for all $n$. So, we have automatically that $\cup K_n \subset \Omega$. Now, let $z \in \Omega$. Define $r = \text{dist}(z, \mathbb{C} \setminus \Omega ) > 0$, which we may do since $\Omega$ is open.

Choose an $n$ such that $|z| < n, 1/n < r$. Then, by construction, we have that $ z \not \in D’(\infty, n)$, but also, by the distance, we have that $z \not \in D(a,1/n)$ for any $a \in \Omega^c$. 

Thus, $z \not \in V_n$, for this choice of $K_n$, and thus, $z \in \cup K_n$. Since we may do this for any $z \in \Omega$, it follows that $\Omega \subset \cup K_n$ and therefore $\Omega = \cup K_n$.

Finally, let $\mathcal{O}$ be a component of $S^2 \setminus K_n$. If $\infty\in \mathcal{O}$, then of course we have that the component of $\Omega^c$ that contains $\infty$ is in $\mathcal{O}$. 

Now, suppose instead that we have that for some $z \in \mathcal{O}$, that $z \in D(a, 1/n)$, for some $z \not \in \Omega$. Because the open disk is connected, $\mathcal{O}$ must contain the whole disk. In particular, $a \in \mathcal{O}$, and thus, the component of $\Omega^c$ containing $a$. 

Making the Cauchy Integral Formula behave well:

Recall that the index part of the formula can be messy. So, how do we control this?

Theorem:

Suppose that $\Omega \subset \mathbb{C}$ is a non-empty open set, with $K \subset \Omega$ compact. Then there exists a cycle $\Gamma$ in $\Omega$ such that the following hold:

1) $\text{Ind}_\Gamma(z) = 1$ for all $z \in K$.

2) $\text{Ind}_\Gamma(w) = 0$ for all $w \in \mathbb{C} \setminus \Omega$.

Why do we care? If this theorem holds, then we would have that for any holomorphic function $f \in \calH(\Omega)$, that $f(z) = \frac{1}{2\pi i} \int_\Gamma f(\zeta)/(\zeta - z) d\zeta$ for any $z \in K$ that is, we can drop the index from the Cauchy integral formula.

Proof:

Since $K$ compact, and $\Omega^c$ closed, define $2r$ as the smallest distance from $K$ to $\Omega^c$, with $r$ being arbitrary if $\Omega^c = \emptyset$. 

Now, consider any lattice of the plane into squares of side length $r$. Consider the collection of squares $Q_1,....,Q_m$ be the squares with non-empty intersection with $K$. Let $\partial Q_1,...,\partial Q_m$ be the positively oriented boundaries of our squares. Let $\Gamma = \partial Q_1 + ... + \partial Q_m$.

Clearly, by the choice of $r$, we know that $\Gamma$ is a cycle contained within $\Omega$. Hence, we have that if $w \not \in \Omega$, then $w \not \in Q_j$ for all $1 \leq j \leq m$, and therefore $\ind_\Gamma(w) = 0$, because:

$$ \ind_\Gamma(w) = \sum_j \ind_{Q_j}(w) = \sum_j 0 = 0$$.

Now, suppose $z \in K$. Then $z \in Q_j$ for at least some $j$. If $z \in Q_j^o$, then of course we have that $\ind_\Gamma(z) = 1$, because the index with respect to all other $Q_i$ is 0, and only 1 at $Q_j$. 

Now, suppose $z$ is not in the interior of any $Q_j$. Then, it must belong to the boundary of at least some $Q_j$. By the choice of $Q_j$, $z$ must also belong to the square(s) that shares this boundary. Hence, $z \not \in \Gamma^*$, so $\ind_\Gamma(z)$ is defined. 

Since $Q_j$ is a closed square, there exists a sequence of points $\{ z_n \}$ in $Q_j$ that converges to $z$, and since they lie in the interior of $Q_j$, $\ind_\Gamma(z_n) = 0$. Since the index is a integral of a continuous function, it itself is continuous, therefore, $\ind_\Gamma(z) = 1$.

\section*{April 3rd}

Runge’s Theorem:

Analysis Preliminaries:

1)

Let $X$ be a normed vector space. 

Define $X^*$, the dual space, as the set of all continuous linear functionals defined on $X$ (that is, $\lambda \in X^*$ is $\lambda: X \to \mathbb{C}$).

Let $M$ be a vector subspace of $X$, and let $x \in X$. Then we have that $x \in \overline{M}$ if and only if for every $\Lambda \in X^*$ such that $\Lambda$ acting on $M$ is identically 0, $\Lambda x = 0$ as well.

2)

Let $K$ be a compact, Hausdorff space. Then, $X = C(K)$ is a normed linear space, with norm $\Vert f \Vert = \max \{ f(x) : x \in K \}$. 

Theorem: Let $ \Lambda \in X^*$. Then, there exists a regular, complex Borel measure on $K$ such that:

$$ \Lambda f = \int_K f(x) d\mu(x) $$

That is, we realize the dual space by integration against arbitrary complex-valued Borel measures.

Theorem:

Let $K \subset \mathbb{C}$ be compact. Let $\{ \alpha_1, \alpha_2,... \}$ be a set that, for each connected component of $C \subset K^c$ (note: we are taking $K^c = S^2 \setminus K$, so we could be using the point at infinity), there exists a $\alpha_i \in C$.

Let $\Omega$ be an open set such that $K \subseteq \Omega$. Let $f \in \calH(\Omega)$. Then, for any $\epsilon > 0$, there exists a rational function $R$, such that $R$ has a pole only at a subset of the $\alpha_i$ and

$$ |R(z) - f(z) |< \epsilon $$ for all $z \in K$.

Explicitly, we can express:

$$R(z) = \frac{P(z)}{\Pi_{i=1}^k (z- \alpha_{j_i})^{m_i}} $$ for some polynomial $P(z)$.

Proof:

Consider the normed space $C(K)$ of all continuous functions on $K$. Let $M$ be the subspace of $C(K)$ consisting of the restriction of rational functions with poles only at a subset of the $\alpha_i$ to $K$. Then, it is enough to say that the restriction of $f$ to $K$ belongs to the closure of $M$.

Thus, we need only show that for any $\Lambda \in C(K)^*$, such that $\Lambda M = 0$, that $\Lambda f = 0$. Let $\mu$ be a regular, complex Borel measure on $K$.

From last time, we have that there exists a cycle $\Gamma$ such that $\ind_\Gamma(z) = 1$ for all $z \in K$ and $\ind_\Gamma(z) = 0$ for all $z \in \Omega^c$.

Then, by the Cauchy integral formula, on $\Gamma$, we have that:

$$ f(z) = \frac{1}{2\pi i} \int_\Gamma \frac{f(\zeta) d\zeta}{\zeta - z}$$

Hence, we wish to consider:

$$ \int_K \frac{1}{2\pi i} \int_\Gamma \frac{f(\zeta) d\zeta}{\zeta - z} d\mu(z) $$

we notice that on $K \times \Gamma^*$, that $\frac{f(\zeta)}{\zeta - z} $ is a continuous function of $(\zeta, z)$. Thus, we may apply Fubini’s theorem to obtain:

$$  \frac{1}{2\pi i} \int_\Gamma f(\zeta) \int_K \frac{d\mu(z)}{\zeta - z} d\zeta $$

We examine the quantity $\int_K \frac{d\mu(z)}{\zeta - z} $ as a function of $z$, for any $\zeta \in S^2 \setminus K$.

We notice that of course, if $\zeta = \infty$, this integral vanishes. So if we needed to, we can look at this as $\mathbb{C} \setminus K$.

Let $\mathcal{O}$ be a component of $S^2 \setminus K$. Then, there exists an $\alpha_j \in \mathcal{O}$. First assume that $\alpha_j$ is not the point at infinity.

We notice that with respect to $\zeta$, $\int_K \frac{d\mu(z)}{\zeta - z}$ is holomorphic. Thus, we need only prove that this integral vanishes on some neighborhood in $\mathcal{O}$. Let $2r = \text{dist}(\alpha_j, K)$, and restrict ourselves to $\zeta \in D(\alpha_j, r)$. 

We notice that:

$$ \frac{1}{ \zeta  -z} = \frac{1}{(\zeta - \alpha_j) - (z - \alpha_j)} = - \frac{1}{z - \alpha_j} \cdot \frac{1}{1 - \frac{\zeta - \alpha_j}{z - \alpha_j}}$$

By the choice of $r$, we see that $\left|\frac{\zeta - \alpha_j}{z - \alpha_j}\right| \leq 1/2$. Then, we may rewrite our fraction as a geometric series with first term $1$, and common ratio $\frac{\zeta - \alpha_j}{z - \alpha_j}$.

Thus we have that:

$$ \frac{1}{ \zeta  -z} = - \frac{1}{z - \alpha_j} \cdot \frac{1}{1 - \frac{\zeta - \alpha_j}{z - \alpha_j}} = -\sum_{k=0}^\infty \frac{(\zeta - \alpha_j)^2}{(z - \alpha_j)^{k+1}} $$

Substituting this back into our integral:

$$\int_K \frac{d\mu(z)}{\zeta - z} = \int_K -\sum_{k=0}^\infty \frac{(\zeta - \alpha_j)^2}{(z - \alpha_j)^{k+1}}  d\mu(z)  = (\zeta - \alpha_j)^2  -\sum_{k=0}^\infty  \int_K \frac{d\mu(z)}{(z - \alpha_j)^{k+1}}$$

where we’ve interchanged sum and integral due to uniform convergence via Weierstrauss M-test.

However, for each $k$, we recall that $\mu$ is a measure such that the integral of rational functions in $z$ disappear. Thus, this is $0$ for $\zeta \in D(\alpha_j, r)$ and therefore $0$ on $\mathcal{O}$ by holomorphicity.

Now, suppose $\alpha_j = \infty$. Choose $r > 0$ such that $K \subset D(0,r)$, which exists due to compactness. Recalling what neighborhoods of $\infty$ look like, we choose $\zeta$ such that $|\zeta| \geq 2r$. In this case:

$$\frac{1}{\zeta - z} = \frac{1}{\zeta} \cdot \frac{1}{1 - \frac{z}{\zeta}} $$

Yet again, $\left| \frac{z}{\zeta}\right| \leq 1/2$, so we expand and do the exact same thing:

$$\frac{1}{\zeta - z} = \frac{1}{\zeta} \cdot \frac{1}{1 - \frac{z}{\zeta}} = \sum_{k=0}^\infty \frac{z^k}{\zeta^{k+1}} $$

So:

$$\int_K \sum_{k=0}^\infty \frac{z^k}{\zeta^{k+1}}  d\mu(z) = \frac{1}{\zeta^{k+1}} \sum_{k=0}^\infty \int_K  z^k d\mu(z) $$

And $z^k$ is certainly a rational function with pole at $\infty$, so this disappears.

Thus, we have that our original iterated integral disappears, and we are done.

Corollary:

Suppose that $S^2 \setminus K$ is connected. (i.e. $K$ is simply connected.) Let $f \in \calH(\Omega)$, where $\Omega \supset K$ is some open set. Then, there exists a sequence of polynomials $\{ P_n \}$ such that $P_n \to f$ uniformly on $K$.

Proof: Just use the theorem, and take $\alpha_j = \infty$. 

Finally:

Runge’s Theorem:

Let $\Omega \subset \mathbb{C}$ be open, and let $A$ be a set containing a point in each connected component of $S^2 \setminus \Omega$. Suppose that $f \in \calH(\Omega)$. Then, there exists a sequence of rational functions $R_n(z)$ such that $R_n(z)$ has poles on $A$, and that $R_n \to f$ uniformly on compact subsets of $\Omega$. 

\section*{April 5th}

Recall our discussion on the last theorem and corollary:

Theorem:

Let $K \subset \mathbb{C}$ be compact. Let $\{ \alpha_1, \alpha_2,... \}$ be a set that, for each connected component of $C \subset K^c$ (note: we are taking $K^c = S^2 \setminus K$, so we could be using the point at infinity), there exists a $\alpha_i \in C$.

Let $\Omega$ be an open set such that $K \subseteq \Omega$. Let $f \in \calH(\Omega)$. Then, for any $\epsilon > 0$, there exists a rational function $R$, such that $R$ has a pole only at a subset of the $\alpha_i$ and

$$ |R(z) - f(z) |< \epsilon $$ for all $z \in K$.

Corollary:

Suppose that $S^2 \setminus K$ is connected. (i.e. $K$ is simply connected.) Let $f \in \calH(\Omega)$, where $\Omega \supset K$ is some open set. Then, there exists a sequence of polynomials $\{ P_n \}$ such that $P_n \to f$ uniformly on $K$.

Question:

If $K$ has holes, can a function $f$, which is holomorphic on a neighborhood of $K$, still be uniformly approximated by polynomials?

No!

Suppose $S^2 \setminus K$ has multiple components. Let $U$ be a bounded component of $S^2 \setminus K$, and fix an $\alpha \in U$. Consider the function $f(z) = \frac{1}{z - \alpha}$, holomorphic on $\mathbb{C} \setminus \{ \alpha \}$, which is a neighborhood of $K$. Suppose that we can find a polynomial approximation to $f$.

Let $r = \max_{z \in K} |z - \alpha|$. Then, $r$ is a finite postiive real value, since $K$ is compact. Let $\epsilon = \frac{1}{r+1}$. By hypothesis, we may find a polynomial such that $\left|P(z) - \frac{1}{z - \alpha}\right|  < \frac{1}{r+1}$. Hence:

$$ \left| (z- \alpha) P(z) - 1 \right|< \frac{|z - \alpha|}{r+1} < 1 $$

for all $z \in K$.

In particular, since $U$ is a bounded component of the compliment, we have that $\partial U \subset K$, and thus we have that we may apply the max modulus principle on $(z - \alpha) P(z) - 1$ to obtain:

$$ |(z - \alpha) P(z) - 1 | < 1$$

But, when $z = \alpha$, we get that $|1| < 1$, a contradiction. Thus, there is no polynomial that approximates $f$ on $K$ given this hole.

Back to Runge’s Theorem:

Let $\Omega \subset \mathbb{C}$ be open, and let $A$ be a set containing a point in each connected component of $S^2 \setminus \Omega$. Suppose that $f \in \calH(\Omega)$. Then, there exists a sequence of rational functions $R_n(z)$ such that $R_n(z)$ has poles on $A$, and that $R_n \to f$ uniformly on compact subsets of $\Omega$. 

Proof:

Recall, we can find a sequence of compact sets $K_n$ such that the following hold:

(a) $K_n \subset \text{int}(K_{n+1})$

(b) each component of $S^2 \setminus K_n$ contains a component of $S^2 \setminus \Omega$

(c) $\cup K_n = \Omega$.

Note that because (b), for a good choice of prescribed set $A = \{ \alpha_1,...,\}$, we can find one that applies for every $K_n$. 

By our theorem, there exists a rational function $R_n$ with poles on $A$ such that $|R_n - f| < 1/n$ for all $z \in K_n$.

Now, let $K$ be an arbitrary compact subset of $\Omega$. Since by condition (c), it implies that $\cup \text{int}(K_n) = \Omega$, this implies that this is an open cover of $K$, and thus since these are nested properly, this implies that for at least one n, $K \subset K_n$. Hence, for $n \geq N$, we have that $| f - R_n | < 1/n$ for all $z \in K \subset K_n$ and we are done.

Mittag-Leffler Theorem:

Let $\Omega$ be open. Let $A \subset \Omega$ be a set without limit points in $\Omega$, which is of course at most countable. Suppose to each $\alpha \in A$, there is an associated integer $m(\alpha) \in \mathbb{Z}$, and a rational function 
$$P_\alpha(z) = \sum_{j=1}^{m(\alpha)} \frac{c_{j,\alpha}}{(z-\alpha)^j} $$

Then there exists a a meromorphic function $f$ on $\Omega$ such that the poles precisely at $A$, such that at each $\alpha \in A$, the principal part of the pole is exactly $P_\alpha(z)$.

Proof:

Introduce the nested sequence as before, that is, compact sets that are nested in the nice way with interiors. Designate $A_1 = A \cap K_1$, and define for $n \geq 2$, $A_n = A \cap (\text{int}(K_{n}) \setminus K_{n-1}) $.

Now, for each $n$, define $Q_n(z) = \sum_{\alpha \in A_n} P_{\alpha}(n) $, which we notice that this must be a finite sum, due to the compactness of $K_n$. For each $n \geq 2$, $Q_n$ is holomorphic in some neighborhood of $K_{n-1}$. Hence, on $K_{n-1}$, we may find a rational function with poles on $S^2 \setminus \Omega$ such that $| R_n - Q_n | < 1/2^n$ for all $z \in K_{n-1}$. 

Define:

$$f(z) = Q_1(z) + \sum_{n=2}^\infty (Q_n(z) - R_n(z))$$

for all $z \in \Omega$.

First, fix some $N$. Then, on $K_N$, we have that:

$$ f(z) = (Q_1 + ... Q_N) + (P_1 + ... + P_N) + \sum_{i = N+1}^\infty (Q_i - P_i) $$.

We notice, that for each $Q_i - P_i$ for each $i \geq N+1$, this is holomorphic on a neighborhood of $K_N$, each with modulus less than $1/2^{i}$. Thus, this sum converges, and the infinite sum is holomorphic on a neighborhood of $K_N$. 

Moreover, the sum $P_1 + ... + P_N$ is holomorphic, because the poles are outside of $\Omega$, and $K_N \subset \Omega$. 

So of course $f - (Q_1 + ... + Q_N)$ is holomorphic on a neighborhood of $K_N$, and the poles of $Q_1 + ... + Q_N$ are in $K_N$. Hence, $f$ must be meromorphic on some neighborhood of $K_N$, with poles in $A_1 \cup .. \cup A_N$. Since the choice of $N$ was arbitrary, this extends to $\Omega$.

Analytic Notions of Simple Connectivity:

Theorem:

Let $\Omega$ be a planar open set. The following are equivalent:

(1) $\Omega$ is homeomorphic to $U = \{ (x,y) : x^2 + y^2 < 1 \}$

(2) $\Omega$ is simply connected

(3) $\ind_\gamma(\alpha) = 0$ for every closed path in $\Omega$ and every $\alpha \not \in  \Omega$

(4) $S^2 \setminus \Omega$ is connected

(5) There is an approximation for every $f \in \calH(\Omega)$ by a sequence of polynomials $R_n \to f$ uniformly on compact subsets of $\Omega$.

(6) For every $f \in \calH(\Omega)$, and every closed path $\gamma \in \Omega$, Cauchy’s theorem holds, that is, $\int_\gamma f(z) dz = 0$. 

(7) For every $f \in \calH(\Omega)$, there exists a function $F \in \calH(\Omega)$ such that $F’ = f$. 

(8) For every $f \in \calH(\Omega)$ such that $f \not = 0$ on $\Omega$, there exists a $g \in \calH(\Omega)$ such that $f = e^{g}$

(9) For every $f \in \calH(\Omega)$ such that $f \not = 0$ on $\Omega$, there exists a $g \in \calH(\Omega)$ such that $f = g^2$

Some proofs:

(1) $\to$ (2)

Obvious, since homeomorphism preserve connectedness. Otherwise, we could push an open set and show $U$ to not be connected.

(2) $\to$ (3)

Obvious, since if $\Omega$ simply connected, then every path homotopic to the constant path, homotopy preserves indices, and the index of any point with respect to the constant path is $0$.

(3) $\to$ (4)

Suppose not. Then $S^2 \setminus \Omega$ is the union of at least 2 disjoint closed sets $K, E$ such that at least one, $K$ is compact. Define $\Omega’ = K \cup \Omega$. Of course, this must be open. Hence, we may find a cycle $\Gamma$ in $\Omega’$ such that $\Gamma^* \cap K = \emptyset$ and $\ind_\Gamma(z) = 1$ for all $z \in K$. But then, $\Gamma$ is actually also a cycle in $\Omega$, contradiction.


\end{document}
