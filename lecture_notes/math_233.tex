\documentclass[10pt]{article}
\usepackage{graphicx}
\usepackage{pst-node,pst-tree,pstricks}
\usepackage{amssymb,amsmath}
\usepackage{hyperref}
\usepackage{pst-node}
\usepackage{mathtools}

% environments shortcuts
\newcommand{\beq}{\begin{equation}}
\newcommand{\eeq}{\end{equation}}
\newcommand{\beqa}{\begin{eqnarray}}
\newcommand{\eeqa}{\end{eqnarray}}
\newcommand{\beqas}{\begin{eqnarray*}}
\newcommand{\eeqas}{\end{eqnarray*}}
\newcommand{\codim}{\text{codim}}

\newcommand{\bit}{\begin{itemize}}
\newcommand{\eit}{\end{itemize}}
\newcommand{\bits}{\begin{itemize*}}
\newcommand{\eits}{\end{itemize*}}
\newenvironment{enumerate*}{\begin{enumerate}
    \setlength{\topsep}{0ex}
    \setlength{\parskip}{0ex}
    \setlength{\partopsep}{-1ex}
    \setlength{\itemsep}{0pt}
    \setlength{\parsep}{0ex}}
{\end{enumerate}}

\newcommand{\benum}{\begin{enumerate*}}
\newcommand{\eenum}{\end{enumerate*}}
%\newcommand{\benums}{\begin{enumerate*}}
%\newcommand{\eenums}{\end{enumerate*}}
\newcommand{\mybullet}{$\bullet$}

% math mode commands

\newcommand{\fracpartial}[2]{\frac{\partial #1}{\partial  #2}}
\newcommand{\rrr}{{\mathbb R}}
\newcommand{\bigOO}{{\cal O}}
\newcommand{\dataset}{{\cal D}}

\newcommand{\X}{\mathbf{X}}
\newcommand{\calB}{\mathcal{B}}
\newcommand{\calF}{\mathcal{F}}
\newcommand{\calG}{\mathcal{G}}
\newcommand{\calN}{\mathcal{N}}
\newcommand{\calT}{\mathcal{T}}
\newcommand{\calH}{\mathcal{H}}

\newcommand{\trace}{\operatorname{trace}}
\newcommand{\diag}{\operatorname{diag}}
\newcommand{\sign}{\operatorname{sgn}}
\newcommand{\onevector}{{\mathbf 1}}
\newcommand{\bbone}[1]{{\mathbf 1}_{[#1]}}

\newcommand {\argmax}[2]{\mbox{\raisebox{-1.7ex}{$\stackrel{\textstyle{\rm #1}}{\scriptstyle #2}$}}\,}  % to replace with the amsmath construction

\newlength{\picwi}
\newcommand{\backskip}{\hspace{-2.5em}} % how much to skip back for an empty item?

% Set up some colors
\definecolor{myblue}{rgb}{0.14,0.11,0.49}
\definecolor{myred}{rgb}{0.74,0.1,0.05}
\definecolor{mygreen}{rgb}{0.,0.52,0.32}
\definecolor{myyellow}{rgb}{0.96,0.92,0.13}
\definecolor{myorange}{rgb}{0.7,0.41,0.1}
\definecolor{mypurple}{rgb}{0.51,0.02,.8}
\definecolor{mygray}{rgb}{0.6,0.6,0.6}

\newcommand{\myblue}[1]{\textcolor{myblue}{#1}}
\newcommand{\myred}[1]{\textcolor{myred}{#1}}
\newcommand{\mygreen}[1]{\textcolor{mygreen}{#1}}
\newcommand{\myorange}[1]{\textcolor{myorange}{#1}}
\newcommand{\myyellow}[1]{\textcolor{myellow}{#1}}
\newcommand{\mypurple}[1]{\textcolor{mypurple}{#1}}
\newcommand{\mygray}[1]{\textcolor{mygray}{#1}}


% Stlyle stuff
% notes are for students , \notes with \mmp{} are for me

\newcommand{\comment}[1]{}
\newcommand{\mmp}[1]{\emph MMP: {#1}}
\newcommand{\mydef}[1]{\myred{\bf {#1}}}
\newcommand{\myemph}[1]{\mygreen{ {#1}}}
\newcommand{\mycode}[1]{\myblue{\tt {#1}}}
\newcommand{\myexe}[1]{{\small \mypurple{Exercise} {#1}}}

\newcommand{\reading}[2]{{\small \myemph{{\bf Reading} CRLS:} {#1}, \myemph{Python APPB4AWD} {#2}}}


\begin{document}
\begin{Large}
\centerline{Math 233}
\centerline{Lecture Notes}  % lecture number here
\centerline{\bf }       % lecture title here
\centerline{}      %date here
\end{Large}


\vspace{2em}
\section*{Jan 18th}

First, settle some notation:

Disks:

Let $ a \in \mathbb{C}$, and $r > 0$. Denote the open disk of radius $r$, centered at $a$ as:

$$ D(a,r) = \{ z \in \mathbb{C} : | z - a | < r \}$$

Similarly, denote the closed disk as:

$$ \overline{D}(a,r) =  \{ z \in \mathbb{C} : | z - a | \leq r \}$$

Connected sets and components:

Let $X$ be a topological space. Call $E \subseteq X$ disconnected when there exist non-empty subsets $A,B \subset E$ such that $A\cup B = E$ and $\overline{A} \cap B = A \cap \overline{B} = \emptyset$. We say that $A,B$ is said to separate $E$.

We call $E \subseteq X$ if it does not admit a separation into subsets.

Now, suppose $E \subseteq X, x_0 \in E$. Then, if we have $A \subset E$ such that $x_0 \in A$, and $A$ connected, then $\cup A$ is connected. Moreover, since we have the (potentially uncountable) union over all connected sets, this must be the largest such connected sets that includes $x_0$. Call this maximal connected set the component of $E$ that contains $x_0$. Call the collection of such connected subsets of $E$ over all $x_0$ the connected components of $E$. It should be clear that the set $E$ must be the disjoint union of the connected components.

Now, let $E \subseteq \mathbb{C}$ be an open set. Let $a \in E$. Because $E$ is open, there exists $r > 0$ such that $D(a,r) \subseteq E$. Then, $D(a,r)$ is in the connected component containing $a$. Thus, the components of $E$ are open.

Note: we will use $\Omega$ to denote open sets in $\mathbb{C}$.

Call an open, connected subset of $\mathbb{C}$ a region.

Derivatives:

Suppose that $\Omega \subseteq \mathbb{C}$ is an open set, and we have $f: \Omega \to \mathbb{C}$. Let $z \in \Omega$. Then, if it exists, we define the derivative of $f$ at $z$ as:

$$f’(z_0) = \lim_{z \to z_0} \frac{f(z) - f(z_0)}{z - z_0}$$

That is, for the limit to exists, we must have that:

$$  \lim_{z \to z_0}  \left| \frac{f(z) - f(z_0)}{z - z_0} - f’(z_0) \right|  = 0 $$

We note that in the $\mathbb{R}^2$ sense of derivatives, complex differentiable implies real differentiable, but not the converse, because the extra structure from the Cauchy-Riemann equations.

Theorem:

If $f’(z_0)$ exists, then $f$ is continuous at $z_0$.

Let $\Omega \subset \mathbb{C}$ be an open set, $f: \Omega \to \mathbb{C}$. If $f$ is differentiable on all of $\Omega$, then we call $f$ holomorphic (or analytic).

We denote the set of all holomorphic functions on $\Omega$ by $\mathcal{H}(\Omega)$.

Let $f,g \in \mathcal{H}(\Omega)$. Then $f+g, fg \in \mathcal{H}(\Omega)$ and $f/g \in \mathcal{H}(\Omega)$ if $0 \not \in g(\Omega)$. The normal rules hold: product rule, quotient rule, etc.

Chain rule:

Suppose that $g \in \mathcal{H}(\Omega)$, $g(\Omega) \subset \Omega_1$, and $f \in \mathcal{H}(\Omega_1)$. Then we claim that $h = f \circ g \in \mathcal{H}(\Omega)$ and $h’ = f’(g(z_0)) * g’(z_0)$

Proof:

Let $z_0 \in \Omega, w_0 = g(z_0) \in \Omega_1$. Since $f’(w_0)$ exists, because $f$ is holomorphic, define $\phi: \Omega_1 \to \mathbb{C}$ as:

$$ \phi(w) = \begin{cases}
\frac{f(w) - f(w_0)}{w - w_0} & \text{ if } w \in \Omega \setminus \{ w_0 \} \\ f’(w_0) & \text{ if } w = w_0 \\ \end{cases}$$

We see that by the definition of the derivative, that $\phi$ is continuous on $\Omega_1$, and we have that $f(w) - f(w_0) = \phi(w) (w - w_0)$. However, we have that $w_0 = g(z_0), w = g(z)$, so we have that:

$$ \frac{f(g(z)) - f(g(z_0)}{z - z_0} = \phi(g(z)) \frac{g(z) - g(z_0)}{z - z_0} $$

If we take the limit of both sides as $z \to z_0$, we see that the right hand side is exactly $\phi(g(z_0)) g’(z_0)$. Further, the left hand side is exactly the definition of the derivative of $h$ at $z_0$. Hence, $h’(z_0)$ exists, and equals $f’(g(z_0)) g’(z_0)$.

Power rule:

By direct computation, we have that $\frac{d}{dz}(z^n) = n z^{n-1}$. This holds for all $n \in \mathbb{Z}$.

Definition:

Let $f: \mathbb{C} \to \mathbb{C}$. If $f$ is holomorphic, then we call $f$ entire. Example: $1, z, z^2$ are entire.

Functions representable by Power Series:

Consider a power series $\Sigma_{n=0}^\infty c_n (z-a_n)^n$.

By the root test, we can say this series has radius of convergence $R$, where $\frac{1}{R} = \limsup_{n \to \infty} | c_n|^{1/n}$. In particular, the series converges absolutely for all $z \in D(a,R)$. Further, the series diverges for all $z \not \in \overline{D}(a,R)$. Now, for $0 \leq r < R$, the power series converges uniformly via the Weierstrauss M-test.

Now, let $f: \Omega \to \mathbb{C}$. We say that $f$ is representable by power series on $\Omega$ provided that for any disk $D(a,r) \subseteq \Omega$, $f$ may be represented by a power series $f(z) = \Sigma_{n=0}^\infty c_n (z-a)^n, z \in D(a,r)$.

Theorem:

Suppose that $f$ is representable by a power series on $\Omega$. Then $f$ is holomorphic on $\Omega$, that is, $f \in \mathcal{H}(\Omega)$. Moreover, if $D(a,r) \subset \Omega$, and $f(z) = \Sigma_{n=0}^\infty c_n (z-a)^n$ on this disk, then for any $z \in D(a,r)$, $f’(z) = \Sigma_{n=1}^\infty nc_n(z-a)^n$. We notice that this is also a power series; as such, $f’$ is also representable by a power series (and, thus, holomorphic).

Proof omitted, long computation, refer to lecture notes or Rudin.

If $f$ is representable by power series on $\Omega$, then $f’$ is also representable, and thus holomorphic. Moreover, $f^{(n)}$ is holomorphic. By term-by-term differentiation, we have that:

$$ f^{(k)} = \Sigma_{n=k}^\infty n(n-1)...(n-k+1) c_n (z-a)^{n-k}$$

Further, we have that at $z = a$, that $f^{(k)}(a) = k! c_k \implies c_k = f^{(k)}(a)/k!$. So we can find every coefficient by successively taking derivatives. 

\end{document}
