\documentclass[10pt]{article}
\setlength{\parskip}{0.25\baselineskip}
\usepackage[margin=1in]{geometry} 
\usepackage{amsmath,amsthm,amssymb, graphicx, multicol, array}
\usepackage[font=small,labelfont=bf]{caption}

\newcommand{\supp}{{\text{supp}}} 
\newcommand{\bv}{{\text{BV}}}
\newcommand{\ac}{{\text{AC}}}

\newenvironment{problem}[2][]{\begin{trivlist}
\item[\hskip \labelsep {\bfseries #1}\hskip \labelsep {\bfseries #2.}]}{\end{trivlist}}

\begin{document}
 
\title{Homework \#3}
\author{Eric Tao\\
Math 123: Homework \#3}
\maketitle

\begin{problem}{Question 1}
Let $x,y \in \mathbb{R}^{d \times 1}$. Prove that $xy^T \in \mathbb{R}^{d \times d}$ has at most rank 1.

\end{problem}
\begin{proof}[Solution]

First, assume we're not in the degenerate case $x = 0$ or $y = 0$, as if either are true, then $xy^T = 0$, the 0 matrix.

Here, we recall that, by definition:

$$(xy^T)_{ij} = \sum_{k=1}^1 x_{ik}y_{kj} = x_i y_j$$ where I drop the second index because, of course, these are vectors. Fix some $i$. Then, looking at the $i$-th row, we notice that we may factor $x_i$ from every term. In particular, as a vector, the $i$-th row looks like:

$$ \begin{pmatrix} x_i y_1 & x_i y_2 & ... & x_i y_d \end{pmatrix} =  x_i \begin{pmatrix} y_1 & y_2 & ... & y_d \end{pmatrix} $$

where we notice that the $i$-th row is a multiple of $y$, as a row vector. Since the choice of $i$ was completely arbitrary, this process can be done for every row, and thus every row vector is a multiple of $y$. Thus, the row space has dimension 1, and since we're a square matrix, we have that the dimension of the row space is equal to that of the column space, and the rank is 1.

\end{proof}

\begin{problem}{Question 2}
Prove that the Euclidean dot product $\langle x, y \rangle = \sum_{i=1}^n x_i y_i, x, y \in \mathbb{R}^n$ is an inner product, where an inner product is a binary function from a (real-valued) vector space $V$ to a field $F$, $\langle \cdot, \cdot \rangle: V \times V \to F$ such that the following hold (in the context of a real vector space):

(a) For all $x,y \in V$, $\langle x,y \rangle =\langle y,x \rangle$

(b) For all $x,y \in V$, $\langle \alpha x,y \rangle =\alpha \langle x,y \rangle$

(c) For all $x,y, z \in V$, $\langle  x + y,z \rangle =\langle x,z \rangle + \langle y,z \rangle$

(d) For all $x\in V$, $\langle x,x \rangle \geq 0$ and $\langle x,x \rangle = 0 \iff x = 0$

\end{problem}

\begin{proof}[Solution]

(a) 

This should be clear by the commutativity of real numbers:

$$ \langle x, y \rangle = \sum_{i=1}^n x_i y_i = \sum_{i=1}^n y_i x_i = \langle y, x \rangle$$

(b)

This should be clear by how multiplication distributes over addition:

$$ \langle \alpha x, y \rangle = \sum_{i=1}^n (\alpha x_i) y_i = \alpha \sum_{i=1}^n y_i x_i = \alpha \langle y, x \rangle$$

(c) 

Same as (b), follows from distributive property of multiplication:

$$ \langle x+y, z \rangle = \sum_{i=1}^n (x_i + y_i) z_i = \sum_{i=1}^n x_i z_i + y_i z_i = \sum_{i=1}^n x_i z_i + \sum_{j=1}^n y_j z_j  = \langle x, z \rangle + \langle y, z \rangle$$

(d) 

First, we consider the expansion of $\langle x ,x \rangle$:

$$\langle x ,x \rangle = \sum_{i=1}^n x_i x_i = \sum_{i=1}^n x_i^2$$

Since we have that for all $x_i \in \mathbb{R}$, $x_i^2 \geq 0$, we have that $ \sum_{i=1}^n x_i^2$ is a sum of non-negative numbers, and thus must be at least 0. Thus, $\langle x ,x \rangle \geq 0$

It is obvious that if $x = 0$, then $\langle x, x \rangle = \sum_{i=1}^n 0 * 0  =0$. Now, suppose $\langle x, x \rangle = 0$. Then, we have that $ \sum_{i=1}^n x_i^2 = 0$. Since, again, these are non-negative numbers, this can only be 0 if $x_i = 0$ for all $i$. But, if $x_i  = 0$ for all $i$, $x$ is the $0$ vector.


\end{proof}

\begin{problem}{Question 3}

(a) Prove that $\langle x,y\rangle_M = x M y^T$ satisfies the properties of an inner product if $M$ is positive definite.

(b) Show that $\langle x,y\rangle_M$ need not be an inner product if $M$ is positive semi-definite.

\end{problem}

\begin{proof}[Solution]

(a)

Suppose $M$ is positive definite. We check each property in turn:

$$\langle x,y\rangle_M =  x M y^T = (x M y^T)^T = y M^T x^T = y M x^T = \langle y,x\rangle_M$$

where we use the fact that since $x M y^T$ is a scalar, and thus may be intepreted as a $1 \times 1$ matrix, it must be symmetric, and that $M$ being positive-definite means that $M$ is symmetric.

$$\langle \alpha x,y\rangle_M =  (\alpha x) M y^T = \alpha(x M y^T) = \alpha \langle x,y\rangle_M$$

where we use the fact that we can pull scalars out from matrix multiplication.

$$ \langle x+y, z \rangle_M  = (x+y) M z^T = xMz^T  + yMz^T = \langle \alpha x,z\rangle_M + \langle \alpha y,z\rangle_M $$

where we use the fact that matrix multiplication distributes over matrix addition.

and the last property comes by the definition of positive definite, as by definition, $x M x^T \geq 0$ for all $x$, and is 0 if and only if $x = 0$. So we notice that:

$$ \langle x, x \rangle_M = x M x^T \geq 0$$

and $$ \langle x,x \rangle = 0 \iff xM x^T = 0 \iff x = 0$$

So,  $\langle x,y\rangle_M$ defines an inner product.

(b)

We show a counter example. Consider the matrix:

$$A =   \begin{pmatrix} 1 & 0 \\ 0 & 0 \end{pmatrix}$$

It should be clear that this matrix is positive semi-definite because we may compute the characteristic polynomial and eigenvalues as $(1-\lambda) \lambda = 0 \implies \lambda = 0, 1 \geq 0$, and since $A$ is symmetric, with non-negative eigenvalues, it must be positive semi-definite.

However, this matrix does not define a inner product. Consider the vector $ v = \begin{pmatrix} 0 & 1 \end{pmatrix}$. We have that:

$$  v A v^T = \begin{pmatrix} 0 & 1 \end{pmatrix} \begin{pmatrix} 1 & 0 \\ 0 & 0 \end{pmatrix} \begin{pmatrix} 0 \\ 1\end{pmatrix} =   \begin{pmatrix} 0 & 1 \end{pmatrix}\begin{pmatrix} 0 \\ 0\end{pmatrix} = 0$$

Thus, because we found a non-0 $v$ such that $ v A v^T = 0$, we have that $v A v^T  = 0 \nRightarrow v = 0$, and thus we do not satisfy the biconditional.

\end{proof}

\begin{problem}{Question 4}

Let $x_1,...,x_n \in \mathbb{R}^d$. Fix some positive integer $K$. Let $C_1,...,C_K$ be a partition of the data with centroids $\mu_1,...,\mu_K$. Let

$$ F(C_1,...,C_k)  \sum_{k=1}^K \sum_{x_i \in C_K} \Vert \mu_k - x_i \Vert_2^2 $$

(a) Prove that, for a fixed $K$, $F$ achieves a minimum value.

(b) What is the minimum value if $K = n$?

\end{problem}

\begin{proof}[Solution]

(a)

Fix some $K \in \mathbb{N}$. Since $n$ is a fixed natural number, we have finitely many data points. In particular, if we have finitely many data points, then we can only have finitely many partitions of $n$ into $K$ subsets. Then, if we consider the values of $F_i$, where we index a partition $\Gamma_i$ by $i$, this is a set of non-negative real numbers, finite. By the least upper bound property of the real numbers, this must have an infimum since this is bounded below by $0$, and further, since this is a finite set, it must achieve the infimum, since finite sets do not have limit points. Thus, $F$ must achieve a minimum for fixed values of $K$.

(b)

If $K= n$, then we allow only the partition$C_1 = \{x_1\}, C_2 = \{ x_2 \} ... C_n = \{ x_n \}$ up to relabeling, since we have $n$ data points and $n$ boxes. Then, the centroid of each is exactly $\mu_j = x_j$, so then we have that for each $C_n$, that $$\sum_{x_i \in C_j  } \Vert \mu_j - x_i \Vert_2^2 = \Vert x_j - x_j \Vert_2^2 = 0$$ 

Thus, $F = \sum_{K=1}^n 0 = 0$, and the minimum and only value of $F$ if $K = n$ is $0$. And since the minimum value of a sum of non-negative real values is $0$, this must be the minimum value.


\end{proof}

\begin{problem}{Question 5}
 
Run the MATLAB script 'Kmeans\_Gaussians'.

(a) Run $K$-means with $K=2, 100$ replicates. Show the output visually.

(b) Plot the error of the $K$-means functional as a function of the number of iterations. Is there convergence?

(c) Do the clusters agree with your intuition?

\end{problem}

\begin{proof}[Solution]


\end{proof}

\begin{problem}{Question 6}
 
Run the MATLAB script 'Kmeans\_Ellipses'.

(a) Run $K$-means with $K=2, 100$ replicates. Show the output visually.

(b) Plot the error of the $K$-means functional as a function of the number of iterations. Is there convergence?

(c) Do the clusters agree with your intuition?

\end{problem}

\begin{proof}[Solution]


\end{proof}

\end{document}