\documentclass[10pt]{article}
\setlength{\parskip}{0.25\baselineskip}
\usepackage[margin=1in]{geometry} 
\usepackage{amsmath,amsthm,amssymb, graphicx, multicol, array}
\usepackage[font=small,labelfont=bf]{caption}

\newcommand{\supp}{{\text{supp}}} 
\newcommand{\bv}{{\text{BV}}}
\newcommand{\ac}{{\text{AC}}}

\newenvironment{problem}[2][]{\begin{trivlist}
\item[\hskip \labelsep {\bfseries #1}\hskip \labelsep {\bfseries #2.}]}{\end{trivlist}}

\begin{document}
 
\title{Homework \#1}
\author{Eric Tao\\
Math 237: Homework \#1}
\maketitle

\begin{problem}{Question 2}

Let $E$ be a normed vector space over $\mathbb{R}$. We call a subspace $H \subseteq E$ a hyperplane if the quotient space $E/H$ has dimension 1.

2.1) Show that the closure of any subspace of $E$ is also a subspace of $E$. Conclude that a hyperplane $H$ is either closed or dense in $E$.

2.2) Let $u$ be a linear functional on $E$. Prove that $u$ is discontinuous if and only if there exists a sequence $\{ x_n  \}$ in $E$ that converges to 0 such that $u(x_n) = 1$ for all $n$.

2.3) Let $x_0 \in E$ be a unit norm vector, and define $H$ as the complement of the span of $x_0$. Show that every $x \in E$ can be uniquely decomposed as $x = t(x)x_0 + y(x)$ where $t: E \to \mathbb{R}$, and $y: E \to H$, linear. Further, prove that $t, y$ are continuous if and only if $H$ is closed.

2.4) Let $u$ be a linear functional on $E$. Prove that $u$ is continuous if and only if the kernel of $u$, $H$, is closed.

\end{problem}
\begin{proof}[Solution]

2.1)

Let $S \subset E$ be a vector subspace, and denote $\overline{S}$ as its closure. Of course, if $S$ is closed, then $\overline{S} = S$, and therefore, the closure is a vector space.

Now, suppose $S \not= \overline{S}$. Then, we may describe $\overline{S}$ as the union of $S$ and the limit points of $S$ in $E$. Since $0 \in S \subset \overline{S}$, we need only show that $\overline{S}$ is closed under addition and scalar multiplication.

To check addition, we may discard the case where $x, y \in S$, as $S$ is already a vector space. Thus, suppose $x$ is a limit point of $S$, and $y \in S$. Since $x$ is a limit point, there exists a sequence $\{ x_n \} \subset S$ such that $\lim_{n \to \infty} x_n = x$. Then, consider the sequence $\{ x_n + y \}$. Clearly, since $x_n \to x$, we have that $\lim_{n \to \infty} x_n + y = x + y$. Since $x \not \in S$, being a limit point, $x + y$ cannot be in $S$, and hence, is a limit point of $S$. Hence, $x + y \in \overline{S}$. Without too much trouble, we see that the same argument holds when $y$ is a limit point, where we leverage the sequences $\{ x_n \}, \{ y_n \}$ and consider their sum $\{ x_n + y_n \}$.

Similarly, we can just check $x \not \in S$ for scalar multiplication; if $x$ is a limit point, $\{  x_n \} \to x$, then of course $\{ a x_n \} \to ax $ for $a \in \mathbb{R}$, and therefore, if $x \in S$, $ax \in S$. Thus, we have that $\overline{S}$ is closed under addition and scalar multiplication, and contains 0. Therefore, $\overline{S}$ is a vector subspace of $E$.

Now, let $H$ be an arbitrary hyperplane. Of course, if $H$ is closed, $\overline{H} = H$. So suppose $H$ is not closed, and therefore $H \subset \overline{H}$ is a proper subset. Looking at $E/H$, since this has dimension $1$, fixing some $z \in E \setminus H$, we may identify $E/H$ as the span of $z + H$. Since $\overline{H}$ is a proper superset of $H$, there exists a $z' \in \overline{H}$ that does not belong to $H$. Under the projection into $E/H$, $\pi(z') = \alpha z + H$ for some $\alpha \in \mathbb{R} \setminus 0$, as otherwise, $z' \in H$, hence there exists a $h \in H$ such that $\alpha z + h = z'$ in $E$. Rearranging, this implies that $z = \frac{1}{\alpha}(z' - h)$. But, since $\alpha \in \mathbb{R}$, $z', h \in \overline{H}$, this implies that $z \in \overline{H}$. Hence, we have that $\overline{H} = E$. Since the closure of $H$ in $E$ is $E$, we have that $H$ is dense in $E$, and we are done.

2.2)

First, we prove the forward direction. Suppose $u$ is discontinuous. In particular then, it is discontinuous at the identity, since $u$ is continuous if and only if it is continuous at the origin. Then, there exists some fixed $\epsilon > 0$, such that we may find a $x_n$ with that $\Vert x_n - 0 \Vert < 1/n$ and with $| u(x_n)  - u(0) | = u(x_n) > \epsilon$. Now, consider the modified sequence $\{ \frac{x_n}{u(x_n)} \}$. We notice that since $u(x_n) > \epsilon$, that term by term, this sequence is smaller in norm than $\{ \frac{x_n}{\epsilon} \}$. Furthermore, since $x_n \to 0$, $\frac{x_n}{\epsilon} \to 0$, since $\Vert \frac{x_n}{\epsilon} \Vert = \frac{1}{\epsilon} \Vert x_n \Vert < \frac{1}{\epsilon} \frac{1}{n}$, which goes to 0 as $n \to \infty$ for a fixed $\epsilon$. Thus, $\frac{x_n}{\epsilon} \to 0$ and therefore, $\{ \frac{x_n}{u(x_n)} \} \to 0$. On the other hand though, since $u$ is linear, $u\left( \frac{x_n}{u(x_n)} \right) = \frac{1}{u(x_n)} u(x_n) = 1$, as desired.

On the other hand, the backwards direction follows fairly easily. Since we have a sequence $\{ x_n \} \to 0$ with $u(x_n) = 1$ for all $n$, of course, $u$ is discontinuous at $0$, because for $\epsilon = 1/2$, for any $\delta > 0$, we can find an $x_n$ such that $\Vert x_n \Vert < \delta$, but by definition, $u(x_n) = 1 > \epsilon$. Hence, $u$ is discontinuous at some point, and thus discontinuous.

2.3)

By the description of $H$, we can identify $E/H$ as spanned by $x_0$. Then, for any $x \in E$, we can consider its image under the projection $\pi: E \to E/H$, $\pi(x) = t(x)x_0  + H$, for some map $t: E \to \mathbb{R}$; moreover, since $\pi$ is linear, so must be $t$. Then, we may identify $y(x) = x - t(x)x_0$. We notice that $\pi(y(x)) = \pi(x - t(x)x_0) = \pi(x) - t(x)\pi(x_0) = t(x) x_0 + H - t(x) x_0 + H = 0 + H$, hence $y(x) \in H$.

We see this decomposition as unique, as $x$ maps to exactly one coset of $E/H$ due to the injectivity of left addition, so $t$ is distinct. The uniqueness of $y$ follows from the uniqueness of $t$. We also notice in what follows, that $t, y$ are either both continuous or both discontinuous due to the definition of $y$.

Now, suppose $t, y$ are continuous. Then, we can identify $H$ as the inverse image $t^{-1}(0)$. Since $t$ is continous, $t^{-1}(0)$ is closed, hence $H = t^{-1}(0)$ is closed.

On the other hand, suppose $t, y$ discontinuous. Then, by 2.2, there exists a sequence $\{ x_n \} \subset E$ such tthat $t(x_n) = 1$, and $x_n \to 0$. By the previous work, we can reexpress this sequence via our decomposition as:

$$ x_n = t(x_n)x_0  + y(x_n) = x_0 + y(x_n) $$

But, since $x_n \to 0$, this implies that $y(x_n) \to -x_0$. Then, $-x_0 \in \overline{H}$, and hence from the work in 2.1, since $\overline{H}$ is a vector subspace, $H$ is dense, i.e. not closed. Therefore, by the contrapositive, $H$ being closed implies that $t$ and thus $y$ is continuous.

2.4)

Let $u$ be a linear functional on $E$.

If $u$ is trivial, then the result is trivial, as then the kernel of $u$ is $E$, always closed, and the trivial map is continuous, because then the preimage of 0 is all of $E$.

Now, suppose $u$ is not trivial. Then, because the kernel has codimension $1$, looking at $E/H$, we may find a representative $z + H$ such that $E/H$ is the span of $z + H$. Then, via 2.3, we may decompose any $x \in E$ as $x = t(x)z + y(x)$.

Then, $u$ acting on any $x$ has the action of $u(x) = u(t(x)z + y(x)) = t(x)u(z)$. Since $u(z)$ is a constant, the continuity of $u(x)$ is equivalent to the continuity of $t$. But, by 2.3, the continuity of $t$ is equivalent to the closure of $H$. Thus, we have that:

$$ u \text{ continuous } \iff t \text{ continuous } \iff H \text{ closed }$$

exactly our desired result.
\end{proof}

\begin{problem}{Question 5}

Let $E$ be a Banach space.

5.1) Suppose $T \in L(E,E)$, with $\Vert I - T \Vert < 1$. Prove that $T$ is invertible, and that the series $\sum_{n=0}^\infty (I - T)^n$ converges in $L(E,E)$ to $T^{-1}$.

5.2) Suppose $T \in L(E,E)$ is invertible and $\Vert S - T \Vert < \Vert T^{-1}\Vert^{-1}$. Prove that $S$ is invertible. Conclude that the set of invertible operators in $L(E,E)$ is open.

\end{problem}

\begin{proof}[Solution]

5.1)

Firstly, we use the fact that since $E$ is complete, so is $L(E,E)$ from Folland 5.4. We notice, that by the definition of the norm, that $\sup \{ \Vert (I -T)x \Vert : \Vert x \Vert = 1 \} < 1$; denote it as $c$. Considering $(I - T)(I - T)(x)$, for $\Vert x \Vert = 1$, call $(I - T)x = y$. Clearly, $\Vert y \Vert \leq c$. Looking at $(I - T)(y) = \Vert y \Vert(I - T)\left(\frac{y}{\Vert y \Vert}\right)$, due to the operator norm again, we see that $\Vert (I - T)(\frac{y}{\Vert y \Vert} \Vert \leq c$. Hence, for all $\Vert x \Vert = 1$, we have that $\Vert (I - T)^2(x) \Vert \leq c^2$. Then, $\sup \{ \Vert (I -T)(I - T)x \Vert : \Vert x \Vert = 1 \} \leq c^2$. Proceeding inductively, by considering $(I - T)^n(x) = (I -T)(I -T)^{n-1}(x)$, and using the same argument on $(I - T)^{n-1}(x)$ as having norm at most $c^{n-1}$ in the same way, we see that $\Vert (I - T)^n \Vert \leq c^n$.

Now, we consider the sum $\sum_{n=0}^\infty \Vert (I - T)^n\Vert$. By the observations above, we have that $\Vert (I - T)^n \Vert \leq \Vert I - T \Vert^n$. So, we have a sum:

$$ \sum_{n=0}^\infty  \Vert (I - T)^n\Vert \leq  \sum_{n=0}^\infty \Vert (I - T)\Vert^n  = \sum_{n=0}^\infty c^{n} = \frac{1}{1 - c} $$

where we've additionally used the fact that $\Vert I \Vert = 1$, which is clear, and identified this as an infinite geometric series with ratio less than 1. Then, since this is an absolutely convergent sum, and $L(E,E)$ is complete, $\sum_{n=0}^\infty (I - T)^{n}$ converges.

Now, we wish to show that $T \sum_{n=0}^\infty (I - T)^n$ acts as the identity, where we note that because $T$ commutes with its powers, and $T$ commutes with $I$, that we can write it on the left or right without ambiguity.

First, we look at the partial sums. We claim that $\sum_{n=0}^k T(I - T)^n = -(I - T)^{k+1} + I$.

The base case is easy. For $k = 1$, we see that this sum is exactly:

$$ TI + T(I - T) = T + T - T^2 = 2T - T^2 = - (I - T)^2 + I$$

Now, suppose this is true for up to $k = m$. Then, we have that:

$$\sum_{n=0}^{m+1} T(I - T)^n = \sum_{n=0}^m T(I - T)^n + T (I - T)^{m+1} = -(I - T)^{m+1} + I + T (I - T)^{m+1} = (I - T)^{m+1} (-I + T) + I = - (I - T)^{m+2} + I$$

as desired. Then, to compute $T \sum_{n=0}^\infty (I - T)^n$, we can take the following limit:

$$ \lim_{m \to \infty} T \sum_{n=0}^m (I - T)^n = \lim_{m \to \infty} -(I - T)^{m+2} + I $$

and because of the the work done with the norm, since $\Vert - (I - T)^{m+2} \Vert \leq \Vert I - T\Vert^{m+2}$, this goes to the 0 map as $m \to \infty$. Hence:

$$\lim_{m \to \infty} T \sum_{n=0}^m (I - T)^n = I$$ and hence, $T$ is bijective with $\sum_{n=0}^\infty (I - T)^n$ as a left and right inverse, with the sum bounded.

5.2)

We consider the related operator $T^{-1}(S - T) = T^{-1}S - I$. By adapting the argument in the first part of 5.1, we see that $\Vert T^{-1}(S- T)\Vert \leq \Vert T^{-1} \Vert \Vert S - T \Vert$, where we do the same trick on considering $T^{-1}[ (S-T)(x)]/\Vert (S-T)(x) \Vert$. So, we have that:

$$ \Vert T^{-1}S - I \Vert  = \Vert T^{-1}(S - T) \Vert \leq \Vert T^{-1} \Vert \Vert S - T \Vert < \Vert T^{-1} \Vert \Vert T^{-1} \Vert^{-1} = 1 $$

Thus, by 5.1 then, $T^{-1}S$ is invertible. But $T$ is already invertible, and the composition of invertible bounded linear operators is invertible (as composition of bijective is bijective, composition of bounded is still bounded pretty easily: $\Vert f \circ g (x)\Vert \leq c_f \Vert g(x) \Vert \leq c_f c_g \Vert x \Vert$, and invertibility comes from, for $f \circ g$, considering $g^{-1} \circ f^{-1}$). Hence, $T \circ T^{-1}S = S$ is invertible.

Thus, we have shown that there exists open ball around any invertible operator $T$ in $B(E,E)$ composed of invertible operators. Hence, by the local criterion for an open set, the set of invertible operators in $B(E,E)$ is open.

\end{proof}

\begin{problem}{Question 8}

Suppose that $\mathcal{H}$ is a Hilbert space, $T \in L(\mathcal{H}, \mathcal{H})$.

8.1) Show that there exists a unique element that we denote $T^* \in L(\mathcal{H}, \mathcal{H})$ such that $\langle Tx, y \rangle = \langle x, T^* y \rangle $ for all $x, y \in \mathcal{H}$. Call $T^*$ the adjoint of $T$.

8.2) Prove that $T^* = V^{-1} T^\dag V$ where $V$ is the conjugate linear isomorphism from $\mathcal{H} \to \mathcal{H}^*$ defined as $(Vy)(x) = \langle x, y \rangle$.

8.3) Prove that $\Vert T^* \Vert = \Vert T \Vert$, $\Vert TT^* \Vert = \Vert T \Vert^2$, $(aS + bT)^* = \overline{a} S^* + \overline{b} T^*$, $(ST)^* = T^* S^*$, and $T^{**} = T$.

8.4) Let $R(T), N(T)$ denote the range and nullspace of $T$, respectively. Prove that $R(T)^\perp = N(T^*)$ and $N(T)^\perp = \overline{R(T^*)}$.

8.5) Show that $T$ is unitary if and only if T is invertible, with $T^{-1} = T^*$.

\end{problem}

\begin{proof}[Solution]

8.1)

Suppose there exists another $T' \in L(\mathcal{H}, \mathcal{H})$ such that $\langle x, T' y \rangle  = \langle Tx, y \rangle = \langle x, T^* y \rangle$.

Then, we consider $\langle x, T'y \rangle - \langle x, T^* y \rangle = 0$. By conjugate symmetry, we have that:

$$ \overline{\langle T'y, x \rangle} - \overline{\langle T^*y, x \rangle} = 0$$

But, the complex conjugate distributes over addition, so:

$$ \overline{\langle T'y, x \rangle - \langle T^*y, x \rangle } = 0$$

Now, using linearity of the first term, we have that:

$$ \overline{\langle T'y - T^*y , x \rangle } = 0 $$

Since $x$ is arbitrary, we may choose $x$ as $T'y - T^* y$. Since the inner product is real in this case, and as a Hilbert space, extends to a norm, we have that

$$\langle T'y - T^*y , T'y - T^*y \rangle = 0 \implies  \Vert T'y - T^*y \Vert^2 = 0 \implies \Vert T'y - T^*y \Vert = 0$$

Hence, by the properties of the norm, we have that $T' y - T^*y = 0 \implies T^*y = T'y$. Since $y$ was arbitrary, this implies that $T' = T^*$ on all of $\mathcal{H}$.

8.2)

We consider the action of $V^{-1} T^\dag V$ on a test vector $y$. By definition, $V(y) = f_y \in \mathcal{H}^*$, which acts via $f_y(x) = \langle x, y \rangle$. Then, again by definition, $T^\dag$ acts on $f_y(x)$, sending it to the functional that acts via $\tilde{f}_y(x) = \langle T(x), y \rangle$. Lastly, $V^{-1}$ takes $\tilde{f}_y$ and sends it back to $\mathcal{H}$ to $z$, such that $z$ is the unique element in $\mathcal{H}$ such that $\langle x, z \rangle = \langle T(x), y \rangle$, due to the definition of $\tilde{f}_y$. But, letting $x, y$ range over $\mathcal{H}$, this is exactly the action of $T^*$. Since $T^*$ is unique, this is an equality of operators.

8.3)

First, we prove that $(T^*)^* = T$. Let $x, y$ be arbitrary elements of $\mathcal{H}$, and consider the equation $\langle T^* x, y \rangle = \langle x, (T^*)^*(y) \rangle$. We have that following string of equalities:

$$ \overline{\langle Ty, x \rangle} = \overline{ \langle y, T^* x \rangle } = \langle T^* x , y \rangle = \langle x, (T^*)^* y \rangle = \overline{\langle (T^*)^* y , x \rangle }$$

which implies then that $\langle Ty, x \rangle = \langle (T^*)^* y , x \rangle \implies \langle [T - (T^*)^*](y), x \rangle = 0$ for all $x, y$. Then, yet again, with the same trick of choosing $x = [T - (T^*)^*](y)$, we see that $T - (T^*)^* = 0$ as operators, and thus $T = (T^*)^*$.

Next, we prove a statement on $V: \mathcal{H} \to \mathcal{H}^*$ that sends $x \mapsto f_x(y) = \langle y, x \rangle$. First, let $y$ be any unit norm vector, and we will consider the norm of $f_y$. Let $x$ be yet another unit norm vector. Then, by the Cauchy-Schwarz inequality, we have that:

$$ \Vert f_y(x) \Vert = | \langle x, y \rangle | \leq \Vert x \Vert \Vert y \Vert \leq 1 $$

where we have used the fact that $\Vert x \Vert, \Vert y \Vert = 1$. Furthermore, by choosing $x = y$, we see that this attains 1. Thus, we have that $\Vert f_y \Vert = 1$. Since this is true for all $y$, we may conclude that $\Vert V \Vert = 1$. Considering the fact that $V^{-1} \circ V$ acts on identity on $\mathcal{H}$ (or, equivalently, $V \circ V^{-1}$ on $\mathcal{H}$), we can conclude that $ \Vert V^{-1} \Vert =1 $.

Finally, we look at $\Vert T^\dag \Vert$.  Letting $f$ be a unit norm vector in $\mathcal{H}^*$. Via the isomorphism that identifies $y \in \mathcal{H}$ with $f_y(x) = \langle x, y \rangle$, it is clear that $\Vert y \Vert = 1 \iff \Vert f_y \Vert = 1$ due to Cauchy-Schwarz. Suppose $\Vert f_y \Vert =1$. Then, for $x \in \mathcal{H}$ with unit norm, we have that:

$$|f_y(x)| \leq \Vert x \Vert \Vert y \Vert \leq   \left\Vert \frac{y}{\Vert y \Vert} \right\Vert \Vert y \Vert = \Vert y \Vert $$

where Cauchy-Schwarz guarantees that we achieve equality at $\frac{y}/\Vert y \Vert$. Then, since this inequality holds for all $x$, and is independent of $x$, we see that $\Vert f_y \Vert = \Vert y \Vert$.

In any case, looking at the action of $T^\dag$ on $f_y$, let $x$ be a unit norm vector in $\mathcal{H}$, then we see that 

$$\Vert T^\dag f_y(x) \Vert = \Vert f_y(T(x)) \Vert = | \langle T(x), y \rangle | \leq \Vert T(x) \Vert \Vert y \Vert \leq \Vert T \Vert$$

where we use the fact that $x, y$ have unit norm. Thus, we may conclude that $\Vert T^\dag \Vert \leq \Vert T \Vert$.

Then, using the same argument as used in 5.1 for showing that the operator norm is submultiplicative, we see that:

$$ \Vert T^* \Vert = \Vert V^{-1} T^\dag V \Vert \leq \Vert V^{-1} \Vert \Vert T^\dag \Vert \Vert V \Vert = \Vert T^\dag \Vert \leq \Vert T \Vert $$

However, we already have that $T =  (T^*)^*$, so we may run this same argument with $\Vert T \Vert = \Vert (T^*)^* \Vert = \Vert (V')^{-1} (T^*)^\dag V' \Vert \leq \Vert T^* \Vert$ with $V'$ as the isomorphism from $\mathcal{H}^* \to \mathcal{H}$ in the same way. Thus, we have that $\Vert T \Vert = \Vert T^* \Vert$.

Now, let $x$ have unit norm. Then, we look at the following string of inequalities:

$$ \Vert Tx \Vert^2 = \langle Tx , Tx \rangle = \langle x , T^* Tx \rangle \leq \Vert x \Vert \Vert T^*T x \Vert \leq \Vert x \Vert \Vert T^* T \Vert \Vert x \Vert = \Vert T^* T \Vert$$

where we notice since $ \langle Tx , Tx \rangle = \langle x , T^* Tx \rangle$, the right side is positive and real, and thus is equal to its absolute value, where we use Cauchy-Schwarz.

Since this is true for all $x$ with unit norm, this implies $\Vert T \Vert^2 \leq \Vert T^* T \Vert$. But by submultiplicativity, we have that $\Vert T^* T \Vert \leq \Vert T^* \Vert \Vert T \Vert = \Vert T \Vert^2$ from $\Vert T^* \Vert = \Vert T \Vert$. Hence, $\Vert T^* T \Vert = \Vert T \Vert^2$. We will see later that since $(T^* T)^* = T^* T$, and $\Vert T \Vert = \Vert T^* \Vert$ will show this to be equivalent to the problem statement.

To see $( aS + bT)^* = \overline{a}S^* + \overline{b}T^*$ is easy via the conjugate linearity of $V$, as clearly, $V^{-1}$ must be conjugate linear itself since if we consider $kf_y(x) = k\langle x,y \rangle = \langle x, \overline{k} y \rangle$, evidently, $V(\overline{k}y) = k f_y$, and so $V^{-1}(kf_y) = \overline{k}f_y$. We see that:

$$ (aS + bT)^*(y) = V^{-1}(aS + bT)^\dag V(y) = V^{-1}(aS + bT)^\dag f_y = V^{-1} (f_y \circ (aS + bT)) = $$
$$V^{-1} [a (f_y \circ S) + b (f_y \circ T)] = \overline{a} V^{-1} f_y \circ S + \overline{b}V^{-1} f_y \circ T = \overline{a} S^* + \overline{b} T^*(y)$$

since this is true for arbitrary $y \in \mathcal{H}$, this is an equality of operators.

Similarly:

$$ (ST)^*(y) = V^{-1} (ST)^\dag V(y) = V^{-1} (ST)^\dag f_y$$

Considering an arbitrary $x\in \mathcal{H}$, we see that:

$$(ST)^\dag f_y(x) = f_y(ST(x)) = S^\dag f_y(T(x)) = T^\dag \circ S^\dag \circ f_y(x)$$

Since this is true for all $x, y$, we have that:

$$(ST)^* = V^{-1} \circ T^\dag \circ S^\dag \circ V$$

On the other hand, by definition, we have that:

$$ T^* S^* = (V^{-1}\circ T^\dag \circ V) \circ (V^{-1} \circ S^\dag \circ V) = V^{-1} \circ T^\dag \circ I \circ S^\dag \circ V = V^{-1} \circ T^\dag \circ S^\dag \circ V$$

completing our proof.

8.4)

Recall that the definition of $R(T)^\perp = \{ x \in \mathcal{H} : \langle x, y \rangle = 0 \forall y \in R(T) \}$.

First, suppose $y \in R(T)^\perp$. Then, by definition, we have that $\langle y, Tx \rangle = \langle Tx, y \rangle = 0$ for all $x \in \mathcal{H}$. Then, we have that $\langle x , T^*y \rangle =  0$. Specifically, this must be true for $x = T^*y$, which implies that $T^*y = 0$. Thus, $R(T)^\perp \subseteq N(T^*)$.

Next, suppose $y \in N(T^*)$. Then, we have that $\langle x, T^*(y) \rangle = 0$ for all $x$, which we can see by the Schwarz inequality, and how $\Vert T^*y \Vert = 0$. Then, we have that $\langle T(x), y \rangle = 0$ for all $x \in \mathcal{H}$, which implies that $\langle y, T(x) \rangle = 0$, and thus by definition again, $y \in R(T)^\perp$.

Now, from the first part, we have that:

$$N(T)^\perp = N(T^{**})^\perp = (R(T^*)^\perp)^\perp $$

It should be clear that for $X$ a subset, that $X \subset (X^\perp)^\perp$, as for any $x \in X$, we have that:

$$ \langle y, x \rangle = 0 = \overline{\langle x, y \rangle} = \langle x, y \rangle $$

for any $ y \in X^\perp$. However, we see that the last expression is exactly the defining statement of $(X^\perp)^\perp$. Hence, $X \subset (X^\perp)^\perp$. So, we have that $R(T^*) \subseteq N(T)^\perp$. In particular, from problem 56, this implies that $N(T)^\perp$ is the smallest closed subspace that contains $R(T^*)$. But from problem 2.1, since $R(T^*)$ is a subspace, $\overline{R(T^*)}$ is a subspace, hence the smallest closed subspace, hence equal to $N(T)^\perp$.

Folland \#56:

Let $E$ be a subset of $\mathcal{H}$. Then $(E^\perp)^\perp$ is the smallest closed subspace containing $E$.

We have already shown that $E \subset (E^\perp)^\perp$. From Proposition 5.21 in Folland, we know that any subset $E^\perp$ is closed. Moreover, from the linearity of the inner product in the first argument, of course this is a vector subspace of $\mathcal{H}$. Thus, we need only prove that it is the smallest such closed subspace.

Suppose we have another closed subspace of $\mathcal{H}$, call it $F$ such that $E \subseteq F$. Then, of course, $F^\perp \subseteq E^\perp$, since if we're orthogonal to all of $F$, and $F$ contains $E$, then we're orthogonal to $E$. Evidently then, $(E^\perp)^\perp \subseteq (F^\perp)^\perp$, substituting $E^\perp$ for $F$, and $F^\perp$ for $E$ above.

Suppose we fix some element $x \in (F^\perp)^\perp$. By theorem 5.24 in Folland, since $F$ is a closed subspace, then we can rewrite $\mathcal{H} = F \oplus F^\perp$, and hence, $x = f + f'$ for $f \in F, f' \in F^\perp$. But, of course, $0 = \langle x, f' \rangle = \langle f + f', f' \rangle = \langle f, f' \rangle + \langle f', f' \rangle = \langle f', f' \rangle$, which implies that $f' = 0$. Hence, $x = f$. Since we can do this for all $x \in (F^\perp)^\perp$, this implies that $(F^\perp)^\perp \subseteq F$. Hence, $(E^\perp)^\perp \subseteq (F^\perp)^\perp \subseteq F$, and therefore, must be the smallest such closed subspace. 


8.5)

The backward direction is easy. We have that:

$$ \langle x, y \rangle = \langle T^{-1} T x ,y \rangle = \langle T^* T x , y \rangle = \langle Tx, T^{**} y \rangle = \langle Tx, Ty \rangle$$

for all $x, y \in \mathcal{H}$. 

On the other hand, suppose $T$ is unitary. Then, we have that:

$$ \langle x, y \rangle = \langle Tx, Ty \rangle = \langle T^* T(x), y \rangle$$

Since $T$ is invertible, we can in particular, choose $x =  T^{-1}(z)$. Then, we have that:

$$ \langle T^{-1}z, y \rangle = \langle T^* TT^{-1}(z), y \rangle = \langle T^* z, y \rangle $$

Since $z$ ranges over all of $\mathcal{H}$ as $T$ is invertible, we can conclude that $T^{-1} - T^* = 0 \implies T^{-1} = T^*$ everywhere. 

\end{proof}

\begin{problem}{Question 12}

Let $M$ be a closed subspace of $L^2([0,1])$, contained in $C([0,1])$.

12.1) Prove that there exists $C > 0 $ such that $\Vert f \Vert_u \leq C \Vert f \Vert_2$ for all $f \in M$.

12.2) For each $x \in [0,1]$, prove that there exists $g_x \in M$ such that $f(x) = \langle f, g_x \rangle$ for all $f \in M$ and that $\Vert g_x \Vert_2 \leq C$.

12.3) Show that the dimension of $M$ is at most $C^2$, by proving that if $\{ f_k \}$ is any orthogonal sequence in $M$, then $\sum_{k} | f_k(x)|^2 \leq C^2$ for all $x \in [0,1]$.

\end{problem}

\begin{proof}[Solution]

12.1)

Consider the inclusion as vector spaces $i: M \to C([0,1])$. Evidently, this map is linear, as the addition and scalar multiplication in $L^2$ and $C([0,1])$ act in the same way. Then, we wish to show it as closed.

Let $\{ f_n \} \to f$ be a convergent sequence of functions in $M$, such that $\{ i(f_n) \} \to y \in C([0,1])$. Suppose $i(f) = f \not = g$. Then, there must exist some $x_0$ such that $| f(x_0) - g(x_0) | > 0$. By continuity then, since $f - g$ is continuous as well, there exists a $\epsilon > 0$ such that for all $| x - x_0 | < \delta$, $|f(x) - g(x) | > \epsilon$. Note that in the case $x_0 - \delta < 0$ or $x_0 + \delta > 1$, we adjust $\delta$ to be the smaller of $\delta$ and the distance to the endpoint. Thus, we have then that:

$$ \Vert f - g \Vert_2  = \sqrt{\int_{[0,1]} | f(x) - g(x)|^2 dx} \geq \sqrt{ \int_{[x_0 - \delta, x_0 + \delta]} | f(x) - g(x) |^2 dx } \geq \sqrt{ 2 \delta \epsilon^2 } $$

On the other hand, we have that:

$$ \Vert f - g \Vert_2  = \Vert f - f_n \Vert_2 + \Vert f_n - g \Vert_2$$

Since $f \to f_n$ in the $L^2$ norm, we may choose $N_1$ such that for all $n > N_1$, $\Vert f - f_n\Vert_2 <  \epsilon\sqrt{2\delta}/2$.

Looking at $\Vert f_n - g \Vert_2 = \sqrt{\int_{[0,1]} | f_n - g |^2}  \leq \sqrt{\int_{[0,1]} \Vert f_n - g \Vert_u^2}$, since $f_n \to g$ in the uniform norm, we may choose $N_2$ such that for all $n > N_2$, $\Vert f_n - g \Vert_u < \epsilon \sqrt{2\delta}/2 $.

Then, choosing $n > \max(N_1, N_2)$, we see that:

$$ \Vert f - g \Vert_2  = \Vert f - f_n \Vert_2 + \Vert f_n - g \Vert_2  <  \epsilon\sqrt{2\delta}/2 + \sqrt{\left( \epsilon\sqrt{2\delta}/2\right)^2} = \epsilon \sqrt{2 \delta}$$

Thus, $\epsilon \sqrt{2 \delta} < \epsilon \sqrt{2 \delta}$, a contradiction. Hence, $f = g$. Therefore, the inclusion is a closed map. Moreover, $C([0,1])$ is a Banach space under $\Vert \cdot \Vert_u$. Moreover, since $M$ is a closed subspace of a Banach space, it is itself a Banach space with the same norm. Hence, by the closed graph theorem (5.12, Folland), we have that because the inclusion is a closed linear map, then it is bounded.

By the definition of a bounded linear map then, we have that there exists a $C> 0$ such that $\Vert i(f) \Vert_u = \Vert f \Vert_u \leq C \Vert f \Vert_2$.

12.2)

First, we note that since $M$ is a closed subspace of a Hilbert space, it too is a Hilbert space with the same inner product as $L^2$, restricted to $M$.

Consider the map that takes a function in $M$ and evaluates it at a point $x \in [0,1]$. Denote this map as $T_x: M \to F$, for $F$ our base field.

Clearly, this map is linear, since $T_x(af + bg) = (af + bg)(x) = a f(x) + bg(x)$, due to how addition and scalar multiplication of functions is defined pointwise. Moreover, of course, $|T_x(f)| = | f(x) | \leq |f|_u$, as the uniform norm is the supremum over all $x \in [0,1]$. But, by 12.1, this is at most $C \Vert f \Vert_2$. Hence, $T_x$ is bounded. Since $T_x$ is a bounded linear functional, it belongs to $M^*$. But then, by Theorem 5.25 (Folland), there exists a unique $g_x \in M$ such that $f(x) = T_x(f) = \langle f, g_x \rangle$, for all $f \in M$. 

In particular, we have that:

$$ \Vert g_x \Vert_2^2 = \langle g_x, g_x \rangle = T_x(g_x) = g_x(x) \leq \Vert g_x \Vert_u \leq C \Vert  g_x \Vert_2$$

Assuming first that $\Vert g_x \Vert_2^2 \not = 0$, this implies after dividing both sides by $\Vert g_x \Vert_2$, that:

$$ \Vert g_x \Vert_2 \leq C $$

and we notice that if $g_x = 0$, then this inequality is still satisfied.

12.3)

Let $\{ f_k \}$ be an orthogonal sequence in $M$. We may replace this with an orthonormal sequence by replacing $f_k$ with $f_k/\Vert f_k \Vert_2$. Further, restrict to a finite sequence, restricting to a subsequence if need be - say that $\{ f_k \}_{k=1}^N$ is our orthonormal subsequence. Fix an $x \in [0,1]$. By 12.2, there exists $g_x$ such that $f(x) = \langle f, g_x \rangle$ for all $f \in M$. Thus, we have that:

$$ \sum_{k=1}^N | f_n(x) |^2 = \sum_{k}^N | \langle f_n, g_x \rangle |^2$$

Now, by Bessel's Inequality, after using the fact that $| \langle f_n, g_x \rangle |^2 = | \langle g_x, f_n \rangle |^2$, since the modulus of the transpose is equal to the original modulus:

$$ \sum_{k=1}^N | \langle g_x, f_k \rangle |^2 \leq \Vert g_x \Vert_2^2$$

and by 12.2, we have that this quantity is at most $ C^2$. Hence, we have that:

$$ \sum_{k=1}^N | f_k(x) |^2 \leq C^2$$ for all $x \in [0,1]$.

Then, we have that:

$$\sum_{k=1}^N \Vert f_k \Vert_2^2 = \sum_{k=1}^N\int_{[0,1]} | f_k |^2 dx = \int_{[0,1]} \sum_{k=1}^N \sum_{k=1}^N |f_k|^2 \leq \int_{[0,1]} C^2 = C^2$$

On the other hand, using the normality, we have that:

$$ \sum_{k=1}^N \Vert f_k \Vert_2^2 = N$.

Hence, we have that $N \leq C^2$, and any finite sequence of orthonormal vectors has at most $C^2$ vectors. Since any orthogonal sequence must give rise to a orthonormal sequence, by dividing out by a norm, and we may always look at finite subsequences of infinite sequence, this must be true for arbitrary orthogonal sequences of $\{ f_k \}$. Thus, the maximal number of distinct elements in an orthogonal sequence is $C^2$, hence the dimensionality of $M$ is at most $C^2$ as a vector space.

\end{proof}


\begin{problem}{Question 20}

Recall that $L^p$ denotes the space of real-valued functions such that their $p$-th power is integrable. Suppose that $\Vert f_0 \Vert_{L_p} = \Vert f_1 \Vert_{L_p} = 1$. Define

$$ f_t = (1- t)f_0 + tf_1 $$

Of course, $\Vert f_t \Vert_{L_p} < 1$ for all $t \in (0,1)$ unless $f_0 = f_1$.

20.1)

Let $f \in L^p, g \in L^q$, with $1/p + 1/q = 1$, $\Vert f \Vert_{L^p} = 1, \Vert g \Vert_{L^q} = 1$. Show that if 

$$ \int fg d\mu = 1$$ 

then f(r) = $\text{sgn}(g(x)) | g(x)|^{q-1}$.

20.2)

Suppose that $\Vert f_{t'} \Vert_{L^p} = 1$ for some $0 < t' < 1$. Suppose that we have $g \in L^q$ with $\Vert g \Vert_{L^q} = 1$, such that:

$$ \int f_{t'}g d\mu = 1$$

and denote $F(t) = \int f_t g d\mu$. Prove that $F(t) = 1$ for all $t \in [0,1]$, and conclude that $f_t = f_0$ for all $t \in [0,1]$.

20.3)

Show that this fails when $p = 1, p = \infty$. What can we say in these cases?

\end{problem}

\begin{proof}[Solution]


\end{proof}

\end{document}