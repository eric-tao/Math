\documentclass[10pt]{article}
\setlength{\parskip}{0.25\baselineskip}
\usepackage[margin=1in]{geometry} 
\usepackage{amsmath,amsthm,amssymb, graphicx, multicol, array}
\usepackage[font=small,labelfont=bf]{caption}

\newcommand{\supp}{{\text{supp}}} 
\newcommand{\bv}{{\text{BV}}}
\newcommand{\ac}{{\text{AC}}}

\newenvironment{problem}[2][]{\begin{trivlist}
\item[\hskip \labelsep {\bfseries #1}\hskip \labelsep {\bfseries #2.}]}{\end{trivlist}}

\begin{document}
 
\title{Homework \#4}
\author{Eric Tao\\
Math 237: Homework \#4}
\maketitle

\begin{problem}{Question 2}

Suppose that $f$ is a continuously differentiable function on $\mathbb{R}$ except at $x_1,...,x_m$, where $f$ suffers jump discontinuities, and that its point-wise derivative, $d f/ d x$, defined except at $x_i$, is $L^1_{loc}(\mathbb{R})$.

Show that the distribution derivative of $f$ is given by:

$$ f' = \left( \frac{d f}{dx} \right) + \sum_{j=1}^m [ f(x_j+) - f(x_j-)] \delta_{x_j} $$

\end{problem}
\begin{proof}[Solution]

Without loss of generality, we may assume that $x_1 < x_2 < ... < x_m$, and otherwise sort these so that this is true, as we have only countably many.

Let $\phi \in C^\infty_c(\mathbb{R})$. Take $K > \max(|x_1|, |x_m|)$ such that the support of $\phi$ is contained within $[-K, K]$. By definition, we have that:

$$ \langle f', \phi \rangle = - \int_{[-K, K]} f(x) \partial_x \phi dx $$

We may break up this integral over each $[x_i, x_{i+1}]$ as well as $[-K, x_1]$ and $[x_m, K]$. Let's focus on a single $[x_i, x_{i+1}]$.

On this interval, we have that $f, \phi$ are differentiable on all of $[x_i,x_{i+1}]$, except for potentially $x_i, x_{i+1}$ for $f$. $\phi$ is evidently continuous, hence bounded, and hence integrable on $[x_i, x_{i+1}]$.

In a similar fashion, let $Z$ be any subset of $[x_i, x_{i+1}]$ of measure 0 that excludes the endpoints and do not have the endpoints as limit points. Then, we may restrict to an $[a,b], a > x_i, b < x_{i+1}$ such that $f$ is differentiable on all of $[a,b]$. Then, by the second growth lemma (6.2.4, Heil), we have that $|f(Z)|_e \leq \int_Z |f'| = 0$. Then, for any set of measure 0 $Z$, we may look at $Z \cap [x_i + 1/n, x_{i+1} - 1/n]$ for $n \in \mathbb{N}$ combined with the fact that at the endpoints, $f(x_i), f(x_{i+1})$ are singletons, hence measure 0, where we interpret this as the limit, not necessarily as the value taken, to conclude that $|f(Z)| = 0$, where we use the fact that $f'$ is locally integrable to ensure that this makes sense.

Then, by Banach-Zaretsky, we may conclude that $f$ is also absolutely continuous on $[x_i, x_{i+1}]$.

Thus, we may apply integration by parts. Looking specifically at the compact set $[x_i + 1/n, x_{i+1} - 1/n]$, to start, by integration by parts, we have that this is exactly equal to:

$$\int_{x_i + 1/n}^{x_{i+1}-1/n} f \partial_x \phi dx =  f(x_{i+1}-1/n) \phi(x_{i+1}-1/n) - f(x_i + 1/n) \phi(x_i + 1/n) - \int_{x_i + 1/n}^{x_{i+1}-1/n} \partial_x(f) \phi dx $$

By a dominated convergence theorem, looking at the sequence $\chi_{[x_i + 1/n, x_{i+1}-1/n]} f \partial_x \phi$, for $n \to \infty$, we see that each one is at most $f \partial_x \phi$ everywhere, and the product of continuous functions being continuous and hence integrable on this compact set, we may take the limit and claim that:
 
$$ \int_{[x_i, x_{i+1}]} f \partial_x \phi = \lim_{n \to \infty}  f(x_{i+1}-1/n) \phi(x_{i+1}-1/n) - f(x_i + 1/n) \phi(x_i + 1/n) - \int_{x_i + 1/n}^{x_{i+1}-1/n} \partial_x(f) \phi dx = $$

$$f(x_{i+1}-)\phi(x_{i+1}) - f(x_i+)\phi(x_i) -  \int_{[x_i, x_{i+1}]} \partial_x(f) \phi dx $$

where we denote that $\lim_{x \to x_{i}^+}f(x)  = f(x_i+)$ and similarly from the negative side, and that since $\partial_x(f)$ is only potentially undefined on a set of measure 0 on the endpoints, it's fine to say this. We note that since the only properties we've used is the compactness of each set, this applies for each of the compact sets earlier.

Repeating this argument for each compact interval then, we have that:

$$ \int_{[-K, K]} f(x) \partial_x \phi dx = \int_{[-K, x_1]} f \partial_x \phi + \sum_{i=1}^{m-1} \int_{[x_i, x_{i+1}]} f \partial_x \phi + \int_{[x_m, K]} f \partial_x \phi =  $$

$$ f(x_1-)\phi(x_1) - f(-K)\phi(-K) - \int_{[-K, x_1]} \partial_x(f) \phi + f(K)\phi(K) - f(x_m+)\phi(x_m) - \int_{[x_m, K]} \partial_x(f) \phi  +$$

$$ \sum_{i=1}^{m-1} f(x_{i+1}-) \phi(x_{i+1}) - f(x_i+)\phi(x_i)- \int_{[x_i, x_{i+1}]} \partial_x(f) \phi dx$$

Since the set of discontinuities is a set of measure 0, we may combine the integrals into:

$$ \int_{[-K, K]} \partial_x(f) \phi dx = \int_{[-K, K]} \frac{df}{dx} \phi dx $$

Further, since $\phi$ is continuous with compact support, since we know that $\phi(K + 1/n) = 0$ for all $n \to \infty$, we may conclude that $\phi(K) = 0$, and similarly for $\phi(-K)$. Thus, we obtain:

$$ - \int_{[-K, K]} \frac{df}{dt} \phi dx +  \sum_{i=1}^{m} f(x_i-)\phi(x_i) - f(x_i+)\phi(x_i)$$

and finally, incorporating the negative sign from the beginning, we have that:

$$ \langle f', \phi \rangle = - \int_{[-K, K]} f(x) \partial_x \phi dx =  \int_{[-K, K]} \frac{df}{dt} \phi dx +  \sum_{i=1}^{m} f(x_i+)\phi(x_i) - f(x_i-)\phi(x_i)$$

Now, compare against the action:

$$ \left \langle \left( \frac{d f}{dx} \right) + \sum_{j=1}^m [ f(x_j+) - f(x_j-)] \delta_{x_j}, \phi \right \rangle $$

Clearly, these are exactly equal, and since this was true for an arbitrary $\phi \in C^\infty_c$, we can conclude that the derivative of $f$ as a distribution is equal to the expression as functionals acting on $C^\infty_c$. 

\end{proof}

\begin{problem}{Question 4}

Define $G$ on $\mathbb{R}^n \times \mathbb{R}$ via $G(x,t) = (4\pi t)^{-n/2} \exp(-|x|^2/4t) \chi_{(0,\infty)}(t)$. Denote $G^\epsilon(x,t) = (4\pi t)^{-n/2} \exp(-|x|^2/4t) \chi_{(\epsilon,\infty)}(t)$.

4.1)

Prove that $G^\epsilon \to G$ in $\mathcal{D}'$.

4.2)

For each $\phi \in C^\infty_c(\mathbb{R}^n \times \mathbb{R})$, compute $\langle (\partial_t - \Delta) G^\epsilon, \phi \rangle$, where $\Delta$ is the Laplacian on $\mathbb{R}^n$ given by $\Delta = \sum_{k=1}^n \partial^2_k$

4.3)

Conclude that $(\partial_t - \Delta)G = \delta$

4.4)

Prove that for $\phi \in C^\infty_c(\mathbb{R}^n \times \mathbb{R})$, then $f = G \ast \phi$ satisfies $(\partial_t - \Delta)f = \phi$.

\end{problem}

\begin{proof}[Solution]

4.1)

First, assume that $G(x,t)$ is $L^1_{loc}$ and we'll come back for this part. Of course then, since $G^\epsilon$ is simply $G$ times a characteristic function, $G^\epsilon$ is $L^1_{loc}$ for any $\epsilon > 0$.

Thus, let $\phi$ be an arbitrary $C^\infty_c$ function on $\mathbb{R}^n \times \mathbb{R}$. Restrict ourselves to some $\cup_{i=1}^n [-K_i, K_i] \times [0, K_t]$ such that this contains the support of $\phi$; call this set $K$. Then, we may consider $\langle G^\epsilon, \phi \rangle$. By the DCT, integrating over $K$, since $G^\epsilon \phi \to G \phi$ pointwise everywhere, and $\phi G^\epsilon$ differs from $\phi G$ only on $\mathbb{R}^n \times (0, \epsilon]$, where $G^\epsilon = 0$, we can say that:

$$ \lim_{\epsilon \to 0} \int_K G^\epsilon \phi = \int_K G \phi $$

where we extend the argument continously on the limit $\epsilon \to 0$ by using the above argument for any sequence of $\epsilon_n \to 0$ as $n \to \infty$. Hence, by definition then, we have that:

$$ \lim_{\epsilon \to 0} \langle G^\epsilon, \phi \rangle = \langle G, \phi \rangle $$

and thus, $G^\epsilon \to G$ in $\mathcal{D}'(\mathbb{R}^n \times \mathbb{R})$.

-- Add something about how $G$ is locally integrable.

4.2)

Here, we need to compute both $\langle \partial_t G^\epsilon, \phi \rangle$ and $\langle \partial_{x_i}^2 G^\epsilon, \phi \rangle$ and we will put this together to retrieve our result.

-- do integration by parts like 3 times, I'll write this out later.

4.3)

4.4)

By proposition 9.3 in Folland, we have that $\partial^\alpha(F \ast \psi) = (\partial^\alpha F) \ast \psi$, for $F \in \mathcal{D}$ and $\psi \in C^\infty_c$.

Thus, we have that, at any point $x$, and any test function $\phi$ that:

$$ ( \partial_t - \Delta)f(x) = (\partial_t - \Delta)(G \ast \phi)(x) = (\partial t - \Delta (G)) \ast \phi(x) = \delta \ast \phi(x)$$

Now, by definition in Folland, we have that:

$$ \delta \ast \phi(x) = \langle \delta, \tau_x(\tilde{\phi}) \rangle$$

where $\tau_x$ sends $f(y) \mapsto f(y - x)$, $\tilde{f}(x) = f(-x)$. Hence, we have that:

$$ \langle \delta, \tau_x(\tilde{\phi}) \rangle = \int \delta(y) \phi(x - y) dy = \phi(x) $$

as this integral disappears except for the value at $y = 0$.

\end{proof}

\begin{problem}{Question 9}

Recall that $l^2 = \{ x = (x_n)_{n=0}^\infty : \sum_{n=0}^\infty |x_n|^2 < \infty \}$. Define the following operators:

$$\begin{cases} Rx = (0, x_0, x_1,....) \\ Lx = (x_1,x_2,...) \end{cases} $$

9.1)

Prove that $LR = I$, but $RL \not = I$, and hence neither $R, L$ are invertible.

9.2)

For an operator $T$, denote $T'$ as its transpose. Prove that $R' = L$, and $L' = R$.

9.3)

Prove that the spectrum of $L, R$ consists of the unit disk $ \{ x \in \mathbb{C} : |x| \leq 1 \}$.

9.4)

Show that $R$ has no eigenvalues.

9.5)

Show that the spectrum of $R, L$ acting on the $l^p$ spaces for $1 \leq p \leq \infty$ consists of every point of the unit circle. 

\end{problem}

\begin{proof}[Solution]

9.1)

Let $x  = (x_n)_{n=0}^\infty  \in l^2$, and consider the action of $LR(x)$. We have that:

$$ LR((x_0,x_1,...,x_n,...)) = L((0,x_0,x_1,....,x_n,...)) = (x_0,x_1,....,x_n) = x $$

Since the choice of $x$ was totally arbitrary, we may conclude that $LR$ acts as identity on $l^2$.

On the other hand, consider the sequence $(1,0,...,0,...)$. This is clearly a $l^2$ sequence, and we have that:

$$ RL((1,0,...0,...)) = R(0,0,...,0,...) = (0,0,...,0,...) \not = (1,0,...,0,...) $$

Hence, $RL \not = I$.

We see that neither $L, R$ are invertible easily, as $L$ is not injective - take $x = (0,x_1,x_2,...)$ and $x' = (1,x_1,x_2,...)$ in $l^2$. Of course $L(x) = L(x') = (x_1,x_2,...)$ but $x \not = x'$.

Similarly, $R$ is not surjective. Take for example the sequence $(1,0,....)$. Since $R$ replaces the first sequence element with $0$, there cannot exist any $x \in l^2$ such that $R(x) = (1,0,0,...)$. Since linear operators are invertible if and only if they are both injective and surjective, we conclude that neither $L, R$ are invertible.

9.2)

First, we recall that we have a natural identification of $(l^2)^*$ as isomorphic to $l^2$. Thus since by definition, we identify the action of the transpose on functional $f$ as $T'(f) = f \circ T$, we can identify the transpose of $T$ as the function $T'$ such that $T'(x)(y) = x(T(y))$, for $x, y \in l^2$.

So first, let $T = L$. Then, we have that for $x = (x_0,x_1,....), y = (y_0,y_1,...)$, that:

$$ x L(y) =(x_0,x_1,...)(y_1,y_2,...) =  \sum_{n=0}^\infty x_n y_{n+1} $$

On the other hand, consider $R(x)$ acting on $y$. We have that:

$$ R(x)(y) = (0, x_0,x_1,...)(y_0,y_1,...) = 0 * y_0 + \sum_{n=1}^\infty x_{n-1}y_n =  \sum_{n=1}^\infty x_{n-1}y_n$$

Reindexing, we see that these sums coincide. Since this is true for arbitrary $x, y$, we conclude that $R = L'$ as functions.

In a similar fashion, letting $T = R$:

$$ x R(y) =(x_0,x_1,...)(0,y_0,y_1,y_2,...) = x_0 * 0 +  \sum_{n=1}^\infty x_n y_{n-1} = \sum_{n=1}^\infty x_n y_{n-1} $$

and looking at $L(x)(y)$:

$$ L(x)(y) = (x_1,x_2,...)(y_0,y_1,...) = \sum_{n=0}^\infty x_{n+1}y_n $$

Again, reindexing, we see these coincide, and iterating across $x, y$ we conclude that $R' = L$ as functions.

9.3)

First, we notice that since $L$ deletes the $0$-th term, we must have that:

$$ \Vert Lx \Vert = \sum_{n=1}^\infty | x_n|^2 \leq \sum_{n=0}^\infty |x_n|^2 = \Vert x \Vert$$

which implies that the norm of $L$ is at most 1. Considering the sequence $y_1 = (0,1,0,...)$, which is clearly in $l^2$ with norm $1$, $L(y_1) = (1,0,0,...)$ which has norm 1, and hence $\Vert L \Vert = 1$.

Moreover, for each $k \in \mathbb{N}$, we may consider the sequence $y_k$ such that the sequence is $0$ except in the $k$-th position. Evidently, $L^k(y_k) = (1,0...)$ and hence, $\Vert L^k \Vert$ achieves $1$, since we know that it is bounded above by $\Vert L\Vert^k$ by submultiplicativity, and we demonstrate a unit vector that achieves this maximum. Since we may do this for all $k \in \mathbb{N}$, we conclude that:

$$ \lim_{k \to \infty} \Vert L^k \Vert^{1/k} = \lim_{k \to \infty} 1^{1/k} = 1$$

that is, the spectral radius of $L$ is $1$.

So, now, consider $\lambda \in \mathbb{C}$ such that $|\lambda | < 1$. We show that each of these $\lambda$ are eigenvalues of $L$.

Consider the sequence $x = (1, \lambda, \lambda^2,...., \lambda^k,...)$. Of course, this is in $l^2$, since $\sum_{n=0}^\infty | x_n|^2 \leq \sum_{n=0}^\infty |x_n| = \frac{1}{1 - \lambda}$, since if $|\lambda| < 1$, we have that $|\lambda|^2 \leq |\lambda|$, and the second sum is a geometric series with ratio with modulus less than 1.

Now, considering $L(x)$, we have that $L(x) = (\lambda, \lambda^2,...,\lambda^{k},...) = \lambda( 1, \lambda, ... \lambda^{k-1},...) = \lambda x$. But, of course, if we consider $(L - \lambda I )(x)$, we have that $(L - \lambda I)x = \lambda x - \lambda x = 0$. Hence, $(L - \lambda I)$ is not injective, thus not invertible, and $\lambda \in \sigma(L)$. Since we may do this for all $\lambda$ with modulus strictly less than $1$, and the spectrum is closed, we conclude $\sigma(L) = \overline{\{ \lambda \in \mathbb{C} : |\lambda| < 1 \}} = \{ \lambda \in \mathbb{C} : |\lambda | \leq 1 \}$.

Here, we argue that a operator is invertible if and only if its transpose is invertible in $l^2$.

Suppose we have that $AB = BA = I$. Then, looking at the transpose, we would see that for $x, y \in l^2$

$$x(y) =  x (AB(y)) = A'(x) B(y)  = B'A'(x) (y)$$

Hence, since this holds for all $y$, for a fixed $x$, since the action of $B'A'(x)$ is identical to the action of $x$, iterating over $x$, this implies that $B' A' = I'$. The same follows for $A'B'$, as we play the same argument with $BA = I$.

Since $l^2$ is a Hilbert space, and hence reflexive, taking another transpose, we identify $A^{ ''} = A$ and same with $B^{''} = B$. Hence, we have our result that in $l^2$, $A$ invertible if and only if $A'$ is invertible.

From 9.2, we have that $L' = R$, and we have already shown that $L - \lambda I$ is not invertible when $| \lambda| < 1$. Hence, for $| \lambda| < 1$, $(L - \lambda I )' = L' - \lambda I' = R - \lambda I$ is not invertible, and thus $\lambda \in \sigma(R)$. Thus, taking the closure again, we conclude that:

$$\sigma(R) = \sigma(L) =  \{ \lambda \in \mathbb{C} : |\lambda | \leq 1 \}$$


9.4)

First, suppose $R(x) =\lambda x$. This equation implies then that $(0,x_0,x_1,...) = (\lambda x_0, \lambda x_1,...)$, which implies that $0 =\lambda  x_0$, and that for $ i \geq 0$, $x_i = \lambda x_{i+1}$. Of course, if $\lambda = 0$, then $x = 0$ immediately. On the other hand, assume $\lambda \not = 0$, then we have that $x_0 = 0$, and due to the recurrence relation, $x_{i+1} = 0$ for each $i \geq 0$. Hence, the only vector that satisfies this equation for any $\lambda$ is $x = 0$, and thus $R$ has no eigenvectors or eigenvalues.

9.5)

Let $p \in [1, \infty]$.

We notice that the arguments in 9.3 for $L$ pass without modification for $l^p$, and that in any $l^p$, $\sigma(L)$ is exactly the unit disk.

Now, we examine $R$ more closely. Firstly, we see that since $R$ merely inserts a $0$, and leaves the rest of the sequence unchanged, we have that for any $x \in l^p$, that $\Vert Rx \Vert = \Vert x \Vert$ after a reindexing of the sum. Thus, we have that $\Vert R^n \Vert \leq 1$.

Now, considering the action of $R^n$ on $y =(1,0,...0)$, this merely sends us to $R^n(y) = (0,0,...,0,1,0,....) = y_n$, that is, $y_n$ has a $1$ in the $n$-th location, where we started our index at $n=0$, and $0$ otherwise. However, this clearly has norm $1$ for any $p$. Hence, $\Vert R^n \Vert =1$.

Thus, for each $n$, we have that $\Vert R^n \Vert ^{1/n} = 1^{1/n} = 1$, and hence, by the spectral radius formula, we have that 

$$\sup_{\lambda \in \sigma(R} |\lambda |= \lim_{n \to \infty} \Vert R^n \Vert^{1/n} = 1$$

Now, let $\lambda \in \mathbb{C}$ such that $|\lambda | < 1$, and consider the preimage of the vector $(1,0,0,...) \in l^p$ under the action of the operator $R - \lambda I$. For such a $y$, we would have that:

$$ (R - \lambda I)y = (1,0,...) \implies (0, y_0, y_1,...) - \lambda(y_0, y_1,...) = (1,0,...)$$

which gives us the following relations, for $i \geq 0$:

$$ \begin{cases} - \lambda y_0 = 1 \\ y_{i} - \lambda y_{i+1} = 0 \end{cases} \implies \begin{cases} y_0 = -\lambda^{-1} \\ y_{i+1} = y_i \lambda^{-1} \end{cases} \implies y_i = -\lambda^{-(i+1)} $$

However, $y \not \in l^p$ for any $p$. Since $|\lambda| < 1$, we must have that $|\lambda^{-1}| > 1$, and hence, the series:

$$ \sum_{n=1}^\infty | \lambda^{-n}| $$

diverges. This immediately gives us the result for $p=1$, and for $p \in [1,\infty)$, we see that since $|\lambda| > 1$, $| \lambda^n|^p > |\lambda^n|$ for each $n \in \mathbb{N}$. Hence, we have that:

$$ \sum_{n=1}^\infty | \lambda^{-n}|^p \geq \sum_{n=1}^\infty |\lambda^{-n}|  = \infty $$

Lastly, of course, for $p = \infty$, viewing $\lambda = r e^{i \theta}$, since $r > 1$ for any $M \in \mathbb{R}$, we may find $N$ such that $r^N > M$. Hence, then, for all $n \geq N$, we have that

$$|\lambda^n| =  | r^n e^{i n \theta} | = r^n > r^N > M $$

completing our result. Thus, under the condition $| \lambda| < 1$, we have that $(R - \lambda I )$ is not surjective, hence, not invertible, and therefore $\lambda \in \sigma(R)$.

Thus, we already have that $\{ \lambda \in \mathbb{C} : |\lambda | < 1 \} \subseteq \sigma(R)$, and since we know the spectrum is closed, we may conclude that the (closed) unit disk is contained within $\sigma(R)$. Moreover, from the spectral radius, we have that the spectrum is contained within the unit disk, and hence $\sigma(R) = \{ \lambda \in \mathbb{C} : | \lambda| \leq 1 \}$.

\end{proof}


\begin{problem}{Question 15}

Let $K$ be a measurable function on $[0,1]^2$, and let $f$ be a measurable function on $[0,1]$. Define $T$ as the map that sends:

$$ T(f(x)) = \int_0^1 K(x,y)f(y) dy $$

15.1)

Prove that if $K$ is continuous, then $T$ is a compact operator from $C[0,1] \to C[0,1]$, and that $\sigma(T) = \{ 0 \}$.

15.2)

Prove that if $K \in L^2$, then $T$ is a compact operator on $L^2[0,1]$.

\end{problem}

\begin{proof}[Solution]

15.1)

We omit the proof that $T$ is linear, since this is trivial, coming out from the linearity of the integral.

First, we should show that $Tf(x)$ is actually a continuous function, for $f \in C[0,1]$.

Let $\epsilon > 0$ be given. 

Since $f$ is a continuous function over a compact set, we have that $\Vert f \Vert_\infty$ is finite over $[0,1]$. If $\Vert f \Vert_\infty = 0$, then $f = 0$, and $Tf = 0$, continuous. So instead, assume $\Vert f \Vert_\infty \not = 0$. Since $K$ is continuous, for $\epsilon/\Vert f \Vert_\infty$, we may find $\delta > 0$ such that $| K(x, y) - K(x’,y) | < \epsilon/\Vert f \Vert_\infty$ whenever $|(x, y) - (x’,y)| < \delta$.

Thus, consider $| x - x’| < \delta$. We have that:

$$ | Tf(x) - Tf(x’) | = \left| \int_0^1 K(x,y) f(y) dy - \int_0^1 K(x’,y) f(y) dy \right| = \left| \int_0^1 (K(x,y) - K(x’,y)) f(y) dy \right|  \leq $$

$$ \int_0^1 | K(x,y) - K(x’, y)| |f(y)| dy \leq \int_0^1 \frac{\epsilon}{\Vert f \Vert_\infty} \Vert f \Vert_\infty dy = \epsilon$$

Since this procedure may be done for any $f \in C[0,1]$, we see $T$ as an operator from $C[0,1] \to C[0,1]$.

Now, consider the unit ball $\mathcal{B} = \{ f \in C[0,1] : \Vert f \Vert_\infty \leq 1 \}$ and in particular, the image in $C[0,1]$ of $T\mathcal{B}$.

We see that $T\mathcal{B}$ is pointwise bounded, as for any fixed $x \in [0,1]$ and every $f \in \mathcal{B}$, we have that:

$$ \left| \int_0^1 K(x,y) f(y) dy \right| \leq \int_0^1 | K(x,y)| | f(y)| dy \leq \int_0^1 \Vert K(x,y) \Vert_{\infty} \Vert f \Vert_\infty = \Vert K(x,y) \Vert_\infty \Vert f \Vert_\infty \leq \Vert K(x,y) \Vert_\infty < \infty $$

where it is understood that $\Vert K(x,y)\Vert$ is the norm over $[0,1]^2$ for $K$ and $\Vert f \Vert$ is the norm over $[0,1]$ for $f$.

Now, again, fix any $x \in [0,1]$, and let $\epsilon > 0$. Since $K$ is continuous, we may find $\delta$ such that $| (x,y) - (x’,y) | < \delta \implies | K(x,y) - K(x’, y) | < \epsilon$. We have that, for any $f \in \mathcal{B}$:

$$ | Tf(x) - Tf(x’) | = \left| \int_0^1 K(x,y) f(y) dy - \int_0^1 K(x’,y) f(y) dy \right| = \left| \int_0^1 (K(x,y) - K(x’,y)) f(y) dy \right|  \leq $$

$$ \int_0^1 | K(x,y) - K(x’, y)| |f(y)| dy \leq \int_0^1 \epsilon \Vert f \Vert_\infty dy \leq \int_0^1 \epsilon = \epsilon $$

Since this procedure can be done for every $x$, we conclude that $T\mathcal{B}$ is equicontinuous, as the neighborhood is independent of the choice of $f \in \mathcal{B}$.

Thus, by the Arzel\`{a}-Ascoli Theorem I (Folland, 4.43), we conclude that the closure of $T \mathcal{B}$ is compact, and hence, $T$ is a compact operator.

-- wait why in the world is the spectrum just 0. Spectral radius?

\end{proof}

\end{document}