\documentclass[10pt]{article}
\setlength{\parskip}{0.25\baselineskip}
\usepackage[margin=1in]{geometry} 
\usepackage{amsmath,amsthm,amssymb, graphicx, multicol, array}
\usepackage[font=small,labelfont=bf]{caption}

\newcommand{\supp}{{\text{supp}}} 
\newcommand{\bv}{{\text{BV}}}
\newcommand{\ac}{{\text{AC}}}

\newenvironment{problem}[2][]{\begin{trivlist}
\item[\hskip \labelsep {\bfseries #1}\hskip \labelsep {\bfseries #2.}]}{\end{trivlist}}

\begin{document}
 
\title{Midterm \#1}
\author{Eric Tao\\
Math 237: Midterm \#1}
\maketitle

\begin{problem}{Question 1}

Let $(X, \rho)$ be a compact metric space, and $f: X \to X$ a function such that:

$$ \rho(f(x), f(y)) < \rho(x,y) $$

for all $x \not = y$.

Define $g: X \to \mathbb{R}$ via $g: x \mapsto \rho(x, f(x))$.

1.1)

Prove that $g$ is Lipschitz, and that $g$ has a minimum value, achieved at a point $x_0 \in X$. Conclude that there exists $x \in X$ such that $g(x) = 0$.

1.2)

Show that $f$ has a unique fixed point $x_0$.

1.3)

Show that the assumption that $X$ is compact may not be omitted.

\end{problem}
\begin{proof}[Solution]

1.1)

Fix some $x \in X$, and let $y \in X$ be arbitrary. By the triangle inequality, we see that:

$$ \begin{cases} \rho(x, f(x)) \leq & \rho(x,y) + \rho (y, f(x)) \\ \rho(y, f(x)) \leq & \rho(y, f(y)) + \rho(f(x), f(y)) \end{cases}$$

Combining these two equations with the property of $f$ by hypothesis, we see that:

$$ \rho(x, f(x)) - \rho(y, f(y)) \leq \rho(x,y) + \rho(f(x), f(y)) < 2 \rho(x,y) $$

However, we notice that we may run the same computation in the triangle inequality, switching the labels of $x, y$, as $\rho(x,y) = \rho(y,x)$. Thus, we can conclude then that

$$ | \rho(x, f(x)) - \rho(y, f(y))| < 2 \rho(x,y)$$

and therefore, since the left side is exactly $d(g(x), g(y))$ with the metric of the real line, we may conclude that $g$ is Lipschitz with Lipschitz constant at most 2.

Now, since $g$ is Lipschitz continuous, it is continuous. Hence, since $X$ is compact, $g$ achieves its extremas. Hence, we may find $x_0 \in X$ such that $g$ achieves its minimum value.

Suppose that $g(x_0) > 0$. Then, of course, we would have that $g(x_0) = \rho(x_0, f(x_0)) > 0$ and hence, $x_0 \not = f(x_0)$. Then, we can consider $g(f(x_0))$. We have that:

$$ g(f(x_0)) = \rho(f(x_0), f(f(x_0))) < \rho(x_0, f(x_0)) = g(x_0)$$

But, this is a contradiction, as we assumed that $g$ attained a minimum at $x_0$. Hence, $g(x_0) = 0$.

1.2)

From 1.1, we've shown that there exists $x_0 \in X$ such that $g(x_0) = 0$. Evidently then:

$$ g(x_0) = 0 \implies \rho(x_0, f(x_0)) = 0 \implies f(x_0) = x_0$$

Furthermore, this point must be unique, as suppose $f(x_1)= x_1$ as well. Assuming that $x_0 \not = x_1$, we have that:

$$ \rho(x_0, x_1) = \rho(f(x_0), f(x_1)) < \rho(x_0, x_1) $$

which is absurd. Hence, $x_0 = x_1$. 

1.3)

Here are some examples to show that we need $X$ to be compact. Consider $X = \mathbb{Z}$, equipped with the standard metric $\rho(x,y) = | x - y | $. Of course, this is not compact, as the sequence $\{ n \}_{n=1}^\infty$ cannot admit any convergent subsequence. If we take $f(x) = \operatorname{round}(x/2)$, where the round function rounds to the integer closer to $0$, then of course, we have that $\rho(f(x), f(y)) < \rho(x,y)$ for $x \not = y$, as it contracts all distances by at least 1/2. On the other hand, it has multiple fixed points, $-1, 0, 1$.

Another example is to take the open interval $(0,1)$, equipped with the standard metric $\rho(x,y)$, and consider the function $f(x) = x/2$. Evidently, in the same fashion, we still have that $\rho(f(x), f(y)) = | x/2 - y/2 | = 1/2 | x- y | = 1/2 \rho(x,y) < \rho(x,y)$. However, $g$ does not attain a minimum and $f$ does not have a fixed point.

We can see $g$ does not have a minimum as for any $\epsilon > 0$, we may choose $N \geq 1$ such that $1/N < \epsilon$. Then, $g(1/N) = \rho(1/N, f(1/N)) = | 1/N - 1/2N | = 1/2N < 1/N < \epsilon$. Hence, $g(x)$ can be arbitrarily small. However, we can see that for $x = 1/2x$, this is satisfied only at $x = 0$, outside of $(0,1)$. Hence, there is no $x$ such that $g(x) = 0$ on $(0,1)$, and no fixed point of $f$ on $(0,1)$.

\end{proof}

\begin{problem}{Question 2}

Let $X, Y$ be Banach spaces. Let $T \in L(X,Y)$. Show that $T$ is surjective if and only if $\operatorname{range}(T)$ is not meager in $Y$.

\end{problem}

\begin{proof}[Solution]

One direction is trivial. Suppose $T$ is surjective. Then, $Y = \operatorname{range}(T)$. But, by the Baire Category Theorem (2.21, Heil), $Y$ is nonmeager in $Y$, and we are done.

Now, suppose $\operatorname{range}(T)$ is not meager. Consider open balls in $X$ centered on the origin, $B_n^X(0) = \{ x \in X : \Vert x \Vert < n \}$, where we use the superscript to remind ourselves this is in $X$. Clearly, $X = \cup_{n=1}^\infty B_n^X(0)$. Therefore, we have that the range of $T$ can be expressed as:

$$ \operatorname{range}(T) = \cup_{n=1}^\infty T(B_n^X(0)) $$

Since $T$ is non-meager, there exists an $m$ such that the closure $\overline{T(B_m^X(0)}$ contains an open ball, as its complement is not dense. We can consider the operator $mT$, and the closure $\overline{mT(B_1^X(0))}$ contains an open ball in $Y$, as $T(B_m^X(0) = mT(B_1^X(0))$ by linearity. Then, by Lemma 2.26 in Heil, we have that $mT(B_1^X(0))$ contains an open ball $B_r^Y(0)$ for some $r > 0$. Again, by linearity then, we have that $T(B_m^X(0)$ contains an open ball $B_{r/m}^Y(0)$.

So now, let $y \in Y$. In particular, consider $\frac{y}{\Vert y \Vert} * \frac{r}{2m}$. Evidently, the norm of this vector is $r/2m$, and hence is contained within $B_{r/m}^Y(0)$. Thus, there exists an $x \in X$ such that $T(x) = \frac{y}{\Vert y \Vert} * \frac{r}{2m}$. By linearity then, we have that:

$$T\left( \frac{2mx \Vert y \Vert}{r} \right) = \frac{2m\Vert y \Vert}{r} T(x) =  \frac{2m\Vert y \Vert}{r} \frac{y}{\Vert y \Vert}  \frac{r}{2m} = y$$

Hence, $Y \subseteq \operatorname{range}(T)$, and therefore, $Y = \operatorname{range}(T)$. Thus, $T$ is surjective. 

\end{proof}

\begin{problem}{Question 3}

Let $C_b(\mathbb{R})$ be the space of bounded, continuous, real-valued functions. Let $C^1_b(\mathbb{R})$ be the space of functions such that $f, f' \in C_b(\mathbb{R})$. Equip both of these spaces with the uniform norm.

3.1)

Show that $C_b$ is complete, and that $C^1_b$ is not complete.

3.2)

Show that the differentiation operator $D: C^1_b(\mathbb{R}) \to C_b(\mathbb{R})$ that sends $D: f \mapsto f'$ is unbounded, but has a closed graph.

\end{problem}

\begin{proof}[Solution]

3.1)

First, consider the family of functions $f_n(x) = 2^{-n} \cos(7^n \pi x)$ for $n \geq 1$, and consider $g_m(x) = \sum_{n=1}^m f_n(x)$.

We have that the sequence of $\{ g_m \}$ is uniformly Cauchy, as if we let $\epsilon > 0$, we may choose $N$ such that $2^{-N+1} < \epsilon$, and then for $m, m' > N$ (WLOG, suppose $m > m'$), we have that:

$$ | g_m(x) - g_{m'}(x) | = | \sum_{n=1}^m f_n(x) - \sum_{n=1}^{m'} f_n(x)| = | \sum_{n=m}^{m'} f_n(x)| \leq | \sum_{n=N}^\infty f_n(x) |  \leq \sum_{n=N}^\infty |f_n(x)|  \leq \sum_{n=N}^\infty 2^{-N} = 2^{-N+1}$$

Since this is independent of the point $x$, this is uniformly Cauchy. Since each $g_m$ is continuous, being the finite sum of continuous functions, and the convergence is uniform, the pointwise limit $g(x) = \lim_{m \to \infty}g_m(x)$ is a continuous function. Moreover, we can see easily that $g$ is bounded, as we can see that each of the partial sums are bounded above by $\sum_{n=1}^\infty 2^{-n} = 2$. However, this is a Weierstrauss function, famously known for being differentiable nowhere. Since we have demonstrated a sequence of functions in $C^1_b$, convergent under the uniform norm to a function not in $C^1_b$, we may conclude that $C^1_b$ is not complete.

On the other hand, let $\{ f_n \}_{n=1}^\infty \subseteq C_b$, with $\sum_{n=1}^\infty \Vert f_n \Vert_u < \infty$. Consider $f = \sum_{n=1}^\infty f_n$, and we will show that $f$ is both bounded, and the uniform limit of the partial sums.

Evidently, $f$ is bounded, as we can look at the partial sums $ \sum_{n=1}^N f_n$. We have that $\Vert \sum_{n=1}^N f_n \Vert_u \leq \sum_{n=1}^N \Vert f_n \Vert_u  < \sum_{n=1}^\infty \Vert f_n \Vert_u < \infty$, where the first inequality comes from the triangle inequality, and the second is simply our hypothesis of being absolutely convergent. Since this bound holds for all $N > 0$, it must hold in the limit as well. Hence, $\Vert f \Vert_u < \sum_{n=1}^\infty \Vert f_n \Vert_u < \infty$.

Now, we wish to show that $\sum_{n=1}^N f_n \to f$ uniformly. Since $\sum_{n=1}^\infty \Vert f_n \Vert_u < \infty$, for $\epsilon > 0$, we may find a $M > 0$ such that for all $m > M$,  $\sum_{n=M}^\infty \Vert f_n \Vert_u < \epsilon$. Now, let $m > M$, and consider $\Vert f - \sum_{n=1}^m f_n \Vert_u$. We see that:

$$ \Vert f - \sum_{n=1}^m f_n \Vert_u = \Vert \sum_{n=m+1}^\infty f_n \Vert_u$$

Now, due to the positivitiy of the norm, since we have for each finite sum: $\Vert \sum_{n=m+1}^p f_n \Vert_u \leq \sum_{n=m+1}^p \Vert f_n \Vert_u \leq \sum_{n=m+1}^\infty \Vert f_n \Vert_u$, we may conclude that this holds in the limit as well.

Hence, we have that:

$$ \Vert \sum_{n=m+1}^\infty f_n \Vert_u \leq \sum_{n=m+1}^\infty \Vert f_n \Vert_u < \epsilon $$

Thus, $f_n \to f$ uniformly, and hence, $f$ is continuous. Therefore, $f \in C_b$, as desired, and $f_n \to f$ under the norm. Since the choice of absolutely convergent sequence was arbitrary, by 5.1 in Folland, since every absolutely convergent sequence converges, $C_b$ must be complete.

3.2)

Evidently, $D$ is unbounded. For example, take the family of functions $f_k = \sin(kx)$, for $k \in \mathbb{N}$. Clearly, this is a continuous function, bounded above by 1, and so $\Vert f_k \Vert_u = 1$. Furthermore, its derivative is $k \cos(kx)$, continuous, and for each $k$, bounded above by $k$. However, $ \Vert D(f_k) \Vert_u = \Vert k \cos (kx) \Vert_u = k$. Since we may choose $k$ arbitrarily large without affecting the norm of $f_k$, $D$ is unbounded.

Now, suppose that we have $f_n \to f \in C^1_b$, and $Df_n = f'_n \to g \in C^1$, uniformly in both cases. Fix an arbitrary point $a \in \mathbb{R}$, and consider, for $x > a$, the closed interval $[a,x]$. Since we have that $f'_n \to g$ uniformly, evidently, $\Vert f'_n \Vert_u$ is bounded. Then, we can take $\sup_{n} \Vert f'_n \Vert_u < \infty$ as an upper bound for all $|f'_n(y)|, y \in [a,x]$. Of course also, if $f'_n \to g$ uniformly, it does so pointwise as well. Therefore, by the Lesbesgue Dominated Convergence Theorem, we have that:

$$ \lim_{n \to \infty} \int_a^x f'_n(y) dy = \int_a^x g(y) dy $$

However, we know that $f_n$ is differentiable on $[a,x]$, and $f'_n$, its derivative is continuous. Thus, we may transform the left hand side via the Fundamental Theorem of Calculus to obtain:

$$ \lim_{n \to \infty} f_n(x) - f_n(a) = \int_a^x g(y) dy$$

Now, since $f_n \to f$ uniformly, it does so pointwise as well, so we have that:

$$ f(x)  - f(a) = \int_a^x g(y) dy$$

and finally, we can apply $D$ to both sides of this equation, and since $g$ is continuous, we can apply the other statement of the FTC to obtain:

$$ D(f(x) - f(a)) = D\left(\int_a^x g(y) dy \right) \implies D(f)(x) = g(x) $$

Since the choice of $a$ were arbitrary, we may repeat this argument for every $x$. Hence, varying across all $x \in \mathbb{R}$, we obtain an equality of functions, and conclude that $Df = g$.

Since this is true for an arbitrary $f_n \to f, f'_n \to g$, this is true for all cases where both sequences simultaneously converge, and hence $D$ has a closed graph.

\end{proof}

\begin{problem}{Question 4}

Let $\mathcal{H} = L^2[0,1]$, the Lebesgue measurable and square-integrable functions defined on $[0,1]$. Let $K$ be a non-empty, closed, convex subset of $\mathcal{H}$. Define $P = P_K$ as the orthogonal projection of $H$ onto $K$.

4.1)

Let $x \in \mathcal{H}$. Prove that the following are equivalent:

i) There exists a unique $z \in K$ such that $\Vert x - z \Vert = \min_{y \in K} \Vert x - y \Vert$.

ii) $z \in K$ and $\langle x - z, y - z \rangle \leq 0$ for all $y \in K$.

4.2)

Let $A$ be a continuous bilinear mapping from $\mathcal{H} \times \mathcal{H} \to \mathbb{R}$ such that, for some $\alpha > 0$, we have:

$$ A(f,f) \geq \alpha \Vert f \Vert_2^2 $$

for every $f \in \mathcal{H}$. We will prove the following statement in parts:

For every $f \in \mathcal{H}$, there exists a unique $u \in K$ such that:

$$ A(u, v-u) \geq \langle f, v - u \rangle $$

for all $v \in K$.

4.2.1)

Fix a $u \in \mathcal{H}$, and prove that there exists a unique $Tu \in \mathcal{H}$ such that $A(u,v) = \langle Tu, v \rangle$ for every $v \in \mathcal{H}$. Prove that $T$ is a bounded linear mapping on $\mathcal{H}$.

4.2.2)

Fix a $\rho > 0$, $f \in \mathcal{H}$, and define a map $S_\rho: K \to K$ that sends $v \mapsto P(\rho f - \rho Tv + v)$. Prove that we may choose $\rho$ such that there exists a $0 < k < 1$ with the property that:

$$ \Vert S_\rho(v_1) - S_\rho(v_2) \Vert \leq k \Vert v_1 - v_2 \Vert $$

for all $v_1, v_2 \in K$.

4.2.3)

Conclude that for the value of $\rho > 0$ chosen in 4.2.2, that $S_\rho$ is a contraction, and therefore has a unique fixed point $u \in K$.

4.2.4)

Note that we can rewrite $\rho f - \rho T u = \rho f - \rho T u + u - u$. Then, use 4.1 to show that:

$$ \langle \rho f - \rho Tu , v - u \rangle \leq 0 $$ 

for every $v \in K$.

4.2.5)

Conclude that, for every $f \in \mathcal{H}$, there exists a unique $u \in K$ such that:

$$ A(u, v-u) \geq \langle f, v - u \rangle $$

\end{problem}

\begin{proof}[Solution]

4.1)

First, we show that if $\langle x - z, y - z \rangle \leq 0$, then we get that $\Vert x - z \Vert = \min \Vert x - y \Vert$.

We have the following sequence of equalities, for arbitrary $y$:

$$ \langle x - z, y - z \rangle = \langle x - z, y + (x - x) - z \rangle = \langle x - z, x - z \rangle + \langle x - z, y - x \rangle = \Vert x - z \Vert^2 + \langle x - z, y - x\rangle$$

Then, we have that:

$$ \langle x - z, y - z \rangle \leq 0 \implies \Vert x - z \Vert^2 + \langle x - z, y - x\rangle \leq 0 \implies \Vert x - z \Vert^2 \leq -\langle x - z, y - x\rangle $$

Since the norm is positive, we may harmlessly replace $\langle x - z, y - x \rangle $ with its absolute value. Then, by the Cauchy-Schwarz inequality, we retrieve:

$$ \Vert x - z \Vert^2 \leq \Vert x - z \Vert \Vert y - x \Vert  \implies \Vert x - z \Vert \leq \Vert y - x \Vert = \Vert x - y \Vert$$

Since this is true for all $y \in K$, including $z$ itself, we conclude that $\Vert x - z \Vert = \min_{y \in K } \Vert x - y \Vert$.

Now, suppose that $z \in K$ is such that $\Vert x - z \Vert = \min_{y \in K} \Vert x - y \Vert$. By convexity, for any $y \in K$, we may reexpress $y = (1-t)z + tw$ for at least some $w \in K, t \in [0,1]$, hence, we have that:

$$ \Vert x - z \Vert \leq \Vert x - (1-t)z + tw \Vert = \Vert x - z - t(w - z) \Vert $$

We may safely square both sides and examine the inner product instead. Thus, we have that:

$$ \langle x - z, x - z \rangle \leq \langle x - z - t(w - z), x - z - t(w - z) \rangle $$

Using the linearity and conjugate linearity of the inner product, we see that the RHS can be rewritten as:

$$   \langle x - z - t(w - z), x - z - t(w - z) \rangle = \langle x - z, x - z \rangle -  t\langle x - z, w - z \rangle - t \langle  w - z, x - z \rangle + t^2 \langle w - z, w - z \rangle$$

Hence, we have that:

$$  \langle x - z, x - z \rangle \leq \langle x - z - t(w - z), x - z - t(w - z) \rangle  \implies \langle x - z, w - z \rangle + \langle w - z, x - z \rangle \leq t \langle w - z , w - z \rangle $$

Wait, uhhhh.

4.2.1)

Let $u  \in \mathcal{H}$. By the bilinearity of $A$, we have that:
\begin{center}
\begin{tabular}{c c} $A_u : \mathcal{H} \to \mathbb{R}$ & $A_u: v \mapsto A(u,v) $\end{tabular}
\end{center}

is a linear functional on $\mathcal{H}$. Moreover, since $A$ is continuous, it is continuous in each variable, and hence $A_u$ is a continuous linear functional. Thus, since $\mathcal{H}, \mathbb{R}$ are normed linear spaces, and $A_u$ is a continuous linear operators, $A_u$ is bounded (1.63, Heil).

Since $\mathcal{H}$ is a Hilbert space, we can identify a $w_u$ such that $A_u(v) = \langle v, w_u \rangle$ by the Riesz Representation Theorem (Folland, 5.25). Since $A$ is real-valued, we can freely pick $w_u$ to be in the first or second argument due to conjugate symmetry - we will from now on use $A_u(v) = \langle w_u, v\rangle$.

So now, we may define $T: \mathcal{H} \to \mathcal{H}$ that sends $u \mapsto w_u$. Evidently, due to the bilinearity of $A$, $T$ is linear:

$$\begin{cases} \langle T(u + u'), v \rangle = A(u + u', v) = A(u, v) + A(u', v) = \langle T(u), v \rangle + \langle T(u'), v \rangle \\ \langle T(ku), v \rangle =A(ku, v) = k A(u, v) = k \langle T(u), v \rangle \end{cases} $$ 

By the Cauchy-Schwarz inequality:

$$\alpha \Vert f \Vert^2 \leq A(f,f) = \langle Tf, f \rangle \leq \Vert Tf \Vert \Vert f \Vert \implies \alpha \Vert f \Vert \leq \Vert Tf \Vert $$

4.2.2)

First of all, using the equivalent statement in 4.1, we see that:

$$ \langle \rho f - \rho Tv + v - S_\rho(v), y - S_\rho(v) \rangle \leq 0$$

for every $y \in K$.

Then, letting $v_1, v_2 \in \mathcal{H}$, we have the following statements:

$$ \begin{cases} \langle \rho f - \rho Tv_1 + v_1 - S_\rho(v_1), S_\rho(v_2) - S_\rho(v_1) \rangle \leq 0 \\ \langle \rho f - \rho Tv_2 + v_2 - S_\rho(v_2), S_\rho(v_1) - S_\rho(v_2) \rangle \leq 0 \end{cases}$$

Summing these equations then, and pulling out a factor of $-1$ from the second argument in the second equation, we find that:

$$ \langle \rho f - \rho Tv_1 + v_1 - S_\rho(v_1) -  \rho f + \rho Tv_2 - v_2 + S_\rho(v_2), S_\rho(v_2) - S_\rho(v_1) \rangle \leq 0  \implies $$

$$ \langle \rho T(v_2 - v_1) + v_2 - v_1 + S_\rho(v_2) - S_\rho(v_1), S_\rho(v_2) - S_\rho(v_1) \rangle \leq 0  \implies \langle \rho T(v_2 - v_1) + v_2 - v_1 + S_\rho(v_2) - S_\rho(v_1), S_\rho(v_1) - S_\rho(v_2) \rangle \geq 0$$

where we've used the linearity of $T$, and then multiplied through by $-1$, bringing it into the second argument.

We examine the square of the norm, to leverage the inner product.

We have that:

$$ \langle S_\rho(v_1) - S_\rho(v_2),S_\rho(v_1) - S_\rho(v_2) \rangle \leq $$

$$ \langle S_\rho(v_1) - S_\rho(v_2),S_\rho(v_1) - S_\rho(v_2) \rangle +  \langle \rho T(v_2 - v_1) + v_2 - v_1 + S_\rho(v_2) - S_\rho(v_1), S_\rho(v_1) - S_\rho(v_2) \rangle =   $$

$$  \langle  S_\rho(v_1) - S_\rho(v_2) + \rho T(v_2 - v_1) + v_2 - v_1 + S_\rho(v_2) - S_\rho(v_1), S_\rho(v_1) - S_\rho(v_2) \rangle = $$

$$ \langle \rho T(v_2 - v_1) + v_2 - v_1, S_\rho(v_1) - S_\rho(v_2) \rangle \leq \Vert \rho T(v_2 - v_1) + v_2 - v_1 \Vert \Vert  S_\rho(v_1) - S_\rho(v_2)  \Vert$$

where we add the positive quantity determined above in line 2, and the final inequality comes from the Cauchy-Schwarz inequality.

Hence, we conclude that:

$$ \Vert  S_\rho(v_1) - S_\rho(v_2)  \Vert^2 \leq \Vert \rho T(v_2 - v_1) + v_2 - v_1 \Vert \Vert  S_\rho(v_1) - S_\rho(v_2)  \Vert \implies  \Vert  S_\rho(v_1) - S_\rho(v_2)  \Vert \leq \Vert \rho T(v_2 - v_1) + v_2 - v_1 \Vert $$

But now, let's examine the right side a bit more. By the triangle inequality and the definition of the operator norm, we find that:

$$ \Vert \rho T(v_2 - v_1) + v_2 - v_1 \Vert \leq \Vert \rho T(v_2 - v_1) \Vert +  \Vert v_2 - v_1 \Vert \leq (\rho \Vert T \Vert + 1) \Vert v_2 - v_1 \Vert $$

Clearly something has gone horribly wrong.

4.2.3)

By definition then, since the $\rho$ in 4.2.2 gives rise to a $k \in (0,1)$ such that $ \Vert  S_\rho(v_1) - S_\rho(v_2)  \Vert \leq k  \Vert v_1 - v_2 \Vert$, we see that $S_\rho$ is a contraction on the metric. Hence, by the Banach fixed-point Theorem, there exists a unique fixed point $u \in K$ such that $S_\rho(u) = u$.

4.2.4)

Identifying $ \rho f - \rho T u + u$ as $x$, $P( \rho f - \rho T u + u) = z = S_\rho(u) = u$, and renaming $y$ to $v$, we see that:

$$\langle \rho f - \rho Tu + u - u, v - u \rangle \leq 0 \implies \langle \rho f - \rho Tu, v - u \rangle \leq 0$$

4.2.5)

Ok, from here, consider $\rho A(u, v - u)$, where $\rho$ is small enough such that we may find $u$, the unique fixed point associated to $S_\rho$ determined by $f$. From 4.2.1, we have that:

$$ \rho A(u, v -u) = \rho \langle Tu, v - u \rangle = \langle \rho f - \rho f + \rho T u, v - u \rangle = $$

$$ \rho \langle f, v - u \rangle + \langle - \rho f + \rho Tu, v - u \rangle$$

But, from 4.2.4, we see that:

$$ \langle - \rho f + \rho Tu, v - u \rangle  = -  \langle \rho f - \rho Tu, v - u \rangle \geq 0 $$

Hence, we conclude that:

$$ \rho \langle f, v - u \rangle \leq \rho A(u, v - u) \implies \langle f, v - u \rangle \leq A(u, v - u) $$





\end{proof}


\end{document}