\documentclass[10pt]{article}
\setlength{\parskip}{0.25\baselineskip}
\usepackage[margin=1in]{geometry} 
\usepackage{amsmath,amsthm,amssymb, graphicx, multicol, array}
\usepackage[font=small,labelfont=bf]{caption}

\newcommand{\supp}{{\text{supp}}} 
\newcommand{\bv}{{\text{BV}}}
\newcommand{\ac}{{\text{AC}}}

\newenvironment{problem}[2][]{\begin{trivlist}
\item[\hskip \labelsep {\bfseries #1}\hskip \labelsep {\bfseries #2.}]}{\end{trivlist}}

\begin{document}
 
\title{Homework \#3}
\author{Eric Tao\\
Math 237: Homework \#3}
\maketitle

\begin{problem}{Question 1}

Show that the only function $f \in L^1(\mathbb{R})$ such that $f  = f \ast f$ is $f = 0$ almost everywhere.

\end{problem}
\begin{proof}[Solution]

Let $f \in L^1(\mathbb{R})$. Suppose we have that $f = f \ast f$. We recall that the Fourier transform distributes pointwise over convolution, hence we have that:

$$ \hat{f}(\zeta ) = \widehat{f \ast f}(\zeta)  = \hat{f}(\zeta) \hat{f}(\zeta) $$

Hence, we already have that at each $\zeta$, $\hat{f} = 0, 1$, by solving the equation $\hat{f}(\zeta)^2 - \hat{f}(\zeta) = 0$. 

From Lemma 9.2.3 in Heil, we have that $\hat{f}$ is continuous. Moreover, from the Riemann-Lesbesgue lemma, since $f \in L^1$, we already know that $\hat{f} \in C_0(\mathbb{R})$. Hence, by continuity, since $\hat{f}$ may only take on values $0,1$, and that it is continuous, $\hat{f} = 0$ everywhere.

Thus, by Corollary 9.2.12 in Heil, since $\hat{f} = 0$ everywhere, it is true almost everywhere of course, and hence $f= 0$ almost everywhere.

\end{proof}

\begin{problem}{Question 6}

Suppose that $f \in \operatorname{AC}(\mathbb{T})$, that is, $1$ periodic and absolutely continuous on $[0,1]$.

6.1)

Prove that $\hat{f}'(n) = 2\pi i n \hat{f}(n)$ for all $n \in \mathbb{Z}$. Conclude that $\lim_{|n| \to \infty} n \hat{f}(n) = 0$.

6.2)

Show that if $\int_0^1 f(x) dx = 0$, then:

$$ \int_0^1 |f(x)|^2 dx \leq \frac{1}{4\pi^2} \int_0^1 |f'(x)|^2 dx $$


\end{problem}

\begin{proof}[Solution]

6.1)

First, we recall that by Corollary 6.1.5 in Heil, that since $f$ is absolutely continuous on $[0,1]$, we have that $f$ is differentiable almost everywhere, and $f' \in L^1[0,1]$. Since $f'$ is at least $L^1$, we may look at its Fourier transform:

$$ \hat{f'}(n) = \int_0^1 f'(x) \exp(-2\pi i n x) dx $$

We notice that for a fixed value of $n \in \mathbb{Z}$, that we may compute the derivative of $g =\exp(-2 \pi i nx)$ as $g' =- 2 \pi i n\exp(-2\pi i nx) $. Hence, $g$ is differentiable everywhere on $[0,1]$, and of course, since $|- 2 \pi i n\exp(-2\pi i nx)| = 2\pi n$ always, it is bounded. Hence, by Lemma 5.2.5, $g$ is Lipschitz on $[0,1]$. Thus, since Lipschitz functions are absolutely continuous (6.1.3), we may apply integration by parts on $f, g$. We have then that:

$$ \int_0^1 f'(x) \exp(-2\pi i nx ) dx = f(1) \exp(-2\pi i n) - f(0) \exp(0) - \int_0^1 f(x) [ - 2 \pi i n \exp(-2\pi i nx)] dx $$

Now, because $f$ is 1-periodic, we have that $f(0) = f(1)$. Further, $\exp(-2\pi in ) = \exp(0)$, of course, since $2\pi n$ is a multiple of $2\pi $ for any $n$. Hence, the first terms vanish.

We are left then with, after using the linearity of the integral:

$$ 2 \pi i n \int_0^1 f(x) \exp(- 2 \pi i nx) dx $$

However, we recognize the integral as exactly the Fourier transform of $f$ at $n$. Hence, we have our result, that:

$$ \hat{f'}(n) = 2 \pi in \hat{f}(n) $$

Again, by the Riemann-Lebesgue lemma, we have that since $f' \in L^1$, that $\lim_{|n| \to \infty} \hat{f'}(n) = 0 $. But, from our result, we have that:

$$ \lim_{|n| \to \infty} 2 \pi i n \hat{f}(n) = \lim_{|n| \to \infty} \hat{f'}(n) =  0 \implies \lim_{|n| \to \infty} n \hat{f}(n) = 0 $$

6.2)

First, we use the result from Heil Problem 9.3.24 (b) that the Plancherel equality holds for $f \in L^1(\mathbb{T})$. and prove that later.

Then, since $f' \in L^1(\mathbb{T})$, we have that:

$$ \sum_{n \in \mathbb{Z}} | \hat{f'}(n)|^2 = \Vert f' \Vert_2^2 = \int_0^1 |f'(x)|^2 dx $$

Now, from 6.1, we have that $\hat{f'}(n) =  2\pi i n \hat{f}(n)$ for each $n$. Hence, we have that:

$$  \sum_{n \in \mathbb{Z}} | \hat{f'}(n)|^2 =  \sum_{n \in \mathbb{Z}} 4 \pi^2 n^2| \hat{f}(n)|^2 = 4 \pi^2 \sum_{n \in \mathbb{Z}} n^2| \hat{f}(n)|^2$$

Now, we notice that since $\int_0^1 f(x) dx = 0$, that evidently, $\hat{f}(0) = 0$, as $\hat{f}(0) = \int_0^1 f(x) \exp(-2\pi i 0 x) dx = \int_0^1 f(x) dx = 0$.

Because this is 0, and for all $n \in \mathbb{Z}, n \not = 0$, we have that $|\hat{f}(n)| \leq | n \hat{f}(n)|$, we can conclude that:

$$  4 \pi^2 \sum_{n \in \mathbb{Z}} n^2| \hat{f}(n)|^2 \geq 4 \pi^2 \sum_{n \in \mathbb{Z}} | \hat{f}(n) |^2 $$

Finally, applying Plancherel's equaity again for $f$, as $f \in L^1(\mathbb{T})$, we see that:

$$ 4 \pi^2 \sum_{n \in \mathbb{Z}} | \hat{f}(n) |^2  = 4 \pi^2 \Vert f \Vert_2^2 = 4 \pi^2 \int_0^1 |f(x)|^2 dx $$

And rewriting all of these inequalities together, we have that:

$$ 4\pi^2 \int_0^1 |f(x)|^2 \leq \int_0^1 |f'(x)|^2 dx \implies \int_0^1 |f(x)|^2 \leq \frac{1}{4\pi^2} \int_0^1 |f'(x)|^2 dx$$

as desired.

Now, we return to proving 9.3.24.

First, we wish to show that if $f \in L^1(\mathbb{T})$, and $\hat{f} \in L^2(\mathbb{Z})$, then $f \in L^2(\mathbb{T})$.

% fill in this argument

Next, we want to show that the Plancherel equality holds, either both sides being finite or infinite. If $f \in L^2 \cap L^1$, then we're fine, by Corollary 9.3.14.

Then, we assume $f \not \in L^2(\mathbb{T})$ but $f \in L^1(\mathbb{T})$. Then, by the previous result, we have that $\hat{f} \not \in L^2(\mathbb{Z})$. Hence, we have that both sides of the Plancherel equality are infinite, and we are done.




\end{proof}

\begin{problem}{Question 12}

Fix a $g \in L^2(\mathbb{R})$. Let $k \in \mathbb{Z}$, and define the operator $T_k$ on $g$ that sends $T_k (g(x)) \mapsto g(x - k)$. Prove that the family $\{ T_k g \}_{k \in \mathbb{Z}}$ is an orthonormal sequence if and only if $\sum_{k \in \mathbb{Z}} | \hat{g}(\zeta - k)|^2 = 1$ almost everywhere.

\end{problem}

\begin{proof}[Solution]

First, suppose $T_k g$ is orthonormal. Consider the function $h(\zeta) = \sum_{k \in \mathbb{Z}} | \hat{g}(\zeta - k)|^2$. Since $k$ varies over all $\mathbb{Z}$, evidently, $h$ is 1-periodic. Moreover, since we know that the Fourier transform brings $L^2$ to $L^2$, we must have that $\hat{g} \in L^2$, and hence, $|\hat{g}|^2$ is in $L^1$. Since this integral converges, by the integral test then, so too must $h$ at each point $\zeta \in [0,1]$. In fact, since the integral over $[0,1]$ is actually:

$$\int_0^1 |h(\zeta)| = \int_0^1 \sum_{k} | \hat{g}(\zeta - k) |^2 d\zeta = \sum_{k} \int_0^1 | \hat{g}(\zeta - k )|^2 d\zeta = \sum_{k} \int_k^{k+1} | \hat{g}(\zeta)|^2 d\zeta = \int_{\mathbb{R}} |\hat{g}(\zeta)|^2 d\zeta$$

Where we justify interchanging the sum and integral by taking partial sums, and noting that of course, for the partial sums, their integral over $[0,1]$ is bounded by above by the integral of $|\hat{g}|^2$ over all of $\mathbb{R}$. Thus, $h \in L^1(\mathbb{T})$.

Then, consider the Fourier transform of $h$. We have that:

$$ \hat{f}(n) = \int_0^1 h(\zeta) \exp(-2 \pi i n \zeta) d\zeta = \int_0^1 \left (\sum_{k} |\hat{g}(\zeta - k)|^2 \right) \exp(-2 \pi i n \zeta) d\zeta $$ 

Here, since we know that $\exp(-2\pi i nk) = 1$ for every $k \in \mathbb{Z}$ regardless of $n$, we may scale the phase such that $\zeta = \zeta - k$ when distributing the exponential. Thus, we have that this equals:

$$ \int_0^1 \sum_{k} \left( |\hat{g}(\zeta - k)|^2 \exp(-2 \pi in (\zeta - k) \right) d\zeta $$

Playing the same trick with interchanging sums and integrals, by using the fact that the exponential has modulus 1, we see that this is simply an integral over all of $\mathbb{R}$ again, and doing some algebra with the modulus, we see:

$$  \int_0^1 \sum_{k} \left( |\hat{g}(\zeta - k)|^2 \exp(-2 \pi in (\zeta - k) \right) d\zeta = \int_{\mathbb{R}} | \hat{g}(\zeta)|^2 \exp(-2\pi i n \zeta) d\zeta = \int_{\mathbb{R}} [\hat{g}(\zeta) \exp(-2\pi i n \zeta)] \overline{\hat{g}(\zeta)} d\zeta$$

Since $\hat{g} \in L^2$, of course $\overline{\hat{g}} \in L^2$, and so too is $\hat{g} \exp(-2\pi i n \zeta)$. Thus, we may view this as, in the lanugage of Heil  9.2.21, if $M_{a}$ is the modulation operator that sends $f(x)$ to $\exp(2\pi i a x) f(x)$, as:

$$ \int_{\mathbb{R}} [\hat{g}(\zeta) \exp(-2\pi i n \zeta)] \overline{\hat{g}(\zeta)} d\zeta = \langle M_{-n} \hat{g}, \hat{g} \rangle $$

Now, by Theorem 9.4.6, the Fourier transform on $L^2$ functions obeys the Parseval Identity. Moreover, by Heil 9.2.21, if $T_b$ is the translation operator that sends $f(x) \mapsto f(x-b)$, then $\hat{T_a f} = M_{-a} \hat{f}$. (We will prove this separately). Thus, we have that:

$$ \langle M_{-n} \hat{g} \hat{g} \rangle = \langle T_n g , g \rangle $$

By hypothesis then, this is exactly $\delta_n^0$, as we had that $\{ T_k g \}$ was orthonormal. We know that the function $1$ has $1$ as the zeroth coefficient, and 0 else. By the uniqueness theorem (9.3.12) then, we must have that $h = 1$ almost everywhere.

Now, suppose that  $\sum_{k \in \mathbb{Z}} | \hat{g}(\zeta - k)|^2 = 1$ almost everywhere. Consider $\langle T_k g, T_{k'} g \rangle $ for some $k \in \mathbb{Z}$. Again, by the Parseval identity, we have that:

$$ \langle T_k g, T_{k'} g \rangle = \langle M_{-k} \hat{g}, M_{-k'} \hat{g} \rangle = \int_{\mathbb{R}} \exp(-2\pi i \zeta k) \hat{g}(\zeta) \overline{ \exp(-2\pi i \zeta k') \hat{g}(\zeta)} d\zeta =\int_{\mathbb{R}} \exp(-2\pi i \zeta (k - k')) |\hat{g}(\zeta)|^2 $$

Now, playing the same trick to convert this into $h$ and changing the integral into an integral over $[0,1]$ by slicing across intervals like $[k, k+1]$, we have this to be equal to:

$$ \sum_{n}  \int_0^1 \exp(-2\pi i \zeta (k - k' -n))| \hat{g}(\zeta - n)|^2 d\zeta = \int_0^1 \sum_{n}\left(  \exp(-2\pi i \zeta (k - k' -n))| \hat{g}(\zeta - n)|^2 \right) d\zeta  $$

Factoring out the exponential, and using again that integer shifts are equivalent to multiplying by $1$, we have that:

$$ \int_0^1 \sum_{n}\left( | \hat{g}(\zeta - n)|^2 \right)  \exp(-2\pi i \zeta (k- k') ) d\zeta  = \hat{h}(k - k') = \delta_{k-k'}^0$$

where we get that $\hat{h}(k-k')$ aligns with $\delta_{k - k'}^0$ almost everywhere, but since we're working over $\mathbb{Z}$, this is true everywhere as the only sets of measure 0 are empty.

Hence, we have that $ \{ T_k g \}$ is orthonormal.

Now, we need to prove that $\hat{T_a f} = M_{-a} \hat{f}$, for $f \in L^1$, as the $L^2$ part comes from the continuity of these operators, and convergence in the $L^2$ Fourier transform sense.

Well, computing this directly, we see that by a change of variables $x \mapsto x +a$

$$ \hat{T_a f}(\zeta) = \int_{\mathbb{R}} f(x-a) \exp(-2\pi i n \zeta x ) dx = \int_{\mathbb{R}} f(x) \exp(-2\pi i n \zeta (x + a)) dx =$$

$$\exp(-2\pi i n \zeta a) \int_{\mathbb{R}} f(x) \exp(-2\pi i n \zeta x) dx =\exp(-2\pi i n \zeta a) \hat{f}(\zeta) = M_{-a} \hat{f}(\zeta) $$

as desired.



\end{proof}

\begin{problem}{Question 14}

Let $p(x) = \chi_{[0,1)}(x)$, $h(x) = \chi_{[0,1/2)}(x) - \chi_{[1/2,1)}(x)$. Let $j, k \in\mathbb{Z}$, and define $I_{jk} = [2^{-j}k, 2^{-j}(k+1))$. Further define the following functions:

$$ \begin{cases} p_{jk} = 2^{j/2} p (2^{j} x - k) \\ h_{jk} = 2^{j/2} h (2^{j} x - k) \end{cases}$$

14.1)

Prove that $\{ h_{jk} \}$ is an orthonormal sequence in $L^2$.

14.2)

For each fixed $j \in \mathbb{Z}$, prove that $\{ p_{jk} \}_{k \in \mathbb{Z}}$ is an orthonormal sequence in $L^2$.  

14.3)

Fix a $j \in \mathbb{Z}$. Let $g_j$ be any step function, constant on each interval $I_{jk}$ for $k \in \mathbb{Z}$. Show that we may express $g_j(x) = g_{j-1}(x) + r_{j-1}(x)$, where

$$ r_{j-1}(x) = \sum_{k \in \mathbb{Z}} a_{j-1}(k)h_{j-1, k}(x) $$

for some coefficients $a_{j-1}(k)$ and some step function $g_{j-1}(x)$, constant on intervals $I_{j-1,k}$.

14.4)

Fix a $J \geq 0$. Consider the set:

$$ \{ p_{Jk} : 0 \leq k \leq 2^J - 1 \} \cup \{ h_{j,k} : j \geq J, 0 \leq k \leq 2^j - 1 \}$$

Prove that this set is an orthonormal sequence in $L^2[0,1]$.

14.5)

For $f \in L^2[0,1]$, and a fixed $J \geq 0$, show that we may find $g_j$ step functions for $j \geq J$, such that they are constant on each $I_{j,k}$, and that $g_j$ approximates $f$ in the $L^2$ norm.

Use this result and the result of 14.4 to show that the set in 14.4 is an orthonormal basis for $L^2[0,1]$.


\end{problem}

\begin{proof}[Solution]

14.1)

%Fix a $j \in \mathbb{Z}$, and consider the family of $\{ h_{jk} \}$ iterating across $k$.

%First, we wish to show that $\Vert h_{jk} \Vert_2 = 1$, regardless of the choice of $j,k$. Fix a choice of $j,k$. We see that, by the definition of $h$, that $h(2^jx - k)$ takes on $1$ on $I_{j+1, 2k}$, as we can see at the endpoints, $h(2^j 2^{-j - 1} 2k - k ) = h(k - k ) = h(0)= 1$. Further, $I_{j+1, 2k}$  has a length $2^{-j-1}$, and we can see that if $h(x) = 1$ on $[0,1/2)$, then $h(2^{j}x)$ takes on $1$ on an interval of length $2^{-j-1}$, from $[0,2^{-j-1})$. Translating this interval over by $k$, we see that this is exactly $I_{j+1, 2k}$. In a similar argument, we see that $h$ takes on $-1$ on $I_{j+1, 2k+1}$. Hence, we may rewrite $h_{jk}$ as:

%$$ h_{jk} = 2^{j/2} (\chi_{I_{j+1, 2k}} - \chi_{I_{j+1, 2k+1}} ) $$

%Thus, we can look at the $L^2$ norm of this function. We have that:

%$$ \Vert h_{jk} \Vert_2^2 = \int_{\mathbb{R}} | h_{jk} | ^2 = \int_{I_{j+1, 2k}} 2^j \chi_{I_{j+1, 2k}} + \int_{I_{j+1, 2k+1}} 2^j \chi_{I_{j+1, 2k}} = 2^j (2^{-j-1}+ 2^{-j-1}) = 1 $$

%where we've used the fact that the square of a characteristic function is itself, the measure of $I_{j+1, 2k}$ is equal to $2^{-j-1}$.

%Now, we want to take $h_{jk}, h_{jk'}$, and look at $\langle h_{jk}, h_{jk'} \rangle$. Computing directly, and dropping the complex conjugate as this family is strictly real-valued, we see that:

%$$ \langle h_{jk}, h_{jk'} \rangle = \int_{\mathbb{R}} 2^{j/2} (\chi_{I_{j+1, 2k}} - \chi_{I_{j+1, 2k+1}} )  2^{j/2} (\chi_{I_{j+1, 2k'}} - \chi_{I_{j+1, 2k'+1}} ) = $$

%$$ 2^j \int_{\mathbb{R}}  \chi_{I_{j+1, 2k}}\chi_{I_{j+1, 2k'}} - \chi_{I_{j+1, 2k}} \chi_{I_{j+1, 2k'+1}} - \chi_{I_{j+1, 2k+1}} \chi_{I_{j+1, 2k'}} + \chi_{I_{j+1, 2k+1}} \chi_{I_{j+1, 2k'+1}} $$

%By definition though, $I_{j+1, 2k} \cap I_{j+1, 2k'} = [2^{-j-1}2k, 2^{-j-1}(2k+1)) \cap [2^{-j-1}k', 2^{-j-1}(2k'+1))$, which is non-empty if and only if $k = k'$, as $j, k$ are integers, and so we have endpoints exactly at multiples of $2^{-j-1}$. We may disregard this case, as this means $h_{jk} = h_{jk'}$, and we wish to look at distinct elements of this family. Hence, if we have distinct elements, we have that $\chi_{I_{j+1, 2k}}\chi_{I_{j+1, 2k'}}, \chi_{I_{j+1, 2k+1}} \chi_{I_{j+1, 2k'+1}}$ are identically 0.

%Then, we need only look at the terms $ - \chi_{I_{j+1, 2k}} \chi_{I_{j+1, 2k'+1}} - \chi_{I_{j+1, 2k+1}} \chi_{I_{j+1, 2k'}}$. However, in a similar fashion to the previous terms, these can be non-0 if and only if either $2k = 2k' + 1$, or $2k' = 2k + 1$. However, $k, k' \in \mathbb{Z}$, and hence there are no values of $k, k'$ such that these are non-0.

%Thus, if we have two distinct elements $h_{jk}, h_{jk'}$, we have that  $\langle h_{jk}, h_{jk'} \rangle = 0$. Hence, the family $\{ h_{jk} \}_{k \in \mathbb{Z}}$ for fixed $j$ is a orthonormal sequence in $L^2$.

Let $j, k \in\mathbb{Z}$, and take two $h_{j,k}, h_{j', k'}$. We first notice by the definition of $h_{jk}$, that we may rewrite $h_{jk}$ in terms of characteristic functions of $I_{j+1, k}$:

$$ h_{jk} = 2^{j/2} (\chi_{I_{j+1, 2k}} - \chi_{I_{j+1, 2k+1}} ) $$


Consider $\langle h_{j,k}, h_{j', k'} \rangle$, where I will drop the complex conjugate as these are real-valued functions. In terms of characteristic functions of $I_{j,k}$, we have that:

$$  \langle h_{j,k}, h_{j', k'} \rangle = \int_{\mathbb{R}}  h_{j,k}  h_{j', k'} = \int_{\mathbb{R}}  2^{j/2} (\chi_{I_{j+1, 2k}} - \chi_{I_{j+1, 2k+1}} ) 2^{j'/2} (\chi_{I_{j'+1, 2k'}} - \chi_{I_{j'+1, 2k'+1}} )$$

Without loss of generality, suppose $j \geq j'$, and switch labels if this is not true. First, suppose $j = j'$. Then, looking at the integrand and factoring out the $2^{j/2}$, we have that:

$$  (\chi_{I_{j+1, 2k}} - \chi_{I_{j+1, 2k+1}} )  (\chi_{I_{j+1, 2k'}} - \chi_{I_{j+1, 2k'+1}} ) = \chi_{I_{j+1, 2k}}\chi_{I_{j+1, 2k'}} - \chi_{I_{j+1, 2k+1}}\chi_{I_{j+1, 2k'}} - \chi_{I_{j+1, 2k}}\chi_{I_{j+1, 2k'+1}} + \chi_{I_{j+1, 2k+1}} \chi_{I_{j+1, 2k'+1}}  $$

First, suppose $k = k'$. Then, we have that this is equal to:

$$ \chi_{I_{j+1, 2k}} +  \chi_{I_{j+1, 2k+1}} $$

as the square of a characteristic function is itself, and the cross terms vanish as $I_{j,k}$ and $I_{j, k'}$ are disjoint unless $k = k'$.

Then, in the case $j = j', k = k'$, we have that the integral evaluates as $2^j \int_{\mathbb{R}}  \chi_{I_{j+1, 2k}} +  \chi_{I_{j+1, 2k+1}} = 2^j( 2^{-j-1} + 2^{ -j -1} ) =  1$.

Else, suppose $k \not = k'$. Then, we have that $\chi_{I_{j+1, 2k}}\chi_{I_{j+1, 2k'}},   \chi_{I_{j+1, 2k+1}} \chi_{I_{j+1, 2k'+1}}$ vanish. Looking at the remaining terms, we have:

$$  - \chi_{I_{j+1, 2k+1}}\chi_{I_{j+1, 2k'}} - \chi_{I_{j+1, 2k}}\chi_{I_{j+1, 2k'+1}} $$

which, again, may be non-0 if and only if either $2k + 1 = 2k'$ or $2k = 2k' + 1$. However, since $k, k' \in \mathbb{Z}$, this is impossible. Hence, for all $k \not = k', j = j'$, this integral vanishes as desired.

Now, suppose $j < j'$. Due to the nested structure of dyadic intervals, for each $I_{j',2k'}, I_{j', 2k'+1}$, these fit exactly within a single $I_{j, l}$ for some $l$. Hence, we can say that at least one of:

$$ \begin{cases} \chi_{I_{j+1, 2k}}[\chi_{I_{j'+1, 2k'}} - \chi_{I_{j'+1, 2k'+1}}] \\ \chi_{I_{j+1, 2k+1}} [\chi_{I_{j'+1, 2k'}} - \chi_{I_{j'+1, 2k'+1}}] \end{cases}$$

vanishes.

If both vanish, then we are done, as the integral disappears as desired. WLOG, suppose the first term survives. Then, we have our integral as:

$$ 2^{j/2 + j'/2} \int_{\mathbb{R}} \chi_{I_{j+1, 2k}}\chi_{I_{j'+1, 2k'}} -  \chi_{I_{j+1, 2k}}\chi_{I_{j'+1, 2k'+1}} $$

Evidently, then, the first term takes on $1$ on an interval of measure $2^{-j'-1}$ and the second term takes on $-1$ on an interval of measure $2^{-j'-1}$. Hence, they cancel out, and we have that for all $j < j'$, $k \in \mathbb{Z}$, that $ \langle h_{j,k}, h_{j', k'} \rangle$. Hence, we have that:

$$  \langle h_{j,k}, h_{j', k'} \rangle = \delta_j^{j'} \delta_k^{k'}$$

as desired.

14.2)

In a similar fashion to 14.1, but maybe slightly cleaner, we do the same procedure. Fix a choice of $j \in \mathbb{Z}$. First, we reexpress $p_{jk}$ in terms of characteristic functions. We see that, in analogy to 14.1, that

$$  p_{jk} = 2^{j/2} \chi_{I_{jk}} $$

as we can see that $2^j x - k$ takes $2^{-j}k$ to $0$ and $2^{-j} k+1$ to $1$, hence takes $[2^{-j}k, 2^{-j}k+1)$ to $[0,1)$ as this is linear in $x$. 

Then, considering $\langle p_{jk}, p_{jk'} \rangle$, dropping the complex conjugate again, we see that:

$$ \langle p_{jk}, p_{jk'} \rangle = \int_{\mathbb{R}} 2^{j/2}  \chi_{I_{jk}} 2^{j/2}  \chi_{I_{jk'}} = 2^{j} \int_\mathbb{R}  \chi_{I_{jk}}  \chi_{I_{jk'}}$$

By definition, we may look at $I_{jk} \cap I_{jk'}$. We have that:

$$ I_{jk} \cap I_{jk'} =  [2^{-j}k, 2^{-j}(k+1)) \cap  [2^{-j}k', 2^{-j}(k'+1))$$

Since $j, k \in \mathbb{Z}$, these intervals have endpoints at multiple of $2^{-j}$, and hence these have overlap if and only if $k = k'$. Hence, we can say that:

$$ 2^{j} \int_\mathbb{R}  \chi_{I_{jk}}  \chi_{I_{jk'}} = 2^{j} \int_\mathbb{R}  \chi_{I_{jk}}  \delta_k^{k'} = 2^j | 2^{-j}(k+1) - 2^{-j} k| \delta_k^{k'} = 2^j 2^{-k} \delta_k^{k'} = \delta_k^{k'} $$

Therefore, $\{ p_{jk} \}_{k \in \mathbb{Z}}$ is an orthonormal sequence in $L^2$ for fixed $j \in \mathbb{Z}$.

14.3)

Fix a $j$, and a step function $g_j$, and consider the dyadic intervals one larger, $I_{j-1, k}$. Of course, each $I_{j-1,k}$ may be broken into $I_{j,2k} \cup I_{j, 2k+1}$, as we may split the interval as:

$$[2^{-j+1}k, 2^{-j+1}(k+1)) = [2^{-j} 2k, 2^{-j}(2k+2))) = [ 2^{-j} 2k, 2^{-j} 2k + 1) \cup [ 2^{-j} 2k+1, 2^{-j} 2k+2) $$

Call $g_j(I_{j, 2k}) = c$ and $g_j(I_{j, 2k+1}) = d$. Define $g_{j-1}(I_{j-1, k}) = \frac{c+d}{2}$ and define $a_{j-1}(k) = \frac{c - d}{2}$. Then, on $I_{j, 2k},I_{j, 2k+1} $, we have that:

$$ \begin{cases} g_{j-1}(I_{j, 2k}) + r_{j-1}(I_{j, 2k}) = \frac{c + d}{2} + \frac{c - d}{2} \chi_{I_{j, 2k}}(I_{j, 2k}) = c \\ g_{j-1}(I_{j, 2k+1}) + r_{j-1}(I_{j, 2k+1}) = \frac{c + d}{2} - \frac{c - d}{2} \chi_{I_{j, 2k}}(I_{j, 2k}) = d \end{cases} $$

We may continue this construction for each $I_{j, k}$, and determine $g_{j-1}, r_{j-1}, a_{j-1}$ on the entire real line, with $g_{j-1}$ a step function constant on $I_{j-1, k}$, $a_{j-1}(k)$ coefficients, and $r_{j-1} = \sum a_{j-1}(k) h_{j-1, k} $

14.4)

We have already shown $\{ p_{Jk} : 0 \leq k \leq 2^J - 1 \}$ to be orthonormal for $k \in \mathbb{Z}$, so of course this is orthonormal on its own, and similarly for $\{ h_{j,k} : j \geq J, 0 \leq k \leq 2^j - 1 \}$ for $j,k \in \mathbb{Z}$. So, we only need to show that $\langle p_{Jk}, h_{j, k'} \rangle = 0$ for any $j \geq J, 0 \leq k \leq 2^J - 1, 0 \leq k' \leq 2^j - 1 $

Thus, we have that:

$$\langle  p_{J,k}, h_{j, k'} \rangle = \int_{\mathbb{R}}p_{J,k} h_{j, k'} = \int_{\mathbb{R}} 2^{J/2} \chi_{I_{Jk}}[ 2^{j/2} (\chi_{I_{j+1, 2k'}} - \chi_{I_{j+1, 2k'+1}} ) = $$

$$ 2^{J/2 + j/2} \int_{\mathbb{R}}  \chi_{I_{Jk}} \chi_{I_{j+1, 2k'}} -  \chi_{I_{Jk}} \chi_{I_{j+1, 2k' + 1}} $$

Now, in the same vein as 14.1 and the nesting of dyadic intervals, since $j+1 > j \geq J$, we must have that either both $I_{j+1, 2k' + 1}, I_{j+1, 2k' + 1} \subseteq I_{Jk}$ or neither are. If neither are, then this integral vanishes, and we are done.

Suppose then that both terms survive. Then, the first term takes on $1$ on the interval $ I_{j+1, 2k'}$, and the second term takes on $-1$ on the interval $I_{j+1, 2k' + 1}$. Since these intervals have the same measure, the integral vanishes due to the opposite sign. Hence, we have that $\langle p_{Jk}, h_{j, k'} \rangle = 0$ for any $p_{Jk}, h_{j, k'}$ and thus, $\{ p_{Jk} : 0 \leq k \leq 2^J - 1 \} \cup \{ h_{j,k} : j \geq J, 0 \leq k \leq 2^j - 1 \}$ is a orthonormal sequence.

14.5)

Since $[0,1]$ is compact, any step functions on dyadic intervals is aleady square integrable, including $p_{jk}, h_{jk}$. 

%First, suppose $f$ is non-negative. Then, define $g_j$ as the step function on $I_{j,k}$ such that $g_j(I_{j,k})$ takes on the $\operatorname{ess\,inf}_{x \in I_{j,k}} f(x)$, the essential infimum of $f$ on $I_{j,k}$. Evidently, the sequence $g_j$ converges to $f$ pointwise almost everywhere. Moreover, since $f \in L^2$, the sequence of $\Vert g_j \Vert_2$ is bounded above, and non-decreasing, hence convergent to $\Vert f \Vert_2$ via the monotone convergence sequence, as $\Vert f \Vert_2$ must be the supremum, due to the pointwise almost everywhere convergence.

%Now, from 14.3), we have already seen that we may express $g_j = g_{j-1} + \sum a_{j-1}(k) h_{j-1, k}$. We may perform this step recursively then to some $g_J$, a step function on $I_{J,k}$ and a sum of $a_{l} h_{l, k_l}$ for $J \leq l \leq j$ and $ 0 \leq k_l \leq 2^l - 1$. Hence, we see that $g_j$ is expressable as a linear combination of the sequence found in 14.4). In fact, we see that this argument works for all $j \geq J$.

%Moreover, from the construction of these step functions, we see that $g_{j-1}$ is nested within

First, suppose $f$ is at least continuous. Construct $g_j$ in the following way. Define $g_J = \sum_{k} \langle f, p_{J,k} \rangle p_{J,k}$. Then, define $g_{j+1} = g_{j} + \sum_{k} \langle f, h_{j, k} \rangle h_{j,k}$. 



\end{proof}


\begin{problem}{Question 16}

Let $\phi$ be a non-0 function in $L^2(\mathbb{R})$. For any $f \in L^2(\mathbb{R})$, define $V_\phi f$ via:

$$ V_\phi(f)(x, \zeta) = \int_{\mathbb{R}} f(t) \overline{\phi(t-x)} \exp(-2 \pi i t \zeta) dt $$

For $a, b \in \mathbb{R}$, let $T_a$ be the translation operator that sends $T_a(f(x)) \mapsto f(x -a )$ and let $M_b$ the modulation operator that sends $M_b(f(x)) \mapsto \exp(2 \pi bx)f(x)$.

16.1)

Prove that for each $f\in L^2$, $V_\phi f$ is uniformly continuous on $\mathbb{R}^2$, and that $\lim_{|(x, \zeta)| \to \infty} V_\phi f = 0$.

16.2)

Recall that the Schwarz space $\mathcal{S}(\mathbb{R})$ is defined as:

$$ \mathcal{S}(\mathbb{R}) = \{ f \in C^\infty(\mathbb{R}) : x^m f^{(n)}(x) \in L^\infty(\mathbb{R}) \text{ for all } m, n \geq 0 \}$$

Prove that if $f \in \mathcal{S}(\mathbb{R})$, then $V_\phi \in S(\mathbb{R}^2)$.

16.3)

Prove that $V_\phi$ acts as an isometry from $L^2(\mathbb{R})$ to $L^2(\mathbb{R}^2)$, and that $\Vert V_\phi f \Vert_{L^2(\mathbb{R}^2)} = \Vert \phi \Vert_{L^2(\mathbb{R})} \Vert f \Vert_{L^2(\mathbb{R})}$ for every $f \in L^2$.

16.4)

Show that the operator $V_\phi^*$ defined by:

$$ V_\phi^* F(t) - \Vert \phi\Vert_2^{-2} \iint_{\mathbb{R}^2} F(x, \zeta) \exp(2\pi i \zeta t) \phi(t - x) dx d\zeta $$

takes $L^2(\mathbb{R}^2)$ to $L^2(\mathbb{R})$, and that for each $f \in L^2(\mathbb{R})$, we can make sense of the following inversion formula:

$$ f(t) = \Vert \phi \Vert_2^{-2} \iint_{\mathbb{R}^2} V_\phi f(x, \zeta) \exp(2 \pi i \zeta t) \phi(t - x) dx d\zeta $$

\end{problem}

\begin{proof}[Solution]

\end{proof}

\end{document}