\documentclass[10pt]{article}
\setlength{\parskip}{0.25\baselineskip}
\usepackage[margin=1in]{geometry} 
\usepackage{amsmath,amsthm,amssymb, graphicx, multicol, array}
\usepackage[font=small,labelfont=bf]{caption}

\newcommand{\supp}{{\text{supp}}} 
\newcommand{\bv}{{\text{BV}}}
\newcommand{\ac}{{\text{AC}}}

\newenvironment{problem}[2][]{\begin{trivlist}
\item[\hskip \labelsep {\bfseries #1}\hskip \labelsep {\bfseries #2.}]}{\end{trivlist}}

\begin{document}
 
\title{Homework \#2}
\author{Eric Tao\\
Math 237: Homework \#2}
\maketitle

\begin{problem}{Question 4}

Let $X$ be the space of $C^1$ functions on $[0,1]$ such that $f(0) = 0$. Define the bilinear function:

$$\langle f, x \rangle = \int_0^1 f'(x) \overline{g'(x)} dx$$

4.1

Prove that $\mathcal{H}$, the completion of $X$, is a reproducing kernel Hilbert space.

4.2

Prove that $K(x,y) = \min(x,y)$.

\end{problem}
\begin{proof}[Solution]

4.1)

We see quickly that due to the linearity of the integral, that this bilinear function is linear in at least the first argument. Moreover, if we consider $\langle g, f \rangle$, denoting $g = g_1 + i g_2, f = f_1 + i f_2$, we see that:

$$ \langle g, f \rangle = \int_0^1 g' \overline{f'} = \int_0^1 (g_1' + i g_2') \overline{f_1' + i f_2'} = \int_0^1 (f_1' g_1' + f_2' g_2') + i(g_2' f_1' - g_1' f_2') =$$

$$ \int_0^1 (f_1' g_1' + f_2' g_2')   + i \int_0^1 (g_2' f_1' - g_1' f_2') = \overline{  \int_0^1 (f_1' g_1' + f_2' g_2')   - i \int_0^1 (g_2' f_1' - g_1' f_2') } = $$

$$\overline{ \int_0^1  (f_1' g_1' + f_2' g_2')   - i (g_2' f_1' - g_1' f_2') } =\overline{\int_0^1 (f_1' + i f_2') \overline{(g_1' + i g_2')}  } = \overline{ \langle f, g \rangle }$$

as desired. Finally, of course, we have that $\langle f, f \rangle = \int f' \overline{f'} = \int |f'|^2 \geq 0 $ since $|f'|^2 \geq 0$ regardless of $x, f$. Furthermore, we have that this integral is 0 if and only if $|f'|^2 = 0$ for all $x$. But, since $f(0) = 0$, this implies that $\int_0^1 |f'|^2 =  0$ if and only if $f = 0$. Hence, we can see that $X$ equipped with this function is a pre-Hilbert space.

First, restrict ourselves to $f \in X$, and fix any $x \in [0,1]$. We wish to show that $|f(x)| \leq C_x \sqrt{ \langle f, f \rangle }$ for some constant $C_x > 0$.

By the Fundamental Theorem of calculus, we have that

$$ | f(x) | \leq \left| \int_0^x f'(y) dy \right| \leq \int_0^x | f'(y)| dy \leq \int_0^1 | f'(y)| dy $$

where the final inequality is reasonable since $|f'(y)| \geq 0$ for all $y \in [0,1]$, and $[0,x] \subseteq [0,1]$.

Now, using the fact that the constant function $1$ is a $L^2$ function on $[0,x]$, we have that by H\"older's inequality

$$ \int_0^1 | f'(y) * 1| dy  \leq \sqrt{\int_0^1 |f'(y)|^2} \sqrt{\int_0^1 |1|^2} =  \sqrt{\int_0^1 |f'(y)|^2} = \sqrt{\langle f, f \rangle }$$

and therefore, for any $x$, we may choose $C_x = 1$.

Now, we need to make an argument for the completion of $X$. Let $f_n \in \mathcal{H} \setminus X$, and take a sequence $f_n \to f$ such that $f_n \in X$ for each $n$. We notice that since $|f(x)| \leq \Vert f \Vert$, then $| f_m(x) - f_n(x) | \leq \Vert f_m - f_n \Vert$ for all $x$, as $f_n$ must be Cauchy. Hence, $\Vert f_m - f_n \Vert_u \leq \Vert f_m - f_n \Vert$, and $f_m, f_n$ actually uniformly converges to $f$, and thus pointwise converges.

Thus, we can say that since $|f_n(x)| \leq \Vert f_n \Vert$ for each $n$, due to both pointwise convergence on the left, and the continuity of the norm in a Hilbert space, we may take the limit of both sides and conclude that $|f(x)| \leq \Vert f \Vert$. 

4.2)

Fix any $x \in [0,1]$. We notice that the function $L_x: \mathcal{H} \to \mathbb{R}$ that sends $f \mapsto f(x)$ is a bounded linear functional, linear due to how evaluating the sum of functions is just the sum of the evaluations, and bounded due to 4.1. Hence, by the Riesz Representation Theorem, there exists a unique element $k_x \in \mathcal{H}$ for each $L_x$ such that:

$$ L_x(f) = \langle f, k_x \rangle$$

We claim that the function $k_x(y) = \max(x,y)$ satisfies this condition. To see that this belongs to $\mathcal{H}$, we consider the family of functions $g_n \in C^1$ given by:

$$ g_n(y) = \begin{cases} y & y \leq x - 1/n \\ x &   x+ 1/n \leq y \leq 1 \end{cases} $$

and for $x - 1/n < y < x + 1/n$, we smoothly join the point $(x-1/n, x-1/n)$ to the point $(x+1/n, 1)$. Evidently, $g_n = k_x$ on $[0,x-1/n] \cup [x+1/n, 1]$. Hence, we have that:

$$ \Vert g_n - k_x \Vert^2 = \int_0^1 (g_n' - k_x')^2 = \int_{[x-1/n, x+1/n]} (g_n' - k_x')^2  $$

Since by definition, $g_n', k_x'$ range from $0, 1$ since they both are $1$ on $[0, x-1/n]$ and $0$ on $[x+ 1/n, 1]$, we must have that $(g_n' - k_x')^2 \leq 1^2 = 1$. Hence, we have that:

$$ \int_{[x-1/n, x+1/n]} (g_n' - k_x')^2 \leq 2/n $$

Taking $n \to \infty$, we see that $\Vert g_n - k_x \Vert^2 \to 0$, and hence, since $\mathcal{H}$ is complete under its norm, $k_x \in \mathcal{H}$.

Now, let $f$ be any function in $C^1$. Then, we may evaluate, since $k'_x(y) = y$  on $[0,x)$ and $x$ on $(x,1]$, with a discontinuity at $x$, a set of measure 0, we have that: 

$$\langle f, k_x \rangle = \int_0^1 f' \overline{k_x'} = \int_0^x f' = f(x) - f(0) = f(x)$$

as desired. Similarly, if $f \in \mathcal{H} \setminus C^1$, then we may take a sequence of $f_n \in C^1$ such that $f_n \to f$ in $\mathcal{H}$. By the work done in the previous section, $f_n \to f$ pointwise as well and due to the continuity of the inner product in both arguments, because we have that $f_n(x) = \langle f_n, k_x \rangle$, taking the limit on both sides as $n \to \infty$, we have that:

$$ f(x) = \langle f, k_x \rangle $$

Since the kernel $K(x,y)$ is the function $K: \mathcal{H} \times \mathcal{H}  \to \mathcal{H}$ that sends $(x,y) \mapsto k_x(y)$, from the work done with $k_x$, we can see that the kernel is exactly $K(x,y)= \max(x,y)$ as desired.

\end{proof}

\begin{problem}{Question 6}

Let $Y = l^1(\mathbb{N})$, and define $X = \{ f \in Y : \sum n |f(n)| < \infty \}$, equipped with the $l^1$ norm.

6.1

Prove that $X$ is a proper dense subspace of $Y$, and hence $X$ is not complete. 

6.2

Define $T: X \to Y$ that sends $f(n) \mapsto nf(n)$. Show that $T$ is a closed map, but not bounded.

6.3

Let $S = T^{-1}$. Prove that $S: Y \to X$ is bounded, surjective, but not open.

\end{problem}

\begin{proof}[Solution]

6.1)

Without too much trouble, we can see $X$ is a vector subspace of $Y$. Of course, the sequence of all 0s satisfies this condition. Further, if $f, g \in X$, then, looking at partial sums, we have that:

$$ \sum_{i=1}^k i|f + g(i) | = \sum_{i=1}^k i | f(i) + g(i) | \leq \sum_{i=1}^k i | f(i)| + i|g(i)| = \sum_{i=1}^k i |f(i)| + \sum_{j=1}^\infty j | g(j) |$$  

Since the right side converges as $ k \to \infty$, due to $f, g \in X$, and all quantities being positive, so too must $\sum i |f + g(i)| < \infty$.

Similarly, we can see scalar multiplication as:

$$ \sum_{i=1}^k i |af(i)| = \sum_{i=1}^k i |a| |f(i)| = |a| \sum_{i=1}^k i |f(i)|$$

and again, since the right side converges, so too must the left.

Furthermore, this inclusion is proper. Consider the sequence $\{ 1/n^2 \} \in Y$. Clearly, via the integral test, the series $\sum_{n=1}^\infty 1/n^2 < \infty$. However, on the other hand, $\sum_{n=1}^\infty n 1/n^2 = \sum_{n=1}^\infty 1/n$ diverges, and hence this sequence is not in $X$.

Lastly, we want to show this subspace is dense. Let $g(n)$ be a sequence in $Y \setminus X$. Because we have that $\sum_{n=1}^\infty |g(n)| < \infty$, for every $\epsilon > 0$, there exists $M$ such that for all $m > M$, $\sum_{n=m}^\infty |g(n)| < \epsilon$. Thus, construct the sequence $f$ such that $f(n) = g(n)$ for all $n \leq M$, and $0$ for all $n > M$. Clearly, $f$ resides in $X$, as only finitely many terms are non-0, hence $\sum_{n=1}^\infty n |f(n)|$ is a finite sum, and therefore finite.

Without too much trouble, by definition, we have that:

$$ \Vert f - g \Vert_1 = \sum_{n=1}^\infty |f(n) - g(n)| = \sum_{n=M+1} | -g(n)| < \epsilon $$

Thus, since $\epsilon$ can be as small as we want, for every open ball around our $g \in Y \setminus X$, we may find an $f$ in such an open ball. Hence, $X$ is dense.

However, $X$ cannot be complete then. As above, take a sequence of $f_n \to g$, where $\Vert f_n - g \Vert_1 < 1/n$. Evidently, this is a Cauchy sequence from the argument above, and it converges to an element in $Y \setminus X$.

6.2)

It should be clear that $T$ must be a closed map. If $f_k \to f$, and $T f_k$ is convergent, we show that $Tf_k \to Tf$. Indeed, we see that:

$$ \Vert Tf - Tf_k \Vert_1 = \sum_{n=1}^\infty | Tf(n) - Tf_k(n) | = \sum_{n=1}^\infty |n f(n) - n f_k(n)| = \sum_{n=1}^\infty |n| |f(n)  - f_k(n) | $$

Now, since $f_k \to f$, we have that $\sum_{n=1}^\infty | f(n) - f_k(n) | \to 0$. In particular, that means we can choose $k$ such that for each n, $| f(n) - f_k(n) | < \frac{1}{n} \frac{\epsilon}{2^n}$. Under this choice, the above inequality becomes:

$$ \sum_{n=1}^\infty |n| |f(n)  - f_k(n) | \leq \sum_{n=1}^\infty |n| \frac{1}{n} \frac{\epsilon}{2^n} = \epsilon$$

Thus, we have that $Tf_k \to Tf$, and thus $T$ is closed.

However, it is clear that $T$ is not bounded. Let $f_i$ be the family of sequences with $0$ everywhere, except for at the $i$-th position. Evidently, $\Vert f_i \Vert_1 = 1$. However, $\Vert Tf_i \Vert_1 = i$, and since in this family, we can choose $i$ arbitrarily large, $T$ cannot be bounded.

6.3)

Clearly, $S$ is surjective, since as we saw, $T$ is a map that takes $f \in X$ to $Tf$, and evidently, $S \circ T(f) = f \in X$. Hence, we can always find a sequence in $Y$ that maps to any sequence in $X$.

Moreover, $S$ must be bounded. If we look at the action of $S$, $S$ sends $f(n) \in Y$ to $\frac{1}{n} f(n) \in X$. Evidently then, we have that:

$$ \Vert Sf \Vert_1 = \sum_{k=1}^\infty | 1/k f(k) | \leq \sum_{k=1}^\infty | f(k)| = \Vert f \Vert_1$$

and so $S$ is bounded, at least by $1$.

However, $S$ is certainly not open. By Folland, we may say that $S$ is open if and only if, for $B$ the open ball of radius 1 around 0 in $Y$, $S(B)$ contains a ball centered around $0$ in $X$.

Fix an $i \in \mathbb{N}$. We can then consider the family of sequences $g_i$ such that $g(i) = 2/i$, and 0 otherwise. Evidently, we may find a sequence as close to 0 as we want, since $\Vert g_i \Vert_1 = 2/i$. However, for no $i$ does there exists a $y \in B$ such that $Sy = g_i$, since under the map $T$, we see that $Tg_i(n)$ is $2$ when $n=i$ and $0$ otherwise, outside of $B$. Hence, $S$ is not open.

\end{proof}

\begin{problem}{Question 16}

Define $\operatorname{Lip}[0,1] = \{ f \in C[0,1] : f \text{ is Lipschitz } \}$. For each $n \geq 1$, define $F_n = \{ f \in C[0,1] : |f(x) - f(y)| \leq n | x - y| \text{ for all } x, y \in [0,1] \}$.

16.1

Prove that for each $n \geq 1$, $F_n$ is a closed, nowhere dense subset of $\operatorname{Lip}[0,1]$.

16.2

Conclude that  $\operatorname{Lip}[0,1]$ is a countable union of nowhere dense subsets of $C[0,1]$.

\end{problem}

\begin{proof}[Solution]

Obviously, $F_n \subset \operatorname{Lip}[0,1]$, as we may just take $n$ to be a Lipschitz constant.

Consider $U_n = (F_n)^c$, the compliment of $F_n$. We wish to show that this set is open.

Let $f \in U_n$. Evidently then, there exists $x, y \in [0,1]$ such that $| f(x) - f(y) | > n | x - y| $. Fix a choice $x_0,y_0$ (where, evidently, $x_0 \not = y_0$ as the inequality always holds then), without loss of generality, we may assume $f(x_0) > f(y_0)$ and otherwise swap labels, and call $d = f(x_0) - f(y_0) - n | x_0 - y_0|$.

Now, let $g$ be in the open ball around $f$ with radius at most $d/2$. From the condition of being in the ball of radius at most $d/2$, we must have that $g(y_0) > g(x_0)$, since $d < | f(x_0) - f(y_0)|$, and hence, from the supremum norm, $g(y_0) \geq f(y_0) - d/2  > [f(x_0) + f(y_0)]/2$ and similarly, $g(x_0) \leq f(x_0) + d/2 < [f(x_0) + f(y_0)]/2$. Thus, we have that at $x_0, y_0$ for $g$, that:

$$ g(y_0) - g(x_0) \geq f(y_0) - d/2  - f(x_0) - d/2  = f(y_0) - f(x_0) - d = n |x_0 - y_0| $$

Since we can arbitrarily shrink the ball to have radius smaller than $d/2$, say $d/3$, this shows that we may find an open ball around any $f \in U_n$, hence $U_n$ is open, and thus $F_n$ is closed.

Now, we need to show that $F_n$ is nowhere dense. Let $f \in F_n$. Let $\epsilon > 0$, and suppose without loss of generality that $f(0) \geq f(\epsilon/n+1)$. Consider the continuous function $g_\epsilon$ such that it takes on $\epsilon$ at 0, joined linearly to $0$ at $\epsilon/n+1$, and constantly $0$ otherwise. Of course, this is continuous, with $\Vert g_\epsilon \Vert_u = \epsilon$. Furthermore, we can consider the function $f + g_\epsilon$, and in particular, compute $f + g_\epsilon(0) - [f + g_\epsilon(\epsilon/n+1)]$. We have that:

$$ f + g_\epsilon(0) - [ f+ g_\epsilon(\epsilon/n+1)] = f(0) - f(\epsilon/n+1) + g_n(0) - g_n(\epsilon/n+1) \geq g_n(0) - g_n(\epsilon/n+1) = \epsilon > \epsilon/n+1$$

where we use the fact that $f(0) \geq f(\epsilon/n+1)$ for one inequality, and the fact that $n \geq 1$ for the other. Hence, $f + g_n$ is not Lipschitz. If $f(0) < f(\epsilon/n+1)$, the same argument holds by taking $-g_\epsilon$.

Since the choice of $f$ was arbitrary, this implies that $F_n$ contains no non-trivial open subsets, and thus, as it is its own closure, $F_n$ is nowhere dense.

Clearly, $\operatorname{Lip}[0,1] = \cup F_n$. We have that $F_n \subseteq \operatorname{Lip}[0,1]$ for each $n$, since we may take $n$ as a Lipschitz constant. Hence, $\cup F_n \subseteq \operatorname{Lip}[0,1]$.

On the other hand, let $g \in \operatorname{Lip}[0,1]$. Then, there exists a constant $K \geq 0$ such that $| g(x) - g(y) | \leq K | x - y | $ for all $x, y \in [0,1]$. Then, since this inequality clearly holds for $K' \geq K$, taking any integer $k \geq K$, we see that $g \in F_k$. Hence, $\operatorname{Lip}[0,1] \subseteq \cup F_n$, and we are done.

\end{proof}

\begin{problem}{Question 17}

Let $f$ be a nonnegative Lebesgue measurable function defined on $\mathbb{R}$. Assume that for all $g \in L^2(\mathbb{R})$, $fg \in L^1(\mathbb{R})$. Prove that $T_f(g) = \int_{\mathbb{R}} gf$ is a bounded linear functional on $L^2$, and conclude that $f \in L^2$.

\end{problem}

\begin{proof}[Solution]

First, we use the $\sigma$-finiteness of $\mathbb{R}$, and consider the compact sets $E_i = [ -i, i ]$. Additionally, for each $E_i$, we define $f_n$ in the following fashion:

$$ f_n(x) = \begin{cases} 0 & x \not \in E_n \\ \min(f(x), n) & x \in E_n \end{cases}$$

Certainly, if $f$ is measurable, so too are each $f_n$, as if we look at, for $a > 0$, $\{ f_n > a \}$, if $ a \geq n$, then this is exactly $\{ f \geq n \}$, if $ a < n$, then it aligns exactly with $ \{ f > a \}$. And lastly, if $a \leq 0$, then the preimage is the entire codomain, hence still measurable.

Moreover, each $f_n$ is clearly in $L^2$, as it has compact support, contained within $E_n$, and is bounded above by $n$. Hence, we can say that:

$$ \Vert f_n \Vert_2^2 = \int_{\mathbb{R}} f_n = \int_{E_n} f_n \leq \int_{E_n} n = 2n^2 < \infty $$

Now, consider the family of linear functionals determined as:

\begin{align*}
T_n: & L^2(\mathbb{R}) \to \mathbb{R} \\
T_n: & g \mapsto \int f_n g
\end{align*}

where this is clearly linear in terms of $g$, due to the linearity of the integral and distributivity of mutliplication.

%Since $f_n \leq f$ everywhere, we have that $|f_n g | \leq |fg|$ everywhere. Hence, we have that $\int f_n g \leq \int fg < \infty$, due to $fg \in L^1$, and so $f_n g \in L^1$.

By H\"older's inequality then, since $f_n, g \in L^2$, we have that:

$$ \Vert f_n g \Vert_1 \leq \Vert f_n \Vert_2 \Vert g \Vert_2$$

However, we've already computed $\Vert f_n \Vert_2 \leq 2n^2$. Hence, we have that:

$$ | T_n(g) | = \left| \int f_n g \right| \leq \int |f_n g | = \Vert f_n g \Vert_1 \leq \Vert f_n \Vert_2 \Vert g \Vert_2 \leq 2n^2 \Vert g \Vert_2 $$

Hence, for each $n$, $T_n$ is a bounded linear functional, with norm at most $2n^2$.

Now, fix any $g \in L^2$, and consider $| T_n(g)|$ again. For every $n$, we have that due to the fact that $f_n \leq f$ everywhere:

$$ | T_n(g) | = \left| \int f_n g \right | \leq \int | f_n g |  \leq \int |f||g| = \Vert fg \Vert_1 < \infty $$

Since this is true for all $n$, we may conclude that for all $g \in L^2$, that $\sup_{n} | T_n(g) | < \infty$. Since $L^2$ is a Banach space, we may now apply the Uniform Boundedness Principle and conclude that

$$ \sup_{n} \Vert T_n \Vert < \infty$$

Thus, we need only show that $ \Vert T_f \Vert = \sup_{n} \Vert T_n \Vert$. Of course, $f_ng \to fg$ almost everywhere, by the definition of $f_n$, and further, $fg$ itself is integrable and then so is $|fg|$, and $| f_n g | \leq |fg|$ by the definition of $f_n$. Hence, we may use an application of the Dominated Convergence Theorem, as well as picking an arbitrary $g \in L^2$ with $\Vert g \Vert = 1$, to see that:

%$$\lim_{n \to \infty} T_n(g) = \lim_{n \to \infty} \int |f_n g| = \int |fg| = T_f(g)$$

%Since this is true for all $g \in L^2$, we may restrict ourselves to where $\Vert g \Vert_2  = 1$, and so we have that:

$$ | T_f(g) | = \left| \int fg \right| = \left|  \lim_{n \to \infty} \int f_n g \right| = \left| \lim_{n \to \infty} T_n(g) \right| = \lim_{n \to \infty} | T_n(g)| \leq \lim_{n \to \infty} \Vert T_n \Vert \leq \sup_{n} \Vert T_n \Vert$$

Since this is true for all $g \in L^2$ with unit norm, we may conclude then that $\Vert T_f \Vert \leq \sup_{n} \Vert T_n \Vert$. However, earlier, we already showed that $|T_n(g)| \leq |T_f(g)|$ due to the fact that $f_n \leq f$ everywhere. Hence, we have that $\Vert T_f \Vert = \sup_{n} \Vert T_n \Vert < \infty$, and hence $T_f$ is a bounded linear functional. 

Now, by the Riesz Representation Theorem, since $L^2$ is a Hilbert space, this implies that we may find an $h \in L^2$ such that $T_n(g) = \langle g, h \rangle$. In particular then, we have that:

$$ T_n(g) = \int fg = \langle g, h \rangle = \int gh \implies \int fg - \int gh = 0 \implies \int g(f-h) = 0 \implies f = h \; a.e.$$

Where because this works for all $g \in L^2$, we are free to pick $g$ to be non-negative, and $f = h$ follows because this single function $h$ must work regardless of the choice of $g$. Hence, $f$ may be identified as a member of an equivalence class in $L^2$.
\end{proof}


\begin{problem}{Question 19}

Suppose $\mathcal{B}$ is a Banach space, and let $S$ be a closed proper subspace. Fix some $f_0 \not \in S$. Show that there exists a continuous linear functional $\gamma$ such that $\gamma(f) = 0$ for all $f \in S$, and $\gamma(f_0) = 1$. Moreover, show that we may choose the linear functional such that $\Vert \gamma \Vert = 1/d$, where $d$ is the distance from $f_0$ to $S$.

\end{problem}

\begin{proof}[Solution]

First, we notice that $d$, the distance between $f_0, S$ must be positive as otherwise, we could find a descending series of $s_n$ such that $\Vert f_0 - s_n \Vert \to 0$. In such a case, $S$ being closed would imply that $f_0 \in S$, a contradiction.

Now, consider the vector subspace $T = \operatorname{span}\{ S, f_0 \}$. We claim that for every $t \in T$, there exists a unique representation $t = s_{t} + a_{t} f_0$,  for $s_{t} \in S$ and $a_{t}$ a scalar.

Suppose we had that $s_{t} + a_{t} f_0 = t = s'_t + a'_t f_0$. Then, we must have that $(s_t - s'_t) + (a_t - a'_t) f_0 = 0$. Since $f_0 \not \in S$, we must have that $(s_t - s'_t) = 0, (a_t - a'_t) f_0 = 0$. Hence, we must have that $s_t = s'_t$, and that $a_t - a'_t = 0 \implies a_t = a'_t$.

Now, define a functional $\lambda: T \to F$ that sends $t \mapsto a_t$, where $a_t$ comes from the representation above. Clearly, we see from the representation that this must be linear:

$$ \lambda( t + t') = \lambda(s_t + a_t f_0 + s_{t'} + a_{t'} f_0) = \lambda((s_t + s_{t'}) + (a_t + a_{t'}) f_0) = a_t + a_{t'} = \lambda(t) + \lambda(t')$$

where we've used the fact that $S$ is a subspace to conclude $s_t + s_{t'} \in S$, and:

$$ \lambda(bt) = \lambda(b(s_t + a_t f_0)) =\lambda(bs_t + ba_t f_0) = b a_t = b\lambda(t)$$

and we omit $\lambda(0)  = 0 $, as of course, $ 0 = 0 + 0f_0$. 

Also, certainly, $\lambda |_{S} = 0$, since for any $s \in S$, we have the representation in $T$, $s = s + 0 f_0$, and of course then, $\lambda(s) = 0$. Finally, in a similar fashion, we have that $f_0 = 0 + f_0$, and hence $\lambda(f_0) = 1$.

Thus, we need only show that $\lambda$ is bounded, with norm $\frac{1}{d}$.

Let $t$ be an arbitrary non-0 vector in $T$, with non-0 component of $f_0$, as of course, if that were true, $\lambda(t) = 0$. We notice that, since $t = s_t + a_t f_0 = a_t( a_t^{-1} s_t + f_0)$, that:

$$ \Vert t \Vert = \Vert a_t( a_t^{-1} s_t + f_0) \Vert = | a_t | \Vert (a_t^{-1} s_t + f_0 \Vert$$

and since $a_t^{-1} s_t \in S$ due to being a scalar multiple, we must have that $\Vert a_t^{-1} s_t + f_0 \Vert \geq d$, as $d$ is the infimum of $\Vert f_0 - s \Vert$, and we can always replace $s$ with $-s_t$, as necessary. Hence, using the fact that $a_t = \lambda(t)$, we have that:

$$ \Vert t \Vert \geq |a_t| d = |\lambda(t) | d \implies |\lambda(t)| \leq \frac{1}{d} \Vert t \Vert $$

Since this is true for all $t$, including when $a_t = 0$ due to the observation earlier, we have that $\lambda$ is bounded, and $\Vert \lambda \Vert_{T^*} \leq \frac{1}{d}$ by definition.

Now, again, since $d$ is the infimum of $\Vert f_0 - s \Vert$ for $s \in S$, we may find $\{ s_n \} \subset S$ such that $\Vert f_0 - s_n \Vert \to d$. Then, considering the action of lambda on each of these, we have that:

$$ 1 = | \lambda(f_0) | = | \lambda(f_0 - s_n) |   \leq \Vert \lambda \Vert_{T^*} \Vert f_0 - s_n \Vert_T  $$

Taking the limit as $n \to \infty$ then, we retrieve the inequality

$$ 1 \leq \Vert \lambda \Vert_{T^*} d \implies \Vert \lambda \Vert_{T^*} \geq \frac{1}{d}$$

Hence, $\Vert \lambda \Vert_{T^*} = \frac{1}{d}$.

Now, by Corollary 2.2 in Heil, since $\mathcal{B}$ is a Banach space, $T$ a subspace of $\mathcal{B}$, $\lambda \in T^*$, there exists a $\gamma \in \mathcal{B}^*$ such that $\gamma(f_0) = \lambda(f_0) = 1$, $\gamma(s) = \lambda(s) = 0$ for all $s \in S$, and $\Vert \gamma \Vert_{\mathcal{B}^*} = \Vert \lambda \Vert_{T^*} = \frac{1}{d}$ as desired.
\end{proof}

\end{document}